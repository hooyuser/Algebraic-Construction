\chapter{Commutative Ring}
\section{Basic Concepts}
A commutative ring $R$ is a commutative $R$-algebra and accordingly a commutative $\mathbb{Z}$-algebra. Furthermore we have a categorical isomorphism 
\[
    \mathsf{CRing}\cong \mathbb{Z}\text{-}\mathsf{CAlg}.
\]



\begin{definition}{Noetherian Commutative Ring}{}
    Let $R$ be a commutative ring. We say $R$ is \textbf{Noetherian} if one of following conditions holds:
    \begin{enumerate}[(i)]
        \item $R$ as an $R$-module is Noetherian.
        \item Every ideal of $R$ is finitely generated.
        \item Every prime ideal of $R$ is finitely generated.
        \item Every ascending chain of ideals of $R$ is eventually constant. That is, if $I_1\subseteq I_2\subseteq I_3\subseteq\cdots$ is a chain of ideals of $R$, then there exists $n\in\mathbb{N}$ such that $I_n=I_{n+1}=\cdots$.
    \end{enumerate}
\end{definition}

\begin{proposition}{Properties of Noetherian rings}{}
    \begin{enumerate}[(i)]
        \item If $R$ is a Noetherian ring, then the polynomial ring $R[X_1,X_2,\cdots,X_n]$ is also Noetherian.
        \item If $R$ is a Noetherian ring, then the  formal power series ring $R[[X_1,X_2,\cdots,X_n]]$ is also Noetherian.
    \end{enumerate}
\end{proposition}


\subsection{Ideals}
\begin{definition}{Ideal}{}
    Let $R$ be a ring. A subset $I\subseteq R$ is called an \textbf{ideal} if
    \begin{enumerate}[(i)]
        \item $I$ is a subgroup of $(R,+)$.
        \item $I$ is closed under multiplication, i.e. $a\in I$ and $b\in R$ implies $ab\in I$.
    \end{enumerate}
\end{definition}


\begin{proposition}{Ideal as Submodule}{}
    Let $R$ be a ring and $I\subseteq R$ be a subset of $R$. Then $I$ is an ideal of $R$ if and only if $I$ is a submodule of $R$ as an $R$-module.
\end{proposition}


\begin{definition}{Prime Ideal}{}
    Let $R$ be a commutative ring. An ideal $I\subseteq R$ is called \textbf{prime} if  and 
    \begin{enumerate}[(i)]
        \item $I\neq R$, i.e. $I$ is a proper ideal.
        \item $ab\in I\implies a\in I\text{ or }b\in I$, i.e. there exist no two elements in $R$ whose product is in $I$ but neither of them is in $I$.
    \end{enumerate}    
\end{definition}

\begin{proposition}{Prime Ideal Equivalent Definition}{}
    Let $R$ be a commutative ring and $I\subseteq R$ be an ideal. Then $I$ is prime if and only if $R/I$ is an integral domain.
\end{proposition}

\begin{definition}{Maximal Ideal}{}
    Let $R$ be a commutative ring. An ideal $I\subseteq R$ is called \textbf{maximal} if
    \begin{enumerate}[(i)]
        \item $I\ne R$, i.e. $I$ is a proper ideal.
        \item There exists no ideal $J\subseteq R$ such that $I\subsetneq J\subsetneq R$.
    \end{enumerate}
\end{definition}

\begin{proposition}{Maximal Ideal Equivalent Definition}{}
    Let $R$ be a commutative ring and $I\subseteq R$ be an ideal. Then $I$ is maximal if and only if $R/I$ is a field.  
\end{proposition}

\begin{proposition}{}{}
    If $R$ is a ring and $I$ an ideal of $R$ such that $I \neq R$, then $R$ contains a maximal ideal $\mathfrak{m}$ such that $I \subset \mathfrak{m}$.
\end{proposition}

\begin{prf}
    Let $\mathcal{A}$ be the set of ideals of $R$ not equal to $R$, ordered by inclusion. We must show that whenever $\mathcal{C}$ is a chain in $\mathcal{A}$ it has an upper bound in $\mathcal{A}$, since then the result follows immediately from Zorn's lemma. So let's take such a chain $\mathcal{C}$.

Let $I=\bigcup_{J \in \mathcal{C}} J$. Now suppose $x_1, x_2$ are in $I$. Then there are $J_1, J_2$ in $\mathcal{C}$ such that $x_i \in J_i$. Either $J_1 \subseteq J_2$ or $J_2 \subseteq J_1$. Without loss of generality, we assume the former follows. Then $x_1 \in J_2$, so $x_1+x_2 \in J_2 \subset I$. Also if $a \in R$ then $a x_i \in J_2 \subseteq I$ for each $i$. It follows that $I$ is an ideal.

It now just remains to check that $I \neq A$. But $1 \notin J$ for each $J \in \mathcal{C}$, so $1 \notin I$ and $I \neq R$ as required.
\end{prf}

\begin{corollary}{}{}
    Every non-unit lies in a maximal ideal.
\end{corollary}

\begin{prf}
    Let $R$ be a commutative ring. If $x\notin R^{\times}$, then $(x)\ne R$. By the proposition above, there exists a maximal ideal $\mathfrak{m}$ such that $(x)\subseteq \mathfrak{m}$.
\end{prf}

\begin{definition}{Ideal generated from subset}{}
    Let $R$ be a commutative ring and $\mathcal I(R)$ be the set of all ideals of $R$. Suppose $S\subseteq R$ be a subset. The \textbf{ideal generated by $S$}, denoted by $(S)$, is the smallest ideal of $R$ containing $S$, i.e. 
    \[
        (S)=\bigcap_{\substack{ I\in \mathcal I(R)\\S\subseteq I}}I=\left\{\sum_{i=1}^n r_is_i\mid n\in\mathbb{Z}_{+},r_i\in R,s_i\in S\right\}.
    \]
    If $S=\{a_1,\dots,a_n\}$, we write 
    \[
        (S)=(a_1,\dots,a_n)=\left\{\sum_{i=1}^n r_ia_i\midv  r_i\in R\right\}.
    \]
\end{definition}

\begin{definition}{Ideal Operations}{}
    \begin{enumerate}[(i)]
        \item Sum: $$I+J=\left\{a+b\mid a\in I,b\in J\right\}=\left(I\cup J\right),$$
        $$
        \sum_{t \in T} I_t=\left\{a_{t_1}+ \cdots +a_{t_n}\mid n\in\mathbb{Z}_{+},t_i\in T,a_{t_i}\in I_{t_i}\right\}.
        $$
        \item Product: $$IJ=\left\{\sum_{i=1}^n a_ib_i\midv n\in\mathbb{Z}_{+},a_i\in I,b_i\in J\right\}=\left(\{ab\mid a\in I,b\in J\}\right).$$
        \item Power: $I^0=R$,
        \[
            I^n=\underbrace{I\cdots I}_{n\text{ times}}=\left(\{a^n\midv a\in I\}\right), 
            \]
        \item Radical: \[
            \sqrt{I} = \left\{ r \in R \mid r^n \in I \text{ for some } n \in \mathbb{Z}_{+} \right\} = \bigcap_{\substack{\mathfrak{p} \in \mathrm{Spec} R \\ I \subseteq \mathfrak{p}}} \mathfrak{p}
            \].
    \end{enumerate}
\end{definition}

\begin{proposition}{}{}
    Let $R$ be a commutative ring and $S$ be a subset of $R$. Then 
    $$(S)=\sum_{s\in S}(s).$$
\end{proposition}

\begin{prf}
    \begin{align*}
        \sum_{s \in S} (s)&=\left\{a_{s_1}+ \cdots +a_{s_n}\mid n\in\mathbb{Z}_{+},s_i\in S,a_{s_i}\in (s_i)\right\}\\
        &=\left\{r_1s_{1}+ \cdots +r_ns_{n}\mid n\in\mathbb{Z}_{+},s_i\in S,r_i\in R\right\}\\
        &=(S).
    \end{align*}
\end{prf}


\begin{proposition}{Properties of Ideal Operations}{}
    \begin{enumerate}[(i)]
        \item $(I\cap J)^2 \subseteq I J \subseteq I \cap J \subseteq I+J$
        \item ${I} \cap({J}+{K}) \supseteq {I} \cap {J}+{I} \cap {K}$
        \item ${I} ({J}+{K}) = {I}  {J}+{I}  {K}$
        \item $$
        \begin{gathered}
        J\sum_{t \in T} I_t=\sum_{t \in T} J I_t.
        \end{gathered}
        $$
        \item $I(J K)=(I J) K$
        \item $(a)^n=(a^n)$
        \item $I^0 \supseteq \sqrt{I} \supseteq I \supseteq I^2 \supseteq I^3 \supseteq \cdots$
        \item $\sqrt{\sqrt{I}} = \sqrt{I}$,
        \item $\sqrt{I^n}=\sqrt{I}$, $\sqrt{I J}=\sqrt{I \cap J}=\sqrt{I} \cap \sqrt{J}$
    \end{enumerate}
\end{proposition}

\begin{prf}
    \begin{enumerate}[(i)]
        \item Since $\{ab\mid a\in I,b\in J\}\subseteq I\cap J$, we see $IJ=\left(\{ab\mid a\in I,b\in J\}\right)\subseteq I\cap J$. Also we can check $I \cap J \subseteq I \cup J\subseteq (I \cup J)=I+J$.
        \item[(vi)] If $x\in(a)^n$, then $x=r_1(r_2a)^n=r_1r_2^na^n\in(a^n)$. If $y\in(a^n)$, then $y=ra^n\in(a)^n$.
    \end{enumerate}
\end{prf}


\begin{proposition}{}{}
    Let $I$ and $J$ be ideals of a commutative ring $R$ and $\mathfrak{p}$ be a prime ideal of $R$. Then 
    \[
        I\cap J\subseteq \mathfrak{p}\iff IJ\subseteq \mathfrak{p}\iff I\subseteq \mathfrak{p}\text{ or }J\subseteq \mathfrak{p}.
    \]
\end{proposition}

\begin{prf}
    We have the following chain of implications:
    \begin{itemize}
        \item $I\cap J\subseteq \mathfrak{p}\implies IJ\subseteq \mathfrak{p}$. Note that $IJ\subseteq I\cap J$. The result follows immediately.
        \item $IJ\subseteq \mathfrak{p}\implies I\subseteq \mathfrak{p}\text{ or }J\subseteq \mathfrak{p}$. Assume $IJ\subseteq \mathfrak{p}$. Suppose $I\subsetneq \mathfrak{p}$ and $J\subsetneq \mathfrak{p}$. Then there exist $a\in I-\mathfrak{p}$ and $b\in J-\mathfrak{p}$. Since $\mathfrak{p}$ is prime, $ab\in IJ\subseteq \mathfrak{p}$ implies $a\in \mathfrak{p}$ or $b\in \mathfrak{p}$, which is a contradiction. Hence $I\subseteq \mathfrak{p}$ or $J\subseteq \mathfrak{p}$.
        \item $I\subseteq \mathfrak{p}\text{ or }J\subseteq \mathfrak{p}\implies I\cap J\subseteq \mathfrak{p}$. Note that $I\cap J\subseteq I$. The result follows immediately.
    \end{itemize}
\end{prf}


\begin{definition}{Radical Ideal}{}
    An ideal $I$ is called a \textbf{radical ideal} if $I=\sqrt{I}$.
\end{definition}

\begin{proposition}{Radical Ideal Equivalent Definition}{}
    Let $R$ be a commutative ring and $I\subseteq R$ be an ideal. Then $I$ is radical if and only if $R/I$ is reduced.
\end{proposition}

\begin{definition}{Nilradical}{}
    The \textbf{nilradical} of $R$, denoted by $\mathfrak{N}_R$, is the radical ideal $\sqrt{0}$ consisting of all the nilpotent elements of $R$. We have
    \[
        \mathfrak{N}_R=\sqrt{0}=\left\{ r \in R \mid r^n=0 \text{ for some } n \in \mathbb{Z}_{+} \right\} = \bigcap_{\substack{\mathfrak{p} \in \mathrm{Spec} R }} \mathfrak{p}
            \]
\end{definition}



\begin{proposition}{Properties of Radical Ideal}{}
    \begin{enumerate}[(i)]
        \item For any ideal $I$, $\sqrt{0}\subseteq \sqrt{I}$.
        \item $\sqrt{I}$ is the smallest radical ideal containing $I$.
        \item $\sqrt{\mathfrak{p}^n}=\sqrt{\mathfrak{p}}=\mathfrak{p}$ for any prime ideal $\mathfrak{p}$, which means prime ideals are radical.
        \item Suppose the natural projection $\pi: R\to R/I$ induces a bijection between the set of ideals of $R$ containing $I$ and the set of ideals of $R/I$, denoted by $\tilde{\pi}:\mathcal{I}(R)\to\mathcal{I}(R/I)$. Then $\tilde{\pi}$ maps $\sqrt{I}$ to $\mathfrak{N}_{R/I}$.
        \item A commutative ring $R$ is reduced if and only if $\mathfrak{N}_R=(0)$. 
    \end{enumerate}
\end{proposition}

In summary, we have the following chain of inclusions:
\[
\left\{\text{maximal ideals of }R\right\} \subseteq \left\{\text{prime ideals of }R\right\} \subseteq \left\{\text{radical ideals of }R\right\} \subseteq \left\{\text{ideals of }R\right\}.
\]
\begin{proposition}{Quotient Preserves Radical, Prime, Maximal Ideals}{}
    Let $R$ be a commutative ring and $I\subseteq R$ be a proper ideal. Then we have bijections between the following sets:
    \begin{align*}
        \left\{\text{ideals of }R\text{ containing }I\right\}&\longleftrightarrow\left\{\text{ideals of }R/I\right\}\\
        J&\longmapsto J/I
    \end{align*}
    The ideal $J\supseteq I$ is radical, prime, or maximal if and only if $J/I$ is radical, prime, or maximal respectively.
\end{proposition}


\subsection{Prime Elements}
\begin{definition}{Divisibility}{}
    Let $R$ be a commutative ring and $a,b\in R$. We say $a$ \textbf{divides} $b$ if there exists $c\in R$ such that $b=ac$, denoted by $a\mid b$. If $a\mid b$. $a$ is called a \textbf{divisor} of $b$, and $b$ is called a \textbf{multiple} of $a$.
\end{definition}


\begin{proposition}{}{}
    Let $R$ be a commutative ring.
    \begin{enumerate}[(i)]
        \item $a \mid b\iff(b) \subseteq (a)$.
        \item $u\in R^\times \iff (u) = R  \iff \forall r\in R,\,u\mid r$.
    \end{enumerate}
\end{proposition}

\begin{definition}{Prime Element}{}
    Let $R$ be a commutative ring. An element $a\in R$ is called \textbf{prime} if
    \begin{enumerate}[(i)]
        \item $a\ne 0$.
        \item $a\notin R^\times$, i.e. $a$ is not a unit.
        \item $a\mid bc\implies a\mid b\text{ or }a\mid c$.
    \end{enumerate}
\end{definition}



\begin{proposition}{Prime Element and Prime Ideal}{}
    Suppose $R$ is a commutative ring and $a\in R$. Then
    \[
        a\text{ is prime }\iff (a)\text{ is a nonzero prime ideal}.
    \]
\end{proposition}

\begin{prf}
    \begin{align*}
        a\text{ is prime }\iff &a\ne 0\text{ and }a\notin R^\times\text{ and }a\mid bc\implies a\mid b\text{ or }a\mid c\\
        \iff &(a)\ne 0\text{ and }(a)\ne R^\times\text{ and }bc\in (a)\implies b\in (a)\text{ or }c\in (a)\\
        \iff &(a)\text{ is a nonzero prime ideal}.
    \end{align*}
\end{prf}

\subsection{Local Commutative Ring}

\begin{definition}{Local Commutative Ring}{local_commutative_ring}
    Let $R$ be a commutative ring. Then the following are equivalent:
    \begin{enumerate}[(i)]
        \item $R$ is a local ring.
        \item $R$ has a unique maximal ideal.
        \item $R$ has a maximal ideal $\mathfrak{m}$ and $R - \mathfrak{m}= R^{\times}$.
        \item $R$ is not the zero ring and for every $x \in R$, $x\in R^{\times}$ or $1-x\in R^{\times}$.
        \item $R$ is not the zero ring and if $\sum_{i=1}^n r_i\in R^{\times}$, then there exist some $i$ such that $r_i\in R^{\times}$. 
        \item $R$ is not the zero ring and the sum of any two non-units in $R$ is a non-unit.
    \end{enumerate}
\end{definition}

\section{Integral Domain}

\begin{definition}{Associate}{}
    Let $R$ be an integal domain. Two elements $a,b\in R$ are called \textbf{associates} if one of the following equivalent conditions holds:
    \begin{enumerate}[(i)]
        \item $a=ub$ for some $u\in R^\times$.
        \item $a\mid b$ and $b\mid a$, i.e. $(a)=(b)$.
    \end{enumerate}
\end{definition}

If $R$ is a general commutative ring, then we only have the implication $(\mathrm i)\implies (\mathrm{ii})$. The converse is not true in general. For example, in $\mathbb{C}[x,y,z]/(x-xyz)$, $\overline{x}\mid \overline{xy}$ and $\overline{xy}\mid \overline{x}$, but there exists no unit $u$ such that $\overline{x}=u\overline{xy}$.

Associatedness can also be described in terms of the action of $R^\times$ on $R$ via multiplication: two elements of $R$ are associates if they are in the same $R^\times$-orbit.
\begin{definition}{Irreducible Element}{}
    Let $R$ be an integal domain. An element $a\in R$ is called \textbf{irreducible} if
    \begin{enumerate}[(i)]
        \item $a\notin R^\times$, i.e. $a$ is not a unit.
        \item $a=bc\implies b\in R^\times\text{ or }c\in R^\times$.
    \end{enumerate}    
\end{definition}

0 is never an irreducible element.
\begin{proposition}{Prime Element $\implies$ Irreducible Element in Integral Domain}{}
    Let $R$ be an integal domain. Then every prime element in $R$ is irreducible.
\end{proposition}

\begin{prf}
    Let $a\in R$ be a prime element. Suppose $a=bc$ for some $b,c\in R$. Then $a\mid bc$. Since $a$ is prime, there must be $a\mid b$ or $a\mid c$. Without loss of generality, we can assume $a\mid b$. Then $b=ad$ for some $d\in R$. Thus we have $$a=bc=adc\implies a(1-dc)=0\implies dc=1\implies c\in R^\times.$$ That implies $a$ is irreducible.
\end{prf}

\begin{proposition}{Prime Ideal Equivalent Definition}{}
    Let $R$ be a commutative ring. An ideal $I\subseteq R$ is prime if and only if $R/I$ is an integal domain.
\end{proposition}


\section{Unique Factorization Domain}
\begin{definition}{Unique Factorization Domain}{}
    An integral domain $R$ is called a \textbf{unique factorization domain} (UFD) if
    \begin{enumerate}[(i)]
        \item every nonzero nonunit element of $R$ can be written as a product of irreducible elements of $R$.
        \item if $p_1\cdots p_n=q_1\cdots q_m$ for some irreducible elements $p_1,\cdots,p_n,q_1,\cdots,q_m\in R$, then $n=m$ and there exists a permutation $\sigma\in S_n$ such that $p_i$ is an associate of $q_{\sigma(i)}$ for all $i=1,\cdots,n$.
    \end{enumerate}
\end{definition}

\begin{proposition}{Irreducible Element $\iff$ Prime Element in UFD}{}
    Let $R$ be a UFD. Then every irreducible element in $R$ is prime.
\end{proposition}

\begin{prf}
    Let $a\in R$ be an irreducible element. Suppose $a\mid bc$ for some $b,c\in R$. Then $bc=ad$ for some $d\in R$. Since $R$ is a UFD, we can write $b=p_1\cdots p_n$ and $c=q_1\cdots q_m$ for some irreducible elements $p_1,\cdots,p_n,q_1,\cdots,q_m\in R$. Then we have $$ad=bc=p_1\cdots p_nq_1\cdots q_m.$$ Since $a$ is irreducible, $a$ must be an associate of one of the $p_i$'s or $q_j$'s. Without loss of generality, we can assume $a\sim p_1$. Then $a\mid b$. That implies $a$ is prime.
\end{prf}


\section{Principal Ideal Domain}
\begin{definition}{Principal Ideal Domain}{}
    An integral domain $R$ is called a \textbf{principal ideal domain} (PID) if every ideal of $R$ is principal.
\end{definition}


\begin{proposition}{PID $\implies$ UFD}{}
    Every PID is a UFD.
\end{proposition}


\begin{proposition}{Prime Ideal $\iff$ Maximal Ideal in PID}{}
    Let $R$ be a PID. Then every nonzero prime ideal in $R$ is maximal.
\end{proposition}

\begin{prf}
    Let $I\subseteq R$ be a prime ideal. We only need to show $R/I$ is a field. Let $\overline{a}\in R/I$ be a nonzero element. Then $a\notin I$. Since $I$ is prime, $a$ is not a multiple of any prime element in $R$. Thus $a$ is irreducible. Since $R$ is a PID, $a$ is prime. Thus $\overline{a}$ is prime in $R/I$. Since $R/I$ is an integral domain, $\overline{a}$ is a maximal ideal in $R/I$. That implies $R/I$ is a field.
\end{prf}


\section{Krull Dimension}
\begin{definition}{Length of a Chain of Prime Ideals}{}
    Let $R$ be a commutative ring and $\mathfrak{p}_0\subsetneq\mathfrak{p}_1\subsetneq\cdots\subsetneq\mathfrak{p}_n$ be a chain of prime ideals of $R$. The \textbf{length} of the chain is defined to be $n$.
\end{definition}


\begin{definition}{Height of a Prime Ideal}{}
    Let $R$ be a commutative ring and $\mathfrak{p}$ be a prime ideal of $R$. The \textbf{height} of $\mathfrak{p}$ is defined to be the supremum of the lengths of all chains of prime ideals of $R$ contained in $\mathfrak{p}$
    \[
    \mathrm{ht}(\mathfrak{p})=\sup\left\{n\in\mathbb{N}\mid\exists\text{ a chain of prime ideals }\mathfrak{p}_0\subsetneq\mathfrak{p}_1\subsetneq\cdots\subsetneq\mathfrak{p}_n=\mathfrak{p}\right\}.
    \]
\end{definition}

\begin{definition}{Height of an Ideal}{}
    Let $R$ be a commutative ring and $I$ be an ideal of $R$. The \textbf{height} of $I$ is defined to be the height of the prime ideal $\mathfrak{p}$ generated by $I$, denoted by $\mathrm{ht}(I)$
\end{definition}


\section{Polynomial Ring}
\begin{definition}{Polynomial Ring}{}
    Let $R$ be a commutative ring. The \textbf{polynomial ring} in $n$ variables over $R$ is the ring $R[x_1,\cdots,x_n]$ defined as the set of all formal sums $$\sum_{\alpha\in\mathbb{N}^n}a_\alpha x^\alpha$$ where $a_\alpha\in R$ satisfies $a_\alpha=0$ for all but finitely many $\alpha\in\mathbb{N}^n$ and $x^\alpha=x_1^{\alpha_1}\cdots x_n^{\alpha_n}$ for $\alpha=(\alpha_1,\cdots,\alpha_n)\in\mathbb{N}^n$. The addition and multiplication are defined as follows: $$\sum_{\alpha\in\mathbb{N}^n}a_\alpha x^\alpha+\sum_{\alpha\in\mathbb{N}^n}b_\alpha x^\alpha=\sum_{\alpha\in\mathbb{N}^n}(a_\alpha+b_\alpha)x^\alpha$$ and $$\left(\sum_{\alpha\in\mathbb{N}^n}a_\alpha x^\alpha\right)\left(\sum_{\beta\in\mathbb{N}^n}b_\beta x^\beta\right)=\sum_{\gamma\in\mathbb{N}^n}\left(\sum_{\alpha+\beta=\gamma}a_\alpha b_\beta\right)x^\gamma.$$
\end{definition}


\begin{proposition}{Properties of Polynomial Ring}{}
    Let $R$ be a commutative ring.
    \begin{enumerate}
        \item If $R$ is a UFD, then $R[x_1,\cdots.x_n]$ is a UFD.
        \item $R$ is a field $\iff$ $R[x]$ is a PID $\iff$ $R[x]$ is an Euclidean domain.
    \end{enumerate}
\end{proposition}


\section{Construction}
\subsection{Free Object}
\begin{definition}{Free Commutative Ring}{}
    Since $\mathsf{CRing}\cong \mathbb{Z}\text{-}\mathsf{CAlg}$, the \textbf{free commutative ring} on a set $X$ is isomorphic to the polynomial ring $\mathbb{Z}[X]$, which coincides with the free commutative $\mathbb{Z}$-algebra on $X$.
\end{definition}

\subsection{Localization}
\begin{definition}{Multuplicative Subset}{}
    Let $R$ be a commutative ring. A subset $S\subseteq R$ is called \textbf{multiplicative} if $S$ is monoid under the multiplication of $R$, i.e.
    \begin{enumerate}[(i)]
        \item $1\in S$.
        \item $a,b\in S\implies ab\in S$.
    \end{enumerate}
\end{definition}



\begin{definition}{Localization of a Ring}{}
    Let $R$ be a commutative ring and $S\subseteq R$ be a multiplicative subset. The \textbf{localization} of $R$ at $S$ is the ring $S^{-1}R$ defined as the set of equivalence classes of the relation $\sim$ on $R\times S$ defined by $$(a,s)\sim (b,t)\iff \exists u\in S\text{ such that }u(at-bs)=0.$$
    The equivalence class of $(a,s)$ is denoted by $\frac{a}{s}$. The addition and multiplication on $S^{-1}R$ are defined as follows:
    \begin{align*}
        \frac{a}{s}+\frac{b}{t}&=\frac{at+bs}{st}\\
        \frac{a}{s}\cdot\frac{b}{t}&=\frac{ab}{st}
    \end{align*}
    The addition identify is $\frac{0}{1}$ and the multiplication identity is $\frac{1}{1}$.
\end{definition}


\begin{proposition}{Universal Property of Localization}{}
    Let $R$ be a commutative ring and $S\subseteq R$ be a multiplicative subset. The ring homomorphism
    \begin{align*}
        \varphi:R&\longrightarrow S^{-1}R\\
         r&\longmapsto \frac{r}{1}
    \end{align*}
    satisfies the following universal property: for any ring homomorphism $\psi:R\to T$ such that $\psi(S)\subseteq T^\times$, there exists a unique ring homomorphism 
    \begin{align*}
        \psi':S^{-1}R&\longrightarrow T\\
        \frac{a}{s}&\longmapsto \psi(a)(\psi(s))^{-1}
    \end{align*}
    such that the following diagram commutes
    \begin{center}
        \begin{tikzcd}[ampersand replacement=\&]
         
            S^{-1}R\arrow[rr, "\psi'", dashed]\&\& T \&  \\             
            \&R \arrow[ru, "\psi"'] \arrow[lu, "\varphi"] \&                          
        \end{tikzcd}
    \end{center}
    Moreover, if $\psi$ is injective, then $\psi'$ is injective. If $\psi$ is surjective, then $\psi'$ is surjective.
\end{proposition}

\begin{prf}
    First let's check $\psi'$ is well-defined. Suppose $\dfrac{a}{s}=\dfrac{b}{t}$. Then there exists $u\in S$ such that $u(at-bs)=0$. Since $\psi$ is a ring homomorphism, we have 
    $$
    0=\psi(u(at-bs))=\psi(u)\left(\psi(a)\psi(t)-\psi(b)\psi(s)\right).
    $$
    Since $u\in S$ and $\psi(S)\subseteq T^\times$, we have $\psi(u)\in T^\times$. Thus
    $$
    \psi'\left(\frac{a}{s}\right)=\psi(a)(\psi(t))^{-1}=\psi(b)(\psi(s))^{-1}= \psi'\left(\frac{b}{t}\right).
    $$ 
    That implies $\psi'$ is well-defined. It is easy to check $\psi'$ is a ring homomorphism
    \[
        \psi'\left(\frac{a}{s}+\frac{b}{t}\right)=\psi'\left(\frac{at+bs}{st}\right)=\psi(at+bs)(\psi(st))^{-1}=\psi(a)(\psi(s))^{-1}+\psi(b)(\psi(t))^{-1}=\psi'\left(\frac{a}{s}\right)+\psi'\left(\frac{b}{t}\right).  
    \]
    The multiplication is similar. The diagram commutes since
    \[
        \psi'\circ\varphi(r)=\psi'\left(\frac{r}{1}\right)=\psi(r)(\psi(1))^{-1}=\psi(r).
    \]
    Now we show $\psi'$ is unique. Suppose there exists another ring homomorphism $\psi'':S^{-1}R\to T$ such that the diagram commutes. Then for any $\frac{a}{s}\in S^{-1}R$, we have 
    \[
        \psi''\left(\frac{a}{s}\right) = \psi''\left(\frac{a}{1}\frac{1}{s}\right) = \psi''\left(\frac{a}{1}\right)\psi''\left(\frac{1}{s}\right) = \psi''\left(\frac{a}{1}\right)\left(\psi''\left(\frac{s}{1}\right)\right)^{-1} = \psi\left(a\right)(\psi(s))^{-1} =    \psi'\left(\frac{a}{s}\right).
    \]
    That implies $\psi''=\psi'$. Thus $\psi'$ is unique.\\
    Now suppose $\psi$ is injective. Then 
    \[
    \ker \psi' = \left\{ \frac{a}{s} \in S^{-1}R \midv \psi(a)(\psi(s))^{-1} = 0 \right\} = \left\{ \frac{a}{s} \in S^{-1}R \midv \psi(a) = 0 \right\} = \left\{ \frac{0}{s} \midv s \in S \right\} = \left\{ 0 \right\},    
    \]
    which implies $\psi'$ is injective.
\end{prf}


Localization is the most economical way to make a multiplicative subset invertible.

\begin{proposition}{}{}
    Let $R$ be a commutative ring and $S \subseteq R$ be a multiplicative subset. The category of $S^{-1} R$ modules is equivalent to the category of $R$-modules $M$ with the property that every $s \in S$ acts as an automorphism on $M$. The following functor $F$ gives a equivalence of categories: 
    \[
        \begin{tikzcd}[ampersand replacement=\&]
            S^{-1} R\text{-}\mathsf{Mod}\&[-25pt]\&[+10pt]\&[-30pt] R\text{-}\mathsf{Mod}\text{ where }S\text{ act as automorphisms}\&[-30pt]\&[-30pt] \\ [-15pt] 
            M  \arrow[dd, "f"{name=L, left}] 
            \&[-25pt] \& [+10pt] 
            \& [-30pt] M\arrow[dd, "f"{name=R}] \&[-30pt]\\ [-10pt] 
            \&  \phantom{.}\arrow[r, "F", squigarrow]\&\phantom{.}  \&   \\[-10pt] 
            N \& \& \&  N\&
        \end{tikzcd}
        \]
\end{proposition}

\begin{prf}
    Assume $S$ is a multiplicative subset of communitative ring $R$ and the localization map is $\varphi:R\to S^{-1}R$. Then $R$ can acts on $S^{-1}R$-module $M$ through 
    \[
        R\xrightarrow{\varphi}S^{-1}R\xrightarrow{\sigma_M'}\mathrm{End}_{\mathsf{Ab}}(M),
    \]
    which enables us to regard $M$ as an $R$-module. Furthermore, since 
    \[
        \sigma_M'(\varphi(S))\subseteq \sigma_M' \left(\left(S^{-1}R\right)^\times\right)\subseteq \left(\mathrm{End}_{\mathsf{Ab}}(M)\right)^\times=\mathrm{Aut}_{\mathsf{Ab}}(M),
    \]
    every $s \in S$ acts as an automorphism on $M$.\\
    Conversely, if $M$ is an $R$-module such that every $s\in S$ acts as an automorphism on $M$, i.e. $\sigma_M:R\to\mathrm{End}_{\mathsf{Ab}}(M)$ satisfies $\sigma_M(S)\subseteq \mathrm{Aut}_{\mathsf{Ab}}(M)$, then by unversal property
    \begin{center}
        \begin{tikzcd}[ampersand replacement=\&]
         
            S^{-1}R\arrow[rr, "\sigma_M'", dashed]\&\& \mathrm{End}_{\mathsf{Ab}}(M) \&  \\             
            \&R \arrow[ru, "\sigma_M"'] \arrow[lu, "\varphi"] \&                          
        \end{tikzcd}
    \end{center}
    we can define a $S^{-1}R$-module structure on $M$ by lifting $\sigma_M$ to $\sigma_M'$. It is easy to check that these two functors are quasi-inverse to each other.
\end{prf}

\begin{proposition}{Properties of Localization of Rings}{prop_of_localization_of_rings}
    Let $R$ be a commutative ring and $S\subseteq R$ be a multiplicative subset. Then
    \begin{enumerate}[(i)]
        \item $S^{-1}R=0$ if and only if $0\in S$.
        \item If $0\notin S$, then $\frac{a}{s}$ is invertible in $S^{-1}R$ if and only if there exists $r\in R$ such that $ra\in S$.
        \item If $0\notin S$, the localization map $\varphi:R\to S^{-1}R$ is injective if and only if $S$ contains no zero divisors.
        \item If $R$ is an integral domain, then $S^{-1}R$ is also an integral domain.
    \end{enumerate}
\end{proposition}

\begin{prf}
    \begin{enumerate}[(i)]
        \item \[
            S^{-1}R=0\iff \frac{1}{1}=\frac{0}{1}\iff \exists s\in S\text{ such that }s\cdot 1=0\iff 0\in S.
            \]
        \item Suppose $0\notin S$. If $\frac{a}{s}$ is invertible in $S^{-1}R$, then there exists $\frac{b}{t}\in S^{-1}R$ such that $\frac{a}{s}\cdot\frac{b}{t}=\frac{1}{1}$, which implies there exists $u\in S$ such that $u(ab-st)=0$. Let $r=ub\in R$ and then we see $ra=ust\in S$. Conversely, suppose there exists $r\in R$ such that $ra\in S$. Then $\frac{a}{s}\cdot\frac{rs}{ra}=\frac{1}{1}$, which implies $\frac{a}{s}$ is invertible.
        \item Suppose $0\notin S$. Given the localization map $\varphi:R\to S^{-1}R$, we have
        \[
            \varphi(r)=0\iff \frac{r}{1}=\frac{0}{1}\iff \exists s\in S\text{ such that }s\cdot r=0.
        \]
        Thus 
        $$
        \varphi\text{ is injective}\iff \ker \varphi=\{0\}\iff \forall s\in S,\forall r\in R-\{0\},sr\ne 0\iff S\text{ contains no zero divisors}.
        $$
    \end{enumerate}
\end{prf}


\begin{definition}{Total Ring of Fractions}{}
    Let $R$ be a commutative ring. Then $S=\left\{r\in R-\{0\}\mid r\text{ is not a zero divisor}\right\}$ is a multiplicative subset. The \textbf{total ring of fractions} of $R$ is the localization $S^{-1}R$, denoted by $\mathrm{Frac}(R)$. The localization map $\varphi:R\hookrightarrow \mathrm{Frac}(R)$ is an injective ring homomorphism.
\end{definition}

\begin{prf}
    Since $0\notin S$ and $S$ contains no zero divisors, $\varphi$ is injective by (iii) of \Cref{th:prop_of_localization_of_rings}.
\end{prf}

\begin{proposition}{Properties of Total Ring of Fractions}{}
    Let $R$ be a commutative ring and $S\subseteq R$ be a multiplicative subset. Then $S^{-1}R$ can be regarded as a subring of $\mathrm{Frac}(R)$. 
\end{proposition}

\begin{prf}
    By the universal property of localization, there exists a unique ring homomorphism $\psi:S^{-1}R\to \mathrm{Frac}(R)$. Since $\varphi:R\hookrightarrow \mathrm{Frac}(R)$ is injective, $\psi$ is also injective.
\end{prf}

\begin{definition}{Field of Fractions}{}
    If $R$ be an integral domain, then $S=R-\{0\}$ is a multiplicative subset. The total ring of fractions $\mathrm{Frac}(R)=S^{-1}R$ is a field, call the \textbf{field of fractions} of $R$.
\end{definition}


\begin{definition}{Localization of an Ideal}{}
    Let $R$ be a commutative ring, $S$ be a multiplicative set in $R$, and $I$ be an ideal of $R$. If we regard $I$ as a $R$-module, the \textbf{localization of the ideal} $I$ by $S$, denoted $S^{-1}I$, is the localization of the module $I$ by $S$. That is,
    \[
        S^{-1}I=\left\{\frac{a}{s}\midv a\in I, s\in S\right\}.
    \]
    $S^{-1}I$ is a $S^{-1}R$-submodule of $S^{-1}R$. Suppose the localization map is $\varphi:R\to S^{-1}R$, $S^{-1}I$ can also defined as the ideal generated by $\varphi(I)$ in $S^{-1}R$
    \[
        S^{-1}I=\langle \varphi(I)\rangle=\left\{\frac{r}{s}\frac{a}{1}\midv a\in I, \frac{r}{s}\in S^{-1}R\right\}.
    \]
\end{definition}

\begin{proposition}{Properties of Localization of Ideals}{}
    Let $R$ be a commutative ring, $S$ be a multiplicative set in $R$, and $0\notin S$. Suppose the localization map is $\varphi:R\to S^{-1}R$. Then we have maps between the sets of ideals of $R$ and $S^{-1}R$:
    \begin{align*}
        \mathcal{I}(R)=\left\{\text{ideals of }R\right\}\xrightleftarrows[\varphi^{-1}]{\quad S^{-1}\quad}
         \left\{\text{ideals of }S^{-1}R\right\}=\mathcal{I}(S^{-1}R)
    \end{align*}
    \begin{enumerate}[(i)]
        \item $S^{-1}\circ \varphi^{-1}=\mathrm{id}_{\mathcal{I}(S^{-1}R)}$. As a result, $S^{-1}$ is surjective and $\varphi^{-1}$ is injective.
        \item For any ideal $J$ of $S^{-1}R$, there exists an ideal $I$ of $R$ such that $S^{-1}I=J$. 
        \item If $I$ is a ideal of $R$, then $S^{-1}I=S^{-1}R\iff I\cap S\ne\varnothing$.
        \item $\varphi$ induces a bijection between the set of prime ideals of $R$ that do not intersect $S$ and the set of prime ideals of $S^{-1}R$. That is, the following restriction of $S^{-1}$ and $\varphi^{-1}$ are bijections:
        \begin{align*}
            \{I \in \operatorname{Spec} R: I \cap S=\varnothing\} \xrightleftarrows[\varphi^{-1}]{\quad S^{-1}\quad}
           \spec S^{-1}R
        \end{align*}
    \end{enumerate}
\end{proposition}

\begin{prf}
    \begin{enumerate}[(i)]
        \item Let $J$ be an ideal of $S^{-1}R$. We have
        \[
            S^{-1}\varphi^{-1}(J)=\left\{\frac{x}{s}\midv x\in \varphi^{-1}(J),s\in S\right\}=\left\{\frac{x}{s}\midv \frac{x}{1}\in J,s\in S\right\}=\left\{\frac{1}{s}\frac{x}{1}\midv \frac{x}{1}\in J,s\in S\right\}=J.
        \]
        \item It is a direct consequence of the surjectivity of $S^{-1}$.
        \item Let $I$ be an ideal of $R$. We have
        \[
            S^{-1}I=S^{-1}R\iff \frac{1}{1} \in S^{-1}I \iff \exists t,s\in S,a\in I, t(a-s)=0\iff ta=ts\in I\cap S\ne\varnothing \iff I\cap S\ne\varnothing.
        \]
    \end{enumerate}
\end{prf}


\begin{example}{Localization at a Prime Ideal}{}
    Let $R$ be a commutative ring and $\mathfrak{p}$ be a prime ideal of $R$. Then $S=R-\mathfrak{p}$ is a multiplicative set. The localization $S^{-1}R$ is called the \textbf{localization of $R$ at $\mathfrak{p}$}, denoted by $R_\mathfrak{p}$. $R_\mathfrak{p}$ is a local ring with unique maximal ideal 
    \[
    \mathfrak{p}R_\mathfrak{p}=S^{-1}\mathfrak{p}=\left\{\frac{x}{s}\midv x\in \mathfrak{p}, s\in R-\mathfrak{p}\right\}.
    \]
    And we have field isomorphism $R_\mathfrak{p}/\mathfrak{p}R_\mathfrak{p}\cong \mathrm{Frac}(R/\mathfrak{p})$.
\end{example}

\begin{prf}
    Since for any ideal $I\in  \{I \in \operatorname{Spec} R: I \cap S=\varnothing\}$, we have
    \[
        I\subseteq \mathfrak{p}\implies S^{-1}I\subseteq  S^{-1}\mathfrak{p}.
    \]
    Thus we see $S^{-1}\mathfrak{p}$ is the unique maximal ideal of $S^{-1}R$.
\end{prf}


\begin{example}{}{}
    Let $R$ be a commutative ring and $f\in R$. Let $S=\{1,f,f^2,\cdots\}$ be the monoid generated by $f$. Then $S$ is a multiplicative set. The localization $S^{-1}R$ is called the \textbf{localization of $R$ at $f$}, denoted by $R_f$ or $R\left[\tfrac{1}{f}\right]$. The notation can be justified by the fact that $R\left[\tfrac{1}{f}\right]\cong R\left[t\right]/(ft-1)$.
    $R_f=\{0\}$ if and only if $f$ is nilpotent.
\end{example}

\begin{prf}
    $R_f=\{0\}\iff 0\in S\iff \exists n\in\mathbb{Z}_{\ge0},\;f^n=0$.
\end{prf}

\section{Absolute Value}
\begin{definition}{Absolute Value on an Integral Ring}{}
    Let $R$ be an integral ring. An \textbf{absolute value} on $R$ is a function $|\cdot|:R\to \mathbb{R}_{\ge0}$ satisfying the following properties:
    \begin{enumerate}[(i)]
        \item (positive definiteness) $|a|=0\iff a=0$.
        \item (multiplicativity) $|ab|=|a||b|$.
        \item (triangle inequality) $|a+b|\le |a|+|b|$.
    \end{enumerate}
\end{definition}

An absolute value on $R$ induces a metric (and thus a topology) by 
\[
    d(a,b)=|a-b|,
\]
which makes $(R,|\cdot|)$ a topological ring. 
\begin{definition}{Equivalent Absolute Value}{}
    Let $R$ be an integral ring and $|\cdot|,|\cdot|'$ be two absolute values on $R$. $|\cdot|$ and $|\cdot|'$ are called \textbf{equivalent} if they induce the same topology on $R$.
\end{definition}

\begin{definition}{Trivial Absolute Value}{}
    Let $R$ be an integral ring. The \textbf{trivial absolute value} on $R$ is defined by 
    \[
        |a|=\begin{cases}
            0, & a=0_R\\
            1, & a\ne 0_R
        \end{cases}
    \]
\end{definition}

On a finite field, the trivial absolute value is the only absolute value.
\begin{definition}{Archimedean Absolute Value}{}
    If an absolute value $|\cdot|$ satisfies the stronger property 
    \[
        |a+b|\le \max\{|a|,|b|\},
    \]
    then $|\cdot|$ is called an \textbf{non-Archimedean absolute value}. Otherwise, $|\cdot|$ is called an \textbf{Archimedean absolute value}.
\end{definition}

\begin{proposition}{}{}
    Let $R$ be an integral ring and $|\cdot|$ be an absolute value on $R$. Then $|\cdot|$ is non-Archimedean if and only if $\left\{ |n1_R|:n\in \mathbb{Z}\right\}$ is bounded.
\end{proposition}

\begin{prf}
    Suppose $|\cdot|$ is non-Archimedean. Then for any $n\in\mathbb{Z}$, we have
    \[
        |n1_R|=|1_R+\cdots+1_R|\le \max\{|1_R|,\cdots,|1_R|\}=|1_R|=1.
    \]
    Thus $\left\{ |n1_R|:n\in \mathbb{Z}\right\}$ is bounded.\\
    Conversely, suppose $\left\{ |n1_R|:n\in \mathbb{Z}\right\}$ is bounded by $M$. Then for any $a,b\in R$, we have 
    \begin{align*}
        |a+b|^n&=|(a+b)^n|\\
        &=\left| \sum_{i=0}^n\binom{n}{i}a^ib^{n-i}\right|\\
        &\le \sum_{i=0}^n\left|\binom{n}{i}1_R\right||a|^i|b|^{n-i}\\
        &\le \sum_{i=0}^n M\max\{|a|,|b|\}^i\max\{|a|,|b|\}^{n-i}\\
        &=(n+1) M \max\{|a|,|b|\}^n.
    \end{align*}
    As $n\to \infty$, we have
    \[
        |a+b|\le \left((n+1) M \right)^{\frac{1}{n}}\max\{|a|,|b|\}\to \max\{|a|,|b|\}.
        \]
    Thus $|\cdot|$ is non-Archimedean.
\end{prf}



\section{Valuation Ring}

\begin{definition}{Dominance of Local Rings}{}
    A \hyperref[th:local_commutative_ring]{local ring} $S$ is said to \textbf{dominate} another local ring $R$ if one of the following equivalent condition holds
    \begin{enumerate}[(i)]
        \item $R\subseteq S$ and $\mathfrak{m}_R=\mathfrak{m}_S\cap R$, where $\mathfrak{m}_R$ and $\mathfrak{m}_S$ are the maximal ideals of $R$ and $S$ respectively.
        \item The inclusion map $i:R\hookrightarrow S$ is a local ring homomorphism.
    \end{enumerate}
  
\end{definition}

\begin{definition}{Valuation Ring}{}
    Suppose $R$ is an integral domain and has field of fractions $K=\mathrm{Frac}\left(R\right)$. We say $R$ is a \textbf{valuation ring} if $R$ satisfies one of the following equivalent conditions:
    \begin{enumerate}[(i)]
        \item For every $x\in K^\times$, either $x\in R$ or $x^{-1}\in R$.
        \item The ideals of $R$ are totally ordered by inclusion.
        \item The principal ideals of $R$ are totally ordered by inclusion (i.e. the elements in $R$ are, up to units, totally ordered by divisibility.)
        \item $R$ is a local ring and $R$ is maximal among all local rings contained in $K$ partially ordered by dominance.
        \item There is a \hyperref[th:totally_ordered_abelian_group]{totally ordered abelian group} $\left(\Gamma,\le\right)$ and a \hyperref[th:valuation_of_field]{valuation} $v:K \rightarrow \Gamma \cup\{\infty\}$ such that $R$ is the valuation ring of $v$, i.e.
        \[
            R=\mathcal{O}_v=\{x \in K \mid v(x) \ge 0\}.
        \]
    \end{enumerate}
\end{definition}

\begin{prf}
    The equivalence of these conditions can be shown as follows.
    \begin{itemize}[leftmargin=*]
        \item (i)$\implies$(ii). Suppose $I,J$ are two distinct ideals of $R$. Without loss of generality, we can assume $I\subsetneq J$ and there exists $a\in I-J$. Suppose $b\in J$. Since $a\notin J$, there must be $ab^{-1}\notin R$, which forces $ba^{-1}\in R$. Thus there exists $r\in R$ such that $b=ra$, implying $b\in I$. Therefore, we show $J\subseteq I$.
        \item (ii)$\implies$(iii). Trivial.
        \item (iii)$\implies$(i). Given any $x\in K^{\times}$, there exists $a,b\in R$ such $x=ab^{-1}$. Then we have
    \begin{align*}
        (a)\subseteq (b)\text{ or }(b)\subseteq (a) \implies ab^{-1}\in R\text{ or } ba^{-1}\in R \implies x\in R \text{ or } x^{-1}\in R.
    \end{align*}
        \item (i)$\implies$(v). Take $\Gamma=K^\times / R^\times$. Define a binary relation $\le$ on $\Gamma$ by $xR^\times\le yR^\times\iff yx^{-1}\in R$. Then we can check $\left(\Gamma, \le\right)$ is a totally ordered abelian group:
    \begin{enumerate}[(a)]
        \item (reflexivity) For any $xR^\times\in \Gamma$, we have $xx^{-1}=1\in R$, which implies $xR^\times\le xR^\times$.
        \item (antisymmetry) Suppose $x R^{\times} \le y R^{\times}$ and $y R^{\times} \le x R^{\times}$. Then $y x^{-1} \in R$ and $\left(y x^{-1}\right)^{-1}=x y^{-1} \in R$, implying $y x^{-1}\in R^\times$. Therefore, $x R^{\times}=y R^{\times}$.
        \item (transitivity) If $x R^{\times} \leq y R^{\times}$ and $y R^{\times} \leq z R^{\times}$, then $y x^{-1} \in R$ and $z y^{-1} \in R$, so $z x^{-1}=\left(z y^{-1}\right)\left(y x^{-1}\right) \in R$, implying $x R^{\times} \leq z R^{\times}$.
        \item (strong connectivity) For any $xR^\times, yR^\times\in \Gamma$, either $xy^{-1}\in R$ or $y^{-1}x\in R$, which means $xR^\times\le yR^\times$ or $yR^\times\le xR^\times$.
        \item (order preservation) For any $xR^\times, yR^\times, zR^\times\in \Gamma$, we have
        \begin{align*}
            xR^\times\le yR^\times\implies yx^{-1}\in R\implies  (yz)\left(xz\right)^{-1}\in R\implies xzR^\times\le yzR^\times.
        \end{align*}
    \end{enumerate}
    Take $v$ to be the natural projection
    \begin{align*}
        v:K&\longrightarrow \Gamma\cup\{\infty\}\\
        x&\longmapsto \begin{cases}
            xR^\times, &\text{ if } x\ne 0\\
            \infty, &\text{ if } x=0
        \end{cases}
    \end{align*}
    Then we can check that $v$ is a valuation of $K$. For any $x,y\in K$, if $x=0$ or $y=0$ or $x+y=0$, then it is clear to see $v(x+y)=\min\{v(x),v(y)\}$. If $x\ne 0$, $y\ne0$ and $x+y\ne 0$, then we have
    \begin{align*}
        v(x)\le v(x+y)&\iff (x+y)y^{-1}\in R\iff xy^{-1}+1\in R\iff xy^{-1}\in R,\\
        v(y)\le v(x+y)&\iff (x+y)x^{-1}\in R\iff yx^{-1}+1\in R\iff yx^{-1}\in R.
    \end{align*}
    which implies either $v(x+y)\ge v(x)$ or $v(x+y)\ge v(y)$. Therefore, we show $v(x+y)\ge \min\{v(x),v(y)\}$.
    \item (ii)$\implies$(iv). Since ideals of $R$ are totally ordered by inclusion, there exists a unique maximal ideal $\mathfrak{m}$ of $R$. Hence $R$ is a local ring.
\end{itemize}
\end{prf}



\begin{proposition}{}{}
    Let $K$ be a field. Let $R \subset K$ be a local subring. Then there exists a valuation ring with fraction field $K$ dominating $R$.
\end{proposition}


\begin{definition}{Discrete Valuation Ring}{}
    Suppose $R$ is an integral domain and has field of fractions $K=\mathrm{Frac}\left(R\right)$. We say $R$ is a \textbf{discrete valuation ring} if $R$ satisfies one of the following equivalent conditions:
    \begin{enumerate}[(i)]
        \item $R$ is a valuation ring such that the induced valuation $v:K\to \Gamma\cup\{\infty\}$ is a discrete valuation.
        \item $R$ is a local PID, and not a field.
        \item $R$ is a PID with a unique non-zero prime ideal.
        \item $R$ is a PID with a unique irreducible element (up to multiplication by units).
        \item $R$ is a UFD with a unique irreducible element (up to multiplication by units).
        \item $R$ is a local Dedekind domain and not a field.
        \item $R$ is a Noetherian local domain whose maximal ideal is principal, and not a field. 
        \item $R$ is an integrally closed Noetherian local ring with Krull dimension one.
        \item $R$ is Noetherian, not a field, and every nonzero fractional ideal of $R$ is irreducible in the sense that it cannot be written as a finite intersection of fractional ideals properly containing it.
    \end{enumerate}
\end{definition}


\section{Integral Element}
\begin{definition}{Integral Element}{}
    Let $R$ be an integral domain and $K$ be its field of fractions. An element $x\in K$ is called \textbf{integral} over $R$ if there exists a monic polynomial $f\in R[x]$ such that $f(x)=0$. The set of all elements in $K$ that are integral over $R$ is called the \textbf{integral closure} of $R$ in $K$, denoted by $\overline{R}$.
\end{definition}


