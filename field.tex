
\chapter{Field}
\section{Field Extension}
\begin{definition}{Field}{}
    A \textbf{field} is a commutative ring $K$ such that $K^{\times}=K-\{0\}$.
\end{definition}

\begin{proposition}{Ideals of Field}{}
    The only ideals of a field $K$ are $\{0\}$ and $K$.
\end{proposition}



\begin{definition}{Field Homomorphism}{}
    A \textbf{field homomorphism} is a ring homomorphism between fields.
\end{definition}

\begin{proposition}{}{}
    Let $\mathsf{Field}$ denote the category of fields. We have
    \begin{enumerate}[(i)]
        \item Monomorphisms in $\mathsf{Field}$ are exactly injective ring homomorphisms of fields.
        \item Isomorphisms in $\mathsf{Field}$ are exactly bijective ring homomorphisms of fields, i.e. ring isomorphisms of fields.
        \item Every morphism in $\mathsf{Field}$ is a monomorphism.
    \end{enumerate}
\end{proposition}
\begin{prf}
    This follows from \Cref{th:nonzero_ring_homomorphism_from_field_is_injective}.
\end{prf}

\begin{definition}{Subfield}{}
    Let $L$ be a field and $K\subseteq L$ is a subset of $L$. If $K$ is a field under the operations inherited from $L$, we say $K$ is a \textbf{subfield} of $L$
\end{definition}


\begin{definition}{Field Extension}{}
    Let $u:L\hookrightarrow K$ be a field monomorphism. We say $K$ is a \textbf{field extension} of $L$. We write $K/L$ to denote the field extension.
\end{definition}
\begin{remark}
    Though in the notation of field extension $K/L$, the function $u:L\hookrightarrow K$ is not given explicitly, $K/L$ is just a simple way to denote $u:L\hookrightarrow K$ and includes totally the same information as $u$. 
    
    Note that $u(L)$ is a subfield of $K$. This allows us to think of $L$ as a subfield of $K$.
\end{remark}


\begin{definition}{$K$-embedding and $K$-isomorphism}{}
 A \textbf{$K$-embedding} is a morphism in the \hyperref[th:coslice_category]{coslice category} $(\mathsf{Field}/K)$. A \textbf{$K$-isomorphism} is an isomorphism in $(\mathsf{Field}/K)$. We say two field extensions $L_1/K$ and $L_2/K$ are \textbf{$K$-isomorphic} if there exists a $K$-isomorphism between them.
\end{definition}

\begin{proposition}{Equivalent Characterizations of $K$-embedding}{equivalent_characterizations_of_K_embedding}
    Let $u_1:F\hookrightarrow K_1$, $u_2:F\hookrightarrow K_2$, $\phi:K_1\hookrightarrow K_2$ be field extensions. Then the following are equivalent:
    \begin{enumerate}[(i)]
        \item $\phi:K_1\hookrightarrow K_2$ is a $K$-embedding between $u_1$ and $u_2$.
        \item We have the following commutative diagram:
        \[
        \begin{tikzcd}
            L_1 \arrow[rr, "\phi", hook] &                                                    & L_2 \\
                                        & K \arrow[lu, "u_1", hook] \arrow[ru, "u_2"', hook] &    
        \end{tikzcd}
        \]
        \item $\phi:K_1\to K_2$ is an $K$-algebra homomorphism.
    \end{enumerate}
\end{proposition}
\begin{prf}
Note $(\mathsf{Field}/K)$ is a full subcategory of $(K/\mathsf{CRing})\cong K\text{-}\mathsf{CAlg}$. 
\end{prf}

\begin{corollary}{$K$-endomorphism Fixes Base Field}{}
    Let $u:K\hookrightarrow L$ be a field extension and $\phi:L\hookrightarrow L$ be a $K$-embedding. Then $\phi|_{u(K)}=\mathrm{id}_{u(K)}$.
\end{corollary}
\begin{prf}
    \Cref{th:equivalent_characterizations_of_K_embedding} implies that $u=\phi\circ u$. So for any $x\in K$, we have $\phi(u(x))=u(x)$. This implies $\phi|_{u(K)}=\mathrm{id}_{u(K)}$.
\end{prf}

\begin{proposition}{$K$-automorphism Acts on Set of Roots}{K_automorphism_acts_on_set_of_roots}
    Let $u:K\hookrightarrow L$ be a field extension and $\phi:L\hookrightarrow L$ be a $K$-embedding. Suppose $f(X)\in K[X]$ is a polynomial and 
    \[
    S_f = \left\{x\in L\mid f(x)=0\right\}
    \]
    be the set of roots of $f$ in $L$. Then $\phi|_{S_f}:S_f\to S_f$ is a bijection and we can define a monoid homomorphism
    \begin{align*}
        \mathrm{End}_{(\mathsf{Field}/K)}(L/K)&\longrightarrow \mathrm{Aut}_{\mathsf{Set}}\left(S_f\right)\\
        \phi&\longmapsto \phi|_{S_f}
    \end{align*}
    Moreover, $\mathrm{Aut}_{(\mathsf{Field}/K)}(L/K)$ acts on $S_f$ through this map.
\end{proposition}
\begin{prf}
    Let $x\in S_f$. Since
    \[
    f(\phi(x))=\phi(f(x))=\phi(0)=0,
    \]
    we have $\phi(x)\in S_f$. Hence $\phi(S_f)\subseteq S_f$. Since $S_f$ is finite set and $\phi$ is injective, we see $\phi|_{S_f}:S_f\to S_f$ is a bijection. It is easy to check
    \begin{align*}
        \mathrm{Aut}_{(\mathsf{Field}/K)}(L/K)&\longrightarrow \mathrm{Aut}_{\mathsf{Set}}\left(S_f\right)\\
        \phi&\longmapsto \phi|_{S_f}
    \end{align*}
    is a group homomorphism. 

\end{prf}

\begin{definition}{Subextension}{subextension}
    Let $u:K\hookrightarrow L$ be a field extension and $M\subseteq L$ be a subfield of $L$. If one of the following equivalent conditions hold:
    \begin{enumerate}[(i)]
        \item $u(K)\subseteq M$,
        \item $M$ is a $K$-subalgebra of $L$,
    \end{enumerate}    
    then we can define a map
    \begin{align*}
        \tilde{u}:K &\longrightarrow M:\\
        x&\longmapsto u(x)
    \end{align*}
    which shrink the codomain of $u$ to $M$. We say $\tilde{u}:K \rightarrow M$ is a \textbf{subextension} of $u:K\hookrightarrow L$.
\end{definition}
\begin{remark}
    If $u(K)\subseteq M$, we can also say $M$ is a subextension of $L/K$, because $\tilde{u}:K \rightarrow M$ is totally determined by $M$. And we have the following commutative diagram:
        \[
        \begin{tikzcd}
            M \arrow[rr, "\iota", hook] &                                                    & L \\
                                        & K \arrow[lu, "\tilde{u}", hook] \arrow[ru, "u"', hook] &    
        \end{tikzcd}
        \]
    where $\iota:M\hookrightarrow K$ is the inclusion map.
\end{remark}





\begin{proposition}{Characteristic of a Field}{characteristic_of_a_field}
    The \hyperref[th:characteristic_of_a_ring]{characteristic} of a field is either $0$ or a prime number.
\end{proposition}
\begin{prf}
    Let $K$ be a field and $\varphi:\mathbb{Z}\to K$ be the unique ring homomorphism. Since $(0)$ is a maximal ideal of $K$, $\ker \varphi$ is a maximal ideal of $\mathbb{Z}$. Therefore, $\ker \varphi$is either $(0)$ or $p\mathbb{Z}$ for some prime number $p$, which implies $\mathrm{char}(K)$ is either $0$ or $p$.
\end{prf}

\begin{proposition}{Field Extension Preserves Characteristic}{field_extension_preserves_characteristic}
    Leet $L/K$ be a field extension. Then $\mathrm{char}(L)=\mathrm{char}(K)$.
\end{proposition}
\begin{prf}
    Since $\mathbb{Z}$ is an initial object in $\mathsf{CRing}$, there exists unique ring homomorphisms $\varphi_K:\mathbb{Z}\to K$ and $\varphi_L:\mathbb{Z}\to L$ such that the following diagram commutes in $\mathsf{CRing}$
    \[
        \begin{tikzcd}
            K \arrow[rr, "u", hook] &                                                                         & L \\
                                    & \mathbb{Z} \arrow[lu, "\varphi_K", hook] \arrow[ru, "\varphi_L"', hook] &  
        \end{tikzcd}
    \]
    Thus we have
    \[
    \ker \varphi_L = \ker\left(u\circ \varphi_K\right)= \varphi_K^{-1}(0)=\ker \varphi_K,
    \]
    which implies $\mathrm{char}(L)=\mathrm{char}(K)$.
\end{prf}

In field category $\mathsf{Field}$, Let $\mathsf{Field}_0$ denote the full subcategory of fields of characteristic $0$ and $\mathsf{Field}_p$ denote the full subcategory of fields of characteristic $p$. $\mathsf{Field}_0$ and all $\mathsf{Field}_p$ are connected components of $\mathsf{Field}$. Though $\mathsf{Field}$ has no initial objects, $\mathsf{Field}_0$ and $\mathsf{Field}_p$ have initial objects.

\begin{proposition}{Initial Objects in $\mathsf{Field}_0$ and $\mathsf{Field}_p$}{initial_objects_in_field}
    \begin{enumerate}[(i)]
        \item $\mathbb{Q}$ is an initial object in $\mathsf{Field}_0$.
        \item $\mathbb{F}_p:=\mathbb{Z}/p\mathbb{Z}$ is an initial object in $\mathsf{Field}_p$.
    \end{enumerate}
\end{proposition}
\begin{prf}
    \begin{enumerate}[(i)]
        \item Omitted.
        \item For any field $K$ of characteristic $p$, there exists a uniqueness ring homomorphism
        \begin{align*}
            \varphi:\mathbb{Z}&\longrightarrow K\\
            n&\longmapsto n\cdot 1_K
        \end{align*}
        and we have $\ker \varphi=\mathbb{F}_p$. Therefore, by ring homomorphism theorem, there exists a unique field homomorphism 
        \begin{align*}
            \psi:\mathbb{F}_p&\longrightarrow K \\
            n+ p\mathbb{Z} &\longmapsto n\cdot 1_K
        \end{align*}
        such that the following diagram commutes
        \[
            \begin{tikzcd}
                \mathbb{Z} \arrow[r, "\varphi", hook] \arrow[d, "\pi"', two heads] & K \\
                \mathbb{F}_p \arrow[ru, "\psi"', dashed]                           &  
                \end{tikzcd}
        \]
        If $\eta:\mathbb{F}_p\to K$ is field homomorphism, then $\eta(1+p\mathbb{Z})=1_K$ implies
        \[
        \eta(n+p\mathbb{Z})=n\cdot \eta(1+p\mathbb{Z})=n\cdot 1_K,
        \]
        which implies $\eta=\psi$.
    \end{enumerate}
\end{prf}

\begin{corollary}{}{}
    There is natural bijection between isomorphisms in $\mathsf{Field}_p$ and isomorphisms in $(\mathbb{F}_p/\mathsf{Field})$. There is natural bijection between isomorphisms in $\mathsf{Field}_0$ and isomorphisms in $(\mathbb{Q}/\mathsf{Field})$.
\end{corollary}
\begin{prf}
    This follows from \Cref{th:initial_objects_in_field}.
\end{prf}

\begin{definition}{Prime Subfield}{prime_subfield}
    If $K$ is a field, the \textbf{prime subfield} of $K$ is the smallest subfield of $K$ containing $1_K$.
\end{definition}

\begin{proposition}{Equivalent Characterizations of Prime Subfield}{}
    Let $K$ be a field and $P$ be a subfield of $K$. The following are equivalent:
    \begin{enumerate}[(i)]
        \item $P$ is the \hyperref[th:prime_subfield]{prime subfield} of $K$.
        \item \begin{itemize}
            \item If $\mathrm{char}(K)=0$, $P$ is the image of the unique field extension $\mathbb{Q}\hookrightarrow K$.
            \item If $\mathrm{char}(K)=p$, $P$ is the image of the unique field extension $\mathbb{F}_p\hookrightarrow K$.
        \end{itemize}
    \end{enumerate}
\end{proposition}
\begin{prf}
    We prove the proposition by discussing the two cases: $\mathrm{char}(K)=0$ and $\mathrm{char}(K)=p$.
    \begin{itemize}
        \item Let $K$ be a field with a positive characteristic $p$. From \Cref{th:initial_objects_in_field} we see $\mathbb{F}_p$ is initial in $\mathsf{Field}_p$ and there exists a unique field extension from $\mathbb{F}_p$ to $K$
        \begin{align*}
            \psi:\mathbb{F}_p&\xhookrightarrow{\quad} K\\
                 n + p\mathbb{Z}&\longmapsto n \cdot 1_K
        \end{align*}
        If $F$ is any subfield of $K$ containing $1_K$, then $F$ must contains $n \cdot 1_K$ for all $n\in \mathbb{Z}$, which implies $\psi(\mathbb{F}_p)\subseteq F$. Therefore, $\psi(\mathbb{F}_p)$ is the smallest subfield of $K$ containing $1_K$. So we proved $\psi(\mathbb{F}_p)$ is the prime subfield of $K$.
        \item Let $K$ be a field with characteristic $0$. From \Cref{th:initial_objects_in_field} we see $\mathbb{Q}$ is initial in $\mathsf{Field}_0$ and there exists a unique field extension from $\mathbb{Q}$ to $K$
        \begin{align*}
            \psi:\mathbb{Q}&\xhookrightarrow{\quad} K\\
                 n &\longmapsto n \cdot 1_K
        \end{align*}
        If $F$ is any subfield of $K$ containing $1_K$, then $F$ must contains $n \cdot 1_K$ for all $n\in \mathbb{Z}$, which implies $\psi(\mathbb{Q})\subseteq F$. Therefore, $\psi(\mathbb{Q})$ is the smallest subfield of $K$ containing $1_K$. So we proved $\psi(\mathbb{Q})$ is the prime subfield of $K$.
    \end{itemize}
    
\end{prf}

\begin{proposition}{Subfield Contains Prime Subfield}{subfield_contains_prime_subfield}
    Let $K$ be a field with prime subfield $P$. If $F\subseteq K$ is a subfield of $K$, then
    \begin{enumerate}[(i)]
    \item $P\subseteq F$.
    \item \begin{itemize}[wide]
        \item If $\mathrm{char}(K)=0$, $F/\mathbb{Q}$ is a subextension of $K/\mathbb{Q}$.
        \item If $\mathrm{char}(K)=p$, $F/\mathbb{F}_p$ is a subextension of $K/\mathbb{F}_p$.
    \end{itemize}
    \end{enumerate}
\end{proposition}
\begin{prf}
    Let $F$ be a subfield of $K$. Since $F$ is a field, $1_K\in F$. Therefore, $P$ is a subfield of $F$.
\end{prf}



\begin{theorem}{Hilbert's Weak Nullstellensatz}{}
    If $\overline{\Bbbk}$ is an algebraically closed field, then the maximal ideals of $\overline{\Bbbk}\left[x_1, \cdots, x_n\right]$ are precisely those ideals of the form $\left(x_1-a_1, \cdots, x_n-a_{n}\right)$, where $a_i\in \overline{\Bbbk}$. 
\end{theorem}


\begin{theorem}{Hilbert's Nullstellensatz}{}
    If $\Bbbk$ is any field and $\mathfrak{m}$ is a maximal ideal of $\Bbbk\left[x_1, \ldots, x_n\right]$, then $\Bbbk\left[x_1, \ldots, x_n\right]/\mathfrak{m}$ is a finite extension of $\Bbbk$.
\end{theorem}

\begin{proposition}{}{}
    If $K$ is a field, then any finite subgroup of $K^\times$ is cyclic.
\end{proposition}

\begin{prf}
    Let $G$ be any subgroup of $K^\times$. By the structure theorem for finite abelian groups, we can write
    \[
     G\cong C_{n_1}\times C_{n_2}\times\cdots \times C_{n_r}
    \]
    with $n_1\divides n_2 \divides\cdots\divides n_r$. Let $n=G=n_1n_2\cdots n_r$. It is sufficient to show that $r=1$. Since for any $g=(g_1,\cdots, g_r)\in C_{n_1}\times C_{n_2}\times\cdots \times C_{n_r}$, we have
    \[
      g^{n_r}-1=(g_1^{n_r},g_2^{n_r},\cdots, g_r^{n_r})-(1,1,\cdots, 1)=0,
    \]
    we know the polynomial $f(X)=X^{n_r}-1$ has $n$ distinct roots in $K$. Since 
    \[
    n\le \deg f= n_r,
    \]
    there must be $n=n_r$ and $r=1$.
\end{prf}




\subsection{Algebraic Extension}

\begin{definition}{Algebraic Element and Transcendental Element}{algebraic_element_and_transcendental_element_for_field_extension}
    Let $L/K$ be a field extension and $\alpha \in L$. Consider the evaluation ring homomorphism
    \begin{align*}
        \mathrm{ev}_\alpha:K[X] &\longrightarrow L\\
        f &\longmapsto f(\alpha).
    \end{align*}
    and $\ker \mathrm{ev}_\alpha=(P_\alpha)$ for some $P_\alpha\in K[X]$. Polynomials in $\ker \mathrm{ev}_\alpha$ are called \textbf{annihilating polynomials} of $\alpha$ over $K$.
    \begin{itemize}
        \item If $P_\alpha=0$, then $\alpha$ is called a \textbf{transcendental element} over $K$. In this case, zero polynomial is the only annihilating polynomial of $\alpha$ over $K$.
        \item If $P_\alpha\ne 0$, then $\alpha$ is called an \textbf{algebraic element} over $K$. Suppose $P_\alpha(X)=\sum_{i=0}^n a_iX^i$ with $a_n\in K^\times$. Then the monic polynomial $m_\alpha(X)=P_\alpha(X)/a_n$ is called the \textbf{minimal polynomial}\index{minimal polynomial} of $\alpha$ over $K$, which is irreducible in $K[X]$.
    \end{itemize}
\end{definition}
\begin{remark}
    This definition is a special case of \Cref{th:algebraic_element_and_transcendental_element}.
\end{remark}

\begin{definition}{Algebraic Extension}{}
    A field extension $L/K$ is \textbf{algebraic} if every element of $L$ is \hyperref[th:algebraic_element_and_transcendental_element]{algebraic} over $K$. That is, for any $\alpha\in L$, there exists a nonzero polynomial $f\in K[X]$ such that $f(\alpha)=0$.
\end{definition}

\begin{proposition}{Simple Algebraic Extension as a Quotient Ring}{simple_algebraic_extension_as_a_quotient_ring}
    Let $L/K$ be a field extension and $\alpha\in L$ be an algebraic element over $K$. Suppose $P(X)\in K[X]$ is an irreducible annihilating polynomial of $\alpha$, or equivalently $P(X)\in K[X]-\{0\}$ is a constant multiple of the minimal polynomial $m_\alpha(X)$ of $\alpha$ over $K$. Then
    \[
    K(\alpha)\cong K[X]/(P(X))
    \]
    and $[K(\alpha):K]=\deg P(X)$. 
\end{proposition}
\begin{prf}
    The equivalence between irreducible annihilating polynomial and constant multiple of minimal polynomial follows from \Cref{th:irreducible_annihilating_polynomial_is_minimal_polynomial}. According to \Cref{th:structure_of_K_a}, we see
    \[
    K[X]/(P(X))\cong K[\alpha]=\left\{ f(\alpha) \in L\midv f\in K[X]
    \right\},
    \]
    $K[\alpha]$ is a field and $[K[\alpha]:K]=\deg P(X)$. It is sufficient to show $K(\alpha)=K[\alpha]$. It is clear that $K[\alpha]\subseteq K(\alpha)$. Since $K(\alpha)$ is the smallest subfield of $L$ containing $K$ and $\alpha$, we have $K(\alpha)\subseteq K[\alpha]$. Therefore, $K(\alpha)=K[\alpha]$.
\end{prf}

\begin{corollary}{Power Basis of Simple Algebraic Extension $K(\alpha)$}{power_basis_of_simple_algebraic_extension}
    Let $L/K$ be a field extension and $\alpha\in L$ be an algebraic element over $K$. 
    Then $1,\alpha,\alpha^2,\cdots, \alpha^{n-1}$ is a $K$-basis of $K(\alpha)$, where $n=[K(\alpha):K]$.
\end{corollary}
\begin{prf}
    Let $n=[K(\alpha):K]$. By \Cref{th:simple_algebraic_extension_as_a_quotient_ring}, the minimal polynomial of $\alpha$ over $K$ has degree $n$. Thus
    we can assume 
    \[
    m_\alpha(X)=X^n+a_{n-1}X^{n-1}+\cdots + a_1 X + a_0\in K[X]
    \]
    is the minimal polynomial of $\alpha$ over $K$. Note 
    \[
    m_\alpha(\alpha)=0\implies \alpha^n =-\left(a_{n-1}\alpha^{n-1}+\cdots + a_1 \alpha + a_0\right).
    \]
    We see $K(\alpha)$ is spanned by $1,\alpha,\alpha^2,\cdots, \alpha^{n-1}$ over $K$, i.e.
    \[
    K(\alpha)=\operatorname{span}_K\{1,\alpha,\dots,\alpha^{n-1}\}.
    \]
    If there exists $b_0,b_1,\cdots, b_{n-1}\in K$ such that
    \[
    b_0 + b_1 \alpha + \cdots + b_{n-1}\alpha^{n-1}=0,
    \]
    then the polynomial
    \[
    f(X)=b_0 + b_1 X + \cdots + b_{n-1}X^{n-1}\in K[X]
    \]
    satisfies $f(\alpha)=0$. Since $\deg f < \deg m_\alpha$, we must have $f(X)=0$ and $b_0=b_1=\cdots = b_{n-1}=0$. This implies $1,\alpha,\alpha^2,\cdots, \alpha^{n-1}$ is linearly independent over $K$. Therefore, $1,\alpha,\alpha^2,\cdots, \alpha^{n-1}$ is a $K$-basis of $K(\alpha)$.
\end{prf}

\begin{proposition}{}{K_embedding_is_K_isomorphism_for_algebraic_extension}
    Let $L/K$ be an algebraic extension. Then any $K$-embedding $L\hookrightarrow L$ is an $K$-isomorphism, namely
    \[
    \mathrm{End}_{\left(\mathsf{Field}/K\right)}\left(L/K\right) = \mathrm{Aut}_{\left(\mathsf{Field}/K\right)}\left(L/K\right).
    \]
\end{proposition}
\begin{prf}
    Let $u:L\hookrightarrow L$ be a $K$-embedding. It is sufficient to show $u$ is surjective. Given $\alpha\in L$, suppose $m_\alpha(X)\in K[X]$ is the minimal polynomial of $\alpha$ over $K$. Let
    \[
    S_{m_\alpha}=\left\{x \in L\midv m_\alpha(x)=0\right\}=\{ \alpha_1,\cdots,\alpha_m\}
    \]
    be the set of roots of $m_\alpha(X)$ in $L$. Let $L_0=K(\alpha_1,\cdots,\alpha_m)$.
    By \Cref{th:K_automorphism_acts_on_set_of_roots}, $u|_{S_{m_\alpha}}:S_{m_\alpha}\to S_{m_\alpha}$ is a bijection. Therefore, $u(L_0)\subseteq L_0$ and $u|_{L_0}:L_0\to L_0$ is a $K$-embedding. Since $L_0/K$ is a finite extension, $u|_{L_0}$ is an $K$-isomorphism. Therefore, $u|_{L_0}$ is surjective and there exists $\beta\in L_0$ such that $u(\beta)=\alpha$. This implies $u$ is surjective.
\end{prf}
If $L/K$ is not algebraic, we have the following counterexample.
\begin{example}{}{}
    $\mathbb{Q}(\pi)/\mathbb{Q}$ is a transcendental extension. We can check that 
    \begin{align*}
        u:\mathbb{Q}(\pi)&\longrightarrow \mathbb{Q}(\pi)\\
        \pi&\longmapsto \pi^2
    \end{align*}
    is a $\mathbb{Q}$-embedding but not a $\mathbb{Q}$-isomorphism.
\end{example}

\subsection{Finitely Generated Extension}


\begin{definition}{Generated Subextension}{generated_subextension}
    Let $L/K$ be a field extension and $S\subseteq L$. The \textbf{subextension generated by $S$} is the smallest subextension of $L/K$ containing $S$, denoted by $K(S)/K$. If $S=\{\alpha_1,\cdots,\alpha_n\}$ is a finite set, we write $K(S)=K(\alpha_1,\cdots,\alpha_n)$.
\end{definition}
\begin{prf}
    Here the smallest means if $M/K$ is another subextension of $L/K$ containing $S$, then $K(S)/K$ is a subextension of $M/K$.
\end{prf}

\begin{proposition}{Equivalent Characterizations of Generated Subextension}{}
    Let $L/K$ be a field extension and $S\subseteq L$. Suppose $M/L$ is a subextension of $L/K$. The following are equivalent:
    \begin{enumerate}[(i)]
        \item $M/K=K(S)/K$.
        \item $M/K$ is the intersection of all subextensions of $L/K$ containing $S$.
        \item 
        \[
        M=\left\{Q(\alpha_1,\cdots,\alpha_n)\in L\midv n\in \mathbb{Z}_{\ge 1}, \; Q\in K(X_1,\cdots,X_n),\; \alpha_i\in S \right\}.
        \]
    \end{enumerate}
\end{proposition}

\begin{definition}{Compositum of Field Extensions}{compositum_of_field_extensions}
    Let $\Omega/ K$ be a field extension and $\left(L_i/K\right)_{i\in I}$ be a collection of subextensions of $\Omega/K$. The \textbf{compositum} of $\left(L_i/K\right)_{i\in I}$ is the \hyperref[th:generated_subextension]{subextension of $\Omega/K$ generated by $\cup_{i\in I} L_i$}, which is denoted by
    \[
    \left(\bigvee_{i\in I} L_i\right)/K:= K\left(\bigcup_{i\in I} L_i\right)/K.
    \]
    We also denote the compositum by $L_1 L_2 \cdots L_n/K$ if $I$ is finite.
\end{definition}

\begin{definition}{Finitely Generated Extension}{}
    A field extension $L/K$ is \textbf{finitely generated} if there exists a finite set $S\subseteq L$ such that $L=K(S)$.
\end{definition}

\begin{definition}{Simple Extension}{}
    A field extension $L/K$ is \textbf{simple} if $L=K(\alpha)$ for some $\alpha\in L$.
\end{definition}


\subsection{Finite Extension}

\begin{definition}{Degree of Field Extension}{}
    Let $L/K$ be a field extension. The \textbf{degree} of $L/K$ is the dimension of $L$ as a $K$-vector space, denoted by $[L:K]=\dim_K L$.
\end{definition}

\begin{definition}{Finite Extension}{}
    A field extension $L/K$ is \textbf{finite} if $[L:K]<\infty$.
\end{definition}

\begin{proposition}{}{}
    Let $L/K$ be a finite generated extension and $L=K(\alpha_1,\cdots,\alpha_n)$. If $\alpha_1,\cdots,\alpha_n$ are algebraic elements over $K$, then $L/K$ is a finite extension and $L=K[\alpha_1,\cdots,\alpha_n]$, where $K[\alpha_1,\cdots,\alpha_n]$ denotes the $K$-subalgebra of $L$ generated by $\alpha_1,\cdots,\alpha_n$.
\end{proposition}

\begin{proposition}{Equivalent Characterization of Finite Extension}{}
    Let $L/K$ be a field extension. The following are equivalent:
    \begin{enumerate}[(i)]
        \item $L/K$ is a finite extension.
        \item $L/K$ is a finitely generated algebraic extension.
    \end{enumerate}
\end{proposition}

\begin{proposition}{}{}
    If $L/K$ is a finite extension and $M/L$ be a subextension of $L/K$, then the following are equivalent:
    \begin{enumerate}[(i)]
        \item $M/K=L/K$
        \item $M/K$ and $L/M$ are $K$-isomorphic.
        \item $[M:L]=[L:K]$.
    \end{enumerate}
\end{proposition}

\begin{lemma}{Zariski's Lemma}{}
    If a field $L$ is a finite-type $K$-algebra, then $L$ is a finite extension of $K$ (that is, $L$ is also finitely generated as a $K$-linear space).
\end{lemma}

\begin{example}{Quadratic Extension}{}
    Suppose $L/K$ is a quadratic extension and $\mathrm{char}(K)\neq 2$. Then there exists $\sqrt{a}\in L$ such that $L=K(\sqrt{a})$ and the minimal polynomial of $\sqrt{a}$ is $X^2-a\in K[X]$.
\end{example}
\begin{prf}
    Since $L/K$ is a quadratic extension. For any $\alpha\in L-K$, suppose the minimal polynomial of $\alpha$ over $K$ is $m_\alpha(X)\in K[X]$ with $\deg m_\alpha\le 2$. If $\deg m_\alpha=1$, then $m_\alpha(X)=X-\alpha$. This forces $\alpha\in K$, which contradicts the assumption. Thus $\deg m_\alpha=2$. Since $\mathrm{char}(K)\neq 2$, we can write
    \[
    m_\alpha(X)=X^2+bX+c=\left(X+\frac{b}{2}\right)^2-\frac{b^2}{4}+c.
    \]
    for some $b,c\in K$ and we have
    \[
    m_\alpha(\alpha)=\left(\alpha+\frac{b}{2}\right)^2-\frac{b^2}{4}+c=0.
    \]
    Take $\sqrt{a}:=\alpha+\frac{b}{2}$ and $f(X):=X^2-b^2/4+c\in K[X]$. We have 
    \[
       f\left(\sqrt{a}\right)=\left(\alpha+\frac{b}{2}\right)^2-\frac{b^2}{4}+c=0.
    \]
    Note that $\sqrt{a}\notin K$. Otherwise, we have $\alpha=\sqrt{a}-\frac{b}{2}\in K$, which leads to a contradiction. Then the minimal polynomial of $\sqrt{a}$ has degree $2$. Therefore, $f$ is the minimal polynomial of $\sqrt{a}$ over $K$. This means $K(\sqrt{a})$ is a quadratic extension of $K$. Therefore, $K(\sqrt{a})$ is a 2-dimensional $K$-subspace of $L$ and accordingly $K(\sqrt{a})=L$.
\end{prf}
\section{Algebraic Closure}

\begin{proposition}{Simple Extension by Adjoining a Root of an Irreducible Polynomial}{}
    Let $K$ be a field and $f\in K[X]$ be a irreducible polynomial. Denote $L:=K[X]/(f)$ and $\alpha:=X+(f(X))\in L$. Then
    \begin{enumerate}[(i)]
        \item $u:K\hookrightarrow K[X]\twoheadrightarrow L$ is a field extension of degree $\deg f$.
        \[
         [L:K]=\deg f.
        \]
        \item $L=K(\alpha)$.
        \item $\alpha$ is algebraic over $K$ with minimal polynomial $f\in K[X]$.
    \end{enumerate}  
\end{proposition}
\begin{proof}
    It is easy to check 
    \[
    \left\{\alpha^k=X^k+(f(X))\midv k=0,1,\cdots, \deg f-1\right\}
    \]
    is a $K$-basis of $L$. Therefore, $[L:K]=\deg f$. Since 
    \[
        f(\alpha)=f(X+(f(X)))=f(X)+(f(X))=0+(f(X)).
    \]
    $\alpha$ is algebraic over $K$ with minimal polynomial $f$.
\end{proof}

\begin{proposition}{}{field_isomorphism_lift_uniquely_to_simple_extension}
    Let $L_1/K_1$ and $L_2/K_2$ be field extensions and $\varphi:K_1\xrightarrow{\sim} K_2$ is a field isomorphism. Let $\widetilde{\varphi}:K_1[X]\to K_2[X]$ be the ring homomorphism induced by $\varphi:K_1\to K_2$ and denote $\leftindex^{\varphi}\!f:=\widetilde{\varphi}(f)\in K_2[X]$ for any $f\in K_1[X]$.
    \begin{enumerate}[(i)]
        \item Suppose $\eta:L_1\hookrightarrow L_2$ is field extension such that the following diagram commutes
        \[
            \begin{tikzcd}
                L_1\arrow[r, "\eta"]                    & L_2                   \\
                K_1 \arrow[u, "u_1", hook] \arrow[r, "\varphi"', "\sim"] & K_2 \arrow[u, "u_2"', hook]
                \end{tikzcd}
        \]
        If $\alpha\in L_1$ is algebraic over $K_1$ with minimal polynomial $f\in K_1[X]$, then $\eta(\alpha)\in L_2$ is algebraic over $K_2$ with minimal polynomial $\leftindex^{\varphi}\!f\in K_2[X]$.
        \item If $\alpha\in L_1$ is algebraic over $K_1$ with minimal polynomial $f_1\in K_1[X]$, and $\beta\in L_2$ is algebraic over $K_2$ with minimal polynomial $f_2:=\varphi(f_1)\in K_2[X]$, then  there exists a field isomorphism $\psi:K_1(\alpha_1)\xrightarrow{\sim} K_2(\alpha_2)$ such that $\psi(\alpha_1)=\alpha_2$ and the following diagram commutes
        the following diagram commutes
    \[
        \begin{tikzcd}
            K_1(\alpha_1) \arrow[r, "\psi", dashed]                    & K_2(\alpha_2)                        \\
            K_1 \arrow[u, "u_1", hook] \arrow[r, "\varphi"', "\sim"] & K_2 \arrow[u, "u_2"', hook]
            \end{tikzcd}
    \]
    \end{enumerate}
    
\end{proposition}
\begin{prf}
    \begin{enumerate}[(i)]
        \item Since $f$ is the minimal polynomial of $\alpha$ over $K_1$, $f$ is monic and irreducible in $K_1[X]$ and we have
        \[
            f(\alpha)=\sum_{k=0}^n c_k\alpha^k=0.
        \]
        Since $\varphi:K_1\to K_2$ is a field isomorphism, we have $ \leftindex^{\varphi}\!f$ is monic and irreducible in $K_2[X]$. Note that
        \[
            \leftindex^{\varphi}\!f(\eta(\alpha))=\sum_{k=0}^n \varphi(c_k)\eta(\alpha)^k=\sum_{k=0}^n \eta(c_k)\eta(\alpha)^k=\eta\left(\sum_{k=0}^n c_k\alpha^k\right)=\eta(0)=0.
        \]
        By \Cref{th:irreducible_annihilating_polynomial_is_minimal_polynomial} we show $\leftindex^{\varphi}\!f$ is the minimal polynomial of $\eta(\alpha)$ over $K_2$.
        \item Suppose $\pi_1:K_1[X]\to K_1[X]/(f_1)$ and $\pi_2:K_2[X]\to K_2[X]/(f_2)$ are the canonical projections. Since for any $g\in K_1[X]$, 
        \[
        \pi_2(\widetilde{\varphi}(gf_1))=\pi_2(\widetilde{\varphi}(g)\widetilde{\varphi}(f_1))=\pi_2(\widetilde{\varphi}(g)\varphi(f_1))=\pi_2(\widetilde{\varphi}(g)f_2)=\pi_2(\widetilde{\varphi}(g))\pi_2(f_2)=0,
        \]
        we see $(f_1)\subseteq \ker (\pi_2\circ \widetilde{\varphi})$. By the universal property of $\pi_1:K_1[X]\to K_1[X]/(f_1)$, there exists a unique ring homomorphism 
        \begin{align*}
            \widehat{\varphi}:K_1[X]/(f_1)&\longrightarrow K_2[X]/(f_2)\\
            g+(f_1)&\longmapsto \widetilde{\varphi}(g)+(f_2)
        \end{align*}
        such that the following diagram commutes
        \[
            \begin{tikzcd}
                K_1(\alpha_1) \arrow[r, "\psi"]                                         & K_2(\alpha_2)                           \\
                {K_1[X]/(f_1)} \arrow[r, "\widehat{\varphi}"] \arrow[u, "\sim"]         & {K_2[X]/(f_2)} \arrow[u, "\sim"']       \\
                {K_1[X]} \arrow[u, "\pi_1", two heads] \arrow[r, "\widetilde{\varphi}"] & {K_2[X]} \arrow[u, "\pi_2"', two heads] \\
                K_1 \arrow[r, "\varphi"'] \arrow[u, hook]                               & K_2 \arrow[u, hook]                    
            \end{tikzcd}
        \]
        Let $\psi:K(\alpha)\hookrightarrow L_2$ be the composition of the following maps
        \[
        \begin{tikzcd}
            K_1(\alpha_1) \arrow[r, "\sim"] & K_1[X]/(f_1) \arrow[r, "\widehat{\varphi}"] & K_2[X]/(f_2)\arrow[r,"{\sim}"] &  K_2(\alpha_2)\\[-1.5em]
            \alpha_1 \arrow[r, mapsto] & X \arrow[r, mapsto] & X \arrow[r, mapsto] & \alpha_2
        \end{tikzcd}
        \]
        Then $\psi$ is a $K$-embedding such that $\psi(\alpha_1)=\alpha_2$. 
    \end{enumerate}
\end{prf}

\begin{corollary}{$K$-embedding Preserves Algebraic Element and Minimal Polynomial}{K_embedding_preserves_algebraic_element_and_minimal_polynomial}
    Let $L_1/K$ and $L_2/K$ be field extensions.
    \begin{enumerate}[(i)]
        \item Suppose $u:L_1\hookrightarrow L_2$ is a $K$-embedding. If $\alpha\in L_1$ is algebraic over $K$ with minimal polynomial $f\in K[X]$, then $u(\alpha)\in L_2$ is also algebraic over $K$ with minimal polynomial $f\in K[X]$.
        \item If $\alpha\in L_1$ and $\beta\in L_2$ are algebraic over $K$ with the same minimal polynomial $f\in K[X]$, then there exists a unique $K$-embedding $\psi:K(\alpha)\hookrightarrow L_2$ such that $\psi(\alpha)=\beta$. Furthermore, $\psi(K(\alpha))=K(\beta)$.
    \end{enumerate}
\end{corollary}
\begin{prf}
    This a direct consequence of \Cref{th:field_isomorphism_lift_uniquely_to_simple_extension} by setting $K_1=K_2=K$.
    \begin{enumerate}[(i)]
        \item Since $f(\alpha)=0$, we have $f(u(\alpha))=u(f(\alpha))=0$. Since $f$ is irreducible over $K$, $f$ is also the minimal polynomial of $u(\alpha)$ over $K$.
        \item Let $\psi:K(\alpha)\hookrightarrow L_2$ be the composition of the following maps
        \[
        \begin{tikzcd}
            K(\alpha) \arrow[r, "\sim"] & K[X]/(f) \arrow[r, "\sim"] & K(\beta)\arrow[r,hook] & L_2\\[-1.5em]
            \alpha \arrow[r, mapsto] & X \arrow[r, mapsto] & \beta\arrow[r, mapsto] & \beta
        \end{tikzcd}
        \]
        Then $\psi$ is a $K$-embedding such that $\psi(K(\alpha))=K(\beta)$. 
        
        Suppose $\eta:K(\alpha)\hookrightarrow L_2$ is $K$-embedding such that $\eta(\alpha)=\beta$. Since $K(\alpha)$ is generated by $\alpha$ over $K$, $\eta$ is totally determined by $\eta(\alpha)$. Therefore, $\eta(\alpha)=\psi(\alpha)\implies \eta=\psi$.
      
    \end{enumerate}
\end{prf}

\begin{corollary}{}{}
    Let $L/K$ be a simple extension and $L=K(\alpha)$ for some $\alpha\in L$. Suppose $m_\alpha(X)\in K[X]$ is the minimal polynomial of $\alpha$ over $K$. Then 
    \begin{align*}
        \mathrm{ev}_{\alpha}:\mathrm{Aut}_{(\mathsf{Field}/K)}(K(\alpha)/K)&\longrightarrow \left\{ x\in K(\alpha)\midv m_\alpha(x)=0\right\}\\
        \sigma&\longmapsto \sigma(\alpha)
    \end{align*}
    is a bijection and we have
    \[
    |\mathrm{Aut}_{(\mathsf{Field}/K)}(K(\alpha)/K)|=\left|\left\{ x\in K(\alpha)\midv m_\alpha(x)=0\right\}\right|\le \deg m_\alpha(X),
    \]
    with equality if and only if $m_\alpha(X)$ splits over $K(\alpha)$ into $\deg m_\alpha(X)$ distinct linear factors.
\end{corollary}
\begin{prf}
    Suppose $\sigma:K(\alpha)\hookrightarrow K(\alpha)$ is a $K$-automorphism. By \Cref{th:K_embedding_preserves_algebraic_element_and_minimal_polynomial}, $u(\alpha)$ is algebraic over $K$ with minimal polynomial $m(X)$. Thus we can define a map
    \begin{align*}
        \mathrm{ev}_{\alpha}:\mathrm{Aut}_{(\mathsf{Field}/K)}(K(\alpha)/K)&\longrightarrow \left\{ x\in L\midv m(x)=0\right\}\\
        \sigma&\longmapsto \sigma(\alpha)
    \end{align*}
    Since $\sigma$ is totally determined by $\sigma(\alpha)$, for any $\sigma_1, \sigma_2\in \mathrm{Aut}_{(\mathsf{Field}/K)}(K(\alpha)/K)$, we have 
    \[
    \sigma_1(\alpha)=\sigma_2(\alpha)\implies \sigma_1=\sigma_2.
    \]
    Therefore, $\mathrm{ev}_{\alpha}$ is injective. Conversely, for any $\beta\in L$ such that $m(\beta)=0$, $m(X)$ is the minimal polynomial of $\beta$ over $K$ because $m(X)$ is irreducible over $K$. By \Cref{th:K_embedding_preserves_algebraic_element_and_minimal_polynomial}, there exists a unique $K$-embedding $\sigma:K(\alpha)\hookrightarrow K(\alpha)$ such that $\sigma(\alpha)=\beta$. By \Cref{th:K_embedding_is_K_isomorphism_for_algebraic_extension}, $\sigma$ is a $K$-automorphism. Therefore, $\mathrm{ev}_{\alpha}$ is surjective. Thus $\mathrm{ev}_{\alpha}$ is a bijection.
\end{prf}

\begin{definition}{Algebraic Closed Field}{}
    A field $K$ is \textbf{algebraically closed} if every nonconstant polynomial $f\in K[X]$ has a root in $K$.
\end{definition}

\begin{proposition}{Equivalent Characterization of Algebraic Closed Field}{equivalent_characterization_of_algebraic_closed_field}
    Let $K$ be a field. The following are equivalent:
    \begin{enumerate}[(i)]
        \item $K$ is algebraically closed.
        \item If $u:K\hookrightarrow L$ is an algebraic extension, then $u$ is an isomorphism.
        \item Every nonconstant polynomial $f\in K[X]$ splits into linear factors.
    \end{enumerate}
\end{proposition}
\begin{prf}
    (i)$\implies$(ii). For any $x\in L$, suppose $f\in K[X]$ is the minimal polynomial of $x$ over $K$. Since $K$ is algebraically closed, $f$ splits into linear factors in $K[X]$. Since $f$ is irreducible over $K$, there must be $f(X)=c(X-a)$ for some $c\in K^\times$ and $a\in K$. 
    \[
    u(f)(x)=u(c)(x-u(a))=0\implies x=u(a).
    \]
    Therefore, $u$ is surjective, which implies $u$ is an isomorphism.
\end{prf}

\begin{definition}{Algebraic Closure}{}
    An \textbf{algebraic closure} of a field $K$ is an algebraic extension $\overline{K}/K$ such that $\overline{K}$ is algebraically closed.
\end{definition}

\begin{proposition}{Existence and Uniqueness of Algebraic Closure}{}
    Let $K$ be a field. There exists an algebraic closure $\overline{K}/K$ and it is unique up to $K$-isomorphism.
\end{proposition}
\begin{prf}
    \textbf{Uniqueness}. Let 
    \[
    \mathcal{T}=\left\{(L,\iota)\midv L/K \text{ is a subextension of } \overline{K}/K \text{ and } \iota:L\hookrightarrow \widetilde{K}  \text{ is a $K$-embedding}\right\}.
    \]
    We can define a partial order on $\mathcal{T}$ by $(L,\iota)\le (L',\iota')$ if and only if $L\subseteq L'$ and $\iota'|_{L}=\iota$, or equivalently, the following diagram commutes
    \[
        \begin{tikzcd}
            & L' \arrow[ld, "\subseteq"', hook] \arrow[rd, "\iota'", hook]                                   &               \\
\overline{K} &                                                                                               & \widetilde{K} \\
            & L \arrow[uu,  "\subseteq"', hook] \arrow[lu, "\subseteq", hook] \arrow[ru, "\iota"', hook]    &               \\
            & K \arrow[u, "u"', hook] \arrow[luu, "\psi", hook, bend left] \arrow[ruu, "\phi"', bend right] &              
\end{tikzcd}
\]
For each chain $\mathcal{C}=\{(L_i,\iota_i)\}_{i\in I}$ in $\mathcal{T}$, we can define $L_{\mathcal{C}}:=\bigcup_{i\in I}L_i$ and $\iota_{\mathcal{C}}:L_{\mathcal{C}}\hookrightarrow \widetilde{K}$ by $\iota_{\mathcal{C}}(x)=\iota_i(x)$ whenever $x\in L_i$ for some $i\in I$. $\iota_{\mathcal{C}}$ is well-defined because for any $x\in L_i\cap L_j$ for some $i,j\in I$, we can assume $i\le j$ without loss of generality and we have $\iota_i(x)=\iota_j(s(x))$ where $s:L_i\hookrightarrow L_j$ is the inclusion map. Then $(L_{\mathcal{C}},\iota_{\mathcal{C}})$ is an upper bound of $\mathcal{C}$ in $\mathcal{T}$. By Zorn's lemma, there exists a maximal element $(L_{\max},\iota_{\max})$ in $\mathcal{T}$. 

We claim $L_{\max}=\overline{K}$ and prove by contradiction. Suppose $L_{\max}\ne \overline{K}$, then there exists $\alpha\in \overline{K}- L_{\max}$. Since $\alpha$ is algebraic over $L_{\max}$, there exists a minimal polynomial $m_\alpha\in L_{\max}[X]$ such that $m_\alpha(\alpha)=0$. Let $\widetilde{m_\alpha}:={\iota_{\max}}(m_\alpha)\in \widetilde{K}[X]$ be the image of $m_\alpha$ under $\iota_{\max}$. Since $\widetilde{K}$ is algebraically closed, $\widetilde{m_\alpha}$ has a root $\beta\in \widetilde{K}$. Since $m_\alpha\in L_{\max}[X]$ is the minimal polynomial of both $\alpha\in \overline{K}$ and $\beta\in \widetilde{K}$, by \Cref{th:K_embedding_preserves_algebraic_element_and_minimal_polynomial}, there exists a $L_{\max}$-embedding $\eta:L_{\max}(\alpha)\hookrightarrow \widetilde{K}$ such that $\eta(\alpha)=\beta$. Then $(L_{\max}(\alpha),\eta)\in \mathcal{T}$ and $(L_{\max}(\alpha),\eta)\le (L_{\max},\iota_{\max})$ does not hold. This contradicts the maximality of $(L_{\max},\iota_{\max})$. Therefore, $L_{\max}=\overline{K}$.

To show $\eta: \overline{K}\hookrightarrow \widetilde{K}$ is a $K$-isomorphism, it is sufficient to show $\eta(\overline{K})= \widetilde{K}$ is surjective. Since $\widetilde{K}$ is algebraically closed, $\eta(\overline{K})$ is algebraically closed. Since $\widetilde{K}/K$ is an algebraic extension, $K/\eta(\overline{K})$ is also an algebraic extension. According to \Cref{th:equivalent_characterization_of_algebraic_closed_field}, the inclusion $\eta(\overline{K})\hookrightarrow \widetilde{K}$ is an isomorphism, which implies $\eta(\overline{K})=\widetilde{K}$. Therefore, $\overline{K}$ is unique up to $K$-isomorphism.
\end{prf}

\begin{proposition}{Embed Algebraic Extension into Algebraic Closure}{embed_algebraic_extension_into_algebraic_closure}
    Let $L/K$ be an algebraic extension and $\overline{K}/K$ be an algebraic closure of $K$. Then there exists a $K$-embedding $\gamma:L\hookrightarrow \overline{K}$, namely
    \[
        \mathrm{Hom}_{(\mathsf{Field}/K)}(L,\overline{K})\ne \varnothing.
    \]
    If $L/K$ is a finite extension, then 
    \[
    |\mathrm{Hom}_{(\mathsf{Field}/K)}(L,\overline{K})|\le \left[L:K\right].
    \]
\end{proposition}
\begin{prf}
    Suppose $\phi:L\hookrightarrow \overline{L}$ is an algebraic closure of $L$. Then $\phi\circ u: K\hookrightarrow \overline{L}$ is an algebraic closure of $K$. By the uniqueness of algebraic closure, there exists a $K$-embedding $\psi:L\hookrightarrow \overline{K}$. Let $\gamma:=\psi\circ \phi$. Since the following diagram commutes
    \[
        \begin{tikzcd}
            L \arrow[r, "\phi", hook]                       & \overline{L} \arrow[d, "\psi"] \\
            K \arrow[u, "u", hook] \arrow[r, "\eta"', hook] & \overline{K}                  
            \end{tikzcd}
    \]
    we see $\gamma:L\hookrightarrow \overline{K}$ is a $K$-embedding.

    If $L/K$ is a finite extension, then $L=K(\alpha_1,\cdots,\alpha_n)$ for some $\alpha_1,\cdots,\alpha_n\in L$ and we have the following tower of field extensions
    \[
        \begin{tikzcd}
            &                                                 &                                                           &                        &                                                                    & \overline{K}                                    \\[+4em]
K \arrow[r, hook] \arrow[rrrrru, hook] & K(\alpha_1) \arrow[r, hook] \arrow[rrrru, hook] & {K(\alpha_1,\alpha_2)} \arrow[r, hook] \arrow[rrru, hook] & \cdots \arrow[r, hook] & {K(\alpha_1,\cdots,\alpha_{n-1})} \arrow[r, hook] \arrow[ru, hook] & {K(\alpha_1,\cdots,\alpha_n)=L} \arrow[u, "\gamma"', hook]
\end{tikzcd}
    \]
    Thus we get a chain of restrictions of $\gamma$:
    \[
        \begin{tikzcd}
            {\mathrm{Hom}_{(\mathsf{Field}/K)}\left(K(\alpha_1,\cdots,\alpha_n)/K,\overline{K}/K\right)} \arrow[d, "\mathrm{res}_n"']         & \gamma \arrow[d, maps to]                                      \\
            {\mathrm{Hom}_{(\mathsf{Field}/K)}\left(K(\alpha_1,\cdots,\alpha_{n-1})/K,\overline{K}/K\right)} \arrow[d, "\mathrm{res}_{n-1}"'] & {\gamma|_{K(\alpha_1,\cdots,\alpha_{n-1})}} \arrow[d, maps to] \\
            {\raisebox{1.5ex}{$\vdots$}}  \arrow[d, "\mathrm{res}_{2}"']                                                                                                 & {\raisebox{1.5ex}{$\vdots$}}  \arrow[d,maps to]                                      \\
            {\mathrm{Hom}_{(\mathsf{Field}/K)}\left(K(\alpha_1)/K,\overline{K}/K\right)} \arrow[d, "\mathrm{res}_1"']                         & \gamma|_{K(\alpha_1)} \arrow[d, maps to]                       \\
            {\mathrm{Hom}_{(\mathsf{Field}/K)}\left(K/K,\overline{K}/K\right)}                                                                & \gamma|_{K}                                                   
            \end{tikzcd}
    \]
\end{prf}


\begin{proposition}{Algebraic Closure of an Algebraic Extension}{algebraic_closure_of_algebraic_extension}
    Let $\iota: K\hookrightarrow L$ be an algebraic extension. 
    \begin{enumerate}[(i)]
        \item If $\tau:K\hookrightarrow \overline{K}$ is an algebraic closure of $K$, then there exists a $K$-embedding $\sigma:L\hookrightarrow \overline{K}$ such that $\sigma$ is an algebraic closure of $L$.
        \item If $\eta:L\hookrightarrow \overline{L}$ is an algebraic closure of $L$, then $\eta\circ \iota:K\hookrightarrow \overline{L}$ is an algebraic closure of $K$.
    \end{enumerate}
    Specially, if $\overline{K}/K$ is an algebraic closure of $K$, and $\overline{L}/L$ is an algebraic closure of $L$, then we have field isomorphism
    \[
       \overline{K}\cong \overline{L}.
    \]
\end{proposition}



\section{Normal Extension}

\begin{definition}{Splitting of Polynomial over a Field}{}
    Let $L/K$ be a field extension and $f\in K[X]$ be a polynomial such that $\deg f \ge 1$. We say $f$ \textbf{splits} over $L$ if $f$ can be written as
    \[
    f(X)=a(X-\alpha_1)\cdots(X-\alpha_n)
    \]
    for some $a\in K^\times$, $\alpha_1,\ldots,\alpha_n\in L$.
\end{definition}

\begin{proposition}{}{inheritance_of_splitting}
    Let $L/K$ be a field extension and $f,g\in K[X]$ be polynomials. If $g$ splits over $L$ and $f\divides g$, then $f$ also splits over $L$.
\end{proposition}
\begin{prf}
    Suppose field embedding is $\iota:K\hookrightarrow L$ and $\tilde{f}$, $\tilde{g}\in L[X]$ are the images of $f,g$ under $\iota$. Since $g$ splits over $L$, we can write
    \[
    g(X)=c(X-\alpha_1)\cdots(X-\alpha_m)\in L[X]
    \]
    for some $c\in L^\times$, $\alpha_1,\ldots,\alpha_m\in L$. Since $f\divides g$, we must have
    \[
    f(X)=d(X-\alpha_{i_1})\cdots(X-\alpha_{i_n})\in L[X]
    \]
    for some $d\in L^\times$ and $1\le i_1<\cdots<i_n\le m$. This shows $f$ splits over $L$.
\end{prf}

\begin{definition}{Splitting Field}{splitting_field}
    Let $K$ be a field. Suppose $\mathcal{P}$ is a family of polynomials in $K[X]$. If $L/K$ is a field extension such that 
    \begin{enumerate}[(i)]
        \item Each $f\in \mathcal{P}$ splits over $L$,
        \item The set of roots of polynomials in $\mathcal{P}$ 
        \[
        S=\{\alpha\in L\mid f(\alpha)=0\text{ for some }f\in \mathcal{P}\}
        \]
        is the generating set of $L/K$, that is, $L=K(S)$,
    \end{enumerate}
    then we say $L/K$ is a \textbf{splitting field of $\mathcal{P}$ over $K$}. If $\mathcal{P}=\{f\}$ is a singleton, then we say $L/K$ is a \textbf{splitting field of $f$ over $K$}.
\end{definition}
\begin{remark}
    Splitting field is a field extension instead of a field. So this terminology is a little bit misleading. But for historical reasons, we still use it.

    If we explicitly state that $\mathcal{P}\subseteq K[X]$, then we can say $L/K$ is a splitting field of $\mathcal{P}$ and the phrase ``over $K$" becomes redundant information that can be omitted.

    We say $L/K$ is ``a" splitting field of $\mathcal{P}$ instead of ``the" splitting field of $\mathcal{P}$ because the splitting field of $\mathcal{P}$ is only unique up to isomorphism in $(\mathsf{Field}/K)$, not unique up to unique isomorphism in $(\mathsf{Field}/K)$.
\end{remark}

\begin{proposition}{Existance and Uniqueness of Splitting Field}{}
    Let $K$ be a field and $\mathcal{P}$ be a family of polynomials in $K[X]$. Then the splitting field of $\mathcal{P}$ over $K$ exists and is unique up to $K$-isomorphism.
\end{proposition}
\begin{prf}
    Suppose $\overline{K}/K$ is an algebraic closure of $K$. Suppose for each $f\in \mathcal{P}$, $f$ has degree $n_f$ and splits into
    \[
    f(X)=c_f(X-\alpha_{f,1})\cdots(X-\alpha_{f,n_f})
    \]
    in $\overline{K}$. Let $L_\mathcal{P}=K(\alpha_{f,k}:f\in \mathcal{P}, 1\le k\le n_f)$. Then $L_\mathcal{P}/K$ is a splitting field of $\mathcal{P}$ over $K$. 
    
    Suppose $L/K$ is another splitting field of $\mathcal{P}$ over $K$. Since $L$ is generated by the roots of polynomials in $\mathcal{P}$, we have $L\subseteq L_\mathcal{P}$. Since $L_\mathcal{P}$ is generated by the roots of polynomials in $\mathcal{P}$, we have $L_\mathcal{P}\subseteq L$. Therefore, $L=L_\mathcal{P}$.
\end{prf}

\begin{proposition}{Properties of Splitting Field}{}
    Let $K$ be a field. Suppose $\mathcal{P}$ is a family of polynomials in $K[X]$ and $L_\mathcal{P}$ is a splitting field of $\mathcal{P}$ over $K$. Then
    \begin{enumerate}[(i)]
        \item $L_\mathcal{P}/K$ is an algebraic extension.
        \item If $\mathcal{P}$ is finite, then $L_\mathcal{P}/K$ is a finite extension.
        \item If $\mathcal{P}=\{f\}$ is a singleton, then
        \[
        \left[L_f:K\right]\le\left(\deg f\right)!\;,
        \]
        where $L_f:=L_{\{f\}}$.
    \end{enumerate}
\end{proposition}

\begin{definition}{Normal Extension}{}
    Let $L/K$ be a field extension. We say $L/K$ is a \textbf{normal extension} if for any $\alpha\in L$, the minimal polynomial $m_\alpha\in K[X]$ of $\alpha$ splits completely into linear factors over $L$, i.e.,
    \[
    m_\alpha(X)=(X-\alpha_1)\cdots(X-\alpha_n), \quad \alpha_1,\ldots,\alpha_n\in L.
    \]
\end{definition}

\begin{proposition}{Equivalent Characterization of Normal Extension}{}
    Let $L/K$ be a field extension. The following are equivalent:
    \begin{enumerate}[(i)]
        \item $L/K$ is a normal extension.
        \item $L$ is a splitting field of a family of nonconstant polynomials in $K[X]$.
        \item Let $\overline{K}/K$ be an algebraic closure of $K$. For any $K$-embedding $\iota_1,\iota_2\in\mathrm{Hom}_{(\mathsf{Field}/K)}(L/K,\overline{K}/K)$, we have $\iota_1(L)=\iota_2(L)$.
    \end{enumerate}
\end{proposition}

\begin{proposition}{}{}
    Let $L/E/K$ be a tower of field extensions given by $\iota_1:K\hookrightarrow E$ and $\iota_2:E\hookrightarrow L$. If $L/K$ is normal, then given any $K$-embedding $\sigma:E\hookrightarrow L$, there exists a $K$-automorphism $\widetilde{\sigma}:L\hookrightarrow L$ such that 
    \[
       \widetilde{\sigma}\circ \iota_2 = \sigma,
    \]
    that is, the following diagram commutes
   \[
    \begin{tikzcd}
                                                         & L                                                                      &                                             \\
E \arrow[ru, "\sigma", hook] \arrow[rr, "\iota_2", hook] &                                                                        & L \arrow[lu, "\widetilde{\sigma}"', dashed] \\
                                                         & K \arrow[lu, "\iota_1", hook] \arrow[ru, "\iota_2\circ\iota_1"', hook] &                                            
\end{tikzcd}
    \]
\end{proposition}
\begin{prf}
    Let $j: L \hookrightarrow \overline{L}$ be an algebraic closure of $L$. By \Cref{th:algebraic_closure_of_algebraic_extension}, $j\circ \iota_1:K\hookrightarrow \overline{L}$ is an algebraic closure of $K$. 
    
    Consider the $K$-embedding $j\circ \sigma: E \hookrightarrow \overline{L}$. 
\end{prf}

\section{Separable Extension}
\begin{definition}{Separable Polynomial}
    Let $K$ be a field and $f\in K[X]$ be a nonzero polynomial. We say $f$ is \textbf{separable} if $f$ has no multiple roots in an algebraic closure of $K$. We say $f$ is \textbf{inseparable} if $f$ is not separable.
\end{definition}

\begin{definition}{Separable Degree of Irreducible Polynomial}{separable_degree_of_irreducible_polynomial}
 Let $K$ be a field and $f\in K[x]$ be an irreducible polynomial. The \textbf{separable degree} of $f$ is the cardinality of the set of roots of $f$ in any algebraic closure $\overline{K}$ of $K$, which is denoted by
 \[
 \deg_s(f):= \left|\left\{\alpha \in \overline{K} : f(\alpha) = 0\right\}\right|
 \]
\end{definition}

\begin{proposition}{Equivalent Characterization of Separable Polynomial}{equivalent_characterization_of_separable_polynomial}
    Let $K$ be a field and $f\in K[X]$ be a nonzero polynomial. The following are equivalent:
    \begin{enumerate}[(i)]
        \item $f$ is separable.
        \item $f$ has no multiple roots in a splitting field of $K$.
        \item $\mathrm{gcd}(f,f')= 1$
    \end{enumerate}
\end{proposition}

\begin{proposition}{Equivalent Characterization of Inseparability for Irreducible Polynomial}{}
    Let $K$ be a field and $f\in K[X]$ be an irreducible polynomial. The following are equivalent:
    \begin{enumerate}[(i)]
        \item $f$ is inseparable.
        \item $f$ has multiple roots in a splitting field of $K$.
        \item $f'=0$.
        \item $\mathrm{char}(K)=p>0$ and $f(X)=g(X^p)$ for some $g\in K[X]$.
    \end{enumerate}
\end{proposition}
\begin{prf}
    (i)$\implies$(ii). Suppose $f$ is inseparable. If $f$ splits into
    \[
    f(X)=c(X-\alpha_1)^{n_1}\cdots(X-\alpha_k)^{n_k}\in \overline{K}[X],
    \]
    over $\overline{K}$, then $K(\alpha_1,\ldots,\alpha_k)$ is a splitting field of $f$ over $K$. Hence $f$ has multiple roots in a splitting field of $K$.

    (ii)$\implies$(iii). Suppose $f$ has multiple roots in a splitting field of $K$ and $\alpha$ is a multiple root of $f$. Then $\gcd (f,f')\ne 1$, which implies there exists a common factor of $f$ and $f'$. Note $f$ is irreducible. The only factors of $f$ are $f$ and $1$, up to a unit. Hence $f|f'$. Since $\deg f'<\deg f$, there must be $f'=0$.

    (iii)$\implies$(iv). Suppose 
    \[
        f(X)=\sum_{n=0}^N a_n X^n\in K[X]
    \]
    is an irreducible polynomial and 
    \[
        f'(X)=\sum_{n=1}^N na_n X^{n-1}=0,
    \]
    Thus for any $n\in \mathbb{Z}_{\ge 1}$, we have
    \begin{align*}
        na_n=0&\implies\left(n\cdot 1_K\right)a_n=0\\
        &\implies n\cdot 1_K=0\text{ or }a_n=0
    \end{align*}
    Note $a_N\ne 0$. There must be $N\cdot 1_K=0$, which implies $\mathrm{char}(K)=p>0$. Furthermore, since $n\cdot 1_K=0\iff p\mid n$, we have
    \[
    p \nmid n\iff n\cdot 1_K\ne 0 \implies a_n=0.
    \]
    So we can write
    \[
    f(X)=\sum_{k=0}^M a_{kp}X^{kp}=g(X^p),
    \]
    where 
    \[
    g(X)=\sum_{k=0}^M a_{kp}X^k\in K[X].
    \]

    (iv)$\implies$(i). Suppose $\mathrm{char}(K)=p>0$ and $f(X)=g(X^p)$ for some 
    \[
    g(X)=\sum_{k=0}^M b_kX^k\in K[X].
    \]
    Suppose $y_1,\cdots, y_M$ are the roots of $g$ in an algebraic closure $\overline{K}$ of $K$. For each $y_i$, we can find $x_i\in \overline{K}$ such that $x_i^p=y_i$. Thus we have
    \[
    f(X)=c\prod_{i=1}^M(X^p-y_i)=c\prod_{i=1}^M(X^p-x_i^p)=c\prod_{i=1}^M(X-x_i)^p\in \overline{K}[X],
    \]
    which implies $f$ has multiple roots in an algebraic closure of $K$.
\end{prf}

\begin{definition}{Separable Element}{}
    Let $L/K$ be a field extension and $\alpha\in L$ be an algebraic element over $K$. We say $\alpha$ is a \textbf{separable element} over $K$ if the minimal polynomial of $\alpha$ over $K$ is separable. 
\end{definition}

\begin{definition}{Separable Extension}{}
    A algebraic extension $L/K$ is \textbf{separable} if every element of $L$ is separable over $K$.
\end{definition}


\begin{lemma}{}{}
    Let $L/K$ be a finite extension such that $L=K(\alpha_1,\cdots,\alpha_n)$ for some $\alpha_1,\cdots,\alpha_n\in L$. Suppose $\overline{K}/K$ is an algebraic closure of $K$. Define $K_0:=K$ and $K_i:=K_{i-1}(\alpha_i)$ for $1\le i\le n$. Denote the minimal polynomial of $\alpha_i$ over $K_{i-1}$ by $m_{\alpha_i}\in K_{i-1}[X]$. 
    \begin{enumerate}[(i)]
        \item If $m_{\alpha_i}$ is separable for any $1\le i\le n$, then 
        \[
        \left|\mathrm{Hom}_{\left(\mathsf{Field}/K\right)}(L,\overline{K})\right| = [L:K]
        \]
        and $L/K$ is separable.
        \item If there exists $1\le j\le n$ such that $m_{\alpha_j}$ is inseparable, then
        \[
        \left|\mathrm{Hom}_{\left(\mathsf{Field}/K\right)}(L,\overline{K})\right| < [L:K].
        \]
    \end{enumerate}
\end{lemma}
\begin{prf}
    If $\alpha_i$ is separable over $K_{i-1}$, then the minimal polynomial $m_{\alpha_i}$ is separable, which implies 
    \[
    \operatorname{deg}_s\left(m_{\alpha_i}\right)=\operatorname{deg}\left(m_{\alpha_i}\right)=\left[K_i: K_{i-1}\right]
    \] 
    (last equality by Lemma 9.9.2). By multiplicativity (Lemma 9.7.7) we have

    $$
    [K: F]=\prod\left[K_i: K_{i-1}\right]=\prod \operatorname{deg}\left(m_{\alpha_i}\right)=\prod \operatorname{deg}_s\left(m_{\alpha_i}\right)=\left|\operatorname{Mor}_F(K, \bar{F})\right|
    $$

    where the last equality is Lemma 9.12.9. By the exact same argument we get the strict inequality $\left|\operatorname{Mor}_F(K, \bar{F})\right|<[K: F]$ if one of the $\alpha_i$ is not separable over $K_{i-1}$.

    Finally, assume again that each $\alpha_i$ is separable over $K_{i-1}$. We will show $K / F$ is separable. Let $\gamma=\gamma_1 \in K$ be arbitrary. Then we can find additional elements $\gamma_2, \ldots, \gamma_m$ such that $K=F\left(\gamma_1, \ldots, \gamma_m\right)$ (for example we could take $\gamma_2=\alpha_1, \ldots, \gamma_{n+1}=\alpha_n$ ). Then we see by the last part of the lemma (already proven above) that if $\gamma$ is not separable over $F$ we would have the strict inequality $\left|\operatorname{Mor}_F(K, \bar{F})\right|<[K: F]$ contradicting the very first part of the lemma (already prove above as well).
\end{prf}
\begin{proposition}{Equivalent Characterization of Finite Separable Extension}{}
    Let $L/K$ be a finite field extension. The following are equivalent:
    \begin{enumerate}[(i)]
        \item $L/K$ is a separable extension.
        \item $L=K(\alpha_1,\cdots,\alpha_n)$ for some separable elements $\alpha_1,\cdots,\alpha_n\in L$ over $K$.
        \item Let $\overline{K}/K$ be an algebraic closure of $K$.  
        \[
        \left|\mathrm{Hom}_{\left(\mathsf{Field}/K\right)}(L,\overline{K})\right| = [L:K]
        \]
        \item The trace pairing is \hyperref[th:nondegenerate_bilinear_form]{nondegenerate}.
        \item $L \otimes_K \overline{K} \;\cong\; \overline{K} \times \cdots \times \overline{K}$
    \end{enumerate}
\end{proposition}

\begin{definition}{Perfect Field}{perfect_field}
    A field $K$ is \textbf{perfect} if every finite extension of $K$ is separable.
\end{definition}

\begin{definition}{Equivalent Characterization of Perfect Field}{}
    Let $K$ be a field. The following are equivalent:
    \begin{enumerate}[(i)]
        \item $K$ is perfect.
        \item Every irreducible polynomial in $K[X]$ is separable.
        \item Every algebraic extension of $K$ is separable.
        \item Either $\mathrm{char}(K)=0$ or $\mathrm{char}(K)=p$ and the Frobenius endomorphism 
        \begin{align*}
            \sigma:K&\longrightarrow K\\
            x&\longmapsto x^p
        \end{align*}
        is a automorphism of $K$.
    \end{enumerate}
    
\end{definition}

\begin{example}{Examples of Perfect Fields}{}
    Examples of perfect fields include
    \begin{itemize}
        \item Field of characteristic $0$.
        \item Finite field.
        \item Algebraically closed field.
        \item Field which is algebraic over a perfect field.
    \end{itemize}
\end{example}


\section{Trace and Norm of Field Extension}


\begin{definition}{Trace and Norm of Finite Extension}{}
    Let $L/K$ be a finite field extension and $\alpha\in L$ be an algebraic element over $K$. We can define a $K$-linear map
    \begin{align*}
        l_{\alpha}:L&\longrightarrow L\\
        x&\longmapsto \alpha x
    \end{align*}
    called the \textbf{left multiplication by $\alpha$}.
    \begin{enumerate}[(i)]
        \item The \textbf{trace} of $\alpha\in L$ over $K$ is defined as
        \[
        \mathrm{Tr}_{L/K}(\alpha):=\mathrm{Tr}\left(l_{\alpha}\right).
        \]
        \item The \textbf{norm} of $\alpha\in L$ over $K$ is defined as
        \[
        \mathrm{N}_{L/K}(\alpha):=\mathrm{det}\left(l_{\alpha}\right).
        \]

    \end{enumerate}
\end{definition}
\begin{remark}
    This definition is a special case of \Cref{th:trace_norm_and_characteristic_polynomial}.
\end{remark}


\begin{definition}{Trace Pairing}{trace_pairing_for_finite_extension}
    Let $L/K$ be a finite extension. The \textbf{trace pairing} is the symmetric $K$-bilinear form
    \begin{align*}
        \langle \cdot,\cdot \rangle_{L/K}: L\times L&\longrightarrow K\\
        (x,y)&\longmapsto \mathrm{Tr}_{L/K}(xy).
    \end{align*}
\end{definition}
\begin{remark}
    This definition is a special case of \Cref{th:trace_pairing}.
\end{remark}


\section{Finite Field}

\begin{definition}{Finite Field}{}
    A \textbf{finite field} is a field with a finite number of elements.
\end{definition}
It is clear that the characteristic of a finite field is a prime number, otherwise the embedding from $\mathbb{Q}$ would force the field to be infinite. 

\begin{lemma}{Existance of Finite Field}{existance_of_finite_field}
    Assume $p$ be a prime number and $m\in\mathbb{Z}_{\ge1}$. Define $\mathbb{F}_p:=\mathbb{Z}/p\mathbb{Z}$. Let $K$ be the splitting field of the polynomial $f(X)=X^{p^m}-X\in \mathbb{F}_p[X]$. Then 
    \begin{enumerate}[(i)]
        \item $f(\alpha)=0$ for all $\alpha\in K$.
        \item $|K|=p^m$.
        \item $K/\mathbb{F}_p$ is an extension of degree $m$.
        
    \end{enumerate}
\end{lemma}
\begin{prf}
    \begin{enumerate}[(i)]
        \item Let $f(X)=X^{p^m}-X$ and $K/\mathbb{F}_p$ be a splitting field of $f$. For any $x,y\in K$, we have $(x+y)^{p^m}=x^{p^m}+y^{p^m}$, which implies the set of roots of $f$ in $K$
        \[
        M:=\{\alpha\in K\mid f(x)=0\}=\left\{\alpha\in K\mid \alpha^{p^m}=\alpha\right\}
        \]
        is a subfield of $K$. By \Cref{th:subfield_contains_prime_subfield}, $M/\mathbb{F}_p$ is a subextension of $K/\mathbb{F}_p$. Note that $f(X)=X^{p^m}-X\in \mathbb{F}_p[X]$ splits over $M$ and $M=\mathbb{F}_p(M)$. We see $M/\mathbb{F}_p$ is a splitting field of $f$ by \hyperref[th:splitting_field]{definition}. By the uniqueness of splitting field, $M/\mathbb{F}_p$ and $K/\mathbb{F}_p$ are $\mathbb{F}_p$-isomorphic. Since $K$ as a splitting field is finite, we get $M=K$. Therefore, for all $\alpha\in K$, there must be $f(\alpha)=0$.
        \item Since
        \[
        f'(X)=p^mX^{p^m-1}-1=-1,
        \]
        we have $\mathrm{gcd}(f,f')=1$. From \Cref{th:equivalent_characterization_of_separable_polynomial} we see $f$ has $p^m$ distinct roots in $K$, which implies $|M|\ge p^m$. On the other hand, $|M|\le \deg f =p^m$. Thus we have $|K|=|M|=p^m$. 
        \item From $|K|=|\mathbb{F}_p|^{[K:\mathbb{F}_p]}=p^{[K:\mathbb{F}_p]}=p^m$, we see $[K:\mathbb{F}_p]=m$.
    \end{enumerate}
  
\end{prf}

\begin{corollary}{Uniqueness of Finite Field}{uniqueness_of_finite_field}
For any prime number $p$ and $m\in\mathbb{Z}_{\ge1}$, there exists a unique finite field of order $q:=p^m$ up to isomorphism, denoted by $\mathbb{F}_{q}$. Any finite field must be isomorphic to $\mathbb{F}_{q}$ for some prime number $p$ and $m\in\mathbb{Z}_{\ge1}$.
\end{corollary}
\begin{prf}
    The existence of $\mathbb{F}_{q}$ follows from \Cref{th:existance_of_finite_field}. The uniqueness of $\mathbb{F}_{q}$ follows the uniqueness of splitting field. Finite field must have positive characteristic, which is a prime number $p$. Then the finite field is an extension of $\mathbb{F}_p$ of degree $m$. Any Characteristic zero field is infinite since it contains an isomorphic copy of $\mathbb{Q}$. Thus finite field $F$ must have positive characteristic, say $p$. Thus $F/\mathbb{F}_p$ is a finite extension and $|F|=p^{[F:\mathbb{F}_p]}$.
\end{prf}

Due to the uniqueness of finite fields, whenever we use the notation $\mathbb{F}_q$, it is understood that the proposition holds for any finite field of order $q$ we choose.

\begin{definition}{Frobenius Endomorphism of Commutative $\mathbb{F}_q$-algebra}{}
    Let $p$ be a prime number and $m\in\mathbb{Z}_{\ge1}$. Let $q=p^m$. The \textbf{Frobenius endomorphism} of $\mathbb{F}_q$-algebra is defined as the following $\mathbb{F}_q$-algebra homomorphism
    \begin{align*}
        \mathrm{Fr}_{q,A}:A&\longrightarrow A\\
        x&\longmapsto x^q
    \end{align*}
\end{definition}
\begin{remark}
    In the definition of \hyperref[th:frobenius_endomorphism_of_a_commutative_ring]{Frobenius endomorphism of a commutative ring}, we see $\mathrm{Fr}_{q,A}$ is a ring homomorphism. To check $\mathrm{Fr}_{q,A}$ is a $\mathbb{F}_q$-algebra homomorphism, we only need to check $\mathrm{Fr}_{q,A}$ is $\mathbb{F}_q$-linear. This is clear because for any $a\in\mathbb{F}_q$ and $x\in A$, we have
    \[
    \mathrm{Fr}_{q,A}(ax)=(ax)^q=a^qx^q=a\mathrm{Fr}_{q,A}(x).
    \]
\end{remark}
\begin{proposition}{Additivity of the Frobenius Powers}{additivity_of_the_frobenius_powers}
    Let $p$ be a prime number and $m\in\mathbb{Z}_{\ge1}$. Let $q=p^m$. Then for any $a,b\in \mathbb{F}_q$ and any $k\in\mathbb{Z}_{\ge 0}$, we have    
    \[
        \left(a+b\right)^{p^k}=a^{p^k}+b^{p^k}.
    \]
\end{proposition}
\begin{prf}
    Consider the Frobenius automorphism
    \begin{align*}
        \mathrm{Fr}_{p, \mathbb{F}_q}: \mathbb{F}_q&\longrightarrow \mathbb{F}_q\\
        x&\longmapsto x^p.
    \end{align*}
    For any $a,b\in \mathbb{F}_q$ and any $k\in\mathbb{Z}_{\ge 0}$, we have
    \[
    \left(a+b\right)^{p^k}=\left(\mathrm{Fr}_{p, \mathbb{F}_q}\right)^k(a+b)=\left(\mathrm{Fr}_{p, \mathbb{F}_q}\right)^k(a)+\left(\mathrm{Fr}_{p, \mathbb{F}_q}\right)^k(b)=a^{p^k}+b^{p^k}.
    \]
\end{prf}

\begin{proposition}{Functoriality of Frobenius Endomorphism}{}
    Let $p$ be a prime number and $m\in\mathbb{Z}_{\ge1}$. Let $q=p^m$. Let $A,B$ be commutative $\mathbb{F}_q$-algebras and $f:A\to B$ be a $\mathbb{F}_q$-algebra homomorphism. Then the following diagram commutes:
    \[
        \begin{tikzcd}
            A \arrow[d, "{\mathrm{Fr}_{q,A}}"'] \arrow[r, "f"] & B \arrow[d, "{\mathrm{Fr}_{q,B}}"] \\
            A \arrow[r, "f"']                                  & B                                 
            \end{tikzcd}
    \]
    This implies Frobenius endomorphism gives a natural transformation $\mathrm{Fr}_{q,-}:\mathrm{id}_{\mathbb{F}_q\text{-}\mathsf{CAlg}}\Rightarrow \mathrm{id}_{\mathbb{F}_q\text{-}\mathsf{CAlg}}$.
    \[
        \begin{tikzcd}[ampersand replacement=\&]
            \mathbb{F}_q\text{-}\mathsf{CAlg}\arrow[r, "\mathrm{id}"{name=A, above}, bend left] \arrow[r, "\mathrm{id}"'{name=B, below}, bend right] \&[+30pt] \mathbb{F}_q\text{-}\mathsf{CAlg}
            \arrow[Rightarrow, shorten <=5.5pt, shorten >=5.5pt, from=A.south-|B, to=B, "\mathrm{Fr}_{q,-}"]
        \end{tikzcd}
    \]
    
\end{proposition}
\begin{prf}
    For any $x\in A$, we have
    \[
    \mathrm{Fr}_{q,B}\circ f(x)=f(x)^p=f(x^p)=f\circ \mathrm{Fr}_{q,A}(x).
    \]
\end{prf}



\begin{proposition}{Properties of Finite Field}{properties_of_finite_field}
    Let $p$ be a prime number and $m\in\mathbb{Z}_{\ge1}$. Let $q=p^m$. Then
    \begin{enumerate}[(i)]
        \item Let $k\in\mathbb{Z}_{\ge 1}$. Then there exists a field extension $\mathbb{F}_{q^k}/\mathbb{F}_q$ such that it is a splitting field of the polynomial $f(X)=X^{q^k}-X\in \mathbb{F}_q[X]$.
        \item $\mathbb{F}_{q}$ is a splitting field of the polynomial $f(X)=X^{q}-X\in \mathbb{F}_p[X]$. So for any $x\in \mathbb{F}_{q}$ and any $r\in \mathbb{Z}_{\ge 0}$, we have
        \[
            x^{q^r}=x.
        \]
        \item $\mathbb{F}_{q^k}/\mathbb{F}_q$ is a Galois extension of degree $k$. 
        \item The characteristic of $\mathbb{F}_q$ is $p$.
        \item Let $a,b\in\mathbb{Z}_{\ge 1}$ and $\mathbb{F}_{q^a}/\mathbb{F}_{q}$, $\mathbb{F}_{q^b}/\mathbb{F}_{q}$ be field extensions. Then 
        \[
        \mathrm{Hom}_{(\mathsf{Field}_p/\mathbb{F}_{q})}(\mathbb{F}_{q^a}/\mathbb{F}_{q},\mathbb{F}_{q^b}/\mathbb{F}_{q})\neq\varnothing\iff a\divides b.
        \]
        If $a\divides b$, then $\left[\mathbb{F}_{q^b}:\mathbb{F}_{q^a}\right]=\frac{b}{a}$.
    \end{enumerate}
\end{proposition}
\begin{prf}
    \begin{enumerate}[(i)]
        \item Suppose $\iota:\mathbb{F}_q\hookrightarrow K$ is a splitting field of $f(X)=X^{q^k}-X\in \mathbb{F}_q[X]$. For any $x,y\in K$, we have 
        \[
        (x+y)^{q^k}=x^{q^k}+y^{q^k},\quad (xy)^{q^k}=x^{q^k}y^{q^k},
        \] which implies the set of roots of $f$ in $K$
        \[
        M:=\{\alpha\in K\mid f(x)=0\}=\left\{\alpha\in K\mid \alpha^{q^k}=\alpha\right\}
        \]
        is a subfield of $K$. By \Cref{th:existance_of_finite_field}, since $\mathbb{F}_q$ is a splitting field of $X^{p^m}-X\in \mathbb{F}_p[X]$, for any $a\in \mathbb{F}_q$, we have $a^{q}=a$, which implies $a^{q^k}=a$ for all $k\in\mathbb{Z}_{\ge 1}$. Thus we have $\iota(\mathbb{F}_q)\subseteq M$ and we can shrink the codomain of $\iota$ to get the field extension $\iota':\mathbb{F}_q\hookrightarrow M$ such that $i\circ \iota'=\iota$, where $i:M\hookrightarrow K$ is the inclusion map. This means $M/\mathbb{F}_q$ is a subextension of $K/\mathbb{F}_q$. 
        
        Note that $f(X)=X^{q^k}-X\in \mathbb{F}_q[X]$ splits over $M$ and $M=\mathbb{F}_q(M)$. We see $M/\mathbb{F}_q$ is a splitting field of $f$ by \hyperref[th:splitting_field]{definition}. By the uniqueness of splitting field, $M/\mathbb{F}_q$ and $K/\mathbb{F}_q$ are $\mathbb{F}_q$-isomorphic. Since $K$ as a splitting field is finite, we see $M=K$. 

        Since
        \[
        f'(X)=q^k X^{q^k-1}-1=-1,
        \]
        we have $\mathrm{gcd}(f,f')=1$. From \Cref{th:equivalent_characterization_of_separable_polynomial} we see $f$ has $q^k$ distinct roots in $K$, which implies $|M|\ge q^k$. On the other hand, $|M|\le \deg f =q^k$. Thus we have $|K|=|M|=q^k$. Thus $\iota: \mathbb{F}_q\hookrightarrow K$ is the desired field extension.
        \item Take $q=p$ and $k=m$. This is a direct consequence of (i).
        \item From (ii) we see $\mathbb{F}_{q^k}/\mathbb{F}_q$ is a Galois extension. Since 
        \[
        q^{[\mathbb{F}_{q^k}:\mathbb{F}_{q}]}=|\mathbb{F}_{q}|^{[\mathbb{F}_{q^k}:\mathbb{F}_{q}]}=|\mathbb{F}_{q^k}|=q^k,
        \]
        we obtain $[\mathbb{F}_{q^k}:\mathbb{F}_{q}]=k$.
        \item By (i), there exists a field extension $\mathbb{F}_{p^m}/\mathbb{F}_p$. Since field extension preserves characteristic, we have $\mathrm{char}(\mathbb{F}_{p^m})=\mathrm{char}(\mathbb{F}_p)=p$.
        \item Let $a,b\in\mathbb{Z}_{\ge 1}$ and $\mathbb{F}_{q^a}/\mathbb{F}_{q}$, $\mathbb{F}_{q^b}/\mathbb{F}_{q}$ be field extensions. 
        \begin{itemize}
            \item $\mathrm{Hom}_{(\mathsf{Field}_p/\mathbb{F}_{q})}(\mathbb{F}_{q^a}/\mathbb{F}_{q},\mathbb{F}_{q^b}/\mathbb{F}_{q})\neq\varnothing\implies a\divides b$. If there exists an $\mathbb{F}_q$-embedding $\mathbb{F}_{q^a}\hookrightarrow \mathbb{F}_{q^b}$, then 
            \[
            b=[\mathbb{F}_{q^b}:\mathbb{F}_{q}]= [\mathbb{F}_{q^b}:\mathbb{F}_{q^a}][\mathbb{F}_{q^a}:\mathbb{F}_{q}]=[\mathbb{F}_{q^b}:\mathbb{F}_{q^a}]\:a,
            \]
            Thus we have $a\divides b$.
            \item $a\divides b \implies \mathrm{Hom}_{(\mathsf{Field}_p/\mathbb{F}_{q})}(\mathbb{F}_{q^a}/\mathbb{F}_{q},\mathbb{F}_{q^b}/\mathbb{F}_{q})\neq\varnothing$. Suppose $a\divides b$. Then we have $q^a-1\divides q^b-1$. So we can asssume there exists $c\in\mathbb{Z}_{\ge1}$ such that $q^b-1 = c(q^a-1)$. Note
            \begin{align*}
            X^{q^b}-X&=X\left(X^{q^b-1}-1\right)\\
            &=X\left(\left(X^{q^a-1}\right)^c-1\right)\\
            &=X\left(X^{q^a-1}-1\right)\left(\sum_{j=0}^{c-1}\left(X^{q^a-1}\right)^j\right)\\
            &=\left(X^{q^a}-X\right)\left(\sum_{j=0}^{c-1}\left(X^{q^a-1}\right)^j\right).
            \end{align*}
            We have
            \[
            \left(X^{q^a}-X\right)\divides \left(X^{q^b}-X\right).
            \]
            According to \Cref{th:inheritance_of_splitting}, since $X^{q^b}-X\in \mathbb{F}_q[X]$ splits over $\mathbb{F}_{q^b}$, we see $X^{q^a}-X\in \mathbb{F}_q[X]$ also splits over $\mathbb{F}_{q^b}$. Recall that $\mathbb{F}_{q^a}$ is a splitting field of $X^{q^a}-X\in \mathbb{F}_q[X]$. By the minimality of splitting field, there exists an $\mathbb{F}_q$-embedding $\mathbb{F}_{q^a}\hookrightarrow \mathbb{F}_{q^b}$.
        \end{itemize}
    \end{enumerate}
\end{prf}



\begin{proposition}{Galois Group of Finite Field}{}
    Let $p$ be a prime number and $m\in\mathbb{Z}_{\ge1}$. Let $q=p^m$. Suppose $L/\mathbb{F}_q$ is a finite extension of degree $n:=[L:\mathbb{F}_q]$. Then
    \begin{enumerate}[(i)]
        \item $L/\mathbb{F}_q$ is a Galois extension.
        \item $\mathrm{Gal}(L/\mathbb{F}_q)$ is a cyclic group of order $n$. The Frobenius automorphism $\mathrm{Fr}_{q,L}$ generates $\mathrm{Gal}(L/\mathbb{F}_q)$, that is,
        \[
            \mathrm{Gal}(L/\mathbb{F}_q)=\left\{  \left(\mathrm{Fr}_{q,L}\right)^k\midv k=0,1,\cdots,n-1\right\}.
        \]
    \end{enumerate}
\end{proposition}
\begin{prf}
    \begin{enumerate}[(i)]
        \item Since $\mathbb{F}_p$ is initial in $\mathsf{Field}_p$, we have a tower of field extensions $L/\mathbb{F}_q/\mathbb{F}_p$. From \Cref{th:properties_of_finite_field} (iii), we see $L/\mathbb{F}_p$ is a Galois extension. According to \Cref{th:properties_of_galois_extension}, $L/\mathbb{F}_q$ is also a Galois extension.
        \item It is clear that $\mathrm{Fr}_{q,L}\in \mathrm{Gal}(L/\mathbb{F}_q)$. Note $|\mathrm{Gal}(L/\mathbb{F}_q)|=[L:\mathbb{F}_q]=n$. We only need to show that the order of $\mathrm{Fr}_{q,L}$ is $n$. For any $x\in L$, we have
        \[
        \left(\mathrm{Fr}_{q,L}\right)^n(x)=x^{q^n}=x,
        \]
        which means $\left(\mathrm{Fr}_{q,L}\right)^n = \mathrm{id}_L$. Hence, the order of $\mathrm{Fr}_{q,L}$ divides $n$. 

        Now we prove $\left(\mathrm{Fr}_{q,L}\right)^k \ne \mathrm{id}_L$ for any $1\le k<n$. Suppose there exists $k\in\mathbb{Z}_{\ge1}$ such that $\left(\mathrm{Fr}_{q,L}\right)^k = \mathrm{id}_L$. Then for any $x\in L$, we have
        \[
        x^{q^k}=x.
        \]
        So every element of $L$ is a root of the polynomial 
        \[
        f(T) = T^{q^k}-T \in \mathbb{F}_q[T],
        \]
        which means $f$ has at least $|L|=q^n$ distinct roots in $L$. Thus we have
        \[
        q^n = |L| \le \deg f = q^k,
        \]
        which implies $k\ge n$. Therefore, we see the order of $\mathrm{Fr}_{q,L}$ is $n$.
    \end{enumerate}
\end{prf}


By Galois correspondence, we can classify all intermediate fields of finite field extension of $\mathbb{F}_q$.

\begin{proposition}{}{}
    Suppose $p$ is a prime number and $m\in\mathbb{Z}_{\ge1}$. Let $q=p^m$. Suppose $L\supseteq \mathbb{F}_q$ is a finite extension of degree $n:=[L:\mathbb{F}_q]$. Then there is a one-to-one correspondence between the set of intermediate fields of $L/\mathbb{F}_q$ and the set of divisors of $n$:
    \begin{align*}
        \left\{ d\in\mathbb{Z}_{\ge1}\mid d\divides n \right\} &\xlongrightarrow{\sim} E_d:=\left\{ E\mid \mathbb{F}_q\subseteq E\subseteq L \right\},\\
        d  &\longmapsto \left\{x\in L \midv x^{q^{\frac{n}{d}}}=x\right\},\\
        [L:E] &\longmapsfrom E .
    \end{align*}
    Moreover, we have $E_d\cong \mathbb{F}_{q^{\frac{n}{d}}}$. For any factors $d_1,d_2$ of $n$,
    \[
    d_1 \divides d_2 \iff E_{d_1} \supseteq E_{d_2}.
    \]
\end{proposition}

\begin{prf}
    By Galois correspondence, there is a one-to-one correspondence 
    \begin{align*}
        f:\left\{ H\le \mathrm{Gal}(L/\mathbb{F}_q) \right\} &\xlongrightarrow{\sim}\left\{ E\mid \mathbb{F}_q\subseteq E\subseteq L \right\},\\
        H  &\longmapsto L^H:=\{x\in L\mid \sigma(x)=x,\forall \sigma\in H\},\\
        \mathrm{Gal}(L/E) &\longmapsfrom E .
    \end{align*}
    And there is a one-to-one correspondence
    \begin{align*}
        g:\left\{ d\in\mathbb{Z}_{\ge1}\mid d\divides n \right\} &\xlongrightarrow{\sim} \left\{ H\le \mathrm{Gal}(L/\mathbb{F}_q) \right\},\\
        d  &\longmapsto \left\langle \left(\mathrm{Fr}_{q,L}\right)^{\frac{n}{d}}\right\rangle,\\
        |H| &\longmapsfrom H .
    \end{align*}
    Note
    \begin{align*}
        f\circ g(d)&=L^{\left\langle \left(\mathrm{Fr}_{q,L}\right)^{\frac{n}{d}}\right\rangle}\\
        &=\left\{x\in L \midv \left(\mathrm{Fr}_{q,L}\right)^{\frac{n}{d}}(x)=x\right\}\\
        &=\left\{x\in L \midv x^{q^{\frac{n}{d}}}=x\right\},
    \end{align*}
   and 
   \begin{align*}
    g \circ f(E)&=\left| \mathrm{Gal}(L/E) \right|=[L:E].
   \end{align*}
    Thus the desired one-to-one correspondence is given by $f\circ g$.
\end{prf}

\[
\begin{tikzcd}[
    row sep=1.2em,     % Adjust vertical spacing
    column sep=1.2em,  % Adjust horizontal spacing
    % Define a wider gap for the central column (column 4) to accommodate the arrow
    /tikz/column 4/.style={column sep=4em} 
]
    % --- Row 1 ---
    & & \mathbb{F}_{q^4} & & & & 4\mathbb{Z} \\  
    % --- Row 2 ---
    & \mathbb{F}_{q^2} 
      % Arrows from F_q^2
      \arrow[ru, phantom, "\subset" sloped] 
      \arrow[rd, phantom, "\subset" sloped] 
    & & & & 2\mathbb{Z} 
      % Arrows from 2Z
      \arrow[ru, phantom, "\supset" sloped] 
      \arrow[rd, phantom, "\supset" sloped] 
    & \\
    % --- Row 3 (Center Line) ---
    \mathbb{F}_q 
      % Arrows from F_q
      \arrow[ru, phantom, "\subset" sloped] 
      \arrow[rd, phantom, "\subset" sloped] 
    & & \mathbb{F}_{q^6} 
      % The central 1:1 Arrow spanning the gap
    \arrow[rr, leftrightarrow, "1:1"', shorten <= 1em, shorten >= 1em]
    & & \mathbb{Z} 
      % Arrows from Z
      \arrow[ru, phantom, "\supset" sloped] 
      \arrow[rd, phantom, "\supset" sloped] 
    & & 6\mathbb{Z} \\
    % --- Row 4 ---
    & \mathbb{F}_{q^3} 
      % Arrows from F_q^3
      \arrow[ru, phantom, "\subset" sloped] 
      \arrow[rd, phantom, "\subset" sloped] 
    & & & & 3\mathbb{Z} 
      % Arrows from 3Z
      \arrow[ru, phantom, "\supset" sloped] 
      \arrow[rd, phantom, "\supset" sloped] 
    & \\
    % --- Row 5 ---
    & & \mathbb{F}_{q^9} & & & & 9\mathbb{Z}
    \end{tikzcd}
\]



