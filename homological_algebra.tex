\chapter{Homological Algebra}
\section{Abelian Category}

\subsection{$\mathsf{Ab}$-enriched Category}
We will refer to \hyperref[th:enriched_functor]{$\mathsf{Ab}$-enriched functors} as \textbf{additive functors}. Some literature refers to $\mathsf{Ab}$-categories as preadditive categories. We will not use this term in this note. 

\begin{definition}{Biproduct}{biproduct}
    Let $\mathsf{C}$ be an $\mathsf{Ab}$-category and $X_1$, $X_2$ be objects in $\mathsf{C}$. A \textbf{biproduct} of $X_1$ and $X_2$ is a diagram
    \[
    \begin{tikzcd}[ampersand replacement=\&]
            X_1 \arrow[r, "i_1", shift left] \& X_1\oplus X_2 \arrow[l, "p_1", shift left] \arrow[r, "p_2"', shift right] \& X_2 \arrow[l, "i_2"', shift right]
    \end{tikzcd}
    \]
    such that
    \begin{itemize}
        \item $p_1\circ i_1=\mathrm{id}_{X_1}$, $p_2\circ i_2=\mathrm{id}_{X_2}$.
        \item $i_1\circ p_1+ i_2\circ p_2=\mathrm{id}_{X_1\oplus X_2}$.
    \end{itemize}
    Empty biproduct is defined to be the zero object.
\end{definition}


\begin{proposition}{Equivalent Conditions of Existance of Biproduct}{}
    Let $\mathsf{C}$ be an $\mathsf{Ab}$-category and $X_1$, $X_2$ be objects in $\mathsf{C}$. Then the following are equivalent
    \begin{enumerate}[(i)]
        \item Product $X_1\times X_2$ exists.
        \item Coproduct $X_1\amalg X_2$ exists.
        \item Biproduct $X_1\oplus X_2$ exists.
    \end{enumerate}
\end{proposition}

\begin{proposition}{Equivalent Conditions of Existance of Zero Object}{}
    Let $\mathsf{C}$ be an $\mathsf{Ab}$-category. Then the following are equivalent
    \begin{enumerate}[(i)]
        \item Initial object (empty product) exists.
        \item Terminal object (empty coproduct) exists.
        \item Zero object (empty biproduct) exists.
    \end{enumerate}
\end{proposition}

\begin{proposition}{$\mathsf{Ab}$-enriched Functors Preserve Biproducts}{}
    $\mathsf{Ab}$-enriched functors preserve finite biproducts.
\end{proposition}


\begin{definition}{Kernel and Cokernel}{}
    Suppose $\mathsf{C}$ is a category with zero object and $f:X\to Y$ is a morphism in $\mathsf{C}$. The \textbf{kernel} of $f$ is the equalizer of $f$ and the zero morphism, which is denoted by $\ker f$. The universal property of kernel is given by the following diagram
    \[
    \begin{tikzcd}[ampersand replacement=\&]
        \& M \arrow[d, "g"] \arrow[rd, "0"] \arrow[ld, "\exists!\widetilde{g}"', dashed] \&   \\
        \ker f \arrow[r] \& X \arrow[r, "f"']                                                             \& Y
    \end{tikzcd}
    \]
    The \textbf{cokernel} of $f$ is the coequalizer of $f$ and the zero morphism, which is denoted by $\operatorname{coker}f$. The universal property of cokernel is given by the following diagram
    \[
    \begin{tikzcd}[ampersand replacement=\&]
        \& M                           \&                                                                     \\
        X \arrow[r, "f"'] \arrow[ru, "0"] \& Y \arrow[u, "g"'] \arrow[r] \& \operatorname{coker} f \arrow[lu, "\exists!\widetilde{g}"', dashed]
        \end{tikzcd}
    \]

\end{definition}


\subsection{Additive Category}
\begin{definition}{Additive Category}{}
    An $\mathsf{Ab}$-category is \textbf{additive} if it has all finite biproducts.
\end{definition}




\subsection{Abelian Category}
\begin{definition}{Abelian Category}{}
    An additive category $\mathsf{A}$ is \textbf{abelian} if 
    \begin{enumerate}[(i)]
        \item Every morphism has both a kernel and a cokernel.
        \item Every monomorphism and every epimorphism is normal. This means that every monomorphism is a kernel of some morphism, and every epimorphism is a cokernel of some morphism.
    \end{enumerate}
\end{definition}


\begin{proposition}{}{}
    If $\mathsf{A}$ is an abelian category, then $\left[\mathsf{C}, \mathsf{A}\right]$ is abelian for any small category $\mathsf{C}$.
\end{proposition}

\begin{proposition}{Exact Functors in Abelian Category}{exact_functors_in_abelian_category}
    Let $\mathsf{A}$ and $\mathsf{B}$ be abelian categories and $F:\mathsf{A}\to\mathsf{B}$ be an additive functor. Then
    \begin{enumerate}[(i)]
        \item $F$ is \hyperref[th:exact_functor]{left exact} if and only if $F$ preserves kernels, or equivalently,
        \[
        0\longrightarrow X\xlongrightarrow{f} Y\xlongrightarrow{g} Z\text{ is exact}\quad\implies\quad 0\longrightarrow F(X)\xlongrightarrow{F(f)} F(Y)\xlongrightarrow{F(g)} F(Z)\text{ is exact}.    
        \]
        \item $F$ is \hyperref[th:exact_functor]{right exact} if and only if $F$ preserves cokernels, or equivalently,
        \[
        X\xlongrightarrow{f} Y\xlongrightarrow{g} Z\longrightarrow 0\text{ is exact}\quad\implies\quad F(X)\xlongrightarrow{F(f)} F(Y)\xlongrightarrow{F(g)} F(Z)\longrightarrow 0\text{ is exact}.
        \]
        \item $F$ is \hyperref[th:exact_functor]{exact} if and only if $F$ preserves kernels and cokernels, or equivalently,
        \[
        0\longrightarrow X\xlongrightarrow{f} Y\xlongrightarrow{g} Z\longrightarrow 0\text{ is exact}\quad\implies\quad 0\longrightarrow F(X)\xlongrightarrow{F(f)} F(Y)\xlongrightarrow{F(g)} F(Z)\longrightarrow 0\text{ is exact}.
        \]
    \end{enumerate}

\end{proposition}


\begin{definition}{Cohomology Functor}{}
    Let $\mathsf{A}$ be an abelian category and 
    \[
    \begin{tikzcd}[ampersand replacement=\&]
        X \arrow[r, "f"] \& Y \arrow[r, "g"] \& Z
    \end{tikzcd}
    \]
    be an exact sequence in $\mathsf{A}$. The \textbf{cohomology functor} $\mathrm{H}:\mathsf{Ch}_{}\left(\mathsf{A}\right)\to\mathsf{Ab}$ is defined by
    \[
    \mathrm{H}(X\xlongrightarrow{f} Y\xlongrightarrow{g} Z):=\operatorname{coker}[\operatorname{im}(f) \rightarrow \operatorname{ker}(g)].
    \]
\end{definition}


    

\begin{proposition}{Exact Functor Preserve Cohomology}{exact_functor_preserve_cohomology}
    Let $F:\mathsf{A}\to\mathsf{B}$ be an exact functor between abelian categories. Suppose 
    \[
    X \xlongrightarrow{f} Y \xlongrightarrow{g} Z 
    \]
    is an exact sequence in $\mathsf{A}$. Then we have the following natural isomorphism
    \[
    F\left(\mathrm{H}\left[X \xrightarrow{f} Y\xrightarrow{g} Z\right]\right)  \cong \mathrm{H}\left[F(X) \xrightarrow{F(f)} F(Y)\xrightarrow{F(g)} F(Z)\right].
    \]
\end{proposition}

\begin{prf}
    $$
        \begin{aligned}
        F\left(\mathrm{H}\left[X \xrightarrow{f} Y\xrightarrow{g} Z\right]\right)& =F \operatorname{coker}[\operatorname{im}(f) \rightarrow \operatorname{ker}(g)] \\
        & \cong \operatorname{coker}[F \operatorname{im}(f) \rightarrow F \operatorname{ker}(g)] \\
        & \cong \operatorname{coker}[\operatorname{im}(F f) \rightarrow \operatorname{ker}(F g)]\\
        &=\mathrm{H}\left[F X \xrightarrow{F f} F Y \xrightarrow{F g} Z\right]
        \end{aligned}
$$
\end{prf}


\begin{proposition}{Hom Functors are Left Exact}{}
    Let $\mathsf{A}$ be an abelian category. Then given any $X\in \mathrm{Ob}(\mathsf{A})$, $\mathrm{Hom}_{\mathsf{A}}(X,-):\mathsf{A}\to\mathsf{Ab}$ and $\mathrm{Hom}_{\mathsf{A}}(-,X):\mathsf{A}^{\mathrm{op}}\to\mathsf{Ab}$ are left exact functors.
\end{proposition}

\begin{prf}
    To prove that $\mathrm{Hom}_{\mathsf{A}}(X,-)$ is left exact, we need to prove that it preserves kernels. Let $f:Y\to Z$ be a morphism in $\mathsf{A}$, the universal property of $\ker f$ is illustrated by the following diagram
    \[
        \begin{tikzcd}[column sep={6em, between origins}, ampersand replacement=\&]
            W \arrow[rd, "g"] \arrow[d, "\exists!h"', dashed] \&                   \&   \\[+3ex]
            \ker f \arrow[r, "\iota"']                \&Y \arrow[r, "f"', shift right=0.7ex] \arrow[r, "0", shift left=0.7ex] \& Z
        \end{tikzcd}
    \]
    It suffices to show that $\iota_*:\mathrm{Hom}_{\mathsf{A}}\left(X,\ker f \right)\hookrightarrow \mathrm{Hom}_{\mathsf{A}}\left(X,Y\right)$ satisfies the universal property of $\ker f_*$. 
    \[
        \begin{tikzcd}[column sep={10em, between origins}, ampersand replacement=\&]
            G \arrow[rd, ""] \arrow[d, "\exists!"', dashed] \&                   \&   \\[+3ex]
            \mathrm{Hom}_{\mathsf{A}}\left(X,\ker f\right) \arrow[r, "\iota_*"']                \&\mathrm{Hom}_{\mathsf{A}}\left(X,Y\right) \arrow[r, "f_*"', shift right=0.7ex] \arrow[r, "0", shift left=0.7ex] \& \mathrm{Hom}_{\mathsf{A}}\left(X,Z\right)
        \end{tikzcd}
    \]
    This reduces to checking that $\ker f_*=\iota_*\left(\mathrm{Hom}_{\mathsf{A}}\left(X,\ker f \right)\right)$. 
    
    If $\iota_*(h)\in\iota_*\left(\mathrm{Hom}_{\mathsf{A}}\left(X,\ker f \right)\right)$, then $f_*\left(\iota_*(h)\right)=f\circ \iota\circ h=0$, which means that $h\in \ker f_*$. Conversely, if $g\in \ker f_*$, then $f_*(g)=f\circ g=0$. By the universal property of $\ker f$, there exists a unique morphism $h:W\to \ker f$ such that $\iota\circ h=g$. This means that $g=\iota_*(h)\in \iota_*\left(\mathrm{Hom}_{\mathsf{A}}\left(X,\ker f \right)\right)$.

    The proof for left-exactness of $\mathrm{Hom}_{\mathsf{A}}(-,X)$ can be obtained by duality.
\end{prf}



\begin{definition}{Split Exact Sequence}{}
    Let $\mathsf{A}$ be an abelian category and 
    \[
    \begin{tikzcd}[ampersand replacement=\&]
        0 \arrow[r] \& X \arrow[r, "f"] \& Y \arrow[r, "g"] \& Z \arrow[r] \& 0
    \end{tikzcd}
    \]
    be an exact sequence in $\mathsf{A}$. The sequence is said to be \textbf{split} if there exists morphisms $r:Y\to X$ and $s:Z\to Y$ such that $Y$ is the \hyperref[th:biproduct]{biproduct} of $X$ and $Z$, which is illustrated by the following diagram
    \[
        \begin{tikzcd}[ampersand replacement=\&]
                X \arrow[r, "f", shift left] \& Y \arrow[l, "r", shift left] \arrow[r, "g"', shift right] \& Z \arrow[l, "s"', shift right]
        \end{tikzcd}
    \]
\end{definition}

\begin{proposition}{Equivalent Characterizations of Split Exact Sequence}{equivalent_split_exact_sequence}
    Let $\mathsf{A}$ be an abelian category and 
    \[
    \begin{tikzcd}[ampersand replacement=\&]
        0 \arrow[r] \& X \arrow[r, "f"] \& Y \arrow[r, "g"] \& Z \arrow[r] \& 0
    \end{tikzcd}
    \]
    be an exact sequence in $\mathsf{A}$. Then the following statements are equivalent
    \begin{enumerate}[(i)]
        \item The sequence is split.
        \item $f$ has left inverse, i.e. there exists a morphism $r:Y\to X$ such that $r\circ f=\mathrm{id}_X$.
        \begin{center}
            \begin{tikzcd}[ampersand replacement=\&, column sep={4em, between origins}, row sep={4em, between origins}]  
               0\arrow[r] \&X\arrow[d, "\mathrm{id}_X"']\arrow[r, "f"] \& Y\arrow[ld, "\exists r", dashed] \arrow[r, "g"] \& Z \arrow[r]                               \& 0\\
               \&X \&   \&   \&   
            \end{tikzcd}
        \end{center}
        \item $g$ has right inverse, i.e. there exists a morphism $s:Z\to Y$ such that $g\circ s=\mathrm{id}_Z$.
        \begin{center}
            \begin{tikzcd}[ampersand replacement=\&, column sep={4em, between origins}, row sep={4em, between origins}]
                \&  \&   \& Z \arrow[d, "\mathrm{id}_Z"] \arrow[ld, "\exists s"', dashed] \&   \\
               0\arrow[r] \&X\arrow[r, "f"'] \& Y\arrow[r, "g"'] \& Z \arrow[r]                               \& 0
            \end{tikzcd}
        \end{center}
        \item The homomorphism $f^*: \mathrm{Hom}_{\mathsf{A}}(Y,A)\to \mathrm{Hom}_{\mathsf{A}}(X,A)$ is surjective for any $A\in \mathrm{Ob}(\mathsf{A})$.
        \item The homomorphism $g_*: \mathrm{Hom}_{\mathsf{A}}(A,Y)\to \mathrm{Hom}_{\mathsf{A}}(A,Z)$ is surjective for any $A\in \mathrm{Ob}(\mathsf{A})$.
    \end{enumerate}
\end{proposition}

\begin{prf}
    (ii) $\implies$ (iv). Suppose $r:Y\to X$ is a left inverse of $f$. Given any $h\in \mathrm{Hom}_{\mathsf{A}}(X,A)$, there exists $h\circ r\in \mathrm{Hom}_{\mathsf{A}}(Y,A)$ such that $f^*(h\circ r)=h\circ r\circ f=h$, which means that $f^*$ is surjective.
    \begin{center}
        \begin{tikzcd}[ampersand replacement=\&, column sep={4em, between origins}, row sep={4em, between origins}]  
           0\arrow[r] \&X\arrow[d, "\mathrm{id}_X"']\arrow[r, "f"] \& Y \arrow[ldd, "h\circ r"]\arrow[ld, "r"', dashed] \arrow[r, "g"] \& Z \arrow[r]                               \& 0\\
           \&X \arrow[d, "h"']   \&   \&   \&   \\
           \&A \&   \&   \&
        \end{tikzcd}
    \end{center}

    (iv) $\implies$ (ii). Suppose $f^*$ is surjective. Given any $h\in \mathrm{Hom}_{\mathsf{A}}(X,A)$, there exists $h'\in \mathrm{Hom}_{\mathsf{A}}(Y,A)$ such that $f^*(h')=h$. This means that $h'\circ f=h$. Let $r:=h'$, then $r\circ f=h$.
\end{prf}



\section{Complex}

\subsection{Complex in Additive Category}
\subsubsection{Cochain Complex}
\begin{definition}{$S$-graded Object of a Category}{}
    Let $\mathsf{C}$ be a category and $S$ be a set. The category of \textbf{$S$-graded object of $\mathsf{C}$} is defined as the functor category $\left[\mathsf{Disc}(S),\mathsf{C}\right]$, denoted by $\mathsf{Gr}_{S}\left(\mathsf{C}\right)$. The explicit description of the category is given as follows
    \begin{itemize}
        \item Object: a collection of objects $\left\{X_s\right\}_{s\in S}$ in $\mathsf{C}$.
        \item Morphism: a collection of morphisms $\left\{f_s:X_s\to Y_s\right\}_{s\in S}$ in $\mathsf{C}$.
    \end{itemize}
\end{definition}


\begin{definition}{Category with Translation}{}
    A \textbf{category with translation} is a category $\mathsf{C}$ equipped with an auto-equivalence functor
    \[
    T:\mathsf{C}\longrightarrow\mathsf{C}
    \]
    called the \textbf{shift functor} or \textbf{translation functor} or suspension functor.
\end{definition}


\begin{example}{Shift Functors on $\mathbb{Z}$-graded Category}{}
    Let $\mathsf{C}$ be a category. Given $k\in \mathbb{Z}$, the \textbf{$k$-th shift functor} on $\mathsf{Gr}_{\mathbb{Z}}\left(\mathsf{C}\right)$ is an endofunctor $[k]$ defined by
    \[
        \begin{tikzcd}[ampersand replacement=\&]
            \mathsf{Gr}_{\mathbb{Z}}\left(\mathsf{C}\right)\&[-25pt]\&[+10pt]\&[-30pt]\mathsf{Gr}_{\mathbb{Z}}\left(\mathsf{C}\right)\&[-30pt]\&[-30pt] \\ [-15pt] 
            X=\left(X^i\right)  \arrow[dd, "f=\left(f^i\right)"{name=L, left}] 
            \&[-25pt] \& [+10pt] 
            \& [-30pt] X[k]=\left(X[k]^i\right)=\left(X^{k+i}\right)\arrow[dd, "{f[k]=\left(f^{k+i}\right)}"{name=R}] \\ [-10pt] 
            \&  \phantom{.}\arrow[r, "{[k]}", squigarrow]\&\phantom{.}  \&   \\[-10pt] 
            Y=\left(Y^i\right)   \& \& \& Y[k]=\left(Y[k]^i\right)=\left(Y^{k+i}\right)
        \end{tikzcd}
    \]

\end{example}


\begin{example}{Shift Functors on $\mathbb{Z}^m$-graded Category}{}
    Let $\mathsf{C}$ be a category. In a similar manner, given $k_1,\cdots,k_m\in \mathbb{Z}$, the \textbf{$(k_1,\cdots,k_m)$-th shift functor} on $\mathsf{Gr}_{\mathbb{Z}^m}\left(\mathsf{C}\right)$ is an endofunctor $[k_1,\cdots,k_m]$ defined by
    \[
        \begin{tikzcd}[ampersand replacement=\&]
            \mathsf{Gr}_{\mathbb{Z}^m}\left(\mathsf{C}\right)\&[-25pt]\&[+10pt]\&[-30pt]\mathsf{Gr}_{\mathbb{Z}^m}\left(\mathsf{C}\right)\&[-30pt]\&[-30pt] \\ [-15pt] 
            A=\left(A^{i_1,\cdots,i_m}\right)  \arrow[dd, "f=\left(f^{i_1,\cdots,i_m}\right)"{name=L, left}] 
            \&[-25pt] \& [+10pt] 
            \& [-30pt] A[k_1,\cdots,k_m]=\left(A[k_1,\cdots,k_m]^{i_1,\cdots,i_m}\right)=\left(A^{i_1+k_1,\cdots,i_m+k_m}\right)\arrow[dd, "\left(f^{i_1+k_1,\cdots,i_m+k_m}\right)"{name=R}] \\ [-10pt] 
            \&  \phantom{.}\arrow[r, "{[k_1,\cdots,k_m]}", squigarrow]\&\phantom{.}  \&   \\[-10pt] 
            B=\left(B^{i_1,\cdots,i_m}\right)   \& \& \& B[k_1,\cdots,k_m]=\left(B[k_1,\cdots,k_m]^{i_1,\cdots,i_m}\right)=\left(B^{i_1+k_1,\cdots,i_m+k_m}\right)
        \end{tikzcd}
    \]
\end{example}


\begin{lemma}{}{}
    If $\mathsf{C}$ is an additive category, then $\mathsf{Gr}_{\mathbb{Z}^m}\left(\mathsf{C}\right)=\left[\mathsf{Disc}(\mathbb{Z}^m),\mathsf{C}\right]$ is an additive category.
\end{lemma}


\begin{definition}{General Differential Object in a Category with Translation}{}
    A \textbf{general differential object} in a category $\mathsf{C}$ with translation $T$ is a pair $\left(X,d_X\right)$ consisting of an object $X$ in $\mathsf{C}$ and a morphism $d_X:X\to T(X)$ in $\mathsf{C}$.
\end{definition}


\begin{definition}{Differential Object in an Additive Category with Translation}{}
    A \textbf{differential object} in an additive category $\mathsf{C}$ with translation $T$ is a pair $\left(X,d_X\right)$ consisting of an object $X$ in $\mathsf{C}$ and a morphism $d_X:X\to T(X)$ in $\mathsf{C}$ such that $T(d_X)\circ d_X=0$.
\end{definition}


\begin{definition}{Differential $\mathbb{Z}$-Graded Object}{}
    Let $\mathsf{A}$ be an additive category. The category of \textbf{differential $\mathbb{Z}$-graded objects} of $\mathsf{A}$ consists of the following data
    \begin{itemize}
        \item Object: differential objects $\left(X,d_X\right)$ in the additive category $\mathsf{Gr}_{\mathbb{Z}}\left(\mathsf{A}\right)$ with translation $[1]$.
        \item Morphism: morphisms $f:\left(X,d_X\right)\to \left(Y,d_Y\right)$ are morphisms $f:X\to Y$ in $\mathsf{Gr}_{\mathbb{Z}}\left(\mathsf{A}\right)$ such that $f[1]\circ d_X=d_Y\circ f$.
    \end{itemize}
\end{definition}


\begin{definition}{Cochain Complex}{}
    Let $\mathsf{A}$ be an additive category. The category of \textbf{cochain complexes} of $\mathsf{A}$, denoted by $\mathsf{Ch}(\mathsf{A})$, consists of the following data
    \begin{itemize}
        \item Object: \textbf{cochain complex} of $\mathsf{A}$, which consists of a collection of objects $X^{\bullet}=\left(X^n\right)_{n\in \mathbb{Z}}$ in $\mathsf{A}$ and a collection of morphisms $d_X^{\bullet}=\left(d^n:X^n\to X^{n+1}\right)_{n\in \mathbb{Z}}$ in $\mathsf{A}$ 
        \[
            \cdots \longrightarrow X^n \xlongrightarrow{d^n_X} X^{n+1} \xlongrightarrow{d^{n+1}_X} X^{n+2} \longrightarrow \cdots
        \]
        such that $d^{n+1}_X\circ d^n_X=0$ for all $n\in \mathbb{Z}$.
        \item Morphism: a collection of morphisms $\left(f^n:X^n\to Y^{n}\right)_{n\in \mathbb{Z}}$ in $\mathsf{A}$ such that $f^{n+1}\circ d^n_X=d^n_Y\circ f^n$ for all $n\in \mathbb{Z}$
        \[
            \begin{tikzcd}[
                ampersand replacement=\&,
                column sep={6em, between origins},
                row sep={5em, between origins}
            ]
                \cdots \arrow[r] \& X^{n} \arrow[r, "d_X^n"] \arrow[d, "f^n"'] \& X^{n+1} \arrow[d, "f^{n+1}"'] \arrow[r, "d_X^{n+1}"] \& X^{n+2} \arrow[d, "f^{n+2}"] \arrow[r] \& \cdots \\
                \cdots \arrow[r] \& Y^{n} \arrow[r, "d_Y^{n}"'] \& Y^{n+1} \arrow[r, "d_Y^{n+1}"'] \& Y^{n+2} \arrow[r] \& \cdots
            \end{tikzcd}
        \]
    \end{itemize}
    $\mathsf{Ch}(\mathsf{A})$ is isomorphic to the category of differential $\mathbb{Z}$-graded objects of $\mathsf{A}$.
\end{definition}


\subsubsection{Chain Complex}
\begin{definition}{Homotopy between Morphisms of Chain Complexes}{}
    A homotopy $h$ between a pair of morphisms of chain complexes $f, g: X_{\bullet} \rightarrow Y_{\bullet}$ is a collection of morphisms $h_i: X_i \rightarrow Y_{i+1}$ such that we have
$$
f_i-g_i=d_{i+1} \circ h_i+h_{i-1} \circ d_i
$$
for all $i$. Two morphisms $f, g: X_{\bullet} \rightarrow Y_{\bullet}$ are said to be homotopic if a homotopy between $f$ and $g$ exists. 
\end{definition}

\subsection{Chain Homotopy}

\begin{definition}{Homotopy between Morphisms of Cochain Complexes}{}
    A \textbf{homotopy $h$ between a pair of morphisms of cochain complexes} $f, g: X^{\bullet} \rightarrow Y^{\bullet}$ is a collection of morphisms $h^i: X^i \rightarrow Y^{i-1}$
    \[
        \begin{tikzcd}[
            ampersand replacement=\&,
            column sep={6em, between origins},
            row sep={5em, between origins}
        ]
            \cdots \arrow[r, "d_X^{i-1}"]  \& X^{i} \arrow[r, "d_X^i"] \arrow[d, "f^i"'] \arrow[ld, "h^i"'] \& X^{i+1} \arrow[d, "f^{i+1}"'] \arrow[r, "d_X^{i+1}"] \arrow[ld, "h^{i+1}"'] \& X^{i+2} \arrow[d, "f^{i+2}"] \arrow[r, "d_X^{i+2}"] \arrow[ld, "h^{i+2}"'] \& \cdots \arrow[ld, "h^{i+3}"] \\
            \cdots \arrow[r, "d_Y^{i-1}"'] \& Y^{i} \arrow[r, "d_Y^{i}"']                                   \& Y^{i+1} \arrow[r, "d_Y^{i+1}"']                                             \& Y^{i+2} \arrow[r, "d_Y^{i+2}"']                                            \& \cdots                      
        \end{tikzcd}
    \]
         such that
$$
f^i-g^i=d^{i-1}_Y \circ h^i+h^{i+1} \circ d^i_X
$$
for all $i\in \mathbb{Z}$. Two morphisms $f, g: X^{\bullet} \rightarrow Y^{\bullet}$ are said to be \textbf{homotopic} if there exists a homotopy between $f$ and $g$. A morphism $f: X^{\bullet} \rightarrow Y^{\bullet}$ is said to be \textbf{null homotopic} if it is homotopic to the zero morphism $0: X^{\bullet} \rightarrow Y^{\bullet}$.
\end{definition}


\begin{proposition}{Additive Functor Preserves Cochain Homotopy}{}
    Let $F:\mathsf{A}\to\mathsf{B}$ be an additive functor between additive categories. Suppose $h$ is a homotopy between a pair of morphisms $f, g: X^{\bullet} \rightarrow Y^{\bullet}$ in $\mathsf{Ch}(\mathsf{A})$. Then $(F(h^i))_{i\in\mathbb{Z}}$ is a homotopy between $F(f),F(g):F\left(X^{\bullet}\right)\to F\left(Y^{\bullet} \right)$ in $\mathsf{B}$.
\end{proposition}



\begin{definition}{Homotopy Equivalence between Cochain Complexes}{}
    A morphism $f: X^{\bullet} \rightarrow Y^{\bullet}$ of cochain complexes is a \textbf{homotopy equivalence} if there exists a morphism $g: Y^{\bullet} \rightarrow X^{\bullet}$ such that $f\circ g$ and $g\circ f$ are homotopic to the identity morphisms of $X^{\bullet}$ and $Y^{\bullet}$, respectively. We say that $X^{\bullet}$ and $Y^{\bullet}$ are \textbf{homotopy equivalent} if there exists a homotopy equivalence between $X^{\bullet}$ and $Y^{\bullet}$.
\end{definition}




\subsection{Cohomology}


\begin{definition}{$n$-th Cohomology Functor}{}
    Let $\mathsf{A}$ be an abelian category. Given any $n\in \mathbb{Z}$, the \textbf{$n$-th cohomology functor} of a cochain complex $X^{\bullet}$ is defined to be the functor
    \[
    \mathrm{H}^n(X^{\bullet}):=\ker d^n/\operatorname{im}d^{n-1}.
    \]
\end{definition}


\begin{definition}{Quasi-isomorphism}{}
    A morphism $f: X^{\bullet} \rightarrow Y^{\bullet}$ of cochain complexes is a \textbf{quasi-isomorphism} if the induced morphism $\mathrm{H}^n(f):\mathrm{H}^n(X^{\bullet})\to \mathrm{H}^n(Y^{\bullet})$ is an isomorphism for all $n\in \mathbb{Z}$.
\end{definition}

\begin{lemma}{}{}
    If $f: X^{\bullet} \rightarrow Y^{\bullet}$ is null homotopic, then $\mathrm{H}^n(f)=0$ for all $n\in \mathbb{Z}$.
\end{lemma}

\begin{prf}
    Let $h$ be a homotopy between $f$ and the zero morphism. Then we have
    \[
    f^n=d^{n-1}_Y \circ h^n+h^{n+1} \circ d^n_X
    \]
    for all $n\in \mathbb{Z}$. This implies that $f^n$ induces the zero morphism on cohomology for all $n\in \mathbb{Z}$.
\end{prf}

\begin{proposition}{Homotopic Morphisms Induce the Same Morphisms on Cohomology}{}
    Let $f, g: X^{\bullet} \rightarrow Y^{\bullet}$ be homotopic morphisms of cochain complexes. Then the induced morphisms $\mathrm{H}^n(f)$ and $\mathrm{H}^n(g)$ on cohomology are the same for all $n\in \mathbb{Z}$.
\end{proposition}

\begin{prf}
    Let $h$ be a homotopy between $f$ and $g$. Then we have
    \[
    f^n-g^n=d^{n-1}_Y \circ h^n+h^{n+1} \circ d^n_X
    \]
    for all $n\in \mathbb{Z}$. This implies that $f^n$ and $g^n$ induce the same morphism on cohomology for all $n\in \mathbb{Z}$.
\end{prf}

\begin{corollary}{Homotopy Equivalences are Quasi-Isomorphisms}{}
    A homotopy equivalence of cochain complexes is a quasi-isomorphism.
\end{corollary}


\begin{definition}{Acyclic Complex}{}
    A cochain complex $X^{\bullet}$ is \textbf{acyclic} if $\mathrm{H}^n(X^{\bullet})=0$ for all $n\in \mathbb{Z}$.
\end{definition}


\begin{proposition}{}{}
    Let $\mathsf{A}$ be an abelian category and $X^{\bullet}\in\mathrm{Ob}\left(\mathsf{Ch}(\mathsf{A})\right)$ be a cochain complex. Then the following are equivalent
    \begin{enumerate}[(i)]
        \item $X^{\bullet}$ is acyclic.
        \item $X^{\bullet}$ is exact, that is, exact at each $X^n$.
        \item The zero morphism $0 \to X^{\bullet}$ is a quasi-isomorphism.
    \end{enumerate}
\end{proposition}

\section{Resolution}

\subsection{Projective and Injective Objects}
\begin{definition}{Projective Object}{}
    An object $P$ in an abelian category $\mathsf{A}$ is \textbf{projective} if for any epimorphism $f: X \rightarrow Y$ and any morphism $g: P \rightarrow Y$, there exists a morphism $h: P \rightarrow X$ such that $f \circ h = g$.
    \begin{center}
        \begin{tikzcd}[ampersand replacement=\&, column sep={5em, between origins}, row sep={5em, between origins}]
            \& P \arrow[d, "g"] \arrow[ld, "\exists h"', dashed] \&   \\
        X \arrow[r, "f"', two heads] \& Y \arrow[r]                               \& 0
        \end{tikzcd}
    \end{center}
\end{definition}


\begin{proposition}{Equivalent Characterizations of Projective Objects}{}
    Let $\mathsf{A}$ be an abelian category and $P$ be an object in $\mathsf{A}$. Then the following are equivalent
    \begin{enumerate}[(i)]
        \item $P$ is projective.
        \item $\mathrm{Hom}_{\mathsf{A}}(P,-)$ is exact.
        \item Every short exact sequence 
        \[
        \begin{tikzcd}[ampersand replacement=\&]
            0 \arrow[r] \& X \arrow[r, "f"] \& Y \arrow[r, "g"] \& P \arrow[r] \& 0
        \end{tikzcd}
        \]
        in $\mathsf{A}$ splits.
        \item $\mathop{\mathrm{Ext}}\nolimits _\mathsf{A}(P, X) = 0$ for all $X \in \mathsf{A}$.
    \end{enumerate}
\end{proposition}

\begin{prf}
    (i)$\implies$(ii). Since $\mathrm{Hom}_{\mathsf{A}}(P,-)$ is a left exact functor, it suffices to show that $\mathrm{Hom}_{\mathsf{A}}(P,-)$ preserves epimorphisms. Let $f: X \rightarrow Y$ be an epimorphism in $\mathsf{A}$. We need to show that $f_*:\mathrm{Hom}_{\mathsf{A}}(P,X)\to \mathrm{Hom}_{\mathsf{A}}(P,Y)$ is surjective. Let $g: P \rightarrow Y$ be a morphism in $\mathsf{A}$. Since $P$ is projective, there exists a morphism $h: P \rightarrow X$ such that $f_*(h)=f \circ h = g$. This means that $f_*$ is surjective.

    (i)$\implies$(iii). If $P$ is projective, then for any short exact sequence $0 \to X \xrightarrow{f} Y \xrightarrow{g} P \to 0$, there exists a morphism $h: P \to Y$ such that $h \circ g = \mathrm{id}_P$.
    \begin{center}
        \begin{tikzcd}[ampersand replacement=\&, column sep={4em, between origins}, row sep={4em, between origins}]
            \&  \&   \& P \arrow[d, "\mathrm{id}_P"] \arrow[ld, "\exists h"', dashed] \&   \\
           0\arrow[r] \&X\arrow[r, "f"'] \& Y\arrow[r, "g"', two heads] \& P \arrow[r]                               \& 0
        \end{tikzcd}
    \end{center}
    By \Cref{th:equivalent_split_exact_sequence}, this implies that the short exact sequence splits.
\end{prf}


\begin{definition}{Injective Object}{}
    An object $I$ in an abelian category $\mathsf{A}$ is \textbf{injective} if for any monomorphism $f: X \rightarrow Y$ and any morphism $g: X \rightarrow I$, there exists a morphism $h: Y \rightarrow I$ such that $h \circ f = g$.
    \begin{center}
        \begin{tikzcd}[ampersand replacement=\&, column sep={5em, between origins}, row sep={5em, between origins}]
            0 \arrow[r] \& X \arrow[r, "f", hook] \arrow[d, "g"'] \& Y \arrow[ld, "\exists h", dashed] \\
            \& I  \&                                  
        \end{tikzcd}
    \end{center}
        
\end{definition}

\begin{proposition}{Equivalent Characterizations of  Injective Objects}{}
    Let $\mathsf{A}$ be an abelian category and $I$ be an object in $\mathsf{A}$. Then the following are equivalent
    \begin{enumerate}[(i)]
        \item $I$ is injective.
        \item $\mathrm{Hom}_{\mathsf{A}}(-,I)$ is exact.
        \item Every short exact sequence 
        \[
        \begin{tikzcd}[ampersand replacement=\&]
            0 \arrow[r] \& I \arrow[r] \& X \arrow[r, "f"] \& Y \arrow[r] \& 0
        \end{tikzcd}
        \]
        in $\mathsf{A}$ splits.
        \item $\mathop{\mathrm{Ext}}\nolimits _\mathsf{A}(I, X) = 0$ for all $X \in \mathsf{A}$.
    \end{enumerate}
\end{proposition}
