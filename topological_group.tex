
\chapter{Topological Group}
\section{Topological Group}
\dfn{Topological Group}{
    A \textbf{topological group} is a group $G$ equipped with a topology $\tau$ such that the group multiplication map
    \begin{align*}
        \mu:G\times G&\longrightarrow G\\
        (g,h)&\longmapsto gh
    \end{align*}
    and the inversion map
    \begin{align*}
        \sigma:G&\longrightarrow G\\
        g&\longmapsto g^{-1}
    \end{align*}
    are continuous maps.
}
\noindent Topological groups are the group objects in the category $\mathsf{Top}$.
\dfn{Topological Group Category}{
    Topological groups form a category $\mathsf{TopGrp}$, where the morphisms are continuous group homomorphisms. 
}
\noindent An isomorphism of topological groups is a group isomorphism that is also a homeomorphism of the underlying topological spaces.
\prop{}{
    The category $\mathsf{TopGrp}$ is complete. Limits in $\mathsf{TopGrp}$ commute with 
    \begin{itemize}
        \item forgetful functor $\mathsf{TopGrp}\to\mathsf{Top}$
        \item forgetful functor $\mathsf{TopGrp}\to\mathsf{Grp}$
    \end{itemize}
}
\pf{ 
    It is enough to prove the existence and commutation for products and equalizers. Let $R_i, i \in I$ be a collection of topological rings. Take the usual product $R=\prod R_i$ with the product topology. Since $R \times R=\prod\left(R_i \times R_i\right)$ as a topological space (because products commutes with products in any category), we see that addition and multiplication on $R$ are continuous. Let $a, b: R \rightarrow R^{\prime}$ be two homomorphisms of topological rings. Then as the equalizer we can simply take the equalizer of $a$ and $b$ as maps of topological spaces, which is the same thing as the equalizer as maps of rings endowed with the induced topology.
}

\prop{Subgroups of Topological Group are Topological Groups}{
    Let $G$ be a topological group and $H$ be a subgroup of $G$. Then $H$ is a topological group with the subspace topology induced by $G$.
}
\pf{
    Since $H$ is a subgroup of $G$, the group multiplication map $\mu:G\times G\to G$ restricts to a map $\mu|_{H\times H}:H\times H\to H$. Since the inclusion $i:H\hookrightarrow G$ is continuous, 
    \[
        \mu|_{H\times H}:H \times H \xrightarrow{i\times i} G\times G\xrightarrow{\mu}\mu(G)  \xrightarrow{i'} H
    \]
    is also continuous. Similarly, the inversion map $\sigma:G\to G$ restricts to a map $\sigma|_H:H\to H$ and $\sigma|_H$ is continuous. Hence $H$ is a topological group.
    
}

\prop{Translation Invariance}{
    For any $a \in G$, left or right multiplication by $a$ yields a homeomorphism $G \rightarrow G$.
}
\pf{
    Given any $a\in G$, let 
    \begin{align*}
        L_a=\mu(a,\cdot):G&\longrightarrow G\\
        g&\longmapsto ag
    \end{align*}
    be the left multiplication map and
    \begin{align*}
        R_a=\mu(\cdot,a):G&\longrightarrow G\\
        g&\longmapsto ga
    \end{align*}
    be the right multiplication map. Since the group multiplication map $\mu:G\times G\to G$ is continuous, $L_a$ and $R_a$ must be continuous. Note that $L_a^{-1}=L_{a^{-1}}$ and $R_a^{-1}=R_{a^{-1}}$. Then we see $L_a^{-1}$ and $R_a^{-1}$ are also continuous maps. Hence $L_a$ and $R_a$ are homeomorphisms.
}

\cor[translation_invariance_cor]{}{
    Given any $a \in G$ and $S \subseteq G$, let's denote $a S:=\{a s: s \in S\}$ and $S a:=\{s a: s \in S\}$. Then
    \begin{itemize}
        \item $S$ is open $\iff$ $a S$ is open $\iff$ $a S$ is open.
        \item $S$ is closed $\iff$ $a S$ is closed $\iff$ $a S$ is closed.
    \end{itemize}
}
\prop{Neighborhood Basis at $1_G$ Determines the Topology of $G$}{
    Given a topological group $G$, if $\mathcal{N}$ is a neighborhood basis of the identity element $1_G$, then for all $x \in X$, 
    \[
        x \mathcal{N}:=\{x N: N \in \mathcal{N}\}
    \] 
    is a neighborhood basis of $x$ in $G$. In particular, the topology on $G$ is completely determined by any neighborhood basis at the identity element.
}
\pf{
    Let $x \in G$ and $V$ be any neighborhood of $x$. There exists an open set $U$ such that $x\in U\subseteq V$. By \Cref{th:translation_invariance_cor}, $x^{-1} U$ is an open neighborhood of $1_G$. Since $\mathcal{N}$ is a neighborhood basis of $1_G$, there exists $N \in \mathcal{N}$ such that $N \subseteq x^{-1} U$. Then there exists $x N \in x \mathcal{N}$ such that $x N \subseteq U$. Hence $x \mathcal{N}$ is a neighborhood basis of $x$.
}

\dfn{Topological Group Action on a Topological Space}{
    Let $G$ be a topological group and $X$ be a topological space. A \textbf{topological group action} of $G$ on $X$ is a group action of $G$ on $X$ such that the action map
    \begin{align*}
        \mu:G\times X&\longrightarrow X\\
        (g,x)&\longmapsto g\cdot x
    \end{align*}
    is continuous.
}
\dfn{Inverse Limit in $\mathsf{TopGrp}$}{
    Let $\mathsf{I}$ be a \hyperref[th:filtered_category]{filtered} \hyperref[th:thin_category]{thin category} and $F:\mathsf{I}^{\mathrm{op}}\to \mathsf{TopGrp}$ be a functor. Similar to the \hyperref[th:inverse_limit_of_groups]{inverse limit in \textsf{Grp}}, we can unpack the information of $F$ into an inverse system $\left(\left(G_i\right)_{i \in I},\left(f_{i j}\right)_{i \leq j \in I}\right)$. The inverse limit of this inverse system is $\varprojlim F$, also denoted by $\varprojlim_{i\in I}G_i$.\\
    To give a concrete construction of $\varprojlim_{i\in I}G_i$, we can take the inverse limit of the underlying group and endow it with the subspace topology induced by the product topology on $\prod_{i\in I}G_i$.
}


\section{Topological Ring}
\dfn{Topological Ring}{
    A \textbf{topological ring} is a ring $R$ equipped with a topology $\tau$ such that the ring addition map
    \begin{align*}
        +:R\times R&\longrightarrow R\\
        (a,b)&\longmapsto a+b
    \end{align*}
    the ring multiplication map
    \begin{align*}
        \cdot:R\times R&\longrightarrow R\\
        (a,b)&\longmapsto a\cdot b
    \end{align*}
    and the addition inverse map
    \begin{align*}
        -:R&\longrightarrow R\\
        a&\longmapsto -a
    \end{align*}
    are continuous maps.
}
