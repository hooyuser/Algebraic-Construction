
\chapter{Topological Group}
\section{Topological Group}
\begin{definition}{Topological Group}{}
    A \textbf{topological group} is a group $G$ equipped with a topology $\tau$ such that the group multiplication map
    \begin{align*}
        \mu:G\times G&\longrightarrow G\\
        (g,h)&\longmapsto gh
    \end{align*}
    and the inversion map
    \begin{align*}
        \sigma:G&\longrightarrow G\\
        g&\longmapsto g^{-1}
    \end{align*}
    are continuous maps.
\end{definition}

\noindent Topological groups are the group objects in the category $\mathsf{Top}$.
\begin{definition}{Topological Group Category}{}
    Topological groups form a category $\mathsf{TopGrp}$, where the morphisms are continuous group homomorphisms. 
\end{definition}

\noindent An isomorphism of topological groups is a group isomorphism that is also a homeomorphism of the underlying topological spaces.
\begin{proposition}{}{}
    The category $\mathsf{TopGrp}$ is complete. Limits in $\mathsf{TopGrp}$ commute with 
    \begin{itemize}
        \item forgetful functor $\mathsf{TopGrp}\to\mathsf{Top}$
        \item forgetful functor $\mathsf{TopGrp}\to\mathsf{Grp}$
    \end{itemize}
\end{proposition}

\begin{prf} 
    It is enough to prove the existence and commutation for products and equalizers. Let $R_i, i \in I$ be a collection of topological rings. Take the usual product $R=\prod R_i$ with the product topology. Since $R \times R=\prod\left(R_i \times R_i\right)$ as a topological space (because products commutes with products in any category), we see that addition and multiplication on $R$ are continuous. Let $a, b: R \rightarrow R^{\prime}$ be two homomorphisms of topological rings. Then as the equalizer we can simply take the equalizer of $a$ and $b$ as maps of topological spaces, which is the same thing as the equalizer as maps of rings endowed with the induced topology.
\end{prf}


\begin{proposition}{Subgroups of Topological Group are Topological Groups}{}
    Let $G$ be a topological group and $H$ be a subgroup of $G$. Then $H$ is a topological group with the subspace topology induced by $G$.
\end{proposition}

\begin{prf}
    Since $H$ is a subgroup of $G$, the group multiplication map $\mu:G\times G\to G$ restricts to a map $\mu|_{H\times H}:H\times H\to H$. Since the inclusion $i:H\hookrightarrow G$ is continuous, 
    \[
        \mu|_{H\times H}:H \times H \xrightarrow{i\times i} G\times G\xrightarrow{\mu}\mu(G)  \xrightarrow{i'} H
    \]
    is also continuous. Similarly, the inversion map $\sigma:G\to G$ restricts to a map $\sigma|_H:H\to H$ and $\sigma|_H$ is continuous. Hence $H$ is a topological group.
    
\end{prf}


\begin{proposition}{Translation Invariance}{}
    For any $a \in G$, left or right multiplication by $a$ yields a homeomorphism $G \rightarrow G$.
\end{proposition}

\begin{prf}
    Given any $a\in G$, let 
    \begin{align*}
        L_a=\mu(a,\cdot):G&\longrightarrow G\\
        g&\longmapsto ag
    \end{align*}
    be the left multiplication map and
    \begin{align*}
        R_a=\mu(\cdot,a):G&\longrightarrow G\\
        g&\longmapsto ga
    \end{align*}
    be the right multiplication map. Since the group multiplication map $\mu:G\times G\to G$ is continuous, $L_a$ and $R_a$ must be continuous. Note that $L_a^{-1}=L_{a^{-1}}$ and $R_a^{-1}=R_{a^{-1}}$. Then we see $L_a^{-1}$ and $R_a^{-1}$ are also continuous maps. Hence $L_a$ and $R_a$ are homeomorphisms.
\end{prf}


\begin{corollary}{}{translation_invariance_cor}
    Given any $a \in G$ and $S \subseteq G$, let's denote $a S:=\{a s: s \in S\}$ and $S a:=\{s a: s \in S\}$. Then
    \begin{itemize}
        \item $S$ is open $\iff$ $a S$ is open $\iff$ $a S$ is open.
        \item $S$ is closed $\iff$ $a S$ is closed $\iff$ $a S$ is closed.
    \end{itemize}
\end{corollary}

\begin{proposition}{Neighborhood Basis at $1_G$ Determines the Topology of $G$}{}
    Given a topological group $G$, if $\mathcal{N}$ is a neighborhood basis of the identity element $1_G$, then for all $x \in X$, 
    \[
        x \mathcal{N}:=\{x N: N \in \mathcal{N}\}
    \] 
    is a neighborhood basis of $x$ in $G$. In particular, the topology on $G$ is completely determined by any neighborhood basis at the identity element.
\end{proposition}

\begin{prf}
    Let $x \in G$ and $V$ be any neighborhood of $x$. There exists an open set $U$ such that $x\in U\subseteq V$. By \Cref{th:translation_invariance_cor}, $x^{-1} U$ is an open neighborhood of $1_G$. Since $\mathcal{N}$ is a neighborhood basis of $1_G$, there exists $N \in \mathcal{N}$ such that $N \subseteq x^{-1} U$. Then there exists $x N \in x \mathcal{N}$ such that $x N \subseteq U$. Hence $x \mathcal{N}$ is a neighborhood basis of $x$.
\end{prf}



\begin{definition}{Inverse Limit in $\mathsf{TopGrp}$}{}
    Let $\mathsf{I}$ be a \hyperref[th:filtered_category]{filtered} \hyperref[th:thin_category]{thin category} and $F:\mathsf{I}^{\mathrm{op}}\to \mathsf{TopGrp}$ be a functor. Similar to the \hyperref[th:inverse_limit_of_groups]{inverse limit in \textsf{Grp}}, we can unpack the information of $F$ into an inverse system $\left(\left(G_i\right)_{i \in I},\left(f_{i j}\right)_{i \leq j \in I}\right)$. The inverse limit of this inverse system is $\varprojlim F$, also denoted by $\varprojlim_{i\in I}G_i$.\\
    To give a concrete construction of $\varprojlim_{i\in I}G_i$, we can take the inverse limit of the underlying group and endow it with the subspace topology induced by the product topology on $\prod_{i\in I}G_i$.
\end{definition}


\section{Continuous Topological Group Action}
\begin{definition}{Compact Open Topology}{}
	Let $X$ be a topological space and $K$ be a compact subset of $X$. The \textbf{compact open topology} on $\mathrm{Hom}_{\mathsf{Top}}(X,Y)$ is the topology generated by the subbasis
	\[
		\mathcal{S}:=\left\{f\in \mathrm{Hom}_{\mathsf{Top}}(X,Y)\midv K\text{ is compact in }X,\;V\text{ is open in }Y,\;f(K)\subseteq V\right\}.
	\]
\end{definition}


\begin{definition}{Group Action on Topological Space by Homeomorphisms}{}
	A \textbf{group action} on a topological space $X$ is a group homomorphism $\rho:G\to \mathrm{Aut}_{\mathsf{Top}}(X)$, where $\mathrm{Aut}_{\mathsf{Top}}(X)$ is the group of all homeomorphisms from $X$ to itself.
\end{definition}


\begin{definition}{Continuous Topological Group Action on Topological Space}{}
	A \textbf{continuous topological group action} on a topological space $X$ is a group homomorphism $\rho:G\to \mathrm{Aut}_{\mathsf{Set}}(X)$, where $G$ is a topological group, such that the following map induced by $\rho$ 
	\begin{align*}
		\varrho:G\times X&\longrightarrow X\\
		(g,x)&\longmapsto \rho(g)(x)
	\end{align*}
	is continuous. In this case, we have $\mathrm{im}\rho \subseteq \mathrm{Aut}_{\mathsf{Top}}(X)$.
\end{definition}

\begin{prf}
    For any $g\in G$, 
    \begin{align*}
      \rho(g): X &\longrightarrow X\\
        x &\longmapsto \varrho(g,x)
    \end{align*}
    is continuous and has a continuous inverse $\rho(g^{-1})$. Hence $\rho(g)\in \mathrm{Aut}_{\mathsf{Top}}(X)$.
\end{prf}

\begin{proposition}{Discrete Group Acts Continuously on Topological Space $\iff$ Acts by Homeomorphisms}{}
    Let $G$ be a group acting on the underlying set of a topological space $X$ through a group homomorphism $\rho:G\to \mathrm{Aut}_{\mathsf{Set}}(X)$. Then the following are equivalent:
    \begin{enumerate}[(i)]
        \item $G$ equipped with discrete topology acts continuously on $X$ 
        \item $G$ acts by homeomorphisms on $X$, i.e.,
        $\mathrm{im}\rho \subseteq \mathrm{Aut}_{\mathsf{Top}}(X)$
    \end{enumerate}
\end{proposition}
\begin{prf}
    We only need to prove (ii)$\implies$ (i). For any open set $U\subseteq X$, we have
    \begin{align*}
        \varrho^{-1}(U)&=\{(g,x)\in G\times X\mid \varrho(g,x)\in U\}\\
        &=\{(g,x)\in G\times X\mid \rho(g)(x)\in U\}\\
        &=\{(g,x)\in G\times X\mid x\in \rho(g)^{-1}(U)\}\\
        &=\bigcup_{g\in G}\left(\left\{g\right\}\times \rho(g)^{-1}(U) \right)
    \end{align*}
    Since $\rho(g)$ is a homeomorphism, $\rho(g)^{-1}(U)$ is open for any open set $U$. Since $G$ is discrete,  each $\left\{g\right\}\times \rho(g)^{-1}(U)$ is open in $G\times X$. Hence $\varrho^{-1}(U)$ is open in $G\times X$, which implies $\varrho$ is continuous.
\end{prf}



% \begin{proposition}{}{}
% 	Let $X$ be a locally compact Hausdorff space and $G$ be a topological group.
% \end{proposition}


\begin{definition}{Orbit Space}{}
	Let $G$ be a group acting on a topological space $X$. The \textbf{orbit space} of $X$ under the action of $G$ is the quotient space $X / G$ obtained by identifying all points in $X$ that are in the same orbit. $X / G$ is equipped with the quotient topology: a subset $U\subseteq X / G$ is open if and only if $\pi^{-1}(U)$ is open in $X$, where $\pi:X\to X / G$ is the quotient map.
\end{definition}



\begin{proposition}{}{}
	For any continuous action of a topological group $G$ on a topological space $E$, the quotient map $p: E \rightarrow E / G$ is an open map.
\end{proposition}

\begin{prf}
	For any $g \in G$ and any subset $U \subseteq M$, we define a set $g \cdot U \subseteq M$ by
$$
g \cdot U=\{g \cdot x: x \in U\} .
$$
If $U \subseteq M$ is open, then $\pi^{-1}(\pi(U))$ is equal to the union of all sets of the form $g \cdot U$ as $g$ ranges over $G$. Since $p \mapsto g \cdot p$ is a homeomorphism, each such set is open, and therefore $\pi^{-1}(\pi(U))$ is open in $M$. Becaues $\pi$ is a quotient map, this implies that $\pi(U)$ is open in $M / G$, and therefore $\pi$ is an open map.
\end{prf}



\begin{definition}{Even Action}{}
    Let $G$ be a topological group acting continuously on a topological space $E$. The action of $E$ is \textbf{even} if each point $y \in Y$ has some neighborhood $U$ such that $gU\cap U=\varnothing$ for all $g\in G-\{1_G\}$. In other words, each point has some neighborhood $U$ such that $U \cap gU \neq \varnothing$ implies $g=1_G$.
\end{definition}

If $G$ acts continuously and evenly on a topological space $E$, then each subgroup $H$ of $G$ also acts continuously and evenly on $E$.

\section{Topological Ring}
\begin{definition}{Topological Ring}{}
    A \textbf{topological ring} is a ring $R$ equipped with a topology $\tau$ such that the ring addition map
    \begin{align*}
        +:R\times R&\longrightarrow R\\
        (a,b)&\longmapsto a+b
    \end{align*}
    the ring multiplication map
    \begin{align*}
        \cdot:R\times R&\longrightarrow R\\
        (a,b)&\longmapsto a\cdot b
    \end{align*}
    and the addition inverse map
    \begin{align*}
        -:R&\longrightarrow R\\
        a&\longmapsto -a
    \end{align*}
    are continuous maps.
\end{definition}

