
\chapter{Set Theory}
\thispagestyle{empty}
\setcounter{page}{1}
\section{Set}
\subsection{Basic Operations}
\begin{definition}{Family of Sets}{}
    Let $I$ be some index set. A \textbf{family of sets index by $I$} is a function that maps each index $i\in I$ to a set $A_i$, denoted by $\left(A_i\right)_{i\in I}$.
\end{definition}

\begin{definition}{Union and Intersection}{}
    Let $I$ be some index set and $(A_i)_{i\in I}$ be a family of sets. The \textbf{union} and \textbf{intersection} of $(A_i)_{i\in I}$ are defined as follows
	\begin{align*}
		\bigcup_{i\in I} A_i&:=\left\{x:\exists\,i\in I,x\in A_i \right\}\\
		\bigcap_{i\in I} A_i&:=\left\{x:\forall\,i\in I,x\in A_i \right\}
	\end{align*}
\end{definition}

\begin{proposition}{Distribution Law}{}
	Suppose that $A_\alpha,A,B_\alpha,B$ are sets and $I$ is some index set. We have the following distribution law:
	\begin{itemize}
        \item $A\cup\left(\bigcap\limits_{\alpha\in I}B_\alpha\right)=\bigcap\limits_{\alpha\in I}\left(A\cup B_\alpha\right)$
		\item $A\cap\left(\bigcup\limits_{\alpha\in I}B_\alpha\right)=\bigcup\limits_{\alpha\in I}\left(A\cap B_\alpha\right)$
	\end{itemize}
\end{proposition}


\begin{proposition}{De Morgan Law}{}
	Suppose that $A_\alpha,A$ are sets and $I$ is some index set. We have
	\begin{itemize}
		\item $\left(\bigcup\limits_{\alpha\in I}A_\alpha\right)^\complement=\bigcap\limits_{\alpha\in I}A_\alpha^{\complement}\ $, $\left(\bigcap\limits_{\alpha\in I}A_\alpha\right)^\complement=\bigcup\limits_{\alpha\in I}A_\alpha^{\complement}$
		\item $E-\bigcup\limits_{\alpha\in I}A_\alpha=\bigcap\limits_{\alpha\in I}\left(E-A_\alpha\right)\ $, $E-\bigcap\limits_{\alpha\in I}A_\alpha=\bigcup\limits_{\alpha\in I}\left(E-A_\alpha\right)$
	\end{itemize}
\end{proposition}


\subsection{Limit of Sequence of Sets}
\begin{definition}{Upper Limit of Sequence of Sets}{}
	Let $\left(A_n\right)_{n=1}^{\infty}$ be a sequence of sets. The \textbf{upper limit} of $\left(A_n\right)_{n=1}^{\infty}$ is defined as follows
	\begin{align*}
		\uplim_{n\to\infty}A_n&:=\bigcap_{n=1}^\infty\bigcup_{k=n}^\infty A_k=\left\{x:x\text{ belongs to infinitely many sets in }\left(A_n\right)_{n=1}^{\infty}\right\}
	\end{align*}
\end{definition}

We have
\[
	x\in \uplim_{n\to\infty}A_n\iff \forall n\ge1,\exists k\ge n,x\in A_k.
\]

\dfn{Lower Limit of Sequence of Sets}{
	Let $\left(A_n\right)_{n=1}^{\infty}$ be a sequence of sets. The \textbf{lower limit} of $\left(A_n\right)_{n=1}^{\infty}$ is defined as follows
	\begin{align*}
		\lowlim_{n\to\infty}A_n&:=\bigcup_{n=1}^\infty\bigcap_{k=n}^\infty A_k=\left\{x:x\text{ belongs to all but finitely many sets in }\left(A_n\right)_{n=1}^{\infty}\right\}\\
	\end{align*}
}
\begin{proposition}{Property of Upper Limit and Lower limit}{}
	\[
		\bigcap_{n=1}^\infty A_n\subseteq\lowlim_{n\to\infty}A_n\subseteq\uplim_{n\to\infty}A_n\subseteq \bigcup_{n=1}^\infty A_n.
	\]
\end{proposition}

\begin{definition}{Limit of Sequence of Sets}{}
	Let $\left(A_n\right)_{n=1}^{\infty}$ be a sequence of sets. If 
	$\lowlim\limits_{n\to\infty}A_n=\uplim\limits_{n\to\infty}A_n$, we define the \textbf{limit} of $\left(A_n\right)_{n=1}^{\infty}$ as follows 
	\[
		\lim\limits_{n\to\infty}A_n:=\lowlim\limits_{n\to\infty}A_n=\uplim\limits_{n\to\infty}A_n
	\]
\end{definition}

\begin{proposition}{}{}
	Let $\left(A_n\right)_{n=1}^{\infty}$ be a sequence of sets.
	\begin{enumerate}
		\item If $\left(A_n\right)_{n=1}^{\infty}$ is nondecreasing, then it has limit
		\[
			\lim\limits_{n\to\infty}A_n=\bigcup_{n=1}^\infty A_n.
		\]
		\item If $\left(A_n\right)_{n=1}^{\infty}$ is nonincreasing, then it has limit
		\[
			\lim\limits_{n\to\infty}A_n=\bigcap_{n=1}^\infty A_n.
		\]
	\end{enumerate}
\end{proposition}


\subsection{Relation}
\begin{definition}{Relation}{}
    An \textbf{$n$-ary relation} $R$ over sets $X_1, \cdots, X_n$ is a subset of the Cartesian product $X_1 \times \cdots \times X_n$. A \textbf{binary relation} $R$ over sets $X$ and $Y$ is a subset of $X \times Y$. We write $xRy$ to denote that $(x, y) \in R$.
\end{definition}

\begin{definition}{Homogeneous Relation}{homogeneous_relation}
    A \textbf{homogeneous relation} $R$ over a set $X$ is a binary relation between $X$ and itself, i.e. a subset of $X \times X$. 
\end{definition}

Here are some important properties that a homogeneous relation $R$ over a set X may have:
\begin{itemize}
    \item \textbf{Reflexivity}: $\forall x \in X$, $xRx$.
    \item \textbf{Irreflexivity}: $\forall x \in X$, not $xRx$.
    \item \textbf{Transitivity}: $\forall x, y, z \in X$, $xRy$ and $yRz$ $\implies$ $xRz$.
    \item \textbf{Symmetry}: $\forall x, y \in X$, $xRy$ $\implies$ $yRx$.
    \item \textbf{Antisymmetry}: $\forall x, y \in X$, $xRy$ and $yRx$ $\implies$ $x = y$.
    \item \textbf{Asymmetry}: $\forall x, y \in X$, $xRy$ $\implies$ not $yRx$.
    \item \textbf{Strong Connectedness}: $\forall x, y \in X$, $xRy$ or $yRx$.
    \item \textbf{Connectedness}: $\forall x, y \in X$, $x\ne y$ $\implies$ $xRy$ or $yRx$.
\end{itemize}
\begin{table}[h]
    \centering
    \renewcommand{\arraystretch}{1.6}
    \begin{tabular}{l|l|l|l|l}
        \toprule
        \textbf{Homogeneous Relation} & \multicolumn{1}{c|}{\textbf{Reflexivity}} & \multicolumn{1}{c|}{\textbf{Symmetry}} & \multicolumn{1}{c|}{\textbf{Transitivity}} & \multicolumn{1}{c}{\textbf{Connectedness}} \\ 
        \midrule
        \midrule
        Directed graph & & & & \\ \midrule
        Undirected graph & & \poscell{Symmetric} & & \\ \midrule
        Dependency & \poscell{Reflexive} & \poscell{Symmetric} & & \\ \midrule
        Tournament & \negcell{Irreflexive} & \negcell{Asymmetric} & & \\ \midrule
        Preorder & \poscell{Reflexive} & & \poscell{Transitive} & \\ \midrule
        Total preorder & \poscell{Reflexive} & & \poscell{Transitive} & \poscell{Strongly Connected} \\ \midrule
        Partial order & \poscell{Reflexive} & \negcell{Antisymmetric} & \poscell{Transitive} & \\ \midrule
        Strict partial order & \negcell{Irreflexive} & \negcell{Asymmetric} & \poscell{Transitive} & \\ \midrule
        Total order & \poscell{Reflexive} & \negcell{Antisymmetric} & \poscell{Transitive} & \poscell{Strongly Connected} \\ \midrule
        Strict total order & \negcell{Irreflexive} & \negcell{Asymmetric} & \poscell{Transitive} & \poscell{Connected} \\ \midrule
        Partial equivalence & & \poscell{Symmetric} & \poscell{Transitive} & \\ \midrule 
        Equivalence & \poscell{Reflexive} & \poscell{Symmetric} & \poscell{Transitive} & \\ \bottomrule
    \end{tabular}
\end{table}

\begin{definition}{Filtered Set}{}
    A \textbf{filtered set} or \textbf{directed set} is a preorder set $(X, \le)$ with an additional property that every pair of elements has an upper bound. In other words, for every $x, y \in X$, there exists $z \in X$ such that $x \le z$ and $y \le z$.
\end{definition}

\begin{proposition}{Intersection of equivalence relations is an equivalence relation}{}
    The intersection of a family of equivalence relations on a set $X$ is an equivalence relation on $X$.
\end{proposition}

\begin{prf}
    Suppose $(R_i)_{i\in i}$ is a family of equivalence relations on $X$. We can check that $\bigcap\limits_{i\in I}R_i$ is an equivalence relation on $X$ by checking the three properties of equivalence relation:
    \begin{enumerate}[(i)]
        \item Reflexivity: For any $x\in X$, since $(x,x)\in R_i$ for all $i\in I$, we have $(x,x)\in \bigcap\limits_{i\in I}R_i$.
        \item Symmetry: For any $x,y\in X$,
        \[
            (x,y)\in \bigcap\limits_{i\in I}R_i\implies\forall i\in I,(x,y)\in R_i\implies\forall i\in I, (y,x)\in R_i\implies(y,x)\in \bigcap\limits_{i\in I}R_i
        \]
        \item Transitivity: For any $x,y,z\in X$,
        \begin{align*}
            (x,y)\in \bigcap\limits_{i\in I}R_i\text{ and }(y,z)\in \bigcap\limits_{i\in I}R_i
            &\implies\forall i\in I,(x,y)\in R_i\text{ and }(y,z)\in R_i\\
            &\implies\forall i\in I, (x,z)\in R_i\\
            &\implies(x,z)\in \bigcap\limits_{i\in I}R_i
        \end{align*}
    \end{enumerate}
\end{prf}


\begin{definition}{Generated Equivalence Relation}{}
    Let $X$ be a set and $R\subseteq X\times X$ be a relation on $X$. The \textbf{generated relation} $\langle R \rangle$ is defined as the smallest equivalence relation on $X$ that contains $R$, or equivalently, the intersection of all equivalence relations on $X$ that contain $R$.
\end{definition}


\section{Function}
\begin{proposition}{}{}
Let $f:X\to Y$ be a map. Suppose that $A_\alpha,A,E\subseteq X$ and $B_\alpha,B,F\subseteq Y$. We have
\begin{itemize}


	\item $f\left(\bigcup\limits_{\alpha\in I}A_\alpha\right)=\bigcup\limits_{\alpha\in I}f\left(A_\alpha\right)\ $, $f\left(\bigcap\limits_{\alpha\in I}A_\alpha\right)\subseteq\bigcap\limits_{\alpha\in I}f\left(A_\alpha\right)\ $, $f(E-A)\supseteq f(E)-f(A)$.
	\item $f^{-1}\left(\bigcup\limits_{\alpha\in I}B_\alpha\right)=\bigcup\limits_{\alpha\in I}f^{-1}\left(B_\alpha\right)\ $, $f^{-1}\left(\bigcap\limits_{\alpha\in I}B_\alpha\right)=\bigcap\limits_{\alpha\in I}f^{-1}\left(B_\alpha\right)\ $, $f^{-1}(F-B)=f^{-1}(F)-f^{-1}(B)$.
	\item $A\subseteq f^{-1}(f(A))\ $, $B\supseteq f(f^{-1}(B))$.
\end{itemize}
\end{proposition}

\begin{proposition}{Equivalent Characterization of Injections}{}
    Let $f:X\to Y$ be a map. The following are equivalent:
    \begin{enumerate}[(i)]
        \item $f$ is injective.
        \item $f$ has a left inverse: there exists a map $h:Y\to X$ such that $h\circ f=\mathrm{id}_X$.
        \item $f$ is left-cancellable: if $f\circ g_1=f\circ g_2$, then $g_1=g_2$.
    \end{enumerate}
\end{proposition}

\begin{proposition}{Equivalent Characterization of Surjections}{}
    Let $f:X\to Y$ be a map. The following are equivalent:
    \begin{enumerate}[(i)]
        \item $f$ is surjective.
        \item $f$ has a right inverse: there exists a map $h:Y\to X$ such that $f\circ h=\mathrm{id}_Y$.
        \item $f$ is right-cancellable: if $g_1\circ f=g_2\circ f$, then $g_1=g_2$.
    \end{enumerate}
\end{proposition}


\begin{proposition}{}{}
    Suppose $f:X\to Y$ and $g:Y\to Z$ are maps. We have
    \begin{enumerate}[(i)]
        \item If $g\circ f$ is injective, then $f$ is injective.
        \item If $g\circ f$ is surjective, then $g$ is surjective.
    \end{enumerate}
\end{proposition}


\section{Grothendieck Universe}
\begin{definition}{Grothendieck Universe}{}
    A \textbf{Grothendieck universe} is a set $\mathscr{U}$ such that
    \begin{enumerate}[(i)]
        \item $\mathscr{U}$ is transitive: $\forall x\in \mathscr{U},\forall y\in x, \;\left(y\in \mathscr{U}\right)$. Or equivalently, $\forall x\in \mathscr{U}, x\subseteq \mathscr{U}$.
        \item $\mathscr{U}$ is closed under power set: $\forall x\in \mathscr{U}, \mathscr{P}(x)\in \mathscr{U}$.
        \item $\mathscr{U}$ is closed under union: $\forall x\in \mathscr{U},\bigcup x\in \mathscr{U}$.
        \item $\mathscr{U}$ is closed under replacement: $\forall x\in \mathscr{U},\forall f\in \mathrm{Hom}(x,\mathscr{U}), f(x)\in \mathscr{U}$.
        \item $\mathscr{U}$ contains all the natural numbers: $\mathbb{N}\in \mathscr{U}$.
    \end{enumerate}
\end{definition}

\begin{definition}{$\mathscr{U}$-Small Set}{small_set}
    Let $\mathscr{U}$ be a Grothendieck universe. A set $X$ is called a \textbf{$\mathscr{U}$-small set} if there exists $Y\in \mathscr{U}$ such that $X$ and $Y$ have the same cardinality.
\end{definition}
