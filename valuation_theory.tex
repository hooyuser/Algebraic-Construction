
\chapter{Valuation Theory}
\section{Valuation of Ring}
\begin{definition}{Totally Ordered Abelian Group}{totally_ordered_abelian_group}
    Suppose $(\Gamma,+)$ is an abelian group and $\le$ is a \hyperref[th:homogeneous_relation]{total order} on $\Gamma$. Totally ordered abelian group is a tuple $(\Gamma,+,\le)$ such that for any $a,b,c\in \Gamma$,
    \[
    a\le b \implies a+c\le b+c.
    \]
    The total order $\le$ can induce a strict total order $<$ on $\Gamma$ by defining $a<b\iff a\le b$ and $a\ne b$.
\end{definition}

\begin{proposition}{Properties of Totally Ordered Abelian Group}{}
    Let $(\Gamma,+,\le)$ be a totally ordered abelian group. Then
    \begin{enumerate}[(i)]
        \item $a\le a',b\le b'\implies a+b\le a'+b'$.
        \item $x\le y\iff -y\le -x$. 
        \item $\Gamma$ is torsion-free. That is, for all $n\in\mathbb{Z}_{\ge1}$ and $a\in\Gamma$, 
        \[
            na= 0\implies 
            a= 0.
        \] 
    \end{enumerate}
\end{proposition}

\begin{prf}
    \begin{enumerate}[(i)]
        \item If $a\le a',b\le b'$, then $a+b\le a'+b\le a'+b'$.
        \item $x\le y\implies x-x-y\le y-x-y\implies -y\le -x$. The other direction is similar.
        \item If $a>0$, then $na>0$ for all $n\in \mathbb{Z}_{\ge 1}$. If $a<0$, then $na<0$ for all $n\in \mathbb{Z}_{\ge 1}$. Therefore, if $a\ne 0$, then $na\ne 0$ for all $n\in \mathbb{Z}_{\ge 1}$.
        
    \end{enumerate}
\end{prf}

\begin{proposition}{Extended Totally Ordered Abelian Group}{}
    Let $(\Gamma,+,\le)$ be a totally ordered abelian group. The total order and group addition on $\Gamma$ are extended to the set $\Gamma \cup\{\infty\}$ by the rules:
    \begin{itemize}
        \item $\alpha\le\infty$ for all $\alpha \in \Gamma\cup\{\infty\}$,
        \item $\infty+\alpha=\alpha+\infty=\infty$ for all $\alpha \in \Gamma\cup\{\infty\}$.
    \end{itemize}
\end{proposition}


\begin{definition}{Valuation of Ring}{}
    Let $R$ be a commutative ring, and $(\Gamma, +, \leq)$ be a totally ordered abelian group. A valuation on $R$ with value group $\Gamma$ refers to a mapping $v: R \rightarrow \Gamma \sqcup\{\infty\}$ satisfying the following properties:
    \begin{itemize}
        \item $v(1)=0$, $v(0)=\infty$,
        \item $v(x y)=v(x)+v(y), \quad\forall x, y \in R$,
        \item $v(x+y) \geq \min \{v(x), v(y)\}, \quad\forall x, y \in R$.        
    \end{itemize}
    Furthermore, we require that $v(R) -\{\infty\}$ generates the group $\Gamma$. If there exists an embedding of the totally ordered abelian group $\Gamma \hookrightarrow \mathbb{R}$, then $v$ is referred to as a rank 1 valuation.
\end{definition}

\begin{proposition}{Properties of Valuation of Ring}{}
    Let $v:R\to \Gamma \cup\{\infty\}$ be a valuation of a commutative ring $R$. Then
    \begin{enumerate}[(i)]
        \item If $x\in R^\times$, then $v(x^{-1})=-v(x)$.
        \item $v(R^\times)$ is a subgroup of $\Gamma$.
        \item If $a^n=1$ for some $n\in\mathbb{Z}_{\ge1}$, then $v(a)=0$. Specially, $v(-1)=0$.
        \item $v(-a)=v(a)$ for all $a\in R$.
        \item $v^{-1}\left(\infty\right)$ is a prime ideal of $R$.
        \item By the universal property quotient set, $v$ induces a map
        \begin{align*}
            \tilde{v}: R/v^{-1}\left(\infty\right)&\longrightarrow \Gamma\cup\{\infty\}\\
            x+v^{-1}\left(\infty\right)&\longmapsto v(x)
        \end{align*}
        Moreover, $\tilde{v}$ is a valuation of $R/v^{-1}\left(\infty\right)$.
        \item For any $x_1,\cdots,x_n\in R$, we have
        \[
            v\left(x_1+\cdots+x_n\right) \geq \min \left\{v\left(x_1\right), \cdots, v\left(x_n\right)\right\}
            \]
        If there exists $j$ such that for all $i\ne j$ we have $v(x_j)<v(x_i)$, then the equality 
        \[
            v\left(x_1+\cdots+x_n\right) = v(x_j)
            \]
            holds.
    \end{enumerate}
\end{proposition}

\begin{prf}
    \begin{enumerate}[(i)]
        \item $v(x^{-1})=v(x^{-1}x)-v(x)=v(1)-v(x)=-v(x)$.
        \item If $x\in R^\times$, then $v(x)+v(x^{-1})=0\implies v(x)\ne \infty$. 
        \item $v(a^n)=nv(a)=0\implies v(a)=0$.
        \item $v(-a)=v(-1)+v(a)=0+v(a)=v(a)$.
        \item For any $x,y\in v^{-1}(\infty)$, we have $v(x+y)\ge \min\left(v(x),v(y)\right)=\infty$, which means $x+y\in v^{-1}(\infty)$. For any $r\in R$ and $x\in v^{-1}(\infty)$, we have $v(rx)=v(r)+v(x)=\infty$, which means $rx\in v^{-1}(\infty)$. Thus $v^{-1}(\infty)$ is a ideal of $R$. If $x,y\in R$ and $xy\in v^{-1}(\infty)$, then $v(xy)=v(x)+v(y)=\infty$, which means $x\in v^{-1}(\infty)$ or $y\in v^{-1}(\infty)$. Thus $v^{-1}(\infty)$ is a prime ideal of $R$.
        \item We first check $\tilde{v}$ is well-defined. If $x-y=a\in v^{-1}(\infty)$, then we have 
        \[
            v(x)=v(y+a)\ge \min\left\{v(y),v(a)\right\}=\min\left\{v(y),\infty\right\}=v(y).
        \]
        Similarly, we have $v(y)=v(x-a)\ge v(x)$, which means $v(x)=v(y)$. Thus $\tilde{v}$ is well-defined. It is easy to check that $\tilde{v}$ is a valuation.
    \end{enumerate}
\end{prf}

% \begin{definition}{Trivial Valuation}{}
%     A valuation $v$ on a commutative ring $R$ is called \textbf{trivial} if 
%     \[
%         v(x)=\begin{cases}
%             0 & \text{if }x\in R^{\times},\\
%             \infty & \text{if }x\notin R^{\times}.
%         \end{cases}
%     \]
% \end{definition}


\section{Valuation of Field}
\begin{definition}{Valuation of Field}{valuation_of_field}
    Suppose $K$ is a field and $(\Gamma,+,\ge)$ is an totally ordered abelian group. 
Then a \textbf{valuation of $K$} is any map
$$
v: K \rightarrow \Gamma \cup\{\infty\}
$$
which satisfies the following properties for all $a, b$ in $K$ :
\begin{itemize}
    \item $v(a)=\infty$ if and only if $a=0$,
    \item $v(a b)=v(a)+v(b)$, i.e. $v$ is a abelian group homomorphism from $K^{\times}$ to $\Gamma$,
    \item $v(a+b) \geq \min (v(a), v(b))$, with equality if $v(a) \neq v(b)$.
\end{itemize}
\end{definition}

\begin{definition}{Value Group}{}
    The \textbf{value group} of a valuation $v$ is the subgroup of $\Gamma$ defined as $v(K^{\times})$.
\end{definition}

\begin{definition}{Discrete Valuation}{}
    A \textbf{discrete valuation} on a field $K$ is a valuation $v: K\to \mathbb{Z} \cup\{\infty\}$.
\end{definition}

\begin{definition}{Valuation Ring}{}
    The \textbf{valuation ring} of a valuation $v$ is the subring of $K$ defined as
    $$
    \mathcal{O}_{v}:=\{a \in K: v(a) \geq 0\}.
    $$
\end{definition}

\begin{proposition}{}{}
    Suppose $v$ is a valuation of a field $K$. Then the unit group of $\mathcal{O}_{v}$ has the form 
    $$
    \mathcal{O}_{v}^{\times}=\{a \in K: v(a)=0\}.
    $$  
\end{proposition}

\begin{prf}
    For any $a\in \mathcal{O}_{v}^{\times}$, we have $a^{-1}\in \mathcal{O}_{v}$ and $v(a)=-v(a^{-1})\le 0$. This forces $v(a)=0$. Conversely, for any $a\in K$ such that $v(a)=0$, we have $v(a^{-1})=-v(a)=0$, which means $a^{-1}\in \mathcal{O}_{v}$. Thus $a\in \mathcal{O}_{v}^{\times}$.
\end{prf}

\begin{definition}{Residue Field of a Valuation}{}
    Suppose $v$ is a valuation of a field $K$. Then
    $$
    \mathfrak{m}_v := \{a \in K: v(a)>0\}
    $$
    is a maximal ideal of $\mathcal{O}_{v}$. The \textbf{residue field} of $v$ is defined as $\kappa_v=\mathcal{O}_{v}/\mathfrak{m}_v$.
\end{definition}

\begin{definition}{Equivalent Valuation}{}
    Suppose a filed $K$ has two valuations $v:K\to\Gamma \cup\{\infty\}$ and $v':K\to\Gamma' \cup\{\infty\}$. we say $v$ and $v'$ are \textbf{equivalent} if there is an order-preserving group isomorphism $\varphi: v(K^\times) \rightarrow v'(K^\times)$ such that $v'=v\circ \varphi$.
\end{definition}

\begin{proposition}{}{}
   Two valuations are equivalent if and only if their valuation rings are equal.
\end{proposition}

\begin{proposition}{}{}
    Suppose $v$ is a valuation of a field $K$ and $\varpi$ is an element in $\mathcal{O}_v-\{0\}$ such that 
    $$
    \sup\left\{n v\left(\varpi\right)\in \Gamma\mid n\in\mathbb{N}\right\}=\infty.
    $$
    Then there is a natural isomorphism $K\cong \mathcal{O}_v\left[\frac{1}{\varpi}\right]$.
\end{proposition}

\begin{prf}
    Since $K=\mathrm{Frac}(\mathcal{O}_v)$, we have an embedding $\mathcal{O}_v\left[\frac{1}{\varpi}\right]\hookrightarrow K$. For any $x\in K$, there exists $n\in \mathbb{N}$ such that $nv(\varpi)\ge -v(x)$. Thus
    \[
    v(x\varpi^n)= v(x)+nv(\varpi)\ge 0\implies x\varpi^n\in \mathcal{O}_v\implies x\in \mathcal{O}_v\left[\frac{1}{\varpi}\right].
    \]
\end{prf}

