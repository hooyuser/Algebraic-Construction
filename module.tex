

\chapter{Module}
\section{Basic Concepts}
\begin{definition}{Module}{}
    Let $R$ be a ring. An left \textbf{$R$-module} is an abelian group $M$ with a binary operation $R\times M\to M$ such that
    \begin{enumerate}[(i)]
        \item $r(m+n)=rm+rn$ for all $r\in R$ and $m, n\in M$.
        \item $(r+s)m=rm+sm$ for all $r, s\in R$ and $m\in M$.
        \item $(rs)m=r(sm)$ for all $r, s\in R$ and $m\in M$.
        \item $1m=m$ for all $m\in M$.
    \end{enumerate}
    If $R$ is a commutative ring, then $M$ is called a \textbf{commutative $R$-module}.
\end{definition}


\begin{definition}{Homomorphism of $R$-modules}{}
    Let $R$ be a ring and $M, N$ be $R$-modules. A map $f:M\to N$ is called an \textbf{$R$-module homomorphism} if
    \begin{enumerate}[(i)]
        \item $f(m+n)=f(m)+f(n)$ for all $m, n\in M$.
        \item $f(rm)=rf(m)$ for all $r\in R$ and $m\in M$.
    \end{enumerate}
    If $f$ is bijective, then $f$ is called an \textbf{$R$-module isomorphism}. If $M=N$, then $f$ is called an \textbf{$R$-module endomorphism}. If $f$ is bijective, then $f$ is called an \textbf{$R$-module automorphism}. Another name for a homomorphism of $R$-modules is an  \textbf{$R$-linear map}.
\end{definition}


\begin{proposition}{Ring Action on an Abelian Group}{}
From the perspective of representation theory, a module is a ring action on an abelian group. To be more precise, a ring $R$ can be regarded as an $\mathsf{Ab}$-enriched category with only one object, called the delooping of $R$, denoted $\mathsf{B}R$. $\mathsf{Ab}$ itself is an $\mathsf{Ab}$-enriched category. Thus a left $R$-module $M$ is a functor between $\mathsf{Ab}$-enriched categories $\mathcal{M}:\mathsf{B}R\to \mathsf{Ab}$. 
\[
    \begin{tikzcd}[ampersand replacement=\&]
        \mathsf{B}R\&[-25pt]\&[+10pt]\&[-30pt] \mathsf{Ab}\&[-30pt]\&[-30pt] \\ [-15pt] 
        *  \arrow[dd, "r\in R"{name=L, left}] 
        \&[-25pt] \& [+10pt] 
        \& [-30pt] M\arrow[dd, "r\cdot\left(-\right)\in \mathrm{End}_{\mathsf{Ab}}(M)"{name=R}] \&[-30pt]\\ [-10pt] 
        \&  \phantom{.}\arrow[r, "\mathcal{M}", squigarrow]\&\phantom{.}  \&   \\[-10pt] 
        * \& \& \&  M\&
    \end{tikzcd}
\]
As a map between objects, $\mathcal{M}$ assigns an abelian group for $\{*\}$. As a map between morphisms, $\mathcal{M}$ specifies a ring homomorphism $\sigma_M:R\to \mathrm{End}_{\mathsf{Ab}}(M)$. Define the ring representation category to be the category of all $\mathsf{Ab}$-enriched functors between $\mathsf{Ab}$-enriched categories $\mathsf{B}R$ and $\mathsf{Ab}$, denoted by
\[
\mathrm{Rep}_{\mathsf{Ab}}(R):=[\mathsf{B}R, \mathsf{Ab}]_{\mathsf{Ab}\text{-}\mathsf{Cat}}.
\]
Then we have category isomorphism 
\[
    \mathrm{Rep}_{\mathsf{Ab}}(R)\cong R\text{-}\mathsf{Mod}.
\]
\end{proposition}

\begin{proposition}{Ring homomorphism $R\to S$ induces functor $S\text{-}\mathsf{Mod}\to R\text{-}\mathsf{Mod}$}{}
    Let $R$ and $S$ be rings with a ring homomorphism $f: R\to S$. Then every $S$-module $M$ is an $R$-module by defining $rm = f(r)m$, or equivalently through $R\to S\to \mathrm{End}_{\mathsf{Ab}}(M)$. This defines a functor $\mathrm{Res}_{R\to S}: S\text{-}\mathsf{Mod}\to R\text{-}\mathsf{Mod}$, which is identify map on objects and morphisms.
    \[
        \begin{tikzcd}[ampersand replacement=\&]
            S\text{-}\mathsf{Mod}\&[-25pt]\&[+10pt]\&[-30pt] R\text{-}\mathsf{Mod}\&[-30pt]\&[-30pt] \\ [-15pt] 
            M  \arrow[dd, "g"{name=L, left}] 
            \&[-25pt] \& [+10pt] 
            \& [-30pt] M\arrow[dd, "g"{name=R}] \&[-30pt]\\ [-10pt] 
            \&  \phantom{.}\arrow[r, "\mathrm{Res}_{R\to S}", squigarrow]\&\phantom{.}  \&   \\[-10pt] 
            N \& \& \&  N\&
        \end{tikzcd}
        \]  
\end{proposition}

In particular, ring homomorphism $R\to S$ makes $S$ an $R$-module.

\begin{definition}{Noetherian Module}{}
Let $R$ be a commutative ring, and let $M$ be an $R$-module. We say $M$ is \textbf{Noetherian} if one of the following equivalent conditions holds:
\begin{enumerate}[(i)]
    \item Every submodule of $M$ is finitely generated.
    \item Every ascending chain of submodules of $M$ stabilizes; that is, if
    $$
    N_1 \subseteq N_2 \subseteq N_3 \subseteq \cdots
    $$
    is a chain of submodules of $M$, then $\exists i$ such that $N_i=N_{i+1}=N_{i+2}=\ldots$.
    \item Every nonempty family of submodules of $M$ has a maximal element w.r.t. inclusion.
\end{enumerate}
\end{definition}


\begin{definition}{Annihilator}{}
    Let $R$ be a ring, and let $M$ be a left $R$-module. Suppose $S$ is a non-empty subset of $M$. The \textbf{annihilator} of $S$, denoted $\operatorname{Ann}_R(S)$, is a left ideal of $R$ defined by 
    $$
    \operatorname{Ann}_R(S)=\{r \in R \mid r s=0 \text { for all } s \in S\}.
    $$
    If $N$ is a submodule of $M$, then 
    \[
        \operatorname{Ann}_R(N)=\{r \in R \mid r s=0 \text { for all } s \in N\}=\ker \left(R\longrightarrow\mathrm{End}_{\mathsf{Ab}}\left(N\right)\right)  
    \]
    is a two-sided ideal of $R$. Through $R/\operatorname{Ann}_R(N)\to \mathrm{End}_{\mathsf{Ab}}\left(N\right)$, we can regard $N$ as an $R/\operatorname{Ann}_R(N)$-module.
\end{definition}

\begin{proposition}{Properties of Annihilator}{}
    Let $R$ be a ring, and let $M$ be a left $R$-module. 
    \begin{enumerate}[(i)]
        \item If $x\in M$, then $\operatorname{Ann}_R(x)=R$ if and only if $x=0$.
        \item If $S$ is a non-empty subset of $M$ and $N$ is the submodule generated by $S$, then $\operatorname{Ann}_R(N)\subseteq \operatorname{Ann}_R(S)$. If $R$ is commutative, then $\operatorname{Ann}_R(N)= \operatorname{Ann}_R(S)$.
    \end{enumerate}
\end{proposition}

\begin{definition}{Faithful Module}{}
    Let $R$ be a ring, and let $M$ be a left $R$-module. We say $M$ is \textbf{faithful} if $\operatorname{Ann}_R(M)=0$, or equivalently, the map $R\to \mathrm{End}_{\mathsf{Ab}}(M)$ is injective.
\end{definition}


\section{Construction}
\subsection{Free Object}
\begin{definition}{Free Module}{}
    Let $R$ be a ring and $S$ be a set. The \textbf{free $R$-module } on $S$, denoted by $\mathrm{Free}_{R\text{-}\mathsf{Mod}}(S)$, together with a map $\iota:S\to \mathrm{Free}_{R\text{-}\mathsf{Mod}}(S)$, is defined by the following universal property: for any $R$-module $M$ and any map $f:S\to M$, there exists a unique $R$-linear map $\widetilde{f}:\mathrm{Free}_{R\text{-}\mathsf{Mod}}(S)\to M$ such that the following diagram commutes
    \begin{center}
        \begin{tikzcd}[ampersand replacement=\&]
            \mathrm{Free}_{R\text{-}\mathsf{Mod}}(S)\arrow[r, dashed, "\exists !\,\widetilde{f}"]  \& M\\[0.3cm]
            S\arrow[u, "\iota"] \arrow[ru, "f"'] \&  
        \end{tikzcd}
    \end{center}
    The free $R$-module $\mathrm{Free}_{R\text{-}\mathsf{Mod}}(S)$ can be contructed as the direct sum of copies of $R$ indexed by $S$, i.e. 
    \[
        \mathrm{Free}_{R\text{-}\mathsf{Mod}}(S)\cong \bigoplus_{s\in S}R.
    \]
\end{definition}


\begin{example}{Forgetful Functor $U$}{}
    Let $R$ be a ring. Then the forgetful functor $U:R\text{-}\mathsf{Mod}\to \mathsf{Set}$ forgets the $R$-module structure of an $R$-module. 
    \begin{enumerate}[(i)]
        \item $U$ is representable by $\left(R, 1_R\right)$. The natural isomorphism $\phi:\mathrm{Hom}_{R\text{-}\mathsf{Mod}}\left(R, -\right)\xRightarrow{\sim} U$ is given by 
        \begin{align*}
            \phi_M:\mathrm{Hom}_{R\text{-}\mathsf{Mod}}\left(R, M\right)&\longrightarrow U(M)\\
            f&\longmapsto f(1_R).
        \end{align*}
        An $R$-linear map $f:R\to M$ is uniquely determined by $f(1_R)$, and vice versa.
        \item $U$ is faithful but not full.
    \end{enumerate}
\end{example}
 

\begin{proposition}{Free-Forgetful Adjunction $\mathrm{Free}_{R\text{-}\mathsf{Mod}}\dashv U$}{}
    Let $R$ be a ring. Then the free $R$-module functor $\mathrm{Free}_{R\text{-}\mathsf{Mod}}$ is left adjoint to the forgetful functor $U$.
    \[
        \begin{tikzcd}[ampersand replacement=\&]
            R\text{-}\mathsf{Mod} \arrow[r, "\mathrm{Free}_{R\text{-}\mathsf{Mod}}"{name=U}, bend left, start anchor=east, yshift=1.7ex, end anchor=west] \&[+18pt] 
            \mathsf{Set} \arrow[l, "{U}"{name=D, anchor=north}, bend left, start anchor=west, yshift=-1.5ex, end anchor=east]
            \arrow[phantom, from=U, to=D, "\dashv"{rotate=-90}]
        \end{tikzcd}    
    \]
    
    For any set $S$ and any $R$-module $M$, we have a natural isomorphism
    \[
        \mathrm{Hom}_{R\text{-}\mathsf{Mod}}\left(\mathrm{Free}_{R\text{-}\mathsf{Mod}}(S), M\right)\cong \mathrm{Hom}_{\mathsf{Set}}\left(S, U(M)\right).
    \]
\end{proposition}





\begin{definition}{Finitely Generated Module}{}
    We say $M$ is a \textbf{finitely generated $R$-module} if 
    one of the following equivariant conditions holds:
    \begin{enumerate}[(i)]
        \item there exist $x_1, \ldots, x_n \in M$ such that every element of $M$ is an $R$-linear combination of the $x_i$. 
        \item there exists an epimorphism $R^{\oplus n} \rightarrow M$ for some $n \in \mathbb{Z}_+$.
        \item there exists an exact sequence
        \[
        R^{\oplus n}  \rightarrow M \rightarrow 0
        \]
        for some $n\in \mathbb{N}$.
        \item $S\cong R^{\oplus n}/M$ for some $n\in \mathbb{Z}_+$ and some submodule $M$ of $R^{\oplus n}$.
    \end{enumerate}
\end{definition}

\begin{definition}{Finitely Presented Module}{}
    We say $M$ is a \textbf{finitely presented $R$-module} if 
    there exists an exact sequence
    \[
    R^{\oplus m} \rightarrow R^{\oplus n} \rightarrow M \rightarrow 0
    \]
    for some $m, n\in \mathbb{Z}_+$.
\end{definition}

\subsection{Tensor Product}

\begin{definition}{Balanced Product}{}
    Let $R$ be a ring, $M$ be a right $R$-module, $N$ be a left $R$-module and $G$ be an abelian group. A map $b:M\times N\to G$ is called a \textbf{$R$-balanced product} if 
    \begin{enumerate}[(i)]
        \item $b(m_1+m_2, n)=b(m_1, n)+b(m_2, n)$ for all $m_1, m_2\in M$ and $n\in N$.
        \item $b(m, n_1+n_2)=b(m, n_1)+b(m, n_2)$ for all $m\in M$ and $n_1, n_2\in N$.
        \item $b(mr, n)=b(m, rn)$ for all $r\in R$, $m\in M$, and $n\in N$.
    \end{enumerate}
    The set of all $R$-balanced products from $M\times N$ to $G$ is denoted by $\mathrm{Bal}_R(M, N; G)$, which is an abelian group with respect to pointwise addition:
    \[
        (b_1+b_2)(m, n)=b_1(m, n)+b_2(m, n).
    \]
\end{definition}

\begin{proposition}{Functor $\mathrm{Bal}_R(M, N; -):\mathsf{Ab}\to \mathsf{Ab}$}{}
    For $M$ and $N$ fixed, we can define a functor $\mathrm{Bal}_R(M, N; -):\mathsf{Ab}\to \mathsf{Ab}$ by
    \[
        \begin{tikzcd}[ampersand replacement=\&]
            \mathsf{Ab}\&[-25pt]\&[+30pt]\&[-30pt] \mathsf{Ab}\&[-30pt]\&[-30pt] \\ [-15pt] 
            G \arrow[dd, "f"{name=L, left}] 
            \&[-25pt] \& [+10pt] 
            \& [-30pt]\mathrm{Bal}_R(M, N; G)\arrow[dd, "{f_*}"{name=R}] \&[-30pt]\ni
            \&[-30pt]b\arrow[dd,mapsto]\&[-30pt]\\ [-10pt] 
            \&  \phantom{.}\arrow[r, "{\mathrm{Bal}_R(M, N; -)}", squigarrow]\&\phantom{.}  \&   \\[-10pt] 
            H \& \& \&  \mathrm{Bal}_R(M, N; H)\&[-30pt]\ni
            \&[-30pt]f\circ b
        \end{tikzcd}
        \]  
\end{proposition}

\begin{definition}{Category $\mathsf{Bal}_R(M,N)$}{}
    Let $R$ be a ring, $M$ be a right $R$-module, and $N$ be a left $R$-module. The \textbf{category of $R$-balanced products} from $M$ and $N$, denoted by $\mathsf{Bal}_R(M, N)$, is defined as follows:
    \begin{itemize}
        \item The objects of $\mathsf{Bal}_R(M, N)$ are $R$-balanced products from $M\times N$ to $G$.
        \item The morphisms from $b_1:M\times N\to G_1$ to $b_2:M\times N\to G_2$ are group homomorphisms $f:G_1\to G_2$ such that $f\circ b_1=b_2$.
        \[
            \begin{tikzcd}[ampersand replacement=\&]
                \& G_1 \arrow[dd, "f"] \\
M\times N \arrow[ru, "b_1"] \arrow[rd, "b_2"'] \&                   \\
                \& G_2                
\end{tikzcd}
        \]
    \end{itemize}
\end{definition}

\begin{definition}{Tensor Product of $R$-modules}{}
    The \textbf{tensor product} of $R$-modules $M$ and $N$, denoted by $\otimes :M\times N\to M\otimes_R N$, is the initial object in the category $\mathsf{Bal}_R(M, N)$. The tensor product satisfies the following universal property: for any $R$-module $G$ and any $R$-balanced product $b:M\times N\to G$, there exists a unique group homomorphism $\widetilde{b}:M\otimes_R N\to G$ such that the following diagram commutes
    \begin{center}
        \begin{tikzcd}[ampersand replacement=\&]
            M\times N \arrow[r, "\otimes"] \arrow[rd, "b"']  \& M\otimes_R N \arrow[d, dashed, "\exists !\,\widetilde{b}"] \\[0.3cm]
            \&  G
        \end{tikzcd}
    \end{center}
    The tensor product can be constructed as follows: let $F:=\mathrm{Free}_{\mathsf{Ab}}(M\times N)$ be the free abelian group on $M\times N$, and let $K$ be the subgroup of $F$ generated by elements of the form
    \begin{enumerate}[(i)]
        \item $(m_1+m_2, n)-(m_1, n)-(m_2, n)$,
        \item $(m, n_1+n_2)-(m, n_1)-(m, n_2)$,
        \item $(mr, n)-(m, rn)$.
    \end{enumerate}
    Then the tensor product $\otimes :M\times N\to M\otimes_R N$ can be constructed as the composition
    \[
        \otimes :M\times N\xrightarrow{\iota}F \xrightarrow{\pi}F/K
    \]
    and $M\otimes_R N:=F/K$.
\end{definition}
\begin{prf}
    To prove the constructed tensor product $F/K$ is the initial object in the category $\mathsf{Bal}_R(M, N)$, we need to check the universal property of tensor product. Let $G$ be an $R$-module and $b:M\times N\to G$ be an $R$-balanced product. 
    \[
        \begin{tikzcd}
            F \arrow[r, "\pi", two heads] \arrow[rd, "\hat{b}", dashed] & F/K \arrow[d, "\tilde{b}", dashed] \\[1.5em]
            M\times N \arrow[u, "\iota", hook] \arrow[r, "b"']          & G                                 
        \end{tikzcd}
    \]
    First by the universal property of free abelian group, there exists a unique group homomorphism
    \begin{align*}
        \hat{b}:F&\longrightarrow G\\
        \sum_{i=1}^k r_i(m_i, n_i)&\longmapsto \sum_{i=1}^k r_ib(m_i, n_i).
    \end{align*}
    such that $\hat{b}\circ\iota=b$. Note that
    \begin{align*}
        \hat{b}\left((m_1+m_2, n)-(m_1, n)-(m_2, n)\right) &= b(m_1+m_2, n)-b(m_1, n)-b(m_2, n)=0,\\
        \hat{b}\left((m, n_1+n_2)-(m, n_1)-(m, n_2)\right) &= b(m, n_1+n_2)-b(m, n_1)-b(m, n_2)=0,\\
        \hat{b}\left((mr, n)-(m, rn)\right) &= b(mr, n)-b(m, rn)=0.
    \end{align*}
    We see for any $k\in K$, $\hat{b}(k)=0$. Thus by the universal property of quotient group, there exists a unique group homomorphism $\tilde{b}:F/K\to G$ such that $\tilde{b}\circ\pi=\hat{b}$. It is easy to check that $\tilde{b}\circ\pi\circ\iota=b$, which means the diagram commutes. 

    To show the uniqueness, assume there exists another group homomorphism $\tilde{b}':F/K\to G$ such that $\tilde{b}'\circ\pi\circ\iota=b$. Since $\hat{b}\circ \iota=b$, by the uniqueness of $\tilde{b}$, we have $\tilde{b}'\circ\pi=\hat{b}$. Then by the uniqueness of $\tilde{b}$, we have $\tilde{b}=\tilde{b}'$. Thus the tensor product $F/K$ is the initial object in the category $\mathsf{Bal}_R(M, N)$.
\end{prf}

\begin{proposition}{Base Change Functor}{}
    Let $\varphi:R\to S$ be a ring homomorphism. 
\end{proposition}

\subsection{Localization}

\begin{proposition}{}{}
    Let $R$ be a commutative ring and $S \subseteq R$ be a multiplicative subset. The category of $S^{-1} R$ modules is equivalent to the category of $R$-modules $M$ with the property that every $s \in S$ acts as an automorphism on $M$. The following functor $F$ gives a equivalence of categories: 
    \[
        \begin{tikzcd}[ampersand replacement=\&]
            S^{-1} R\text{-}\mathsf{Mod}\&[-25pt]\&[+10pt]\&[-30pt] R\text{-}\mathsf{Mod}\text{ where }S\text{ act as automorphisms}\&[-30pt]\&[-30pt] \\ [-15pt] 
            M  \arrow[dd, "f"{name=L, left}] 
            \&[-25pt] \& [+10pt] 
            \& [-30pt] M\arrow[dd, "f"{name=R}] \&[-30pt]\\ [-10pt] 
            \&  \phantom{.}\arrow[r, "F", squigarrow]\&\phantom{.}  \&   \\[-10pt] 
            N \& \& \&  N\&
        \end{tikzcd}
        \]
\end{proposition}

\begin{prf}
    Assume $S$ is a multiplicative subset of communitative ring $R$ and the localization map is $\varphi:R\to S^{-1}R$. Then $R$ can acts on $S^{-1}R$-module $M$ through 
    \[
        R\xrightarrow{\varphi}S^{-1}R\xrightarrow{\sigma_M'}\mathrm{End}_{\mathsf{Ab}}(M),
    \]
    which enables us to regard $M$ as an $R$-module. Furthermore, since 
    \[
        \sigma_M'(\varphi(S))\subseteq \sigma_M' \left(\left(S^{-1}R\right)^\times\right)\subseteq \left(\mathrm{End}_{\mathsf{Ab}}(M)\right)^\times=\mathrm{Aut}_{\mathsf{Ab}}(M),
    \]
    every $s \in S$ acts as an automorphism on $M$.\\
    Conversely, if $M$ is an $R$-module such that every $s\in S$ acts as an automorphism on $M$, i.e. $\sigma_M:R\to\mathrm{End}_{\mathsf{Ab}}(M)$ satisfies $\sigma_M(S)\subseteq \mathrm{Aut}_{\mathsf{Ab}}(M)$, then by unversal property
    \begin{center}
        \begin{tikzcd}[ampersand replacement=\&]
         
            S^{-1}R\arrow[rr, "\sigma_M'", dashed]\&\& \mathrm{End}_{\mathsf{Ab}}(M) \&  \\             
            \&R \arrow[ru, "\sigma_M"'] \arrow[lu, "\varphi"] \&                          
        \end{tikzcd}
    \end{center}
    we can define a $S^{-1}R$-module structure on $M$ by lifting $\sigma_M$ to $\sigma_M'$. It is easy to check that these two functors are quasi-inverse to each other.
\end{prf}


\begin{definition}{Localization of a Module}{}
    Let $R$ be a commutative ring, $S$ be a multiplicative set in $R$, and $M$ be an $R$-module. The \textbf{localization of the module} $M$ by $S$, denoted $S^{-1}M$, is an $S^{-1}R$-module that is constructed exactly as the localization of $R$, except that the numerators of the fractions belong to $M$. That is, as a set, it consists of equivalence classes, denoted $\frac{m}{s}$, of pairs $(m, s)$, where $m\in M$ and $s\in S$, and two pairs $(m, s)$ and $(n, t)$ are equivalent if there is an element $u$ in $S$ such that
    \[u(sn-tm)=0.\]
    Addition and scalar multiplication are defined as for usual fractions (in the following formula, $r\in R$, $s,t\in S$, and $m,n\in M$):
    \[\frac{m}{s} + \frac{n}{t} = \frac{tm+sn}{st},\]
    \[\frac{r}{s} \frac{m}{t} = \frac{r m}{st}.\]
\end{definition}



\begin{proposition}{Universal Property of Localization}{}
    Let $R$ be a commutative ring and $S\subseteq R$ be a multiplicative subset, an $M$ be an $R$-module. The $R$-linear map
    \begin{align*}
        l_S:M&\longrightarrow S^{-1}M\\
         m&\longmapsto \frac{m}{1}
    \end{align*}
    satisfies the any of following universal properties: 
    \begin{itemize}
        \item for any $R$-linear map $\psi:M\to N$ such that $S$ act as automorphisms on $N$ (i.e. the induced ring homomorphism $\sigma_{N}:R\to\mathrm{End}_{\mathsf{Ab}}(N)$ satisfies $\sigma_{N}(S)\subseteq \mathrm{Aut}_{\mathsf{Ab}}(N)$), there exists a unique $R$-linear map
        \begin{align*}
            \psi':S^{-1}M&\longrightarrow N\\
            \frac{m}{s}&\longmapsto \sigma_{N}(s)^{-1}(\psi(m))
        \end{align*}
        such that the following diagram commutes in $R\text{-}\mathsf{Mod}$
        \begin{center}
            \begin{tikzcd}[ampersand replacement=\&]
             
                S^{-1}M\arrow[rr, "\psi'", dashed]\&\& N \&  \\             
                \&M \arrow[ru, "\psi"'] \arrow[lu, "l_S"] \&                          
            \end{tikzcd}
        \end{center}
        \item for any $S^{-1}R$-module $N$ and $R$-linear map $\psi:M\to \mathrm{Res}(N)$, there exists a unique $S^{-1}R$-linear map $\psi':S^{-1}M\to N$ such that the following diagram commutes in $R\text{-}\mathsf{Mod}$
        \begin{center}
            \begin{tikzcd}[ampersand replacement=\&]
             
                \mathrm{Res}(S^{-1}M)\arrow[rr, "\mathrm{Res}(\psi')", dashed]\&\& \mathrm{Res}(N) \&  \\             
                \&M \arrow[ru, "\psi"'] \arrow[lu, "l_S"] \&                          
            \end{tikzcd}
        \end{center}
    \end{itemize}
\end{proposition}


\begin{proposition}{Localization is a Left Adjoint Functor}{} 
    Let $R$ be a commutative ring, $S$ be a multiplicative set in $R$, and $M$ be an $R$-module. Define the localization functor as follows
    \[
        \begin{tikzcd}[ampersand replacement=\&]
           R\text{-}\mathsf{Mod}\&[-25pt]\&[+10pt]\&[-30pt] S^{-1}R\text{-}\mathsf{Mod}\&[-30pt]\&[-30pt] \\ [-15pt] 
            M  \arrow[dd, "f"{name=L, left}] 
            \&[-25pt] \& [+10pt] 
            \& [-30pt] S^{-1}M\arrow[dd, "S^{-1}(f)"{name=R}] \&[-30pt]\ni
            \&[-30pt]\frac{m}{s}\arrow[dd,mapsto]\&[-30pt]\\ [-10pt] 
            \&  \phantom{.}\arrow[r, "S^{-1}", squigarrow]\&\phantom{.}  \&   \\[-10pt] 
            N \& \& \&  S^{-1}N\&[-30pt]\ni
            \&[-30pt]\frac{f(m)}{s}
        \end{tikzcd}
        \]  
        where $S^{-1}(f)$ is defiend as the composition $S^{-1}M\xrightarrow{f'} N\to S^{-1}N$. \\
        Let $\mathrm{Res}_{R\to S^{-1}R}: S^{-1}R\text{-}\mathsf{Mod}\to R\text{-}\mathsf{Mod}$ be the functor that regards $S^{-1}R$-modules as $R$-modules. Then we have a pair of adjoint functors    
        \[
            \begin{tikzcd}[ampersand replacement=\&]
                R\text{-}\mathsf{Mod} \arrow[r, "S^{-1}"{name=U}, bend left, start anchor=east, yshift=1.7ex, end anchor=west] \&[+12pt] 
                S^{-1}R\text{-}\mathsf{Mod} \arrow[l, "{\mathrm{Res}}"{name=D, anchor=north}, bend left, start anchor=west, yshift=-1.5ex, end anchor=east]
                \arrow[phantom, from=U, to=D, "\dashv"{rotate=-90}]
            \end{tikzcd}    
        \]
         and natural isomorphism
        \[
        \mathrm{Hom}_{S^{-1}R\text{-}\mathsf{Mod}}(S^{-1}M, N)\cong \mathrm{Hom}_{R\text{-}\mathsf{Mod}}(M, \mathrm{Res}_{R\to S^{-1}R}(N)).    
        \]
\end{proposition}

\begin{proposition}{Localization is an Exact Functor}{}
    Let $R$ be a commutative ring, $S$ be a multiplicative set in $R$. If 
    \[
        L\xlongrightarrow {u} M\xlongrightarrow {v} N    
    \]
    is an exact sequence of $R$-module, then 
    \[
        S^{-1}L\xlongrightarrow {S^{-1}(u)} S^{-1}M\xlongrightarrow {S^{-1}(v)} S^{-1}N
    \]
    is an exact sequence of $S^{-1}R$-module.
\end{proposition}

\begin{prf}
    Suppose $\frac{m}{s}\in \ker S^{-1}(v) $. Then we have
    \[
        S^{-1}(v)\left(\frac{m}{s}\right)=\frac{v(m)}{s}=\frac{0}{1},
    \]
    which imples that there exists $t\in S$ such that $tv(m)=v(tm)=0$. Thus we have $tm\in \ker v$. By exactness, there exists $l\in L$ such that $u(l)=tm$. Since 
    \[
        S^{-1}(u)\left(\frac{l}{ts}\right)=\frac{u(l)}{ts}=\frac{tm}{ts}=\frac{m}{s},
    \]
    we see that $\frac{m}{s}\in \operatorname{im}S^{-1}(u)$, which means $\operatorname{im}S^{-1}(u)=\ker S^{-1}(v)$. Hence $S^{-1}$ is exact.
\end{prf}

\begin{proposition}{Localization as Tensor Product}{}
    Let \(R\) be a ring, \(S \subseteq R\) a multiplicative subset. The localization functor $S^{-1}:R\text{-}\mathsf{Mod}\to S^{-1}R\text{-}\mathsf{Mod}$ is isomorphic to the tensor product functor $ S^{-1}R \otimes_R -$.
\end{proposition}
\begin{prf}
Define the map
\begin{align*}
    b_M: S^{-1}R \times M &\longrightarrow S^{-1}M\\
    \left(\frac{a}{s}, m\right) &\longmapsto \frac{am}{s}.
\end{align*}
This map is well-defined because for any \(\frac{a}{s} = \frac{a'}{s'}\), there exists \(u \in S\) such that \(u(s'a - sa') = 0\), so that 
\[
u(s'a - sa')m = u(s'(am) - s(a'm))=0 \quad \implies \frac{am}{s} = \frac{a'm}{s'}.
\]
It is straightforward to check that \(b_M\) is an \(R\)-balanced product. By the universal property of the tensor product, there exists a unique \(R\)-linear map
\begin{align*}
    \theta_M: S^{-1}R\otimes_R M &\longrightarrow S^{-1}M\\
    \frac{a}{s} \otimes m &\longmapsto \frac{am}{s}\qquad(a \in A, s \in S, m \in M).
\end{align*}
such that the following diagram commutes
\[
    \begin{tikzcd}
        S^{-1}R\otimes_R M \arrow[rr, "\theta_M", dashed] &                                                          & S^{-1}M \\[1.5em]
                                                        & S^{-1}R\times M \arrow[ru, "b_M"'] \arrow[lu, "\otimes"] &        
        \end{tikzcd}
\]
\[
    \begin{tikzcd}
        S^{-1}M \arrow[rr, "\psi_M", dashed] &                                          & S^{-1}R\otimes_R M \\
                                             & M \arrow[lu, "l"] \arrow[ru, "1\otimes -"'] &                   
        \end{tikzcd}
\]

\vspace{1em}
\textbf{Step 2. Constructing the Inverse}

Define the inverse map
\[
\psi_M: S^{-1}M \longrightarrow S^{-1}R \otimes_R M
\]
by
\[
\psi_M\left(\frac{m}{s}\right) = \frac{1}{s} \otimes m.
\]

\textbf{Verification that \(\theta_M\) and \(\psi_M\) are Inverses:}

\textbf{(i) Composition \(\theta_M \circ \psi_M\):} For any \(\frac{m}{s} \in S^{-1}M\),
\[
(\theta_M \circ \psi_M)\left(\frac{m}{s}\right) = \theta_M\left(\frac{1}{s} \otimes m\right) = \frac{1\cdot m}{s} = \frac{m}{s}.
\]

\textbf{(ii) Composition \(\psi_M \circ \theta_M\):} For any \(\frac{a}{s} \otimes m \in S^{-1}R \otimes_R M\),
\[
(\psi_M \circ \theta_M)\Big(\frac{a}{s} \otimes m\Big) = \psi_M\left(\frac{am}{s}\right) = \frac{1}{s} \otimes am.
\]
By \(A\)-linearity of the tensor product, we have
\[
\frac{1}{s} \otimes am = \frac{a}{s} \otimes m,
\]
so that \(\psi_M \circ \theta_M\) is the identity on \(S^{-1}R \otimes_R M\).

\vspace{1em}
\textbf{Step 3. Naturality}

Let \(f: M \to N\) be a morphism of \(A\)-modules. We need to show that the following diagram commutes:
\[
\begin{array}{ccc}
S^{-1}R \otimes_R M & \xrightarrow{\theta_M} & S^{-1}M \\[1ex]
\downarrow{1\otimes f} &  & \downarrow{S^{-1}f} \\[1ex]
S^{-1}R \otimes_R N & \xrightarrow{\phi_N} & S^{-1}N.
\end{array}
\]
For any \(\frac{a}{s} \otimes m \in S^{-1}R \otimes_R M\), we have
\[
\phi_N\Big((1\otimes f)\Big(\frac{a}{s} \otimes m\Big)\Big) = \phi_N\Big(\frac{a}{s} \otimes f(m)\Big) = \frac{af(m)}{s},
\]
and
\[
(S^{-1}f)\Big(\theta_M\Big(\frac{a}{s} \otimes m\Big)\Big) = (S^{-1}f)\Big(\frac{am}{s}\Big) = \frac{af(m)}{s}.
\]
Thus, the diagram commutes, and the isomorphism is natural.

\vspace{1em}
\textbf{Conclusion}

We have constructed a natural isomorphism
\[
\theta_M: S^{-1}R \otimes_R M \xrightarrow{\sim} S^{-1}M,
\]
given by
\[
\theta_M\left(\frac{a}{s} \otimes m\right) = \frac{am}{s}.
\]
This establishes that the functors
\[
S^{-1}(-) \quad \text{and} \quad S^{-1}A\otimes_A(-)
\]
are naturally isomorphic, i.e.,
\[
\boxed{S^{-1}A\otimes_R M \cong S^{-1}M \quad \text{naturally for all } A\text{-modules } M.}
\]
\end{prf}


\begin{proposition}{Localization Respects Quotients}{localization_respects_quotients}
    Let $M$ be an $R$-module and $N$ be a submodule of $M$. Then we have an isomorphism $S^{-1}(M/N)\cong (S^{-1}M)/(S^{-1}N)$ and the following commutative diagram
    \[
        \begin{tikzcd}
            M \arrow[r, "\pi_M"] \arrow[d, "S^{-1}"'] &[+5em] M/N \arrow[d, "S^{-1}"]              \\[+2em]
            S^{-1}M \arrow[r, "\pi_{S^{-1}N}"']       & S^{-1}(M/N)\cong (S^{-1}M)/(S^{-1}N)
            \end{tikzcd}
    \]
\end{proposition}

\begin{prf}
    Since localization is exact, from the exact sequence
    \[
        0\longrightarrow N\longrightarrow M\longrightarrow M/N\longrightarrow 0,
    \]
    we obtain the following exact sequence
    \[
        0\longrightarrow S^{-1}N\longrightarrow S^{-1}M\longrightarrow S^{-1}(M/N)\longrightarrow 0.
    \]
\end{prf}

\begin{proposition}{Localization as Colimit}{}
    Let $R$ be a commutative ring, $S$ be a multiplicative set in $R$, and $M$ be an $R$-module. Then we have an isomorphism
    \[
        S^{-1}M\cong \varinjlim_{f\in S}M_f,
    \]
    where $M_f$ is the localization of $M$ by the multiplicative set $\langle f\rangle=\{f^n\mid n\in \mathbb{Z}_{\ge0}\}$. 
    
    Formally, $S$ can be endowed with a preorder relation: $f\mid g$ if and only if $fh=g$ for some $h\in S$, which makes $S$ a \hyperref[thin_category]{thin category} $\mathsf{S}$. Then we can define a functor $M_{\text{\textbf{-}}}:\mathsf{S}\to R\text{-}\mathsf{Mod}$
    \[
        \begin{tikzcd}[ampersand replacement=\&]
            \mathsf{S}\&[-25pt]\&[+10pt]\&[-30pt] R\text{-}\mathsf{Mod}\&[-30pt]\&[-30pt] \\ [-15pt] 
            f  \arrow[dd, ""{name=L, left}] 
            \&[-25pt] \& [+10pt] 
            \& [-30pt]M_f\arrow[dd, "l_g'"{name=R}] \&[-30pt]\ni
            \&[-30pt]\frac{m}{f^n}\arrow[dd,mapsto]\&[-30pt]\\ [-10pt] 
            \&  \phantom{.}\arrow[r, "G", squigarrow]\&\phantom{.}  \&   \\[-10pt] 
            g \&\hspace{-3pt}=fh \& \&  M_g\&[-30pt]\ni
            \&[-30pt]\frac{mh^n}{g^n}
        \end{tikzcd}
        \]  
        where $l_g'$ is given by the following  universal property
        \begin{center}
            \begin{tikzcd}[ampersand replacement=\&]
             
                M_f\arrow[rr, "l_g'", dashed]\&\& M_g \&  \\             
                \&M \arrow[ru, "l_g"'] \arrow[lu, "l_f"] \&                          
            \end{tikzcd}
        \end{center}
        And we have
        \[
            S^{-1}M\cong \varinjlim M_{\text{\textbf{-}}}
        \]
\end{proposition}

\begin{prf}
First let's show that the $l'_g$ induced by universal property can be writen as $l'_g:\frac{m}{f^n}\mapsto\frac{mh^n}{g^n}$. Suppose $R$ acts on $M_g$ through
\[
    \sigma_{M_g}:R\xrightarrow{}S^{-1}_gR\xrightarrow{\sigma_{M_g'}}\mathrm{End}_{\mathsf{Ab}}(M_g),
\]
Then we can check for any $f^n \in S_f$,
\[
    \sigma_{M_g}(f^n)\sigma_{M_g'}\left(\frac{h^n}{g^n}\right)=\sigma_{M_g'}\left(\frac{f^nh^n}{g^n}\right)=\sigma_{M_g'}\left(1\right)=1\implies \sigma_{M_g}(f^n)\in \mathrm{Aut}_{\mathsf{Ab}}(M_g),
\]
which means $\sigma_{M_g}(S_f)\subseteq  \mathrm{Aut}_{\mathsf{Ab}}(M_g)$. Thus by universal property of $M_f$, we have
\[
    l'_g\left(\frac{m}{f^n}\right)=\sigma_{M_g}(f^n)^{-1}(m)=\sigma_{M_g'}\left(\frac{h^n}{g^n}\right)(m)=\frac{mh^n}{g^n}.
\]
In a similar way, we can check that $S_f$ can act on $S^{-1}M$ as automorphisms and induce $\psi_f$ by the following universal property
\begin{center}
    \begin{tikzcd}[ampersand replacement=\&]
     
        M_f\arrow[rr, "\psi_f", dashed]\&\& S^{-1}M\&  \\             
        \&M \arrow[ru, "l_S"'] \arrow[lu, "l_f"] \&                          
    \end{tikzcd}
\end{center}
And we are going to show that $\left(\psi_f:M_f\to S^{-1}M\right)_{f\in S}$ is the colimit of $G$. 
\[
\begin{tikzcd}[ampersand replacement=\&]
        \& N                                 \&                                                             \\[+15pt]
        \& S^{-1}M \arrow[u, "\nu"', dashed] \&                                                             \\[+10pt]
M_f \arrow[rr, "l_g'"] \arrow[ru, "\psi_f"] \arrow[ruu, "\mu_f", bend left] \&                                   \& M_g \arrow[lu, "\psi_g"'] \arrow[luu, "\mu_g"', bend right]
\end{tikzcd}
\]
We can prove
\[
   \psi_f=\psi_g\circ l_g'
\]
by checking
\[
    \left(\psi_g\circ l_g'\right)\circ l_f=\psi_g\circ l_g =l_S=\psi_f\circ l_f
\]
and utilizing the uniqueness of the universal property. \\
Given any $\left(\mu_f:M_f\to S^{-1}M\right)_{f\in S}$ such that $\mu_f=\mu_g\circ l_g'$, note that $\mu_f\circ l_f=\mu_f\circ \mu_g\circ l_g'=\mu_g\circ l_g$. Thus we can define $\nu$ to be the unique map such that $\nu\circ l_S=\mu_f\circ l_f$.
\[
\begin{tikzcd}[ampersand replacement=\&]
    \& N                                                  \&                                    \\[+12pt]
M_f \arrow[rr, "\psi_f"] \arrow[ru, "\mu_f"] \&                                                    \& S^{-1}M \arrow[lu, "\nu"', dashed] \\[+12pt]
    \& M \arrow[lu, "l_f"] \arrow[ru, "l_S"'] \&                                   
\end{tikzcd}
\]
Hence we have 
\[
\left(\nu\circ \psi_f\right)\circ l_f=\nu\circ l_S =\mu_f\circ l_f.
\]
By the uniqueness of the universal property of $M_f$, we have $\mu_f=\nu\circ \psi_f$. If there exists another $\nu'$ such that $\mu_f=\nu'\circ \psi_f$, there must be $\nu'\circ l_S=\nu'\circ \psi_f \circ l_f=\mu_f \circ l_f=\nu\circ l_S$. The uniqueness of such $\nu$ forces $\nu=\nu'$.\\
Therefore we show that $ S^{-1}M\cong \varinjlim_{f\in S}M_f$.
\end{prf}

\begin{proposition}{}{associative_localization}
    Suppose $R$ is a commutative ring, $S,S'$ are multiplicative sets in $R$, and $M$ is an $R$-module. View $S^{\prime-1} M$ as an $R$-module, then $S^{-1}\left(S^{\prime-1} M\right)$ is isomorphic to $\left(S S^{\prime}\right)^{-1} M$ as $R$-modules.
\end{proposition}

\begin{prf}
    Define 
    $$
    \begin{aligned}
     f: S^{-1}\left(S^{\prime-1} M\right) &\longrightarrow\left(S S^{\prime}\right)^{-1} M\\
     \quad \frac{x / s^{\prime}}{s} &\longmapsto\frac{x}{ss'} 
    \end{aligned}
    $$
    To show that $f$ is well-defined, suppose that $\frac{x/s^{\prime}}{s}=\frac{y/t^{\prime}}{t}$, which means there exists $v\in S$ such that 
    \[
        v\left(t\frac{x}{s'} - s\frac{y}{t'}\right)=\frac{vtx}{s'}-\frac{vsy}{t'}=0.
    \]
    This further implies that there exists $w\in S'$ such that $w(vtt'x-vss'y)=0$.
     Then we see there exists $vw\in S S^{\prime}$ such that $vw(tt'x-ss'y)=0$, which means $\frac{x}{ss'}=\frac{y}{tt'}$. Thus $f$ is well-defined.

    Define
     \[
    \begin{aligned}
     g:\left(S S^{\prime}\right)^{-1} M& \longrightarrow S^{-1}\left(S^{\prime-1} M\right)\\
      \frac{x}{ss'} &\longmapsto \frac{x / s^{\prime}}{s} \text { for some } s \in S, s^{\prime} \in S^{\prime}
    \end{aligned}
    \]
    and we can check that $g$ is well-defined in a similar way. It is clear that $f$ and $g$ are linear maps inverse to each other.
\end{prf}



\begin{proposition}{}{module_localization_glueing_property}
    Let $R$ be a commutative ring and $M$ be a $R$-module. Suppose $x\in M$. Then the following are equivalent:
    \begin{enumerate}[(i)]
        \item $x=0$.
        \item $x$ maps to $0$ in $M_{\mathfrak{p}}$ for all $\mathfrak{p}\in \spec{R}$. 
        \item $x$ maps to $0$ in $M_{\mathfrak{m}}$ for all $\mathfrak{m}\in \mathrm{Max}\left(R\right)$. 
    \end{enumerate}
    As a consequence, $M\to \prod\limits_{\mathfrak{p}\in \spec{R}}M_{\mathfrak{p}}$ is an injective ring homomorphism.
\end{proposition}

\begin{prf}
    (i)$\implies$(ii) and (ii)$\implies$(iii) are clear. It is left to show (iii)$\implies$(i). Let $x\in M$ and 
    \[
        \mathrm{Ann}_M\left(x\right)=\left\{r\in R\midv rx=0\right\} 
    \] 
    be the annihilator of $x$ in $R$, which is an ideal of $R$. Note $\frac{x}{1}=\frac{0}{1}$ in $M_{\mathfrak{m}}$ if and only if 
    \[
        \mathrm{Ann}_M -\mathfrak{m}=\left\{r\in R-\mathfrak{m}\midv rx=0\right\}\neq \varnothing\iff \mathrm{Ann}_M \subsetneq \mathfrak{m}.
    \]
    If $x$ maps to $0$ in $M_{\mathfrak{m}}$ for all $\mathfrak{m}\in \mathrm{Max}\left(R\right)$, then $\mathrm{Ann}_M$ is not contained in any maximal ideal of $R$, which means $\mathrm{Ann}_M(x)=R$. Hence $x=0$.
\end{prf}

\begin{corollary}{}{module_localization_glueing_property_cor}
    Given an $R$-module $M$, the following are equivalent:
    \begin{enumerate}[(i)]
        \item $M$ is zero,
        \item $M_{\mathfrak{p}}$ is zero for all $\mathfrak{p} \in \operatorname{Spec}(R)$,
        \item $M_{\mathfrak{m}}$ is zero for all $\mathfrak{m}\in \mathrm{Max}\left(R\right)$.
    \end{enumerate} 
\end{corollary}

\begin{prf}
    (iii)$\implies$(i). Suppose $M_{\mathfrak{m}}$ is zero for all $\mathfrak{m}\in \mathrm{Max}\left(R\right)$. Given any $x\in M$, since $x$ maps to $0$ in $M_{\mathfrak{m}}$ for all $\mathfrak{m}\in \mathrm{Max}\left(R\right)$, there must be $x=0$ by \Cref{th:module_localization_glueing_property}. Thus $M$ is zero.\\
\end{prf}

\begin{corollary}{Exactness is a Local Property}{exactness_is_a_local_property}
    Given a sequence of $R$-modules $M\to M'\to M''$, the following are equivalent:
    \begin{enumerate}[(i)]
        \item $M\to M'\to M''$ is exact,
        \item $M_{\mathfrak{p}}\to M'_{\mathfrak{p}}\to M''_{\mathfrak{p}}$ is exact for all $\mathfrak{p} \in \operatorname{Spec}(R)$,
        \item $M_{\mathfrak{m}}\to M'_{\mathfrak{m}}\to M''_{\mathfrak{m}}$ is exact for all $\mathfrak{m}\in \mathrm{Max}\left(R\right)$.
    \end{enumerate}
\end{corollary}

\begin{prf}
    (i)$\implies$(ii) because localization is an exact functor. (ii)$\implies$(iii) is clear. It is left to show (iii)$\implies$(i). Let $H=\ker\left(M'\to M''\right)/\operatorname{im}\left(M\to M'\right)$. Since \hyperref[th:exact_functor_preserve_cohomology]{exact functor preserve cohomology}, we have 
    $$
    H_{\mathfrak{m}}\cong\ker \left(M'_{\mathfrak{m}}\to M''_{\mathfrak{m}}\right)/\operatorname{im}\left(M_{\mathfrak{m}}\to M'_{\mathfrak{m}}\right)=0
    $$ 
    for all $\mathfrak{m}\in \mathrm{Max}\left(R\right)$. Thus by \Cref{th:module_localization_glueing_property_cor} we have $H=0$, which implies $M\to M'\to M''$ is exact.
\end{prf}


\begin{proposition}{Glueing Functions}{}
    Let $R$ be a ring. Let $f_1, \ldots, f_n$ be elements of $R$ generating the unit ideal. Let $M$ be an $R$-module. The sequence
$$
\begin{aligned}
& 0 \longrightarrow M \xlongrightarrow{\alpha} \bigoplus_{i=1}^n M_{f_i} \xlongrightarrow{\beta} \bigoplus_{i, j=1}^n M_{f_i f_j} \\
& 
\end{aligned}
$$
is exact, where $\alpha(m)=\left(\frac{m}{1}, \cdots, \frac{m}{1}\right)$ and $\beta\left(\dfrac{m_1}{f_1^{e_1}}, \cdots, \dfrac{m_n}{f_n^{e_n}}\right)=\left(\dfrac{m_i}{f_i^{e_i}}-\dfrac{m_j}{f_j^{e_j}}\right)_{(i, j)}$.
\end{proposition}

\begin{prf}
 According to \Cref{th:exactness_is_a_local_property}, it suffices to show that the localization of the sequence at any maximal ideal $\mathfrak{m}$ is exact. Given any maximal ideal $\mathfrak{m}$ of $R$, since $f_1, \ldots, f_n$ generate the unit ideal, there is an $i$ such that $f_i \notin \mathfrak{m}$. Without loss of generality we may assume $f_1\notin \mathfrak{m}$. Note that \Cref{th:associative_localization} guarantees $\left(M_{f_i}\right)_{\mathfrak{m}}=\left(M_{\mathfrak{m}}\right)_{f_i}$ and $\left(M_{f_i f_j}\right)_{\mathfrak{m}}=\left(M_{\mathfrak{m}}\right)_{f_i f_{j}}$. In particular we have $\left(M_{f_1}\right)_{\mathfrak{m}}=M_{\mathfrak{m}}$ and $\left(M_{f_1 f_i}\right)_{\mathfrak{m}}=\left(M_{\mathfrak{m}}\right)_{f_i}$, because $f_1\in M_{\mathfrak{m}}^{\times}$. Thus it is suffices to show that the sequence
\[
    0 \longrightarrow M_{\mathfrak{m}} \xlongrightarrow{\alpha_{\mathfrak{m}}} \bigoplus_{i=1}^n \left(M_{\mathfrak{m}}\right)_{f_i} \xlongrightarrow{\beta_{\mathfrak{m}}} \bigoplus_{i, j=1}^n \left(M_{\mathfrak{m}}\right)_{f_i f_j}
\]
 is exact for $f_1=1$.

 Injectivity of $\alpha_{\mathfrak{m}}$ is trivial because the first component of $\alpha_{\mathfrak{m}}$ is the identity map on $M_{\mathfrak{m}}$. 

 For any $\mathbf{x}=\left(x_1, \dfrac{x_2}{f_2^{e_n}},\cdots, \dfrac{x_n}{f_n^{e_n}}\right)\in\ker \beta_{\mathfrak{m}}$ we have $\beta_{\mathfrak{m}}(\mathbf{x})=0$. Consider the $(1,i)$-component of $\beta_{\mathfrak{m}}\left(\mathbf{x}\right)$ for $i=2, \cdots, n$. Then we get
 \[
    x_1-\frac{x_i}{f_i^{e_i}}=0\implies \mathbf{x}=\left(x_1,x_1,\cdots,x_1\right)=\alpha_{\mathfrak{m}}\left(x_1\right)\implies \ker \beta_{\mathfrak{m}}\subseteq \operatorname{im}\alpha_{\mathfrak{m}}.
 \]
For any $\mathbf{y}=\left(y, y, \cdots, y\right)\in \operatorname{im}\alpha_{\mathfrak{m}}$, we have $\beta_{\mathfrak{m}}\left(\mathbf{y}\right)=0$, which means $\operatorname{im}\alpha_{\mathfrak{m}}\subseteq \ker \beta_{\mathfrak{m}}$. Thus the sequence is exact and we complete the proof.
\end{prf}


\subsection{Graded Object}

\begin{definition}{$I$-Graded Module (External Definition)}{}    
    Let $R$ be a ring and $I$ be a set. An \textbf{$I$-graded $R$-module} is a family of $R$-modules $\left(M_i\right)_{i\in I}$. The category of $I$-graded $R$-modules, denoted by $R\text{-}\mathsf{Mod}^I$, is simply the functor category $[I, R\text{-}\mathsf{Mod}]$, where $I$ is regarded as a discrete category.
\end{definition}

\begin{definition}{$I$-Graded Module (Internal Definition)}{graded_module_over_ring_internal}
    Let $R$ be a ring and $I$ be a set. An \textbf{$I$-graded $R$-module} is an $R$-module $M$ together with a family of submodules $\left(M_i\right)_{i\in I}$ such that
    \[
        M=\bigoplus_{i\in I}M_i.
    \]
\end{definition}

\begin{proposition}{$R\text{-}\mathsf{Mod}^I$ is a Monoidal Category}{}
    Let $R$ be a ring and $I$ be a commutative monoid. Then $\left(R\text{-}\mathsf{Mod}^I,\otimes\right)$ is a monoidal category with tensor product defined as
    \[
        \left(M\otimes N\right)_i=\bigoplus_{j+k=i}M_j\otimes N_k.
    \]
\end{proposition}



\begin{definition}{$I$-Graded Module over an Graded Ring (Internal Definition)}{graded_module_over_graded_ring_internal}
    Let $(I,+)$ be a monoid and $R$ be a $I$-graded ring with grading $(R_i)_{i\in I}$. An \textbf{$I$-graded module over graded ring $R$} is an $R$-module $M$ together with a family of submodules $\left(M_i\right)_{i\in I}$ such that
    \begin{enumerate}[(i)]
        \item $M=\bigoplus_{i\in I}M_i$.
        \item $R_iM_j\subseteq M_{i+j}$ for all $i, j\in I$.
    \end{enumerate}
\end{definition}

When $I$ is a monoid, \Cref{th:graded_module_over_ring_internal} is a special case of \Cref{th:graded_module_over_graded_ring_internal} because any ring $R$ can be regarded as a graded ring with trivial grading $R_0=R$ and $R_i=0$ for all $i\ne0$.


