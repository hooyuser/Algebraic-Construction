
\chapter{Associative Algebra}

\section{Basic Properties}
\begin{definition}{Associative Algebra over Commutative Ring}{}
    Let $R$ be a commutative ring. An \textbf{associative $R$-algebra} is a ring $A$ together with a ring homomorphism $\sigma:R\to Z(A)$, which makes $A$ an $R$-module by defining 
    $$
    r\cdot a=\sigma(r)a
    $$
    for all $r\in R$ and $a\in A$. \\
    We can check that 
\[
  r\cdot (ab)=\sigma(r)ab=(r\cdot a)b=\sigma(r)ab=a\left(\sigma(r)b\right) =a(r\cdot b),
\]
which justifies the naming ``associative". 
\end{definition}

We usually call associative $R$-algebra as $R$-algebra for short.

\begin{proposition}{Commutative Ring homomorphism $R\to S$ induces functor $S\text{-}\mathsf{Alg}\to R\text{-}\mathsf{Alg}$}{}
    Let $R$ and $S$ be commutative rings with a ring homomorphism $f: R\to S$. Then every $S$-algebra $A$ is an $R$-algebra by defining $ra = f(r)a$, or equivalently through $R\to S\to Z(A)$. This defines a functor $F: S\text{-}\mathsf{Alg}\to R\text{-}\mathsf{Alg}$, which is identify map on objects and morphisms.
    \[
        \begin{tikzcd}[ampersand replacement=\&]
            S\text{-}\mathsf{Alg}\&[-25pt]\&[+10pt]\&[-30pt] R\text{-}\mathsf{Alg}\&[-30pt]\&[-30pt] \\ [-15pt] 
            A  \arrow[dd, "g"{name=L, left}] 
            \&[-25pt] \& [+10pt] 
            \& [-30pt] A\arrow[dd, "g"{name=R}] \&[-30pt]\\ [-10pt] 
            \&  \phantom{.}\arrow[r, "F", squigarrow]\&\phantom{.}  \&   \\[-10pt] 
            B \& \& \&  B\&
        \end{tikzcd}
        \]  
\end{proposition}

In particular, commutative ring homomorphism $R\to S$ makes $S$ an $R$-algebra.



\section{Construction}
\subsection{Quotient Object}

\begin{definition}{Quotient Algebra}{}
    Let $A$ be an $R$-algebra and $\mathfrak{a}$ be a two-sided ideal of $A$. Since $\mathfrak{a}$ is an $R$-submodule of $A$, the quotient ring $A/\mathfrak{a}$ can also be endowed with an $R$-module structure, which makes $A/\mathfrak{a}$ an $R$-algebra. We call $A/\mathfrak{a}$ the \textbf{quotient algebra} of $A$ by $\mathfrak{a}$.
\end{definition}


\subsection{Free Object}
\begin{definition}{Free $R$-Algebra}{}
    Let $X$ be a set and $R$ be a commutative ring. The \textbf{free $R$-algebra} on $X$, denoted by $\mathrm{Free}_{R\text{-}\mathsf{Alg}}(X)$, together with a function $\iota:X\to \mathrm{Free}_{R\text{-}\mathsf{Alg}}(X)$, is defined by the following universal property: for any $R$-algebra $A$ and any function $f:X\to A$, there exists a unique $R$-algebra homomorphism $\widetilde{f}:\mathrm{Free}_{R\text{-}\mathsf{Alg}}(X)\to A$ such that the following diagram commutes
    \begin{center}
        \begin{tikzcd}[ampersand replacement=\&]
            \mathrm{Free}_{R\text{-}\mathsf{Alg}}(X)\arrow[r, dashed, "\exists !\,\widetilde{f}"]  \& A \\[0.3cm]
            X\arrow[u, "\iota"] \arrow[ru, "f"'] \&  
        \end{tikzcd}
    \end{center}
    The free $R$-algebra $\mathrm{Free}_{R\text{-}\mathsf{Alg}}(X)$ can be contructed by direct sum of copies of $R$
    \[
        \mathrm{Free}_{R\text{-}\mathsf{Alg}}(X)\cong\bigoplus_{w\in\mathrm{Free}_{\mathsf{Mon}}(X)}Rw.  
    \]
\end{definition}


\subsection{Graded Object}

\begin{definition}{$I$-Graded Algebra over an Graded Commutative Ring}{}
    Let $(I,+)$ be a monoid and $R$ be a $I$-graded commutative ring with grading $(R_i)_{i\in I}$. An \textbf{$I$-graded algebra over graded ring $R$} is an $R$-algebra $A$ together with a family of subalgebras $\left(A_i\right)_{i\in I}$ such that
    \begin{enumerate}[(i)]
        \item $A=\bigoplus_{i\in I}A_i$.
        \item $A_iA_j\subseteq A_{i+j}$ for all $i, j\in I$.
        \item $R_iA_j\subseteq A_{i+j}$ for all $i, j\in I$.
    \end{enumerate}
    Elements in $A_i$ are called \textbf{homogeneous elements of degree $i$}.
\end{definition}


\begin{proposition}{Graded Algebra Quotients out Graded Ideal}{}
    Let $A$ be an $I$-graded algebra over graded ring $R$ with grading $(A_i)_{i\in I}$ and $\mathfrak{a}$ be a graded two-sided ideal of $A$. Then $A/\mathfrak{a}$ has decomposition
    \[
    A/ \mathfrak{a}=\bigoplus_{i\in I}A_i/\left(\mathfrak{a}\cap A_i\right),    
    \]
    which makes $A/\mathfrak{a}$ an $I$-graded algebra.
\end{proposition}



\begin{example}{Polynomial Algebra $R[X_1,\cdots,X_n]$}{}
    Let $R$ be a commutative ring and $X_1,\cdots,X_n$ be indeterminates. Then $R[X_1,\cdots,X_n]$ is an $\mathbb{N}$-graded $R$-algebra with grading $R[X_1,\cdots,X_n]_i$ being the set of homogeneous polynomials of degree $i$.
\end{example}



\section{Tensor Algebra}
\begin{definition}{Tensor Algebra $T^{\bullet}(M)$}{}
    Given a $R$-module $M$, the \textbf{$k$-th tensor power of $M$} is defined as
    \begin{align*}
        T^k(M)&=M^{\otimes k}=\underbrace{M\otimes_R\cdots\otimes_R M}_{k\text{ times}},\\
        T^0(M)&=R.
    \end{align*}
    The \textbf{tensor algebra} of $M$ is defined as
    \[
        T^{\bullet}(M)=\bigoplus_{k=0}^{\infty}T^k(M)
    \]
    with multiplication $\otimes$ defined as
    \[
        (m_1\otimes\cdots\otimes m_k)\otimes(m_{k+1}\otimes\cdots\otimes m_{k+l})=m_1\otimes\cdots\otimes m_{k+l}
    \]
    $T^{\bullet}(M)$ is an $\mathbb{N}$-graded $R$-algebra with grading $(T^k(M))_{k\ge 0}$.
\end{definition}


\begin{proposition}{Tensor Algebra Functor $T^{\bullet}:R\text{-}\mathsf{Mod}\to R\text{-}\mathsf{Alg}$}{}
    Let $R$ be a commutative ring. The tensor algebra construction $T^{\bullet}:R\text{-}\mathsf{Mod}\to R\text{-}\mathsf{Alg}$ is a functor. 
    \[
        \begin{tikzcd}[ampersand replacement=\&]
            R\text{-}\mathsf{Mod}\&[-25pt]\&[+10pt]\&[-30pt] R\text{-}\mathsf{Alg}\&[-30pt]\&[-30pt] \\ [-15pt] 
            M  \arrow[dd, "g"{name=L, left}] 
            \&[-25pt] \& [+10pt] 
            \& [-30pt] T^{\bullet}(M)\arrow[dd, "T^{\bullet}(g)"{name=R}] \&[-20pt]\ni\& [+10pt]m_1\otimes\cdots\otimes m_k \arrow[dd, mapsto, "g\otimes g\cdots\otimes g"{name=L, right}] 
            \\ [-10pt] 
            \&  \phantom{.}\arrow[r, "T^{\bullet}", squigarrow]\&\phantom{.}  \&   \\[-10pt] 
            N \& \& \&  T^{\bullet}(N)\&[-0pt]\ni\& g(m_1)\otimes\cdots\otimes g(m_k)
        \end{tikzcd}
        \]  
\end{proposition}


\begin{proposition}{Adjunction $T^{\bullet}\dashv U_{\mathsf{R\text{-}\mathsf{Mod}}}$}{}
    Let $R$ be a commutative ring. Suppose $U:R\text{-}\mathsf{Alg}\to R\text{-}\mathsf{Mod}$ is the forgetful functor. Then the tensor algebra functor $T^{\bullet}:R\text{-}\mathsf{Mod}\to R\text{-}\mathsf{Alg}$ is left adjoint to $U$.
\end{proposition}


\section{Exterior Algebra and Symmetric Algebra}
\begin{definition}{Exterior Algebra $\Largewedge^{\bullet} (M)$}{}
    Given an $R$-module $M$, 
    \begin{align*}
        I_{\largewedge}(M) & :=\langle x \otimes x: x \in M\rangle
    \end{align*}
    is a graded two-sided ideal of $T^{\bullet}(M)$. The \textbf{exterior algebra} of $M$ is defined as
    \[
        \largewedge^{\bullet} (M)=T^{\bullet}(M)/I_{\largewedge}(M).
    \]
    The multiplication of $\Largewedge^{\bullet} (M)$ is denoted by $\wedge$ and is called the \textbf{wedge product}.
    The grading of $\Largewedge^{\bullet} (M)$ is given by 
    \begin{align*}
        \largewedge^{\bullet} (M)=\bigoplus_{k=0}^{\infty}\largewedge^k(M),
    \end{align*}
    where 
    \[
        \largewedge^k(M)=T^k(M)/\left(I_{\largewedge}(M)\cap T^k(M)\right)
    \] 
    is called the \textbf{$k$-th exterior power of $M$}.
\end{definition}


\begin{definition}{Exterior Algebra Functor: $\Largewedge^{\bullet}:R\text{-}\mathsf{Mod}\to R\text{-}\mathsf{AcAlg_{\mathbb{Z}}}$}{}
    The exterior algebra construction $\Largewedge^{\bullet}:R\text{-}\mathsf{Mod}\to R\text{-}\mathsf{Alg}$ is a functor defined as follows
    \[
        \begin{tikzcd}[ampersand replacement=\&]
            R\text{-}\mathsf{Mod}\&[-25pt]\&[+10pt]\&[-30pt] R\text{-}\mathsf{AcAlg_{\mathbb{Z}}}\&[-30pt]\&[-30pt] \\ [-15pt] 
            M  \arrow[dd, "g"{name=L, left}] 
            \&[-25pt] \& [+10pt] 
            \& [-30pt] \Largewedge^{\bullet}(M)\arrow[dd, "\Largewedge^{\bullet}(g)"{name=R}] \&[-20pt]\ni\& [+20pt]m_1\wedge\cdots\wedge m_k \arrow[dd, mapsto, "g\wedge g\cdots\wedge g"{name=L, right}] 
            \\ [-10pt] 
            \&  \phantom{.}\arrow[r, "\Largewedge^{\bullet}", squigarrow]\&\phantom{.}  \&   \\[-10pt] 
            N \& \& \&  \Largewedge^{\bullet}(N)\&[-0pt]\ni\& g(m_1)\wedge\cdots\wedge g(m_k)
        \end{tikzcd}
    \]  
\end{definition}


\begin{proposition}{Adjunction $\Largewedge^{\bullet}\dashv U_{\mathsf{R\text{-}\mathsf{Mod}}}$}{}
    Let $R$ be a commutative ring. Suppose $U:R\text{-}\mathsf{AcAlg_{\mathbb{Z}}}\to R\text{-}\mathsf{Mod}$ is the forgetful functor. Then the exterior algebra functor $\Largewedge^{\bullet}:R\text{-}\mathsf{Mod}\to R\text{-}\mathsf{AcAlg_{\mathbb{Z}}}$ is left adjoint to $U$.
\end{proposition}


\begin{proposition}{}{}
    Suppose $R$ is a commutative ring and $M=\bigoplus_{x\in X}Rx$ is a free $R$-module. Then
    \begin{enumerate}[(i)]
        \item $\Largewedge^{\bullet}(M)$ has a basis $\{x_1\wedge\cdots\wedge x_k: x_1,\cdots,x_k\in X, x_i\ne x_j\text{ for all }i\ne j\}$.
        \item If $M$ has a basis $\{x_1,\cdots,x_n\}$, then we have an $R$-linear isomorphism
        \begin{align*}
            \largewedge^{n}(M)&\xlongrightarrow{\sim} R\\
            x_1 \wedge \cdots \wedge x_n& \longmapsto 1_R.
        \end{align*}
        Moreover, we have $\Largewedge^{m}(M)=0$ for all $m>n$.
    \end{enumerate}
\end{proposition}


\section{Determinant and Trace}



\begin{definition}{Determinant of $R$-linear Transformation on Free Module of Finite Rank}{}
    Let $R$ be a commutative ring and $M$ be a free $R$-module of finite rank $n$. The \textbf{determinant} of an $R$-linear map $f:M\to M$ can be defined as one of the following equivalent ways:
    \begin{enumerate}[(i)]
        \item The functor $\Largewedge^n:R\text{-}\mathsf{Mod}\to R\text{-}\mathsf{Mod}$ induces a hom-set morphism 
        \begin{align*}
            \largewedge^n:\mathrm{End}_{R\text{-}\mathsf{Mod}}\left(M\right) &\longrightarrow\mathrm{End}_{R\text{-}\mathsf{Mod}}\left(\largewedge^n(M)\right)\\
        f &\longmapsto f\wedge\cdots\wedge f.
        \end{align*}
        Note $\mathrm{End}_{R\text{-}\mathsf{Mod}}\left(\largewedge^n(M)\right)$ is a 1-dimensional $R$-module with basis $\mathrm{id}_{\largewedge^n(M)}$. And $R$ is a also a 1-dimensional $R$-module with basis $1_R$. We have the following $R$-module isomorphism
        \begin{align*}
            s:\mathrm{End}_{R\text{-}\mathsf{Mod}}\left(\largewedge^n(M)\right)&\xlongrightarrow{\sim} R\\
            \mathrm{id}_{\largewedge^n(M)} &\longmapsto 1_R.
        \end{align*}
        The \textbf{determinant} of $f$ is defined as the composition $\det:=s\circ \largewedge^n$.
        \item Suppose $\{e_1,\cdots,e_n\}$ is a baisis of $M$, then for any $f\in \mathrm{End}_{R\text{-}\mathsf{Mod}}\left(M\right)$, it can be uniquely represented by a matrix $A=(a_{ij})$ with respect to the basis $\{e_1,\cdots,e_n\}$. Then the \textbf{determinant} of $f$ is defined as follows
        \begin{align*}
            \largewedge^n:\mathrm{End}_{R\text{-}\mathsf{Mod}}\left(M\right) &\longrightarrow R\\
        f &\longmapsto f\wedge\cdots\wedge f.
        \end{align*}
    \end{enumerate}
\end{definition}

\begin{example}{Left Multiplication Endomorphism}{}
    Let $R$ be a commutative ring and $A$ be an $R$-Algebra. Suppose $A$ is a free $R$-module of finite rank $n$. Given any $a\in A$, the left multiplication endomorphism $l_a\in\mathrm{End}_{R\text{-}\mathsf{Mod}}(A)$ is defined by
    \begin{align*}
        l_a:A &\longrightarrow A\\
        x &\longmapsto ax.
    \end{align*}
\end{example}


\begin{definition}{Trace of Elements in $R$-algebra}{}
    Let $A$ be an $R$-algebra. The \textbf{trace} of an element $a\in A$ is defined as the trace of the left multiplication endomorphism $l_a\in\mathrm{End}_{R\text{-}\mathsf{Mod}}(A)$.
\end{definition}

\section{Algebra over Field}
\begin{lemma}{Nonzero Ring Homomorphism from Field is Injective}{nonzero_ring_homomorphism_from_field_is_injective}
    If $K$ is a field, $R$ is a ring, a ring homomorphism $f:K\to R$ is either injective or the zero map. Furthermore, If $R$ is not a zero ring, then $f$ is injective.
\end{lemma}
\begin{prf}
    Since the only ideals of $K$ are $\{0\}$ and $K$, the kernel of $f$ is either $\{0\}$ or $K$. If $\ker f=\{0\}$, then $f$ is injective. If $\ker f=K$, then $f$ is the zero map. By \Cref{th:kernel_of_ring_homomorphism_is_an_ideal}, if $R$ is not a zero ring, then $\ker f$ is not $K$, so $f$ is injective.
\end{prf}

\begin{corollary}{}{}
    If $K$ is a field and $A$ is a nonzero $K$-algebra, then the ring homomorphism $K\to Z(A)$ is injective.
\end{corollary}
\begin{prf}
    This is a direct consequence of \Cref{th:nonzero_ring_homomorphism_from_field_is_injective}.
\end{prf}


\begin{proposition}{}{}
    Let $K$ be a field, $A$ be a $K$-algebra and $a\in A$. Consider the evaluation ring homomorphism
    \begin{align*}
        \mathrm{ev}_a:K[X] &\longrightarrow A\\
        f &\longmapsto f(a).
    \end{align*}
    Since $K[X]$ is a PID, we can suppose $\ker \mathrm{ev}_a=(P_a)$ for some $P_a\in K[X]$. Since $\operatorname{im}\mathrm{ev}_a=K[a]$, we have the following isomorphism in $K$-$\mathsf{Alg}$
    \[
        K[a]\cong K[X]/(P_a(X)).
    \]
    And it can be divided into two cases:
    \begin{enumerate}[(i)]
        \item If $P_a=0$, then $\mathrm{ev}_a$ is injective and $K[a]\cong K[X]$.
        \item If $P_a\ne 0$, then $\mathrm{ev}_a$ is not injective. If we further assume $A$ is a domain, then $P_a(X)$ is irreducible, $K[a]$ is a field and
        \[
        \left[ K[a]:K \right]=\deg P_a(X).
        \]
    \end{enumerate}
\end{proposition}
\begin{prf}
    \begin{enumerate}[(i)]
        \item If $P_a=0$, then $\ker \mathrm{ev}_a=\{0\}$, so $\mathrm{ev}_a$ is injective. And $K[a]\cong K[X]$.
        \item If $A$ is a domain, then $K[a]$ as a subring of $A$ is an integral domain. This implies $(P_a(X))$ is a nonzero prime ideal of $K[X]$. By \Cref{th:nonzero_prime_ideal_iff_maximal_ideal_in_PID}, $P_a(X)$ is irreducible. Since $K[X]/(P_a(X))$ as $K$-vector space has a basis $\{1,X,X^2,\cdots,X^{\deg P_a(X)-1}\}$, we have $\left[ K[a]:K \right]=\deg P_a(X)$.
    \end{enumerate}
   
\end{prf}

\begin{definition}{Algebraic Element and Transcendental Element}{}
    Let $K$ be a field and $A$ be a $K$-algebra. Consider the evaluation ring homomorphism
    \begin{align*}
        \mathrm{ev}_a:K[X] &\longrightarrow A\\
        f &\longmapsto f(a).
    \end{align*}
    and $\ker \mathrm{ev}_a=(P_a)$ for some $P_a\in K[X]$.
    \begin{itemize}
        \item If $P_a=0$, then $a$ is called a \textbf{transcendental element} over $K$. $a$ is not the root of any nonzero polynomial in $K[X]$. 
        \item If $P_a\ne 0$, then $a$ is called an \textbf{algebraic element} over $K$. Suppose $P_a(X)=\sum_{i=0}^n a_iX^i$. Then $m_a(X)=P_a(X)/a_n$ is called the \textbf{minimal polynomial} of $a$ over $K$.
    \end{itemize}
\end{definition}




\section{Commutative Algebra}
\begin{definition}{Commutative Algebra}{}
    Let $R$ be a commutative ring. A \textbf{commutative $R$-algebra} is an $R$-algebra where the multiplication is commutative. Or equivalently, a commutative $R$-algebra is a commutative ring $A$ together with a ring homomorphism $R\to A$. Hence there is a category isomorphism $R\text{-}\mathsf{CAlg}\cong \left(R/\mathsf{CRing}\right)$.
\end{definition}

\subsection{Construction}

\begin{definition}{Free Commutative Algebra}{free_commutative_algebra}
    Let $X$ be a set and $R$ be a commutative ring. The \textbf{free commutative $R$-algebra} on $X$, denoted by $\mathrm{Free}_{R\text{-}\mathsf{CAlg}}(X)$, together with a map $\iota:X\to \mathrm{Free}_{R\text{-}\mathsf{CAlg}}(X)$, is defined by the following universal property: for any commutative $R$-algebra $A$ and any map $f:X\to A$, there exists a unique homomorphism $\widetilde{f}:\mathrm{Free}_{R\text{-}\mathsf{CAlg}}(X)\to A$ such that the following diagram commutes
    \begin{center}
        \begin{tikzcd}[ampersand replacement=\&]
            \mathrm{Free}_{R\text{-}\mathsf{CAlg}}(X)\arrow[r, dashed, "\exists !\,\widetilde{f}"]  \& A\\[0.3cm]
            X\arrow[u, "\iota"] \arrow[ru, "f"'] \&  
        \end{tikzcd}
    \end{center}
    The free commutative $R$-algebra $\mathrm{Free}_{R\text{-}\mathsf{CAlg}}(X)$ can be contructed as the polynomial algebra $R[X]$.

    And we can define a functor
    \[
        \begin{tikzcd}[ampersand replacement=\&]
            \mathsf{CRing}\&[-25pt]\&[+10pt]\&[-30pt] \mathsf{CRing}\&[-30pt]\&[-30pt] \\ [-15pt] 
            R  \arrow[dd, "\varphi"{name=L, left}] 
            \&[-25pt] \& [+10pt] 
            \& [-30pt] R[X]\arrow[dd, "{{}^{\varphi}\!(-)}"{name=R}] \&[-20pt]\ni\& [+10pt]f(X)=\sum\limits_{\beta} a_\beta x^\beta \arrow[dd, mapsto, ""{name=L, right}] 
            \\ [-10pt] 
            \&  \phantom{.}\arrow[r, "{\mathrm{Free}_{\bullet\text{-}\mathsf{CAlg}}(X)}", squigarrow]\&\phantom{.}  \&   \\[-10pt] 
            S\& \& \&  S[X]\&[-0pt]\ni\& ~^\varphi\!f(X)=\sum\limits_{\beta} \varphi(a_\beta) x^\beta
        \end{tikzcd}
        \]  
\end{definition}
\begin{prf}
    We can check that $\mathrm{Free}_{\bullet\text{-}\mathsf{CAlg}}(X)$ is a functor
    \[
        \leftindex^{\psi\circ \varphi}f(X)=\sum\limits_{\beta} (\psi\circ \varphi)(a_\beta) x^\beta=\sum\limits_{\beta} \psi(\varphi(a_\beta)) x^\beta=\leftindex^\psi(\leftindex^\varphi f)(X).
    \]
\end{prf}




