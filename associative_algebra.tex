
\chapter{Associative Algebra}

\section{Basic Properties}
\dfn{Associative Algebra over Commutative Ring}{
    Let $R$ be a commutative ring. An \textbf{associative $R$-algebra} is a ring $A$ together with a ring homomorphism $\sigma:R\to Z(A)$, which makes $A$ an $R$-module by defining 
    $$
    r\cdot a=\sigma(r)a
    $$
    for all $r\in R$ and $a\in A$. \\
    We can check that 
\[
  r\cdot (ab)=\sigma(r)ab=(r\cdot a)b=\sigma(r)ab=a\left(\sigma(r)b\right) =a(r\cdot b),
\]
which justifies the naming ``associative". 
}
We usually call associative $R$-algebra as $R$-algebra for short.

\prop{Commutative Ring homomorphism $R\to S$ induces functor $S\text{-}\mathsf{Alg}\to R\text{-}\mathsf{Alg}$}{
    Let $R$ and $S$ be commutative rings with a ring homomorphism $f: R\to S$. Then every $S$-algebra $A$ is an $R$-algebra by defining $ra = f(r)a$, or equivalently through $R\to S\to Z(A)$. This defines a functor $F: S\text{-}\mathsf{Alg}\to R\text{-}\mathsf{Alg}$, which is identify map on objects and morphisms.
    \[
        \begin{tikzcd}[ampersand replacement=\&]
            S\text{-}\mathsf{Alg}\&[-25pt]\&[+10pt]\&[-30pt] R\text{-}\mathsf{Alg}\&[-30pt]\&[-30pt] \\ [-15pt] 
            A  \arrow[dd, "g"{name=L, left}] 
            \&[-25pt] \& [+10pt] 
            \& [-30pt] A\arrow[dd, "g"{name=R}] \&[-30pt]\\ [-10pt] 
            \&  \phantom{.}\arrow[r, "F", squigarrow]\&\phantom{.}  \&   \\[-10pt] 
            B \& \& \&  B\&
        \end{tikzcd}
        \]  
}
In particular, commutative ring homomorphism $R\to S$ makes $S$ an $R$-algebra.



\section{Construction}
\subsection{Quotient Object}

\dfn{Quotient Algebra}{
    Let $A$ be an $R$-algebra and $\mathfrak{a}$ be a two-sided ideal of $A$. Since $\mathfrak{a}$ is an $R$-submodule of $A$, the quotient ring $A/\mathfrak{a}$ can also be endowed with an $R$-module structure, which makes $A/\mathfrak{a}$ an $R$-algebra. We call $A/\mathfrak{a}$ the \textbf{quotient algebra} of $A$ by $\mathfrak{a}$.
}

\subsection{Graded Object}

\dfn{$I$-Graded Algebra over an Graded Commutative Ring}{
    Let $(I,+)$ be a monoid and $R$ be a $I$-graded commutative ring with grading $(R_i)_{i\in I}$. An \textbf{$I$-graded algebra over graded ring $R$} is an $R$-algebra $A$ together with a family of subalgebras $\left(A_i\right)_{i\in I}$ such that
    \begin{enumerate}[(i)]
        \item $A=\bigoplus_{i\in I}A_i$.
        \item $A_iA_j\subseteq A_{i+j}$ for all $i, j\in I$.
        \item $R_iA_j\subseteq A_{i+j}$ for all $i, j\in I$.
    \end{enumerate}
    Elements in $A_i$ are called \textbf{homogeneous elements of degree $i$}.
}

\prop{Graded Algebra Quotients out Graded Ideal}{
    Let $A$ be an $I$-graded algebra over graded ring $R$ with grading $(A_i)_{i\in I}$ and $\mathfrak{a}$ be a graded two-sided ideal of $A$. Then $A/\mathfrak{a}$ has decomposition
    \[
    A/ \mathfrak{a}=\bigoplus_{i\in I}A_i/\left(\mathfrak{a}\cap A_i\right),    
    \]
    which makes $A/\mathfrak{a}$ an $I$-graded algebra.
}


\ex{Polynomial Algebra $R[X_1,\cdots,X_n]$}{
    Let $R$ be a commutative ring and $X_1,\cdots,X_n$ be indeterminates. Then $R[X_1,\cdots,X_n]$ is an $\mathbb{N}$-graded $R$-algebra with grading $R[X_1,\cdots,X_n]_i$ being the set of homogeneous polynomials of degree $i$.
}


\section{Tensor Algebra}
\dfn{Tensor Algebra $T^{\bullet}(M)$}{
    Given a $R$-module $M$, the \textbf{$k$-th tensor power of $M$} is defined as
    \begin{align*}
        T^k(M)&=M^{\otimes k}=\underbrace{M\otimes_R\cdots\otimes_R M}_{k\text{ times}},\\
        T^0(M)&=R.
    \end{align*}
    The \textbf{tensor algebra} of $M$ is defined as
    \[
        T^{\bullet}(M)=\bigoplus_{k=0}^{\infty}T^k(M)
    \]
    with multiplication $\otimes$ defined as
    \[
        (m_1\otimes\cdots\otimes m_k)\otimes(m_{k+1}\otimes\cdots\otimes m_{k+l})=m_1\otimes\cdots\otimes m_{k+l}
    \]
    $T^{\bullet}(M)$ is an $\mathbb{N}$-graded $R$-algebra with grading $(T^k(M))_{k\ge 0}$.
}

\section{Exterior Algebra and Symmetric Algebra}
\dfn{Exterior Algebra $\Largewedge^{\bullet} (M)$}{
    Given an $R$-module $M$, 
    \begin{align*}
        I_{\largewedge}(M) & :=\langle x \otimes x: x \in M\rangle
    \end{align*}
    is a graded two-sided ideal of $T^{\bullet}(M)$. The \textbf{exterior algebra} of $M$ is defined as
    \[
        \largewedge^{\bullet} (M)=T^{\bullet}(M)/I_{\largewedge}(M).
    \]
    The multiplication of $\Largewedge^{\bullet} (M)$ is denoted by $\wedge$ and is called the \textbf{wedge product}.
    The grading of $\Largewedge^{\bullet} (M)$ is given by 
    \begin{align*}
        \largewedge^{\bullet} (M)=\bigoplus_{k=0}^{\infty}\largewedge^k(M),
    \end{align*}
    where 
    \[
        \largewedge^k(M)=T^k(M)/\left(I_{\largewedge}(M)\cap T^k(M)\right)
    \] 
    is called the \textbf{$k$-th exterior power of $M$}.
}

\section{Commutative Algebra}
\dfn{Commutative Algebra}{
    Let $R$ be a commutative ring. A \textbf{commutative $R$-algebra} is an $R$-algebra where the multiplication is commutative. Or equivalently, a commutative $R$-algebra is a commutative ring $A$ together with a ring homomorphism $R\to A$. Hence there is a category isomorphism $R\text{-}\mathsf{CAlg}\cong \left(R/\mathsf{CRing}\right)$.
}


\dfn{Free Commutative Algebra}{
    Let $X$ be a set and $R$ be a commutative ring. The \textbf{free commutative $R$-algebra} on $X$, denoted by $\mathrm{Free}_{R\text{-}\mathsf{CAlg}}(X)$, together with a map $\iota:X\to \mathrm{Free}_{R\text{-}\mathsf{CAlg}}(X)$, is defined by the following universal property: for any commutative $R$-algebra $A$ and any map $f:X\to A$, there exists a unique homomorphism $\widetilde{f}:\mathrm{Free}_{R\text{-}\mathsf{CAlg}}(X)\to A$ such that the following diagram commutes
    \begin{center}
        \begin{tikzcd}[ampersand replacement=\&]
            \mathrm{Free}_{R\text{-}\mathsf{CAlg}}(X)\arrow[r, dashed, "\exists !\,\widetilde{f}"]  \& A\\[0.3cm]
            X\arrow[u, "\iota"] \arrow[ru, "f"'] \&  
        \end{tikzcd}
    \end{center}
    The free $R$-module $\mathrm{Free}_{R\text{-}\mathsf{CAlg}}(X)$ can be contructed as the polynomial algebra $R[X]$.
}
\dfn{Finite-type Commutative Algebra}{
    Let $R\to A$ be a commutative ring homomorphism. We say $A$ is a \textbf{finite-type $R$-algebra}, or that $R\to A$ is \textbf{of finite type}, if one of the following equivalent conditions holds:
    \begin{enumerate}[(i)]
        \item there exists a finite set of elements $a_1,\cdots,a_n$ of A such that every element of $A$ can be expressed as a polynomial in $a1,\cdots,an$, with coefficients in $K$.
        \item there exists a finite set $X$ such that $A\cong R[X]/I$ as $R$-algebra where $I$ is an ideal of $R[X]$.
    \end{enumerate}
    
}

\prop{}{
    Let $A$ be a $R$-algebra. If $A$ is finitely generated as an $R$-module, then $A$ is a finite-type $R$-algebra.
}
\pf{
    This holds because if each element of $A$ can be expressed as an $R$-linear combination of finitely many elements of $A$, then each element of $A$ can also be expressed as a polynomial in finitely many elements of $A$ with coefficients in $R$.

    An alternative proof can be given by utilizing the universal property of the free contruction. Suppose $A$ is finitely generated as an $R$-module. Then there exists some a finite set $X=\{x_1,\cdots,x_n\}$ and a surjective $R$-linear map $\varphi:R^{\oplus X}\to A$. Define $f=\varphi\circ \iota$, where $\iota:X\to R^{\oplus X}$ is the inclusion map. 
     \begin{center}
        \begin{tikzcd}[ampersand replacement=\&]
            R^{\oplus X}\arrow[r, dashed, "\exists !\,\widetilde{j}"]  \&R[X]\arrow[r, dashed, "\exists !\,\widetilde{f}"]  \& A\\[0.3cm]
            \& X\arrow[ul, "\iota"] \arrow[u, "j"] \arrow[ru, "f:=\varphi\circ \iota"'] \&  
        \end{tikzcd}
    \end{center}
    The universal property of free $R$-module induces a unique $R$-linear map $\widetilde{j}:R^{\oplus}\to R[X]$ such that $j=\widetilde{j}\circ \iota$. And the universal property of free commutative $R$-algebra induces a unique $R$-algebra homomorphism $\widetilde{f}:R[X]\to A$ such that $f=\widetilde{f}\circ j$. Note $f=\varphi\circ \iota=\left(\widetilde{f}\circ \widetilde{j}\right)\circ \iota$. By the uniqueness of the universal property of $R^{\oplus}$, we have $\widetilde{f}\circ \widetilde{j}=\varphi$. Since $\widetilde{f}\circ \widetilde{j}$ is surjective, $\widetilde{f}$ must be surjective, which implies $A$ is a finite-type $R$-algebra.
}