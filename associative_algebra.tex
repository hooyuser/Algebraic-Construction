
\chapter{Associative Algebra}

\section{Basic Properties}
\begin{definition}{Associative Algebra over Commutative Ring}{}
    Let $R$ be a commutative ring. An \textbf{associative $R$-algebra} is a ring $A$ together with a ring homomorphism $\varphi:R\to Z(A)$, which makes $A$ an $R$-module by defining the scalar multiplication as
    \begin{align*}
        R\times A &\longrightarrow A\\
        (r,a) &\longmapsto r\cdot a:=\varphi(r)a.
    \end{align*}
    $\varphi:R\to Z(A)$ is called the \textbf{structure homomorphism} of $A$.
\end{definition}
\begin{remark}
    We can check that 
    \[
      r\cdot (ab)=\sigma(r)ab=(r\cdot a)b=\sigma(r)ab=a\left(\sigma(r)b\right) =a(r\cdot b),
    \]
    which justifies the naming ``associative". 
\end{remark}

We usually call associative $R$-algebra as $R$-algebra for short.

\begin{proposition}{Commutative Ring homomorphism $R\to S$ induces functor $S\text{-}\mathsf{Alg}\to R\text{-}\mathsf{Alg}$}{}
    Let $R$ and $S$ be commutative rings with a ring homomorphism $f: R\to S$. Then every $S$-algebra $A$ is an $R$-algebra by defining $ra = f(r)a$, or equivalently through $R\to S\to Z(A)$. This defines a functor $F: S\text{-}\mathsf{Alg}\to R\text{-}\mathsf{Alg}$, which is identify map on objects and morphisms.
    \[
        \begin{tikzcd}[ampersand replacement=\&]
            S\text{-}\mathsf{Alg}\&[-25pt]\&[+10pt]\&[-30pt] R\text{-}\mathsf{Alg}\&[-30pt]\&[-30pt] \\ [-15pt] 
            A  \arrow[dd, "g"{name=L, left}] 
            \&[-25pt] \& [+10pt] 
            \& [-30pt] A\arrow[dd, "g"{name=R}] \&[-30pt]\\ [-10pt] 
            \&  \phantom{.}\arrow[r, "F", squigarrow]\&\phantom{.}  \&   \\[-10pt] 
            B \& \& \&  B\&
        \end{tikzcd}
        \]  
\end{proposition}

In particular, commutative ring homomorphism $R\to S$ makes $S$ an $R$-algebra.



\section{Construction}
\subsection{Quotient Object}

\begin{definition}{Quotient Algebra}{quotient_algebra}
    Let $A$ be an $R$-algebra and $\mathfrak{a}$ be a two-sided ideal of $A$. Since $\mathfrak{a}$ is an $R$-submodule of $A$, the quotient ring $A/\mathfrak{a}$ can also be endowed with an $R$-module structure, which makes $A/\mathfrak{a}$ an $R$-algebra. We call $A/\mathfrak{a}$ the \textbf{quotient algebra} of $A$ by $\mathfrak{a}$.
\end{definition}
\begin{proposition}{Universal Property of Quotient Algebra}{universal_property_of_quotient_algebra}
    Let $A$ be an $R$-algebra and $\mathfrak{a}$ be a two-sided ideal of $A$. Let $\pi:A\to A/\mathfrak{a}$ be the canonical projection. For any $R$-algebra homomorphism $f:A\to B$ such that $\mathfrak{a}\subseteq \ker(f)$
    or equivalently $f(\mathfrak{a})=\{0\}$, there exists a unique $R$-algebra homomorphism $\widetilde{f}:A/\mathfrak{a}\to B$ such that the following diagram commutes
    \begin{center}
        \begin{tikzcd}
                                                                 & A \arrow[ld, "\pi"'] \arrow[rd, "f"] &   \\
A/\mathfrak{a} \arrow[rr, "{\exists !\,\widetilde{f}}"', dashed] &                                      & B
\end{tikzcd}
    \end{center}
    The unique $R$-algebra homomorphism $\widetilde{f}:A/\mathfrak{a}\to B$ is defined as
    \begin{align*}
        \widetilde{f}: A/\mathfrak{a} &\longrightarrow B\\
        a + \mathfrak{a} &\longmapsto f(a).
    \end{align*}
\end{proposition}

\subsection{Free Object}
\begin{definition}{Free $R$-Algebra}{}
    Let $X$ be a set and $R$ be a commutative ring. The \textbf{free $R$-algebra} on $X$, denoted by $\mathrm{Free}_{R\text{-}\mathsf{Alg}}(X)$, together with a function $\iota:X\to \mathrm{Free}_{R\text{-}\mathsf{Alg}}(X)$, is defined by the following universal property: for any $R$-algebra $A$ and any function $f:X\to A$, there exists a unique $R$-algebra homomorphism $\widetilde{f}:\mathrm{Free}_{R\text{-}\mathsf{Alg}}(X)\to A$ such that the following diagram commutes
    \begin{center}
        \begin{tikzcd}[ampersand replacement=\&]
            \mathrm{Free}_{R\text{-}\mathsf{Alg}}(X)\arrow[r, dashed, "\exists !\,\widetilde{f}"]  \& A \\[0.3cm]
            X\arrow[u, "\iota"] \arrow[ru, "f"'] \&  
        \end{tikzcd}
    \end{center}
    The free $R$-algebra $\mathrm{Free}_{R\text{-}\mathsf{Alg}}(X)$ can be contructed by direct sum of copies of $R$
    \[
        \mathrm{Free}_{R\text{-}\mathsf{Alg}}(X)\cong\bigoplus_{w\in\mathrm{Free}_{\mathsf{Mon}}(X)}Rw.  
    \]
\end{definition}


\subsection{Graded Object}

\begin{definition}{$I$-Graded Algebra over an Graded Commutative Ring}{}
    Let $(I,+)$ be a monoid and $R$ be a \hyperref[th:I_graded_ring]{$I$-graded commutative ring} with grading $(R_i)_{i\in I}$. An \textbf{$I$-graded algebra over graded ring $R$} is an $R$-algebra $A$ together with a family of subalgebras $\left(A_i\right)_{i\in I}$ such that
    \begin{enumerate}[(i)]
        \item $A=\bigoplus_{i\in I}A_i$.
        \item $A_iA_j\subseteq A_{i+j}$ for all $i, j\in I$.
        \item $R_iA_j\subseteq A_{i+j}$ for all $i, j\in I$.
    \end{enumerate}
    Elements in $A_i$ are called \textbf{homogeneous elements of degree $i$}. 
\end{definition}

\begin{definition}{Degree-preserving $R$-algebra homomorphism}{degree_preserving_R_algebra_homomorphism}
    Let $(I,+)$ be a monoid and let $R=\bigoplus_{i\in I} R_i$ be an $I$-graded commutative ring. Let $A=\bigoplus_{i\in I} A_i$ and $B=\bigoplus_{i\in I} B_i$ be $I$-graded $R$-algebras. 
    An $R$-algebra homomorphism $f:A\to B$ is called \textbf{degree-preserving} (or \textbf{graded}) if for every $i\in I$,
    \[
    f(A_i)\subseteq B_i.
    \]
    We write $\mathsf{GrAlg}^I_R$ for the category whose objects are $I$-graded $R$-algebras and whose morphisms are degree-preserving $R$-algebra homomorphisms.
\end{definition}

\begin{proposition}{Graded Algebra Quotients out Graded Ideal}{graded_algebra_quotients_out_graded_ideal}
    Let $A$ be an $I$-graded algebra over graded ring $R$ with grading $(A_i)_{i\in I}$ and $\mathfrak{a}$ be a \hyperref[th:graded_ideal]{$I$-graded two-sided ideal} of $A$. Then we have an $R$-module isomorphism
    \begin{align*}
        A_i/\left(\mathfrak{a}\cap A_i\right)&\xlongrightarrow{\sim} \left( A_i+\mathfrak{a}\right)/\mathfrak{a}\\
         a + \left(\mathfrak{a}\cap A_i\right) &\longmapsto a + \mathfrak{a},
    \end{align*}
    and the \hyperref[th:quotient_algebra]{quotient algebra} $A/\mathfrak{a}$ has a decomposition
    \[
    A/ \mathfrak{a}=\bigoplus_{i\in I}\left( A_i+\mathfrak{a}\right)/\mathfrak{a}\cong \bigoplus_{i\in I}A_i/\left(\mathfrak{a}\cap A_i\right),    
    \]
    which makes $A/\mathfrak{a}$ an $I$-graded $R$-algebra.
\end{proposition}
\begin{prf}
    By the definition of graded ideal, we have
    \[
    \mathfrak{a}=\bigoplus_{i\in I}\mathfrak{a}_i,\quad \mathfrak{a}_i:=\mathfrak{a}\cap A_i.
    \]
    Let $\pi:A\to A/\mathfrak{a}$ be the canonical projection. Restrict $\pi$ to $A_i$, we have an $R$-module homomorphism
    \begin{align*}
        \pi|_{A_i}: A_i &\longrightarrow \pi(A_i)\\
    \end{align*}
    with kernel 
    \[
        \ker\left(\pi|_{A_i}\right)=\left\{ x\in A_i \midv \pi(x)=0 + \mathfrak{a} \right\}=\left\{ x\in A_i \midv x\in  \mathfrak{a} \right\} = A_i \cap \mathfrak{a} = \mathfrak{a}_i.
    \]
    Thus we have an $R$-module isomorphism
    \begin{align*}
        A_i/\mathfrak{a}_i&\xlongrightarrow{\sim} \pi(A_i)\\
         a + \mathfrak{a}_i &\longmapsto a + \mathfrak{a}.
    \end{align*}
    Take any $\pi(a)\in A/\mathfrak{a}$, where $a\in A$. Since $A=\bigoplus_{i\in I} A_i$, we can write $a=\sum_{i\in I} a_i$ with $a_i\in A_i$ and only finitely many $a_i$ being nonzero. Thus
    \[
    \pi(a)=\pi\left(\sum_{i\in I} a_i\right)=\sum_{i\in I} \pi(a_i)\in \sum_{i\in I}\pi(A_i),
    \]
    which means $A/\mathfrak{a}\subseteq \sum_{i\in I}\pi(A_i)$. On the other hand, we have $\sum_{i\in I}\pi(A_i)\subseteq A/\mathfrak{a}$. Therefore, we have $A/\mathfrak{a}=\sum_{i\in I}\pi(A_i)$.
    
    Moreover, if $\sum_{i\in I}\pi(a_i)=0$ with $a_i\in A_i$, then $\pi\left(\sum_{i\in I} a_i\right)=0$, which means $\sum_{i\in I} a_i\in \mathfrak{a}$. According to \Cref{th:membership_criterion_for_graded_ideals}, this implies $a_i\in \mathfrak{a}\cap A_i $ for all $i\in I$. Thus we have $\pi(a_i)=0$ for all $i\in I$. Therefore, the sum $A/\mathfrak{a}=\sum_{i\in I}\pi(A_i)$ is direct, which means
    \[
     A/\mathfrak{a}=\bigoplus_{i\in I}\pi(A_i).
    \]
    Note 
    \[
    \pi(A_i) =\left\{x+\mathfrak{a}\midv x\in A_i\right\} = \left\{x+a +\mathfrak{a}\midv x\in A_i, a\in \mathfrak{a}\right\}= \left( A_i+\mathfrak{a}\right)/\mathfrak{a}.
    \]
    Combining these results, we have the $R$-module isomorphism
    \[
    A/ \mathfrak{a}=\bigoplus_{i\in I}\pi(A_i)=\bigoplus_{i\in I}\left( A_i+\mathfrak{a}\right)/\mathfrak{a}\cong \bigoplus_{i\in I}A_i/\left(\mathfrak{a}\cap A_i\right).
    \]
    Finally, we can check that the multiplication and scalar multiplication respect the grading:
    \[
        \left(a_i + \mathfrak{a}\right)\left(a_j + \mathfrak{a}\right)=a_ia_j + \mathfrak{a}\in \left(A_{i+j}+\mathfrak{a}\right)/\mathfrak{a},\quad \forall a_i\in A_i, a_j\in A_j,
    \]
    \[
        r\cdot\left(a_i + \mathfrak{a}\right)=ra_i + \mathfrak{a}\in \left(A_{i+j}+\mathfrak{a}\right)/\mathfrak{a},\quad \forall r\in R_j, a_i\in A_i.
    \]

\end{prf}


\begin{example}{Polynomial Algebra $R[X_1,\cdots,X_n]$}{}
    Let $R$ be a commutative ring and $X_1,\cdots,X_n$ be indeterminates. Then $R[X_1,\cdots,X_n]$ is an $\mathbb{N}$-graded $R$-algebra with grading $R[X_1,\cdots,X_n]_i$ being the set of homogeneous polynomials of degree $i$.
\end{example}

\subsection{Tensor Product}
\begin{definition}{Tensor Product of Algebras}{tensor_product_of_algebras}
    Let $R$ be a commutative ring and $A$, $B$ be $R$-algebras. The \textbf{tensor product of $R$-algebras $A$ and $B$} is defined by the following universal property: for any triple $(C, f_A, f_B)$, where $C$ is an $R$-algebra and $f_A:A\to C$, $f_B:B\to C$ are $R$-algebra homomorphisms which satisfy
    \[
        f_A(a)f_B(b)=f_B(b)f_A(a),\quad \forall a\in A, b\in B,
    \]
    the tensor product
    \[
    (A\otimes_R B, \iota_A: A \times B \to A\otimes_R B, \iota_B: A \times B \to A\otimes_R B)
    \]
    is initial among such triples, i.e. there exists a unique $R$-algebra homomorphism \begin{align*}
        \phi: A\otimes_R B &\longrightarrow C
    \end{align*}
    
    such that the following diagram commutes
    \[
        \begin{tikzcd}
        A \arrow[r, "\iota_A"] \arrow[rd, "f_A"'] & A\otimes_R B \arrow[d, "\exists! \phi", dashed] & B \arrow[l, "\iota_B"'] \arrow[ld, "f_B"] \\[0.5em]
                                                & C                                               &                                          
        \end{tikzcd}
    \]

    Concretely, $A\otimes_R B$ can be constructed as the tensor product of $R$-modules $A\otimes_R B$ together with multiplication defined as
    \[
        (a_1\otimes b_1)(a_2\otimes b_2):=(a_1a_2)\otimes(b_1b_2),\quad \forall a_1,a_2\in A, b_1,b_2\in B
    \]
    and unity 
    \[
          1_{A\otimes_R B}:=1_A\otimes 1_B.
    \]
    And the $R$-algebra homomorphisms $\iota_A$, $\iota_B$ are defined as
    \begin{align*}
        \iota_A : A &\longrightarrow A\otimes_R B\\
        a &\longmapsto a\otimes 1_B,
    \end{align*}
    \begin{align*}
        \iota_B : B &\longrightarrow A\otimes_R B\\
        b &\longmapsto 1_A\otimes b.
    \end{align*}
    The unique $R$-algebra homomorphism $\phi:A\otimes_R B\to C$ is defined as
    \begin{align*}
        \phi: A\otimes_R B &\longrightarrow C\\
        a\otimes b &\longmapsto f_A(a)f_B(b).
    \end{align*}
\end{definition}
\begin{remark}
    It is straightforward to check that the multiplication defined above is well-defined and makes $A\otimes_R B$ an $R$-algebra. According to \Cref{th:pure_tensors_generate_tensor_product}, since $(a,b)\mapsto f_A(a)f_B(b)$ is $\mathbb{Z}$-bilinear, $\phi$ is a well-defined abelian group homomorphism. We can further check that $\phi$ is an $R$-algebra homomorphism:
    \[
    \phi(r(a\otimes b))=\phi((r a)\otimes b)=f_A(r a)f_B(b)=r f_A(a)f_B(b)=r \phi(a\otimes b)
    \]
    \[
    \phi\left((a_1\otimes b_1)(a_2\otimes b_2)\right)=\phi\left((a_1a_2)\otimes(b_1b_2)\right)= f_A(a_1a_2)f_B(b_1b_2)=f_A(a_1)f_A(a_2)f_B(b_1)f_B(b_2)=\phi(a_1\otimes b_1)\phi(a_2\otimes b_2)
    \]
    \[
\phi\left(1_{A\otimes_R B}\right)=\phi\left(1_A\otimes 1_B\right)=f_A(1_A)f_B(1_B)=1_C
    \]
\end{remark}

\begin{definition}{Tensor Product of $R$-algebra Homomorphisms}{tensor_product_of_R_algebra_homomorphisms}
    Let $R$ be a commutative ring and $A_1$, $A_2$, $B_1$, $B_2$ be $R$-algebras. Given two $R$-algebra homomorphisms $f:A_1\to A_2$ and $g:B_1\to B_2$, the \textbf{tensor product of $R$-algebra homomorphisms} is defined as the $R$-algebra homomorphism 
    \begin{align*}
        f\otimes_R g: A_1\otimes_R B_1 &\longrightarrow A_2\otimes_R B_2\\
        a\otimes b &\longmapsto f(a)\otimes g(b).
    \end{align*}
    which is induced by the universal property of tensor product $A_1\otimes_R B_1$ through the following commutative diagram:
    \[
    \begin{tikzcd}
        A_1 \arrow[r, "\iota_{A_1}"] \arrow[d, "f"']                              & A_1\otimes_R B_1 \arrow[d, "f\otimes_R g", dashed] & B_2 \arrow[l, "\iota_{B_1}"'] \arrow[d, "g"] \\[0.7em]
        A_2  \arrow[r, "\iota_{A_2}"'] & A_2\otimes_R B_2                                   & B_2 \arrow[l, "\iota_{B_2}"]                
    \end{tikzcd}
    \] 
\end{definition}


\begin{proposition}{Symmetric Monoidal Structure on $R$-$\mathsf{Alg}$}{}
    Let $R$ be a commutative ring. The tensor product $\otimes_R$ defines a symmetric monoidal structure on the category $R$-$\mathsf{Alg}$, with unit object $R$.
    \begin{enumerate}[(i)]
        \item Tensor product: the tensor product functor is 
        \[
        \begin{tikzcd}[ampersand replacement=\&]
            R\text{-}\mathsf{Alg}\times R\text{-}\mathsf{Alg}\&[-25pt]\&[+10pt]\&[-30pt] R\text{-}\mathsf{Alg}\&[-30pt]\&[-30pt] \\ [-15pt] 
            (A_1, B_1)  \arrow[dd, "f\times g"{name=L, left}] 
            \&[-25pt] \& [+10pt] 
            \& [-30pt] A_1\otimes_R B_1\arrow[dd, "f\otimes_R g"{name=R}] \&[-30pt]\\ [-10pt] 
            \&  \phantom{.}\arrow[r, "\otimes_R", squigarrow]\&\phantom{.}  \&   \\[-10pt] 
            (A_2, B_2) \& \& \&  A_2\otimes_R B_2\&
        \end{tikzcd}
        \]  
        \item Associator: for any $R$-algebras $A$, $B$, $C$, there is a natural isomorphism
        \begin{align*}
            \alpha_{A,B,C}:(A\otimes_R B)\otimes_R C &\xlongrightarrow{\sim}A\otimes_R (B\otimes_R C)\\
            (a\otimes b)\otimes c &\longmapsto a\otimes (b\otimes c)
        \end{align*}
        \item Unit object: $R$.
        \item An isomorphism in $R$-$\mathsf{Alg}$:
        \begin{align*}
            \iota: R \otimes_R R &\xlongrightarrow{\sim} R\\
            r\otimes r' &\longmapsto rr'
        \end{align*}
        \item Symmetry: for any $R$-algebras $A$, $B$, there is a natural isomorphism
        \begin{align*}
            \gamma_{A,B}: A\otimes_R B &\xlongrightarrow{\sim} B\otimes_R A\\
            a\otimes b &\longmapsto b\otimes a
        \end{align*}
    \end{enumerate}
\end{proposition}

% \begin{proposition}{}{}
%     Let $R$ be a commutative ring, and $A$, $B$, $C$ be $R$-algebras. Suppose $\otimes_R:A\times B\to A\otimes_R B$ is the tensor product map. Let $f:\mathrm{im}(\otimes_R)\to C$ be a map. Define
%     \[
%     \beta:A\times B\longrightarrow C,\qquad \beta(a,b)\coloneqq f(a\otimes b).
%     \]
% then find equivalent consitions for that there exists a $R$-algebra homomorphism $\tilde{f}:A\otimes_R B\to C$ such that the following diagram commutes 
%     \[ 
%     \begin{tikzcd}[ampersand replacement=\&] \mathrm{im}(\otimes) \arrow[r, "i", hook] \arrow[rd, "f"'] \& A\otimes_R B \arrow[d, dashed, "\tilde{f}"] \\[0.3cm] \& C 
%     \end{tikzcd} 
%     \]
% \end{proposition}

\begin{proposition}{Tensor Product of Quotient Algebras}{tensor_product_of_quotient_algebras}
    Let $R$ be a commutative ring and $A_1$, $A_2$ be $R$-algebras. Let $I_1\subseteq A_1$, $I_2\subseteq A_2$ be two-sided ideals of $A_1$, $A_2$ respectively. Then we have an $R$-algebra isomorphism
    \begin{align*}
        (A_1/I_1)\otimes_R (A_2/I_2) &\xlongrightarrow{\sim}(A_1\otimes_R A_2)/(I_1\otimes_R A_2 + A_1\otimes_R I_2)\\
        \overline{a_1}\otimes \overline{a_2} &\longmapsto \overline{a_1\otimes a_2}
    \end{align*}
    Here, given inclusion $i_1: I_1\hookrightarrow A_1$, the $R$-module $I_1\otimes_R A_2$ is identified as the image of $i_1\otimes_R \mathrm{id}_{A_2}:I_1\otimes_R A_2 \to A_1\otimes_R A_2$
    \[
        \mathrm{im}\left(i_1\otimes_R \mathrm{id}_{A_2}\right)=\left\{\sum_{n=1}^m x_n\otimes y_n\in A_1\otimes_R A_2\midv m\in\mathbb{Z}_{\ge 1},\,x_n\in I_1,\, y_n\in A_2\right\},
    \]
    which is a two-sided ideal of $A_1\otimes_R A_2$. The similar identification applies to $A_1\otimes_R I_2$. 
\end{proposition}
\begin{prf}
    Let $J:=I_1\otimes_R A_2 + A_1\otimes_R I_2$. Define
    \begin{align*}
        \iota_1: A_1/I_1 &\longrightarrow (A_1\otimes_R A_2)/J\\
        a_1 + I_1 &\longmapsto (a_1\otimes 1_{A_2}) + J,
    \end{align*}
    If $a_1, a_1'\in A_1$ satisfy $a_1-a_1'\in I_1$, then 
    \[
    (a_1\otimes 1_{A_2}) - (a_1'\otimes 1_{A_2}) = (a_1 - a_1')\otimes 1_{A_2} \in I_1\otimes_R A_2 \subseteq J\implies (a_1\otimes 1_{A_2}) + J = (a_1'\otimes 1_{A_2}) + J,
    \]
    which shows that $\iota_1$ is well-defined. And we can check that $\iota_1$ is an $R$-algebra homomorphism. Similarly, we can define an $R$-algebra homomorphism
    \begin{align*}
        \iota_2: A_2/I_2 &\longrightarrow (A_1\otimes_R A_2)/J\\
        a_2 + I_2 &\longmapsto (1_{A_1}\otimes a_2) + J.
    \end{align*}
    Moreover, the images of $\iota_1$ and $\iota_2$ commute in $(A_1\otimes_R A_2)/J$: for any $a_1\in A_1$, $a_2\in A_2$,
    \begin{align*}
        \iota_1(a_1 + I_1)\iota_2(a_2 + I_2) &= \left((a_1\otimes 1_{A_2}) + J\right)\left((1_{A_1}\otimes a_2) + J\right)\\
        &= (a_1\otimes a_2) + J\\
        &= \left((1_{A_1}\otimes a_2) + J\right)\left((a_1\otimes 1_{A_2}) + J\right) \\
        &= \iota_2(a_2 + I_2)\iota_1(a_1 + I_1).
    \end{align*}   
    Thus by the \hyperref[th:tensor_product_of_algebras]{universal property of tensor product}, there exists a unique $R$-algebra homomorphism
    \begin{align*}
        \varphi: (A_1/I_1)\otimes_R (A_2/I_2) &\longrightarrow (A_1\otimes_R A_2)/J\\
        (a_1 + I_1)\otimes (a_2 + I_2) &\longmapsto (a_1\otimes a_2) + J.
    \end{align*}
    Next, we construct the inverse of $\varphi$. Given the quotient maps $\pi_1:A_1\to A_1/I_1$ and $\pi_2:A_2\to A_2/I_2$, we can define an $R$-algebra homomorphism $\psi:=\pi_1\otimes_R \pi_2$ as
    \begin{align*}
        \psi: A_1\otimes_R A_2 &\longrightarrow (A_1/I_1)\otimes_R (A_2/I_2)\\
        a_1\otimes a_2 &\longmapsto (a_1 + I_1)\otimes (a_2 + I_2).
    \end{align*}
    Since for any $x\in I_1$, $y\in A_2$, we have
    \[
        \psi(x\otimes y)=(0 + I_1)\otimes (y + I_2)=0
    \]
    and for any $x\in A_1$, $y\in I_2$, we have
    \[
        \psi(x\otimes y)=(x + I_1)\otimes (0 + I_2)=0,
    \]
    we have $J\subseteq \ker(\psi)$. Thus, by the universal property of quotient algebra, there exists a unique $R$-algebra homomorphism
    \begin{align*}
        \widetilde{\psi}: (A_1\otimes_R A_2)/J &\longrightarrow (A_1/I_1)\otimes_R (A_2/I_2)\\
        (a_1\otimes a_2) + J &\longmapsto (a_1 + I_1)\otimes (a_2 + I_2).
    \end{align*}
    We can check that $\widetilde{\psi}$ is the inverse of $\varphi$:
    \begin{align*}
        \widetilde{\psi}\circ \varphi\left((a_1 + I_1)\otimes (a_2 + I_2)\right) &= \widetilde{\psi}\left((a_1\otimes a_2) + J\right) = (a_1 + I_1)\otimes (a_2 + I_2),\\
        \varphi\circ \widetilde{\psi}\left((a_1\otimes a_2) + J\right) &= \varphi\left((a_1 + I_1)\otimes (a_2 + I_2)\right) = (a_1\otimes a_2) + J.
    \end{align*}
    Therefore, we show that $\varphi$ is an $R$-algebra isomorphism.
\end{prf}

\begin{corollary}{$mod\; I$ Reduction of $R$-Algebras}{mod_I_reduction_of_R_algebras}
    Let $R$ be a commutative ring and $I\subseteq R$ be an ideal of $R$. For any $R$-algebra $A$, there is an isomorphism of $R$-algebras
    \begin{align*}
        A/IA &\xlongrightarrow{\sim} A\otimes_R (R/I) \\
        \overline{a} &\longmapsto  a\otimes \overline{1_R}\\
        \overline{ra}&\longmapsfrom a\otimes \overline{r}
    \end{align*}
    where
    \[
    IA:= \left\{ r a \in A\midv r\in I,\, a\in A\right\}
    \]
    is the two-sided ideal of $A$ generated by $I$.
\end{corollary}
\begin{prf}
    Apply \Cref{th:tensor_product_of_quotient_algebras} with $A_1=A$, $A_2=R$, $I_1=\{0\}$, $I_2=I$. We obtain an $R$-algebra isomorphism
    \begin{align*}
        A \otimes_R (R/I) &\xlongrightarrow{\sim} (A\otimes_R R)/( 0\otimes_R R+A\otimes_R I)\\
        a \otimes (r+I) &\longmapsto (a\otimes r) + (A\otimes_R I).
    \end{align*}
    Under the canonical isomorphism 
    \begin{align*}
        \phi:  A\otimes_R R &\xlongrightarrow{\sim} A\\
        a\otimes r &\longmapsto r a,
    \end{align*}
    $A \otimes_R I$ is mapped to $IA$. Thus we have an $R$-algebra isomorphism
    \begin{align*}
        A \otimes_R (R/I) &\xlongrightarrow{\sim} A/IA\\
        a \otimes (r+I) &\longmapsto r a + IA.
    \end{align*}
\end{prf}

\begin{corollary}{}{}
   Let $R$ be a commutative ring and $I\subseteq R$ be an ideal of $R$. We have a cononical isomorphism of $R$-algebras
    \begin{align*}
          (R/I)\otimes_R (R/I) &\xlongrightarrow{\sim} R/I\\
          (r+I)\otimes (r'+I) &\longmapsto rr' + I.
    \end{align*}
\end{corollary}
\begin{prf}
    By \Cref{th:mod_I_reduction_of_R_algebras}, we have an $R$-algebra isomorphism
    \begin{align*}
        (R/I)\otimes_R (R/I) &\xlongrightarrow{\sim} (R/I)/ (I(R/I))\\
        (r+I)\otimes (r'+I) &\longmapsto rr' + I(R/I).
    \end{align*}
    Since for any $r\in I$, $r'+I\in R/I$, we have
    \[
        r(r'+I)=rr' + I = 0 + I,
    \]
    which shows that $I(R/I)=\{0\}$. Thus we obtain the desired isomorphism.
\end{prf}

\subsection{Tensor Algebra}
\begin{definition}{Tensor Algebra $T^{\bullet}(M)$}{}
    Given a $R$-module $M$, the \textbf{$k$-th tensor power of $M$} is defined as
    \begin{align*}
        T^k(M)&:=M^{\otimes k}=\underbrace{M\otimes_R\cdots\otimes_R M}_{k\text{ times}},\\
        T^0(M)&:=R.
    \end{align*}
    The \textbf{tensor algebra} of $M$ is defined as
    \[
        T^{\bullet}(M):=\bigoplus_{k=0}^{\infty}T^k(M)
    \]
    with multiplication $\otimes$ defined as
    \[
        (m_1\otimes\cdots\otimes m_k)\otimes(m_{k+1}\otimes\cdots\otimes m_{k+l})=m_1\otimes\cdots\otimes m_{k+l}
    \]
    $T^{\bullet}(M)$ is an $\mathbb{N}$-graded $R$-algebra with grading $(T^k(M))_{k\ge 0}$.
\end{definition}


\begin{proposition}{Tensor Algebra Functor $T^{\bullet}:R\text{-}\mathsf{Mod}\to \mathsf{GrAlg}^{\mathbb{N}}_R$}{}
    Let $R$ be a commutative ring. The tensor algebra construction $T^{\bullet}:R\text{-}\mathsf{Mod}\to \mathsf{GrAlg}^{\mathbb{N}}_R$ is a functor defined as follows
    \[
        \begin{tikzcd}[ampersand replacement=\&]
            R\text{-}\mathsf{Mod}\&[-25pt]\&[+10pt]\&[-30pt] \mathsf{GrAlg}^{\mathbb{N}}_R\&[-30pt]\&[-30pt] \\ [-15pt] 
            M  \arrow[dd, "g"{name=L, left}] 
            \&[-25pt] \& [+10pt] 
            \& [-30pt] T^{\bullet}(M)\arrow[dd, "T^{\bullet}(g)"{name=R}] \&[-20pt]\ni\& [+10pt]m_1\otimes\cdots\otimes m_k \arrow[dd, mapsto, "g\otimes g\cdots\otimes g"{name=L, right}] 
            \\ [-10pt] 
            \&  \phantom{.}\arrow[r, "T^{\bullet}", squigarrow]\&\phantom{.}  \&   \\[-10pt] 
            N \& \& \&  T^{\bullet}(N)\&[-0pt]\ni\& g(m_1)\otimes\cdots\otimes g(m_k)
        \end{tikzcd}
        \]  
\end{proposition}
\begin{prf}
    According to \Cref{th:tensor_product_of_R_algebra_homomorphisms}, for each $k\ge 0$, we can define an $R$-module homomorphism $T^k(g):=g^{\otimes k}$ on degree-$k$ component:
    \begin{align*}
        T^k(g): T^k(M) &\longrightarrow T^k(N)\\
        m_1\otimes\cdots\otimes m_k &\longmapsto g(m_1)\otimes\cdots\otimes g(m_k).
    \end{align*}
    Then we can define a \hyperref[th:degree_preserving_R_algebra_homomorphism]{degree-preserving $R$-algebra homomorphism} $T^{\bullet}(g):=\bigoplus_{k=0}^{\infty} T^k(g)$ as follows:
    \begin{align*}
        T^{\bullet}(g): T^{\bullet}(M) &\longrightarrow T^{\bullet}(N)\\
        (x_0, x_1, x_2, \cdots) &\longmapsto \left(T^0(g)(x_0),\, T^1(g)(x_1),\, T^2(g)(x_2), \cdots\right).
    \end{align*}
    It is straightforward to check that $T^{\bullet}(\mathrm{id}_M)=\mathrm{id}_{T^{\bullet}(M)}$ and $T^{\bullet}(g_2\circ g_1)=T^{\bullet}(g_2)\circ T^{\bullet}(g_1)$ for any $R$-module homomorphisms $g_1:M\to N$, $g_2:N\to P$. Thus, $T^{\bullet}$ is a functor.
\end{prf}


\begin{proposition}{Adjunction $T^{\bullet}\dashv U_{\mathsf{R\text{-}\mathsf{Mod}}}$}{}
    Let $R$ be a commutative ring. Suppose $U:R\text{-}\mathsf{Alg}\to R\text{-}\mathsf{Mod}$ is the forgetful functor. Then the tensor algebra functor $T^{\bullet}:R\text{-}\mathsf{Mod}\to R\text{-}\mathsf{Alg}$ is left adjoint to $U$.
\end{proposition}


\subsection{Exterior Algebra and Symmetric Algebra}
\begin{definition}{Exterior Algebra $\Largewedge^{\bullet} (M)$}{}
    Given an $R$-module $M$, 
    \begin{align*}
        I_{\largewedge}(M) & :=\langle x \otimes x: x \in M\rangle=\left\{\sum_{i=1}^m a_i (x_i \otimes x_i) b_i \midv m\in\mathbb{Z}_{\ge 1},\, a_i,b_i\in T^{\bullet}(M),\, x_i\in M\right\}
    \end{align*}
    is a graded two-sided ideal of $T^{\bullet}(M)$. The \textbf{exterior algebra} of $M$ is defined as
    \[
        \largewedge^{\bullet} (M)=T^{\bullet}(M)/I_{\largewedge}(M).
    \]
    According to \Cref{th:graded_algebra_quotients_out_graded_ideal}, $\largewedge^{\bullet} (M)$ is a graded $R$-algebra with grading 
    \[
        \largewedge^{\bullet} (M)\cong \bigoplus_{k=0}^{\infty} \largewedge^k(M)
    \]
    where
    \[
        \largewedge^k(M):=T^k(M)/\left(I_{\largewedge}(M)\cap T^k(M)\right)
    \] 
    is an $R$-module and is called the \textbf{$k$-th exterior power of $M$}. Especially, we have $\Largewedge^0(M)\cong R$ and $\Largewedge^1(M)\cong M$ as $R$-modules, and we identify them directly.
    
    The multiplication of $\Largewedge^{\bullet} (M)$ is denoted by 
    \begin{align*}
        \wedge: \Largewedge^{\bullet} (M)\times \Largewedge^{\bullet} (M) &\longrightarrow \Largewedge^\bullet (M)\\
        (a + I_{\wedge}(M), b + I_{\wedge}(M)) &\longmapsto (a\otimes b) + I_{\wedge}(M).
    \end{align*}
    and is called the \textbf{wedge product}. The graded version of the wedge product for degree-1 elements is given by
    \begin{align*}
        \wedge: \Largewedge^1 (M)\times \Largewedge^1 (M) &\longrightarrow \Largewedge^{2} (M)\\
        (m_1, m_2) &\longmapsto m_1\wedge m_2 := (m_1\otimes m_2) + I_{\wedge}(M)\cap T^{2}(M)
    \end{align*}
    We can prove $\Largewedge^{k} (M)$ is an $R$-module generated by the elements of the form
    \[
        m_1\wedge m_2\wedge \cdots \wedge m_k = (m_1\otimes m_2\otimes \cdots \otimes m_k) + I_{\wedge}(M)\cap T^k(M)
    \]
    for $m_1, m_2, \cdots, m_k \in M$. The wedge product for degree-$k$ and degree-$l$ elements is given by
    \begin{align*}
        \wedge: \Largewedge^k (M)\times \Largewedge^l (M) &\longrightarrow \Largewedge^{k+l} (M)\\
        (m_1\wedge\cdots\wedge m_k,\, m_{k+1}\wedge\cdots\wedge m_{k+l}) &\longmapsto m_1\wedge\cdots\wedge m_k \wedge m_{k+1}\wedge\cdots\wedge m_{k+l}
    \end{align*}
    where $m_i\in M$ for $1\le i \le k+l$.
\end{definition}
\begin{remark}
    Since $T^k(M)$ is an $R$-module generated by the pure tensors of the form $m_1\otimes m_2\otimes \cdots \otimes m_k$ for $m_i\in M$, the quotient module $\Largewedge^k(M)=T^k(M)/\left(I_{\largewedge}(M)\cap T^k(M)\right)$ is generated by the elements of the form $(m_1\otimes m_2\otimes \cdots \otimes m_k) + I_{\wedge}(M)\cap T^k(M)$. And by induction on $k$, we can show that 
    \[
    m_1 \wedge m_2 \wedge \cdots m_k = (m_1\otimes m_2\otimes \cdots \otimes m_k) + I_{\wedge}(M)\cap T^k(M).
    \]
\end{remark}

\begin{proposition}{}{}
     Given an $R$-module $M$ and $m_1, m_2 \in M$, we have
    \[
        m_1 \wedge m_2 = - m_2 \wedge m_1.
    \]
    For any homogeneous elements $x, y \in \Largewedge^{\bullet}(M)$ , we have
    \[
        x \wedge y = (-1)^{\deg(x)\deg(y)} y \wedge x.
    \]
\end{proposition}



\begin{definition}{Exterior Algebra Functor: $\Largewedge^{\bullet}:R\text{-}\mathsf{Mod}\to \mathsf{GrAlg}^{\mathbb{N}}_R$}{}
    The exterior algebra construction $\Largewedge^{\bullet}:R\text{-}\mathsf{Mod}\to \mathsf{GrAlg}^{\mathbb{N}}_R$ is a functor defined as follows
    \[
        \begin{tikzcd}[ampersand replacement=\&]
            R\text{-}\mathsf{Mod}\&[-25pt]\&[+10pt]\&[-30pt] \mathsf{GrAlg}^{\mathbb{N}}_R\&[-30pt]\&[-30pt] \\ [-15pt] 
            M  \arrow[dd, "f"{name=L, left}] 
            \&[-25pt] \& [+10pt] 
            \& [-30pt] \Largewedge^{\bullet}(M)\arrow[dd, "\Largewedge^{\bullet}(f)"{name=R}] \&[-20pt]\ni\& [+20pt]m_1\wedge\cdots\wedge m_k \arrow[dd, mapsto, "f\wedge f\wedge\cdots\wedge f"{name=L, right}] 
            \\ [-10pt] 
            \&  \phantom{.}\arrow[r, "\Largewedge^{\bullet}", squigarrow]\&\phantom{.}  \&   \\[-10pt] 
            N \& \& \&  \Largewedge^{\bullet}(N)\&[-0pt]\ni\& f(m_1)\wedge\cdots\wedge f(m_k)
        \end{tikzcd}
    \]  
    where $\Largewedge^{\bullet}(f)$ is induced by the \hyperref[th:universal_property_of_quotient_algebra]{universal property of the quotient algebra} $T^{\bullet}(M)/I_\wedge(M)$ through the following commutative diagram
    \[
        \begin{tikzcd}
    T^{\bullet}(M) \arrow[r, "T^{\bullet}(f)"] \arrow[d, "\pi_M"']                &[1em] T^{\bullet}(N) \arrow[d, "\pi_N"] \\[2em]
    T^{\bullet}(M)/I_\wedge(M)  \arrow[r, "\wedge^\bullet(f)"',dashed] & T^{\bullet}(N)/I_\wedge(N)       
    \end{tikzcd}
    \]
\end{definition}
\begin{remark}
    Since for any $x\in M$,
    \begin{align*}
        \pi_N\circ T^{\bullet}(f)(x\otimes x) &= \pi_N(f(x)\otimes f(x))=f(x)\wedge f(x) + I_\wedge(N)=0+ I_\wedge(N),
    \end{align*}
    we see each generator of $I_\wedge(M)$ is mapped to $0$ in $\pi_N\circ T^{\bullet}(f)$. Thus, we have $I_\wedge(M)\subseteq \ker(\pi_N\circ T^{\bullet}(f))$, which guarantees that there exists a unique $R$-algebra homomorphism 
    \begin{align*}
        \Largewedge^{\bullet}(f): \Largewedge^{\bullet}(M) &\longrightarrow \Largewedge^{\bullet}(N)\\
        m_1\wedge\cdots\wedge m_k &\longmapsto f(m_1)\wedge\cdots\wedge f(m_k).
    \end{align*}
    such that $\Largewedge^{\bullet}(f)\circ \pi_M = \pi_N \circ T^{\bullet}(f)$.
\end{remark}

\begin{proposition}{Adjunction $\Largewedge^{\bullet}\dashv U_{\mathsf{R\text{-}\mathsf{Mod}}}$}{}
    Let $R$ be a commutative ring. Suppose $U_{\mathsf{R\text{-}\mathsf{Mod}}}:\mathsf{GrAlg}^{\mathbb{N}}_R\to R\text{-}\mathsf{Mod}$ is the forgetful functor. Then the exterior algebra functor $\Largewedge^{\bullet}:R\text{-}\mathsf{Mod}\to \mathsf{GrAlg}^{\mathbb{N}}_R$ is left adjoint to $U$.
\end{proposition}


\begin{example}{Take Degree $k$ Functor}{take_degree_k_functor}
    Let $R$ be a commutative ring and $M$ be an $R$-module. The \textbf{take degree $k$ functor} is defined as
     \[
        \begin{tikzcd}[ampersand replacement=\&]
            \mathsf{GrAlg}^{\mathbb{N}}_R\&[-25pt]\&[+10pt]\&[-30pt]R\text{-}\mathsf{Mod} \\ [-15pt] 
            A=\bigoplus\limits_{i=0}^{\infty} A_i\arrow[dd, "g"{name=L, left}] 
            \&[-25pt] \& [+10pt] 
            \& [-30pt]  A_k\arrow[dd, "g|_{A_k}"{name=R}]  \\ [-10pt] 
            \&  \phantom{.}\arrow[r, "\left(-\right)_k", squigarrow]\&\phantom{.}  \&   \\[-10pt] 
              B=\bigoplus\limits_{i=0}^{\infty} B_i \& \& \&  B_k 
        \end{tikzcd}
    \]  
    In particular, we have the composition functor $\Largewedge^{k} := (-)_k \circ \bigwedge^\bullet : R\text{-}\mathsf{Mod}\to R\text{-}\mathsf{Mod}$ defined as follows
    \[
        \begin{tikzcd}[ampersand replacement=\&]
            R\text{-}\mathsf{Mod}\&[-25pt]\&[+10pt]\&[-30pt]R\text{-}\mathsf{Mod} \&[-30pt]\&[-30pt] \\ [-15pt] 
            M \arrow[dd, "f"{name=L, left}] 
            \&[-25pt] \& [+10pt] 
            \& [-30pt]  \Largewedge^{k}(M)\arrow[dd, "\Largewedge^{k}(f)"{name=R}] \&[-20pt]\ni\& [+20pt]m_1\wedge\cdots\wedge m_k \arrow[dd, mapsto, "f\wedge f\cdots\wedge f"{name=L, right}] 
            \\ [-10pt] 
            \&  \phantom{.}\arrow[r, "\Largewedge^{k}", squigarrow]\&\phantom{.}  \&   \\[-10pt] 
             N \& \& \&  \Largewedge^{k}(N)\&[-0pt]\ni\& f(m_1)\wedge\cdots\wedge f(m_k)
        \end{tikzcd}
    \]  
    
\end{example}


\begin{proposition}{}{}
    Suppose $R$ is a commutative ring and $M=\bigoplus_{x\in X}Rx$ is a free $R$-module. Then
    \begin{enumerate}[(i)]
        \item $\Largewedge^{\bullet}(M)$ has a basis $\{x_1\wedge\cdots\wedge x_k: x_1,\cdots,x_k\in X, x_i\ne x_j\text{ for all }i\ne j\}$.
        \item If $M$ has a basis $\{x_1,\cdots,x_n\}$, then we have an $R$-linear isomorphism
        \begin{align*}
            \largewedge^{n}(M)&\xlongrightarrow{\sim} R\\
            x_1 \wedge \cdots \wedge x_n& \longmapsto 1_R.
        \end{align*}
        Moreover, we have $\Largewedge^{m}(M)=0$ for all $m>n$.
    \end{enumerate}
\end{proposition}
\section{Integral Element}

\begin{definition}{Integral Element}{}
    Let $R$ be a commutative ring and $A$ be an $R$-algebra with structure homomorphism $\varphi:R\to Z(A)$. An element $x\in A$ is called \textbf{integral} over $R$ if there exists a monic polynomial $f\in R[T]$ such that $\leftindex^{\varphi}\!f(x)=0$.
\end{definition}


\begin{definition}{Generated Subalgebra}{generated_subalgebra}
    Let $R$ be a commutative ring and $A$ be an $R$-algebra. By the universal property of $R\langle T\rangle$, there exists a unique $R$-algebra homomorphism $\psi:R\langle T\rangle\to A$ such that 
    $\psi(T)=x$.
    \begin{center}
        \begin{tikzcd}[ampersand replacement=\&]
            R\langle T\rangle\arrow[r, dashed, "\exists !\,\psi"]  \& A\\[0.3cm]
            \{T\}\arrow[u, "\iota"] \arrow[ru, "\mathrm{const}_x"'] \&  
        \end{tikzcd}
    \end{center}
    The \textbf{$R$-subalgebra of $A$ generated by $x$} is defined as 
    \[
    R[x]:=\psi\left(R\langle T\rangle\right)=\left\{\sum_{k=0}^n r_k x^k \in A\;\middle|\; r_k\in R\right\}.
    \]
\end{definition}

\begin{proposition}{Equivalent Definition of Integral Element}{equivalent_definition_of_integral_element}
    Let $R$ be a commutative ring and $A$ be an $R$-algebra. Let $R[x]$ be the \hyperref[th:generated_subalgebra]{$R$-subalgebra of $A$ generated by $x$}. Then $A$ is an $R[x]$-module. And the following statements are equivalent:
    \begin{enumerate}[(i)]
        \item $x$ is integral over $R$.
        \item $R[x]$ is a finitely generated $R$-module.
        \item There exists a faithful $R[x]$-submodule of $A$ that is finitely generated as an $R$-module and contains $x$.
    \end{enumerate}
\end{proposition}

\section{Trace and Norm}

\begin{lemma}{Left Multiplication Endomorphism}{}
    Let $R$ be a commutative ring and $A$ be an $R$-algebra. For any $a\in A$, we can define the left multiplication endomorphism $l_a\in\mathrm{End}_{R\text{-}\mathsf{Mod}}(A)$ by
    \begin{align*}
        l_a:A &\longrightarrow A\\
        x &\longmapsto ax.
    \end{align*}
    Moreover,
    \begin{align*}
        l_{-}:A &\longrightarrow \mathrm{End}_{R\text{-}\mathsf{Mod}}(A)\\
        a &\longmapsto l_a
    \end{align*}
    is an $R$-algebra homomorphism.
\end{lemma}
\begin{prf}
    For any $r\in R$, $a,b\in A$ and $x\in A$, we have
    \begin{align*}
        l_{ra+b}(x)&=(ra+b)x=r(ax)+bx=(ra)x+bx=l_{ra}(x)+l_b(x),\\
        l_{ab}(x)&=(ab)x=a(bx)=l_a(l_b(x)),\\
        l_{1_A}(x)&=1_Ax=x.
    \end{align*}
    Hence $l_{-}$ is an $R$-algebra homomorphism.
\end{prf}


\begin{definition}{Trace, Norm and Characteristic Polynomial}{trace_norm_and_characteristic_polynomial}
    Let $R$ be a commutative ring and $A$ be an $R$-algebra. Suppose $A$ as an $R$-module is free of finite rank. For any $a\in A$, we can define the left multiplication endomorphism $l_a\in\mathrm{End}_{R\text{-}\mathsf{Mod}}(A)$ by
    \begin{align*}
        l_a:A &\longrightarrow A\\
        x &\longmapsto ax.
    \end{align*}
    \begin{itemize}
        \item The \textbf{trace} of $a\in A$ is defined as the trace of $l_a$, denoted by 
        \[
        \mathrm{Tr}_{A|R}(a):=\mathrm{Tr}(l_a) \in R.
        \]
        That is, $\mathrm{Tr}_{A|R}:A\to R$ is an $R$-module homomorphism through the following composition
        \[
            \mathrm{Tr}_{A|R}: A\xrightarrow{l_{-}} \mathrm{End}_{R\text{-}\mathsf{Mod}}(A)\xrightarrow{\mathrm{Tr}} R.
        \]
        \item The \textbf{norm} of $a\in A$ is defined as the determinant of $l_a$, denoted by
        \[
        \mathrm{N}_{A|R}(a):=\det(l_a) \in R.
        \]
        That is, $\mathrm{N}_{A|R}:A\to R$ is a multiplicative monoid homomorphism through the following composition
        \[
            \mathrm{N}_{A|R}: A\xrightarrow{l_{-}} \mathrm{End}_{R\text{-}\mathsf{Mod}}(A)\xrightarrow{\det} R.
        \]
        \item The \textbf{characteristic polynomial} of $a\in A$ is defined as the characteristic polynomial of $l_a$, denoted by
        \[
        \mathrm{char}_{A|R}(a; X):=\mathrm{char}(l_a; X)=\det(X\cdot\mathrm{id}_A-l_a)= \mathrm{N}_{A[X]|R[X]}(X-a)\in R[X]
        \]
    \end{itemize}
\end{definition}

\begin{proposition}{Trace, Norm, and Characteristic Polynomial under Change of Base Ring}{trace_norm_and_characteristic_polynomial_under_change_of_base_ring}
    Let $R$ be a commutative ring, $A$ be an commutative $R$-algebra and $M$ be a free $R$-module of finite rank. Suppose $A$ as an $R$-module is free of finite rank. Given any $A$-linear transformation $\varphi\in\mathrm{End}_{A\text{-}\mathsf{Mod}}(M)$, by applying the functor of restriction of scalars to $R\to A$, we can regard $\varphi$ as an $R$-linear transformation on $M$ through $\mathrm{End}_{A\text{-}\mathsf{Mod}}(M)\hookrightarrow\mathrm{End}_{R\text{-}\mathsf{Mod}}(M)$. And we have
    \begin{enumerate}[(i)]
        \item $\mathrm{Tr}_{R}(\varphi)=\mathrm{Tr}_{A|R}(\mathrm{Tr}_{A}(\varphi))$.
        \item $\mathrm{N}_{R}(\varphi)=\mathrm{N}_{A|R}(\mathrm{N}_{A}(\varphi))$.
        \item $\mathrm{char}_{R}(\varphi;X)=\mathrm{N}_{A[X]|R[X]}(\mathrm{char}_{A}(\varphi;X))$.
    \end{enumerate}
\end{proposition}
\begin{corollary}{}{}
    Let $R$ be a commutative ring, $A$ be an commutative $R$-algebra and $B$ be an $A$-algebra. Suppose $A$ as an $R$-module is free of finite rank and $B$ as an $A$-module is free of finite rank. Then for any $b\in B$, we have
    \begin{enumerate}[(i)]
        \item $\mathrm{Tr}_{B|R}(b)=\mathrm{Tr}_{A|R}(\mathrm{Tr}_{B|A}(b))$.
        \item $\mathrm{N}_{B|R}(b)=\mathrm{N}_{A|R}(\mathrm{N}_{B|A}(b))$.
        \item $\mathrm{char}_{B|R}(b;X)=\mathrm{N}_{A[X]|R[X]}(\mathrm{char}_{B|A}(b;X))$.
    \end{enumerate}
    
\end{corollary}
\begin{prf}
    This is a direct consequence of \Cref{th:trace_norm_and_characteristic_polynomial_under_change_of_base_ring} by taking $M=A$ and $\varphi=l_b$.
\end{prf}

\begin{definition}{Trace Pairing}{trace_pairing}
    Let $R$ be a commutative ring and $A$ be an $R$-algebra. Suppose $A$ as an $R$-module is free of finite rank. The \textbf{trace pairing} is the symmetric $R$-bilinear form
    \begin{align*}
        \langle \cdot,\cdot \rangle_{A|R}:A\times A&\longrightarrow R\\
        (x,y)&\longmapsto \mathrm{Tr}_{A|R}(xy).
    \end{align*}
\end{definition}

\subsection{Discriminant}
We first define the discriminant of a polynomial.

\begin{definition}{Resultant}{}
    Let $R$ be a commutative ring and 
    \begin{align*}
        f(X)&=a_n X^n + a_{n-1}X^{n-1} + \cdots + a_1 X + a_0 \in R[X],\\
        g(X)&=b_m X^m + b_{m-1}X^{m-1} + \cdots + b_1 X + b_0 \in R[X]
    \end{align*}
    be two polynomials of degree $n$ and $m$ respectively. The \textbf{resultant} of $f(X)$ and $g(X)$ is defined as
    \[
    \mathrm{Res}(f,g):= \det \left[
\begin{array}{c@{\;}c}
  \underbrace{
    \begin{matrix}
      a_n     & 0       & \cdots & 0       \\
      a_{n-1} & a_n     & \cdots & 0       \\
      a_{n-2} & a_{n-1} & \ddots & \vdots  \\
      \vdots  & \vdots  & \ddots & a_n     \\
      a_0     & a_1     & \cdots & a_{n-1} \\
      0       & a_0     & \cdots & a_{n-2} \\
      \vdots  & \vdots  & \ddots & \vdots  \\
      0       & 0       & \cdots & a_0
    \end{matrix}
  }_{m \text{ columns}}
  &
  \underbrace{
    \begin{matrix}
      b_m     & 0       & \cdots & 0       \\
      b_{m-1} & b_m     & \cdots & 0       \\
      b_{m-2} & b_{m-1} & \ddots & \vdots  \\
      \vdots  & \vdots  & \ddots & b_m     \\
      b_0     & b_1     & \cdots & b_{m-1} \\
      0       & b_0     & \cdots & b_{m-2} \\
      \vdots  & \vdots  & \ddots & \vdots  \\
      0       & 0       & \cdots & b_0
    \end{matrix}
  }_{n \text{ columns}}
\end{array}
\right]_{(m+n) \times (m+n)}.
    \]
    If $R$ is an integral domain with field of fractions $K$ and $f(X),g(X)$ have roots $\alpha_1,\alpha_2,\cdots,\alpha_n$ and $\beta_1,\beta_2,\cdots,\beta_m$ in some algebraic closure $\overline{K}$ of $K$ respectively, then
    \[
    \mathrm{Res}(f,g)=a_n^m b_m^n \prod_{i=1}^n \prod_{j=1}^m (\alpha_i - \beta_j)=a_n^m \prod_{i=1}^n g(\alpha_i)=(-1)^{mn} b_m^n \prod_{j=1}^m f(\beta_j).
    \]
\end{definition}
\begin{remark}
    Suppose $R$ is an integral domain with field of fractions $K$ and $f(X),g(X)$ have roots $\alpha_1,\alpha_2,\cdots,\alpha_n$ and $\beta_1,\beta_2,\cdots,\beta_m$ in some algebraic closure $\overline{K}$ of $K$ respectively. Then we have
    \begin{align*}
        \frac{1}{a_n}f(X)&=(X-\alpha_1)(X-\alpha_2)\cdots (X-\alpha_n),\\
        \frac{1}{b_m}g(X)&=(X-\beta_1)(X-\beta_2)\cdots (X-\beta_m).
    \end{align*}
    Let 
    \[
    V = 
\begin{bmatrix}
\alpha_1^{m+n-1} & \alpha_1^{m+n-2} & \cdots & \alpha_1 & 1 \\
\vdots & \vdots & \ddots & \vdots & \vdots \\
\alpha_n^{m+n-1} & \alpha_n^{m+n-2} & \cdots & \alpha_n & 1 \\
\beta_1^{m+n-1} & \beta_1^{m+n-2} & \cdots & \beta_1 & 1 \\
\vdots & \vdots & \ddots & \vdots & \vdots \\
\beta_m^{m+n-1} & \beta_m^{m+n-2} & \cdots & \beta_m & 1
\end{bmatrix},\quad S=\begin{bmatrix}
a_n     & 0       & \cdots & 0       & b_m     & 0       & \cdots & 0       \\
a_{n-1} & a_n     & \cdots & 0       & b_{m-1} & b_m     & \cdots & 0       \\
a_{n-2} & a_{n-1} & \ddots & \vdots  & b_{m-2} & b_{m-1} & \ddots & \vdots  \\
\vdots  & \vdots  & \ddots & a_n     & \vdots  & \vdots  & \ddots & b_m     \\
a_0     & a_1     & \cdots & a_{n-1} & b_0     & b_1     & \cdots & b_{m-1} \\
0       & a_0     & \cdots & a_{n-2} & 0       & b_0     & \cdots & b_{m-2} \\
\vdots  & \vdots  & \ddots & \vdots  & \vdots  & \vdots  & \ddots & \vdots  \\
0       & 0       & \cdots & a_0     & 0       & 0       & \cdots & b_0
\end{bmatrix}.
    \]
    Then 
    \[
    {\renewcommand{\arraystretch}{1.5}}
    V S =  \begin{bmatrix}
    \alpha_1^{m-1}f(\alpha_1) & \alpha_1^{m-2}f(\alpha_1) & \cdots & f(\alpha_1) & \alpha_1^{n-1}g(\alpha_1)  & \alpha_1^{n-2}g(\alpha_1) &\cdots & g(\alpha_1) \\
     \alpha_2^{m-1}f(\alpha_2) & \alpha_2^{m-2}f(\alpha_2) & \cdots & f(\alpha_2) & \alpha_2^{n-1}g(\alpha_2)  & \alpha_2^{n-2}g(\alpha_2) &\cdots & g(\alpha_2)  \\
    \vdots & \vdots & \ddots & \vdots & \vdots & \vdots & \ddots & \vdots \\
    \alpha_n^{m-1}f(\alpha_n) & \alpha_n^{m-2}f(\alpha_n) & \cdots & f(\alpha_n) & \alpha_n^{n-1}g(\alpha_n)  & \alpha_n^{n-2}g(\alpha_n) &\cdots & g(\alpha_n)  \\
    \beta_1^{m-1}f(\beta_1) & \beta_1^{m-2}f(\beta_1) & \cdots & f(\beta_1) & \beta_1^{n-1}g(\beta_1)  & \beta_1^{n-2}g(\beta_1) &\cdots & g(\beta_1)  \\
    \vdots & \vdots & \ddots & \vdots & \vdots & \vdots & \ddots & \vdots \\
    \beta_m^{m-1}f(\beta_m) & \beta_m^{m-2}f(\beta_m) & \cdots & f(\beta_m) & \beta_m^{n-1}g(\beta_m)  & \beta_m^{n-2}g(\beta_m) &\cdots & g(\beta_m)
    \end{bmatrix}=\begin{bmatrix}
    \mathbf{0}_{n\times m} & D_g \\
    D_f & \mathbf{0}_{m\times n}.
    \end{bmatrix}
    \]
    This implies that
    \begin{align*}
        \det(V)\det(S)&=\det(V S)\\
        &=(-1)^{mn}\det(D_g)\det(D_f)\\
        &=(-1)^{mn} \left(\prod_{i=1}^ng(\alpha_i)\prod_{1\le i<j\le n}(\alpha_i-\alpha_j)\right) \left(\prod_{j=1}^m f(\beta_j)\prod_{1\le i<j\le m}(\beta_i-\beta_j)\right).
    \end{align*}
    Since $\det(V)$ is a Vandermonde determinant, we have
    \[
    \det(V)=\prod_{1\le i<j\le n}(\alpha_i-\alpha_j)\prod_{1\le i<j\le m}(\beta_i-\beta_j)\prod_{i=1}^n\prod_{j=1}^m(\alpha_i-\beta_j)
    \]
    Thus
    \[
    \det(S)=\frac{(-1)^{mn}\displaystyle\prod_{i=1}^n g(\alpha_i)\prod_{j=1}^m f(\beta_j)}{\displaystyle\prod_{i=1}^n\prod_{j=1}^m(\alpha_i-\beta_j)}.
    \]
    Note 
    \begin{align*}
        &\;\prod_{i=1}^n\prod_{j=1}^m(\alpha_i-\beta_j)=\prod_{i=1}^n \frac{g(\alpha_i)}{b_m}=\frac{1}{b_m^n} \prod_{i=1}^n g(\alpha_i)\\
         =&\;(-1)^{mn}\prod_{j=1}^m\prod_{i=1}^n(\beta_j-\alpha_i)=(-1)^{mn}\prod_{j=1}^m \frac{f(\beta_j)}{a_n}=\frac{(-1)^{mn}}{a_n^m} \prod_{j=1}^m f(\beta_j).
    \end{align*}
    We can conclude that
    \[
    \det(S)=a_n^m b_m^n \prod_{i=1}^n \prod_{j=1}^m (\alpha_i - \beta_j)=a_n^m \prod_{i=1}^n g(\alpha_i)=(-1)^{mn} b_m^n \prod_{j=1}^m f(\beta_j).
    \]
\end{remark}

\begin{definition}{Discriminant of Polynomial}{discriminant_of_polynomial}
    Let $R$ be an integral domain and $f(X)\in R[X]$ be a polynomial. The \textbf{discriminant} of $f(X)$ is defined as
    \[
    \mathrm{Disc}(f):=(-1)^{\frac{n(n-1)}{2}}\frac{\mathrm{Res}(f,f')}{a_n}.
    \]
\end{definition}

\begin{proposition}{}{}
    Let $K$ be a field and $f(X)\in K[X]$ be a polynomial of degree $n$ with roots $\alpha_1,\alpha_2,\cdots,\alpha_n$ in some algebraic closure $\overline{K}$ of $K$. Then
    \[
    \mathrm{Disc}(f)=(-1)^{\frac{n(n-1)}{2}}   a_n^{n-2} \prod_{i=1}^n f'(\alpha_i) = (-1)^{\frac{n(n-1)}{2}} a_n^{2n-2} \prod_{\substack{i,j \\ i \ne j}}(\alpha_i-\alpha_j)=a_n^{2n-2}\prod_{1\le i<j\le n}(\alpha_i-\alpha_j)^2.
    \]
\end{proposition}
\begin{prf}
    By definition of resultant, we have
    \[
      \mathrm{Disc}(f)=(-1)^{\frac{n(n-1)}{2}}\frac{\mathrm{Res}(f,f')}{a_n}=(-1)^{\frac{n(n-1)}{2}}\frac{a_n^{n-1} \prod\limits_{i=1}^n f'(\alpha_i)}{a_n}=(-1)^{\frac{n(n-1)}{2}} a_n^{n-2} \prod_{i=1}^n f'(\alpha_i).
    \]
    Since 
    \begin{align*}
        f'(X)&=\frac{\mathrm{d}}{\mathrm{d}X}\left(a_n \prod_{i=1}^n (X - \alpha_i) \right)= a_n \sum_{i=1}^n \prod_{j\ne i} (X - \alpha_j)\implies f'(\alpha_i)= a_n \prod_{j\ne i} (\alpha_i - \alpha_j),
    \end{align*}
    we have
    \[
    \prod_{i=1}^n f'(\alpha_i)= \prod_{i=1}^n \left( a_n \prod_{j\ne i} (\alpha_i - \alpha_j) \right)= a_n^{n} \prod_{\substack{i,j \\ i \ne j}} (\alpha_i - \alpha_j).
    \]
    Thus
    \[
    \mathrm{Disc}(f)=(-1)^{\frac{n(n-1)}{2}} a_n^{n-2} \prod_{i=1}^n f'(\alpha_i)=(-1)^{\frac{n(n-1)}{2}} a_n^{2n-2} \prod_{\substack{i,j \\ i \ne j}} (\alpha_i - \alpha_j).
    \]
    Note
    \begin{align*}
        \prod_{\substack{i,j \\ i \ne j}} (\alpha_i - \alpha_j)&=\prod_{1\le i<j\le n} (\alpha_i - \alpha_j)(\alpha_j - \alpha_i)\\
        &=\prod_{1\le i<j\le n} -(\alpha_i - \alpha_j)^2\\
        &=(-1)^{\frac{n(n-1)}{2}} \prod_{1\le i<j\le n} (\alpha_i - \alpha_j)^2.
    \end{align*}
    We get
    \[
    \mathrm{Disc}(f)=(-1)^{\frac{n(n-1)}{2}} a_n^{2n-2} \prod_{\substack{i,j \\ i \ne j}} (\alpha_i - \alpha_j)= a_n^{2n-2}\prod_{1\le i<j\le n}(\alpha_i-\alpha_j)^2.
    \]
\end{prf}

\begin{definition}{Discriminant of Bilinear Form $M\times M\to R$}{}
    Let $R$ be a commutative ring and $M$ be an free $R$-module of rank $n$. Let $Q:M\times M\to R$ be a $R$-bilinear form. Then
    \begin{align*}
        \varphi_Q:M &\longrightarrow M^*\\
        x &\longmapsto Q(x,-)
    \end{align*}
    is an $R$-module homomorphism. Taking top exterior powers gives an $R$-module homomorphism
    \begin{align*}
        \Largewedge^{n}(\varphi_Q):\Largewedge^{n}(M) &\longrightarrow \Largewedge^{n}(M^*)\\
        x_1\wedge\cdots\wedge x_n &\longmapsto \varphi_Q(x_1)\wedge\cdots\wedge \varphi_Q(x_n).
    \end{align*}
    Note we have natural isomorphisms
    \begin{align*}
        \psi_M:\Largewedge^{n}(M^*)&\xlongrightarrow{\sim} (\Largewedge^{n}(M))^*\\
        f_1\wedge\cdots\wedge f_n &\longmapsto \left( x_1\wedge\cdots\wedge x_n \mapsto \det\left( [f_i(x_j)]_{1\le i,j\le n} \right) \right),
    \end{align*}
    By composing $\Largewedge^{n}(\varphi_Q)$ with $\psi_M$, we get an $R$-module endomorphism of the rank-$1$ free $R$-module $\Largewedge^{n}(M)$
    \begin{align*}
        \theta:=\psi_M \circ \Largewedge^{n}(\varphi_Q):\Largewedge^{n}(M) &\longrightarrow (\Largewedge^{n}(M))^*\\
        x_1\wedge\cdots\wedge x_n &\longmapsto \left(x_1\wedge\cdots\wedge x_n\mapsto \det\left( [Q(x_i,x_j)]_{1\le i,j\le n} \right) \right).
    \end{align*}
    Let $\varepsilon$ be a basis of $\Largewedge^{n}(M)$ and
    \begin{align*}
        \varepsilon^*:\Largewedge^{n}(M) &\longrightarrow R\\
        \varepsilon &\longmapsto 1_R.
    \end{align*}
    be the dual basis of $\varepsilon$. Then there exists a unique $\lambda\in R$ such that
    \[
        \theta(\varepsilon) = \lambda \cdot \varepsilon^*.
    \]
    Then the \textbf{discriminantof $Q$} is defined as the image of $\lambda$ under the canonical projection $R\to R/(R^{\times})^2$
    \[
    \mathrm{Disc}(Q):=\lambda \mod (R^{\times})^2 \in R/(R^{\times})^2,
    \]
    where $R/(R^{\times})^2$ is the quotient monoid of the multiplicative monoid $R$ by modulo the congruence relation 
    \[
    a \sim b \iff \exists u\in R^{\times}, a = u^2 b.
    \]
    We can check that this definition is independent of the choice of basis $\varepsilon$. If we choose another basis $\tilde{\varepsilon} = c \varepsilon$ for some $c\in R^{\times}$, then $\tilde{\varepsilon}^* = c^{-1} \varepsilon^*$ and 
    \[
        \theta(\tilde{\varepsilon}) = \theta(c \varepsilon) = c \theta(\varepsilon) = c \lambda \varepsilon^* =c \lambda (c \tilde{\varepsilon}^*) = (c^2 \lambda)\cdot \tilde{\varepsilon}^*.
    \]
    If $E=\{e_1,e_2,\cdots,e_n\}$ is a basis of $M$, then the \textbf{discriminant of $Q$ with respect to the basis $\{e_1,e_2,\cdots,e_n\}$} is defined as
    \[
    \mathrm{Disc}(Q; e_1,e_2,\cdots,e_n):=\det\left( [Q(e_i,e_j)]_{1\le i,j\le n} \right)\in R.
    \]
    The relation between $\mathrm{Disc}(Q)$ and $\mathrm{Disc}(Q; e_1,e_2,\cdots,e_n)$ is given by
    \[
    \mathrm{Disc}(Q)=\mathrm{Disc}(Q; e_1,e_2,\cdots,e_n) \mod (R^{\times})^2 \in R/(R^{\times})^2.
    \]
    The ideal of $R$ generated by $\mathrm{Disc}(Q; e_1,e_2,\cdots,e_n)$ is called the \textbf{discriminant ideal of $Q$}, which is independent of the choice of basis $\{e_1,e_2,\cdots,e_n\}$.
\end{definition}
\begin{remark}
    Here we check the relation between $\mathrm{Disc}(Q)$ and $\mathrm{Disc}(Q; e_1,e_2,\cdots,e_n)$. Let $\langle \cdot,\cdot \rangle :M^*\times M \to R$ be the dual pairing defined by $\langle f,x \rangle := f(x)$. Note $\varepsilon:=e_1\wedge e_2 \wedge \cdots \wedge e_n$ is a basis of $\Largewedge^{n}(M)$. Then we have 
    \[
    \langle\theta(\varepsilon),\varepsilon \rangle= \langle \lambda\varepsilon^*,\varepsilon \rangle =\lambda \langle \varepsilon^*,\varepsilon \rangle = \lambda.
    \]
    Denote
    \[
    f_i:=\varphi_Q(e_i) \in M^*.
    \]
    We have
    \begin{align*}
        \langle\theta(\varepsilon),\varepsilon \rangle &= \langle \psi_M(f_1\wedge f_2 \wedge \cdots \wedge f_n), e_1\wedge e_2 \wedge \cdots \wedge e_n \rangle \\
        &= \det\left( [f_i(e_j)]_{1\le i,j\le n} \right) \\
        &= \det\left( [\varphi_Q(e_i)(e_j)]_{1\le i,j\le n} \right) \\
        &= \det\left( [Q(e_i,e_j)]_{1\le i,j\le n} \right).
    \end{align*}
\end{remark}

\begin{definition}{Discriminant of Algebra}{discriminant_of_algebra}
    Let $R$ be a commutative ring and $A$ be an $R$-algebra. Suppose $A$ is a free $R$-module of rank $n$. Then the \textbf{discriminant} of $A$ over $R$ is defined as
    \[
       d_A=\mathrm{Disc}\left(\langle \cdot ,\cdot \rangle_{A|R}\right)
    \]
    where $\langle \cdot ,\cdot \rangle_{A|R}$ is the \hyperref[th:trace_pairing]{trace pairing} of $A$ over $R$. 
    
    Assume $\{e_1,e_2,\cdots,e_n\}$ is a basis of $A$ over $R$. The \textbf{discriminant} of $A$ with respect to the basis $\{e_1,e_2,\cdots,e_n\}$ is defined as
    \begin{align*}
        d_A(e_1,e_2,\cdots,e_n):&=\det\left( \left[ \langle e_i,e_j \rangle_{A|R} \right]_{1\le i,j\le n} \right)\\
        &= \det \begin{bmatrix}
        \mathrm{Tr}_{A|R}(e_1 e_1) & \mathrm{Tr}_{A|R}(e_1 e_2) & \cdots & \mathrm{Tr}_{A|R}(e_1 e_n) \\
        \mathrm{Tr}_{A|R}(e_2 e_1) & \mathrm{Tr}_{A|R}(e_2 e_2) & \cdots & \mathrm{Tr}_{A|R}(e_2 e_n) \\
        \vdots & \vdots & \ddots & \vdots \\
        \mathrm{Tr}_{A|R}(e_n e_1) & \mathrm{Tr}_{A|R}(e_n e_2) & \cdots & \mathrm{Tr}_{A|R}(e_n e_n)
        \end{bmatrix}\in R.
    \end{align*}
    The relation between $d_A$ and $d_A(e_1,e_2,\cdots,e_n)$ is given by
    \begin{align*}
        d_A=d_A(e_1,e_2,\cdots,e_n)\mod (R^{\times})^2.
    \end{align*}
    The ideal of $R$ generated by $d_A(e_1,e_2,\cdots,e_n)$ is called the \textbf{discriminant ideal
    of $A$ over $R$} and is denoted by $\mathfrak{d}_A$.
\end{definition}



\begin{definition}{Order}{}
    Let $R$ be an integral domain with field of fractions $K$ and $A$ be a finite-dimensional $K$-algebra. An $R$-\textbf{order} in $A$ is a unital $R$-subalgebra $\mathcal{O}\subseteq A$ which is finitely generated as an $R$-module and satisfies $\mathcal{O}\otimes_R K \cong A$.
\end{definition}
\begin{remark}
    Since the $K$-algebra $A$ can be regarded as an $R$-algebra through the restriction of scalars along $R\hookrightarrow K$, it makes sense to talk about $R$-subalgebras of $A$.
\end{remark}

\begin{lemma}{}{discriminant_relation}
    Let $R$ be a commutative ring, $M$ be a free $R$-module of rank $n$ and $N\subseteq M$ be a submodule which is also a free $R$-module of rank $n$. Let $Q:M\times M\to R$ be a $R$-bilinear form. 
    \begin{enumerate}[(i)]
    \item Given any basis $\{e_1,e_2,\cdots,e_n\}$ of $M$ over $R$ and any basis $\{f_1,f_2,\cdots,f_n\}$ of $N$ over $R$, there exists a matrix $P\in M_n(R)$ such that
    \[
    \begin{bmatrix} 
f_1 & f_2 & \cdots & f_n
\end{bmatrix} = \begin{bmatrix} 
e_1 & e_2 & \cdots & e_n
\end{bmatrix} P.
    \]
    The discriminants with respect to the bases $\{e_1,e_2,\cdots,e_n\}$ and $\{f_1,f_2,\cdots,f_n\}$ satisfy
    \[
    \mathrm{Disc}\left(Q|_{N\times N};f_1,f_2,\cdots,f_n\right)=\det(P)^2\; \mathrm{Disc}\left(Q;e_1,e_2,\cdots,e_n\right).
    \]
    \item Let $\mathrm{Fitt}_0(M/N)$ be the 0-th Fitting ideal of the $R$-module $M/N$. Then we have equality of ideals
    \[
    \left(\mathrm{Disc}(Q|_{N\times N})\right)= \mathrm{Fitt}_0(M/N)^2 \left(\mathrm{Disc}(Q)\right).
    \]
    \end{enumerate}
\end{lemma}
\begin{prf}
     \begin{enumerate}[(i)]
    \item Let $\{e_1,e_2,\cdots,e_n\}$ be a basis of $M$ over $R$. Since $N$ is a unital $R$-submodule of $M$ which is also a free $R$-module of rank $n$, we can assume that $\{f_1,f_2,\cdots,f_n\}$ is a basis of $N$ over $R$. Note each $f_i$ can be expressed as a linear combination of $e_1,e_2,\cdots,e_n$. Thus there exists a matrix $M\in M_n(R)$ such that
    \[
    \begin{bmatrix}
    f_1 & f_2 & \cdots & f_n
    \end{bmatrix} = \begin{bmatrix}
    e_1 & e_2 & \cdots & e_n
    \end{bmatrix} P.
    \]
    Thus we have
    \begin{align*}
        \mathrm{Disc}\left(Q|_{N\times N};f_1,f_2,\cdots,f_n\right)&=\det\left( \left[ \langle f_i,f_j \rangle_{Q|_{N\times N}} \right]_{1\le i,j\le n} \right)\\
        &=\det\left( P^\top \left[ \langle e_i,e_j \rangle_{Q} \right]_{1\le i,j\le n} P \right)\\
        &=\det(P)^2 \det\left( \left[ \langle e_i,e_j \rangle_{Q} \right]_{1\le i,j\le n} \right)\\
        &=\det(P)^2\: \mathrm{Disc}\left(Q;e_1,e_2,\cdots,e_n\right).
    \end{align*}
    \item By identifying matrix $P$ and its corresponding linear map $P:R^n\to R^n$, the $R$-module $M/N$ has a presentation
    \[
    R^n \xlongrightarrow{P} R^n \xlongrightarrow{\pi} M/N \longrightarrow 0
    \]
    where $\pi$ is defined by 
    \[ 
    \pi(0, \cdots, \underset{\text{i-th component}}{1_R}, \cdots, 0) = e_i + N 
    \]
    for $1\le i\le n$. By definition, the 0-th Fitting ideal of $M/N$ is generated by $\det(P)$
    \[
    \mathrm{Fitt}_0(M/N) = (\det(P)).
    \]
      \end{enumerate}
\end{prf}

\begin{corollary}{Discriminant Relation}{}
    Let $R$ be a commutative ring and $A$ an $R$-algebra which is a free $R$-module of rank $n$. Let $B\subseteq A$ be a unital $R$-subalgebra of $A$ which is also a free $R$-module of rank $n$. Then for any basis $\{e_1,e_2,\cdots,e_n\}$ of $A$ over $R$ and any basis $\{b_1,b_2,\cdots,b_n\}$ of $B$ over $R$, there exists a matrix $M\in M_n(R)$ such that
    \[
    \begin{bmatrix}
    b_1 & b_2 & \cdots & b_n
    \end{bmatrix} = \begin{bmatrix}
    e_1 & e_2 & \cdots & e_n
    \end{bmatrix} M.
    \]
    The discriminants with respect to the bases $\{e_1,e_2,\cdots,e_n\}$ and $\{b_1,b_2,\cdots,b_n\}$ satisfy
    \[
    d_B(b_1,b_2,\cdots,b_n)=\det(M)^2\; d_A(e_1,e_2,\cdots,e_n).
    \]
    In particular, if $R=\mathbb{Z}$, then $\mathbb{Z}/(\mathbb{Z}^{\times})^2 \cong \mathbb{Z}$ and we have
    \[
    d_B = [A:B]^2 d_A,
    \]
    where $[A:B]$ is the index of $B$ in $A$ as abelian groups.
\end{corollary}
\begin{prf}
    Note 
    \[
    \left(\langle \cdot ,\cdot \rangle_{A|R}\right)|_{B\times B} = \langle \cdot ,\cdot \rangle_{B|R}.
    \]
    $d_B(b_1,b_2,\cdots,b_n)=\det(M)^2\; d_A(e_1,e_2,\cdots,e_n)$ is a direct consequence of \Cref{th:discriminant_relation}. If $R=\mathbb{Z}$, then $\det(M)=\pm [A:B]$ and $\mathbb{Z}/(\mathbb{Z}^{\times})^2 \cong \mathbb{Z}$. Thus
\end{prf}

\section{Algebra over Field}
\begin{lemma}{Nonzero Ring Homomorphism from Field is Injective}{nonzero_ring_homomorphism_from_field_is_injective}
    If $K$ is a field, $R$ is a ring, a ring homomorphism $f:K\to R$ is either injective or the zero map. Furthermore, If $R$ is not a zero ring, then $f$ is injective.
\end{lemma}
\begin{prf}
    Since the only ideals of $K$ are $\{0\}$ and $K$, the kernel of $f$ is either $\{0\}$ or $K$. If $\ker f=\{0\}$, then $f$ is injective. If $\ker f=K$, then $f$ is the zero map. By \Cref{th:kernel_of_ring_homomorphism_is_an_ideal}, if $R$ is not a zero ring, then $\ker f$ is not $K$, so $f$ is injective.
\end{prf}

\begin{corollary}{}{}
    If $K$ is a field and $A$ is a nonzero $K$-algebra, then the ring homomorphism $K\to Z(A)$ is injective.
\end{corollary}
\begin{prf}
    This is a direct consequence of \Cref{th:nonzero_ring_homomorphism_from_field_is_injective}.
\end{prf}


\begin{proposition}{Structure of $K[a]$}{structure_of_K_a}
    Let $K$ be a field, $A$ be a $K$-algebra and $a\in A$. Consider the evaluation ring homomorphism
    \begin{align*}
        \mathrm{ev}_a:K[X] &\longrightarrow A\\
        f &\longmapsto f(a).
    \end{align*}
    Since $K[X]$ is a PID, we can suppose $\ker \mathrm{ev}_a=(P_a)$ for some $P_a\in K[X]$. Since $\operatorname{im}\mathrm{ev}_a=K[a]$, we have the following isomorphism in $K$-$\mathsf{Alg}$
    \[
        K[a]\cong K[X]/(P_a(X)).
    \]
    And it can be divided into two cases:
    \begin{enumerate}[(i)]
        \item If $P_a=0$, then $\mathrm{ev}_a$ is injective and $K[a]\cong K[X]$.
        \item If $P_a\ne 0$, then $\mathrm{ev}_a$ is not injective. If we further assume $A$ is a \hyperref[th:domain]{domain}, then $P_a(X)$ is irreducible, $K[a]$ is a field and
        \[
        \left[ K[a]:K \right]=\deg P_a(X).
        \]
    \end{enumerate}
    Moreover, 
    \[
    \deg P_a(X) = 0\iff A \text{ is a zero ring}.
    \]
\end{proposition}
\begin{prf}
    \begin{enumerate}[(i)]
        \item If $P_a=0$, then $\ker \mathrm{ev}_a=\{0\}$, so $\mathrm{ev}_a$ is injective. And $K[a]\cong K[X]$.
        \item If $A$ is a domain, then $K[a]$ as a subring of $A$ is an integral domain. This implies $(P_a(X))$ is a nonzero prime ideal of $K[X]$. By \Cref{th:nonzero_prime_ideal_iff_maximal_ideal_in_PID}, $P_a(X)$ is irreducible. Since $K[X]/(P_a(X))$ as $K$-vector space has a basis $\{1,X,X^2,\cdots,X^{\deg P_a(X)-1}\}$, we have $\left[ K[a]:K \right]=\deg P_a(X)$.
    \end{enumerate}
   And we have
   \[
   \deg P_a(X) = 0\iff P_a\in K^\times \iff \ker \mathrm{ev}_a=(P_a)=K[X] \iff \mathrm{ev}_a \text{ is the zero map } \iff A \text{ is a zero ring}.
   \]
\end{prf}

\begin{definition}{Algebraic Element and Transcendental Element}{algebraic_element_and_transcendental_element}
    Let $K$ be a field, $A$ be a $K$-algebra, and $a \in A$. Consider the evaluation ring homomorphism
    \begin{align*}
        \mathrm{ev}_a:K[X] &\longrightarrow A\\
        f &\longmapsto f(a).
    \end{align*}
    and $\ker \mathrm{ev}_a=(P_a)$ for some $P_a\in K[X]$. Polynomials in $\ker \mathrm{ev}_a$ are called \textbf{annihilating polynomials} of $a$ over $K$.
    \begin{itemize}
        \item If $P_a=0$, then $a$ is called a \textbf{transcendental element} over $K$. $a$ is not the root of any nonzero polynomial in $K[X]$. 
        \item If $P_a\ne 0$, then $a$ is called an \textbf{algebraic element} over $K$. Suppose $P_a(X)=\sum_{i=0}^n a_iX^i$ with $a_n\in K^\times$. Then the monic polynomial $m_a(X)=P_a(X)/a_n$ is called the \textbf{minimal polynomial}\index{minimal polynomial} of $a$ over $K$. According to \Cref{th:structure_of_K_a}, if we further assume $A$ is a \hyperref[th:domain]{domain}, then $m_a(X)$ is irreducible.
    \end{itemize}
\end{definition}

If $K$ is a field and $A=\{0_A\}$ is a zero $K$-algebra, then $0_A\in A$ is algebraic over $K$ with minimal polynomial $m_{0_A}(X)=1_K$.

\begin{proposition}
{Algebraic Element is Integral}{algebraic_element_is_integral}
    Let $K$ be a field and $A$ be a $K$-algebra. Then $a\in A$ is algebraic over $K$ if and only if $a$ is integral over $K$.
\end{proposition}
\begin{prf}
Suppose $a$ is algebraic over $K$. Then the minimal polynomial $m_a(X)\in K[X]$ is a monic polynomial such that $m_a(a)=0$. Thus $a$ is integral over $K$.
\end{prf}

\begin{proposition}{Monic Irreducible Annihilating Polynomial is Minimal Polynomial}{irreducible_annihilating_polynomial_is_minimal_polynomial}
    Let $K$ be a field, $A$ be a nonzero $K$-algebra and $a\in A$ be an algebraic element over $K$. If $f_a(X)\in K[X]$ is a monic irreducible annihilating polynomial of $a$ over $K$, then $f_a(X)$ is the minimal polynomial of $a$ over $K$.
\end{proposition}
\begin{prf}
    Let $m_a(X)$ be the minimal polynomial of $a$ over $K$. Since $f_a$ is irreducible and $f_a \in \ker \mathrm{ev}_a = (m_a)$, by \Cref{th:dividing_irreducible_element_in_PID_implies_associate_or_unit}, either $m_a\in K[X]^\times=K^\times$ or $f_a$ and $m_a$ are associates. If $m_a\in K^\times$, then $\deg m_a=0$, which implies that $A$ is a zero ring by \Cref{th:structure_of_K_a}, contradicting the assumption. Thus $f_a$ and $m_a$ are associates. Since both $f_a$ and $m_a$ are monic, we have $f_a=m_a$.
\end{prf}



\chapter{Commutative Unital Algebra}
\section{Basic Properties}
\begin{definition}{Commutative Algebra}{}
    Let $R$ be a commutative ring. A \textbf{commutative $R$-algebra} is an $R$-algebra where the multiplication is commutative. Or equivalently, a commutative $R$-algebra is a commutative ring $A$ together with a ring homomorphism $R\to A$. 
\end{definition}

\begin{remark}
    There is a category isomorphism $R\text{-}\mathsf{CAlg}\cong \left(R/\mathsf{CRing}\right)$.
\end{remark}

\section{Polynomial Algebra}
\begin{definition}{Polynomial Ring}{}
    Let $R$ be a commutative ring. The \textbf{polynomial ring} in $n$ variables over $R$ is the ring $R[x_1,\cdots,x_n]$ defined as the set of all formal sums $$\sum_{\alpha\in\mathbb{N}^n}a_\alpha x^\alpha$$ where $a_\alpha\in R$ satisfies $a_\alpha=0$ for all but finitely many $\alpha\in\mathbb{N}^n$ and $x^\alpha:=x_1^{\alpha_1}\cdots x_n^{\alpha_n}$ for $\alpha=(\alpha_1,\cdots,\alpha_n)\in\mathbb{N}^n$. The addition and multiplication are defined as follows: $$\sum_{\alpha\in\mathbb{N}^n}a_\alpha x^\alpha+\sum_{\alpha\in\mathbb{N}^n}b_\alpha x^\alpha=\sum_{\alpha\in\mathbb{N}^n}(a_\alpha+b_\alpha)x^\alpha$$ and $$\left(\sum_{\alpha\in\mathbb{N}^n}a_\alpha x^\alpha\right)\left(\sum_{\beta\in\mathbb{N}^n}b_\beta x^\beta\right)=\sum_{\gamma\in\mathbb{N}^n}\left(\sum_{\alpha+\beta=\gamma}a_\alpha b_\beta\right)x^\gamma.$$
\end{definition}


\begin{proposition}{Properties of Polynomial Ring}{properties_of_polynomial_ring}
    Let $R$ be a commutative ring.
    \begin{enumerate}[(i)]
        \item If $R$ is a UFD, then $R[x_1,\cdots.x_n]$ is a UFD.
        \item $R$ is a field $\iff$ $R[x]$ is a PID $\iff$ $R[x]$ is an Euclidean domain.
        \item $R$ is an integral domain $\iff$ $R[x]$ is an integral domain.
        \item $R$ is Noetherian $\implies$ $R[x]$ is Noetherian.
        \item $R$ is reduced $\implies$ $R[x]$ is reduced.
    \end{enumerate}
\end{proposition}



\begin{proposition}{Division Algorithm in Polynomial Ring}{division_algorithm_in_polynomial_ring}
    Let $R$ be a commutative ring and $f,g\in R[x]$ be nonzero polynomials. If the leading coefficient of $g$ is in $R^\times$, then there exist unique polynomials $q,r\in R[x]$ such that $f=qg+r$ and $\deg r<\deg g$.
\end{proposition}
\begin{prf}
    We can prove this by induction on $\deg f$. The base case is $\deg f=0$. If $g=a_0\in R^\times$, then we can take $q=f/a_0$ and $r=0$. If $\deg g\ge 1$, then we can take $q=0$ and $r=f$.

    Suppose the statement holds for any $h\in R[x]$ with $\deg h < n$. Let $f\in R[x]$ be a polynomial of degree $n$. If $\deg f<\deg g$, then we can take $q=0$ and $r=f$. If $\deg f\ge \deg g$. Suppose
    \[
        f=\sum_{i=0}^n a_ix^i\quad\text{and}\quad g=\sum_{i=0}^m b_ix^i
    \]
    where $a_n,b_m\ne 0$. Let $h=f-\frac{a_n}{b_m} x^{n-m}g$. Then $\deg h<\deg f$. By induction hypothesis, there exist $\tilde{q},\tilde{r}\in R[x]$ such that $h=\tilde{q}g+\tilde{r}$ and $\deg \tilde{r}<\deg g$. Thus there exist 
    \[
        q=\tilde{q}+\frac{a_n}{b_m} x^{n-m}\quad\text{and}\quad r=\tilde{r}
    \]
    such that $f=qg+r$ and $\deg r<\deg g$. If there are $Q$ and $R$ such that $f=Qg+R$ and $\deg R<\deg g$, then we have
    \[
    h=f-\frac{a_n}{b_m} x^{n-m}g=\left(Q-\frac{a_n}{b_m} x^{n-m}\right)g+R=\tilde{q}g+\tilde{r}.
    \]
    By uniqueness, we have $Q-\frac{a_n}{b_m} x^{n-m}=\tilde{q}$ and $R=\tilde{r}$, which implies $Q=q$ and $R=r$.
\end{prf}

\begin{corollary}{Polynomial Remainder Theorem}{polynomial_remainder_theorem}
    Let $R$ be a commutative ring and $f(x)\in R[x]$ be a polynomial. If $a\in R$, then there exist a unique polynomial $q(x)\in R[x]$ such that $f(x)=q(x)(x-a)+f(a)$.
\end{corollary}
\begin{prf}
    This is a direct application of \Cref{th:division_algorithm_in_polynomial_ring}.
\end{prf}


\begin{corollary}{}{}
    Let $R$ be a commutative ring.
    \begin{enumerate}[(i)]
        \item Let $a\in R$ and $f_1(x),\cdots, f_r(x)\in R[x]$ be polynomials. Then we have
        \[
        (f_1(x),\cdots,f_r(x),x-a)=(f_1(a),\cdots,f_r(a),x-a).
        \]
        \item Let $a\in R$ and $f_1(x),\cdots, f_r(x)\in R[x]$ be polynomials. Then we have 
    \[
        \frac{R[x]}{(f_1(x),\cdots,f_r(x),x-a)}\cong \frac{R}{\left(f_1(a),\cdots,f_r(a) \right)}.
    \]
    \item Suppose $a_1,\cdots,a_n\in R$. Then we can define a \textbf{evaluation homomorphism} 
    \begin{align*}
        \mathrm{ev}_{a_1,\cdots,a_n}:R[x_1,\cdots,x_n]&\longrightarrow R\\
        \sum_{\alpha\in\mathbb{N}^n}r_\alpha x^\alpha&\longmapsto \sum_{\alpha\in\mathbb{N}^n}r_\alpha a_1^{\alpha_1}\cdots a_n^{\alpha_n}.
    \end{align*}
    The kernel of $\mathrm{ev}_{a_1,\cdots,a_n}$ is 
    \[
    \ker \mathrm{ev}_{a_1,\cdots,a_n}=(x_1-a_1,\cdots,x_n-a_n)
    \]
    and we have
    \[
    \frac{R[x_1,\cdots,x_n]}{(x_1-a_1,\cdots,x_n-a_n)}\cong R.
    \]
    \end{enumerate}
    
    
\end{corollary}
\begin{prf}
    \begin{enumerate}[(i)]
        \item 
        By the \hyperref[th:polynomial_remainder_theorem]{polynomial remainder theorem}, we have
        \[
        f_i(x)=q_i(x)(x-a)+f_i(a),\quad i=1,2,\cdots,r,
        \]
        which implies
        \[
        (f_1(x),\cdots,f_r(x),x-a)=(f_1(a),\cdots,f_r(a),x-a).
        \]
        \item First we show the kernel of 
        \begin{align*}
            \mathrm{ev}_a:R[x]&\longrightarrow R\\
            f(x)&\longmapsto f(a)
        \end{align*}
        is $\ker \mathrm{ev}_a=(x-a)$.
        By the \hyperref[th:polynomial_remainder_theorem]{polynomial remainder theorem}, we have
        \[
        f(x)=q(x)(x-a)+f(a)
        \]
        Note
        \[
        f(x)\in \ker \mathrm{ev}_{a} \iff f(a)=0\iff f(x)\in (x-a).
        \]
        We have $\ker \mathrm{ev}_a=(x-a)$ and $R[x]/(x-a)\cong R$. 

        From \Cref{ex:coset_of_generated_ideals_quotient_generates_ideals}, we have the following equality of ideals in $R[x]/(x-a)$
        \[
            (f_1(x),\cdots,f_r(x),x-a)/(x-a)=\left(f_1(x)+(x-a),\cdots,f_r(x)+(x-a)\right).
        \]
        By the third isomorphism theorem, we have
        \[
        \frac{R[x]}{(f_1(x),\cdots,f_r(x),x-a)}\cong \frac{R[x]/(x-a)}{\left(f_1(x)+(x-a),\cdots,f_r(x)+(x-a)\right)}.
        \]
        Apply the isomorphism 
        \begin{align*}
            \overline{\mathrm{ev}_a}:R[x]/(x-a)&\longrightarrow R\\
            f(x)+(x-a)&\longmapsto f(a)
        \end{align*}
        we get
        \[
            \frac{R[x]/(x-a)}{\left(f_1(x)+(x-a),\cdots,f_r(x)+(x-a)\right)} \cong \frac{R}{\left(f_1(a),\cdots,f_r(a) \right)}.
        \]
        \item We can prove 
        \[
        \ker \mathrm{ev}_{a_1,\cdots,a_n}=(x_1-a_1,\cdots,x_n-a_n)
        \]
        by induction on $n$. The base case is $n=1$, which has been proved in (ii). Suppose the statement holds for $n-1$. Let $f(x_1,\cdots,x_n)\in R[x_1,\cdots,x_n]$.
        \[
        \ker \mathrm{ev}_{a_1,\cdots,a_n}=\ker \left(\mathrm{ev}_{a_n}\circ \mathrm{ev}_{a_1,\cdots,a_{n-1}}\right)= \mathrm{ev}_{a_n}^{-1}(\left(x_1-a_1,\cdots,x_n-a_n\right)).
        \]
        
        By the \hyperref[th:polynomial_remainder_theorem]{polynomial remainder theorem}, we have
        and we have
    \[
    \frac{R[x_1,\cdots,x_n]}{(x_1-a_1,\cdots,x_n-a_n)}\cong R.
    \]
    \end{enumerate}
\end{prf}

\begin{example}{$R$-algebra Endomorphisms of $R[x]$}{}
    Let $R$ be a commutative ring. By the universal property of free commutative $R$-algebra, we have the following isomorphism
    \[
    \mathrm{End}_{R\text{-}\mathsf{CAlg}}(R[x])\cong \mathrm{Hom}_{\mathsf{Set}}(x,R[x])\cong R[x].
    \]
    For $f\in R[x]$, or for any function $\mathbf{1}_f:\{x\}\to R[x]$, there exists a unique $R$-algebra homomorphism 
    \begin{align*}
        \widetilde{f}:R[x]&\longrightarrow R[x]\\
        \sum_{k=0}^n a_k x^k &\longmapsto \sum_{k=0}^n a_k f(x)^k
    \end{align*}
    
    If $\deg f\le 0$
\end{example}

\section{Construction}

\subsection{Free Object}

\begin{definition}{Free Commutative Algebra}{free_commutative_algebra}
    Let $X$ be a set and $R$ be a commutative ring. The \textbf{free commutative $R$-algebra} on $X$, denoted by $\mathrm{Free}_{R\text{-}\mathsf{CAlg}}(X)$, together with a map $\iota:X\to \mathrm{Free}_{R\text{-}\mathsf{CAlg}}(X)$, is defined by the following universal property: for any commutative $R$-algebra $A$ and any map $f:X\to A$, there exists a unique homomorphism $\widetilde{f}:\mathrm{Free}_{R\text{-}\mathsf{CAlg}}(X)\to A$ such that the following diagram commutes
    \begin{center}
        \begin{tikzcd}[ampersand replacement=\&]
            \mathrm{Free}_{R\text{-}\mathsf{CAlg}}(X)\arrow[r, dashed, "\exists !\,\widetilde{f}"]  \& A\\[0.3cm]
            X\arrow[u, "\iota"] \arrow[ru, "f"'] \&  
        \end{tikzcd}
    \end{center}
    The free commutative $R$-algebra $\mathrm{Free}_{R\text{-}\mathsf{CAlg}}(X)$ can be contructed as the polynomial algebra $R[X]$.

    And we can define a functor
    \[
        \begin{tikzcd}[ampersand replacement=\&]
            \mathsf{CRing}\&[-25pt]\&[+10pt]\&[-30pt] \mathsf{CRing}\&[-30pt]\&[-30pt] \\ [-15pt] 
            R  \arrow[dd, "\varphi"{name=L, left}] 
            \&[-25pt] \& [+10pt] 
            \& [-30pt] R[X]\arrow[dd, "{{}^{\varphi}\!(-)}"{name=R}] \&[-20pt]\ni\& [+10pt]f(X)=\sum\limits_{\beta} a_\beta x^\beta \arrow[dd, mapsto, ""{name=L, right}] 
            \\ [-10pt] 
            \&  \phantom{.}\arrow[r, "{\mathrm{Free}_{\bullet\text{-}\mathsf{CAlg}}(X)}", squigarrow]\&\phantom{.}  \&   \\[-10pt] 
            S\& \& \&  S[X]\&[-0pt]\ni\& ~^\varphi\!f(X)=\sum\limits_{\beta} \varphi(a_\beta) x^\beta
        \end{tikzcd}
        \]  
\end{definition}
\begin{prf}
    We can check that $\mathrm{Free}_{\bullet\text{-}\mathsf{CAlg}}(X)$ is a functor
    \[
        \leftindex^{\psi\circ \varphi}f(X)=\sum\limits_{\beta} (\psi\circ \varphi)(a_\beta) x^\beta=\sum\limits_{\beta} \psi(\varphi(a_\beta)) x^\beta=\leftindex^\psi(\leftindex^\varphi f)(X).
    \]
\end{prf}

\subsection{Coproduct}

\begin{definition}{Coproduct of Commutative Algebras}{coproduct_of_commutative_algebras}
    Let $R$ be a commutative ring and $\left(A_i\right)_{i\in I}$ be a family of commutative $R$-algebras. The \textbf{coproduct} of $\left(A_i\right)_{i\in I}$ in $R$-$\mathsf{CAlg}$, denoted by $\bigotimes_{i\in I} A_i$, together with a family of $R$-algebra homomorphisms
    \[
    \left(\iota_j:A_j\to \bigotimes_{i\in I} A_i\right)_{j\in I},
    \]
    is the tensor product of $\left(A_i\right)_{i\in I}$ over $R$. 
    
    It is defined by the following universal property: for any commutative $R$-algebra $B$ and any family of $R$-algebra homomorphisms $\{f_j:A_j\to B\}_{j\in I}$, there exists a unique $R$-algebra homomorphism $\widetilde{f}:\bigotimes_{i\in I} A_i\to B$ such that the following diagram commutes for all $j\in I$
    \begin{center}
        \begin{tikzcd}[ampersand replacement=\&]
            \bigotimes_{i\in I} A_i\arrow[r, dashed, "\exists !\,\widetilde{f}"]  \& B\\[0.3cm]
            A_j\arrow[u, "\iota_j"] \arrow[ru, "f_j"'] \&  
        \end{tikzcd}
    \end{center}
\end{definition}

\begin{proposition}{Diagonal Base Change for Commutative Algebras}{}
    If $\varphi:R\to S$, $f:S\to A$, $g:S\to B$ are ring homomorphisms of commutative rings. Denote $T:=S\otimes_{R} S$. Then we have the following isomorphism in $S$-$\mathsf{CAlg}$
    \begin{align*}
        \theta:A\otimes_{S} B &\xlongrightarrow{\sim} \left(A\otimes_{R} B\right) \otimes_{S\otimes_{R} S} S\\
        a\otimes_S b &\longmapsto (a\otimes_R b)\otimes_T 1_S\\
        (a\cdot s)\otimes_S b= a\otimes_S (s\cdot b) &\longmapsfrom (a\otimes_R b)\otimes_T s.
    \end{align*}
\end{proposition}
\begin{prf}
    This is a direct consequence of \Cref{th:diagonal_base_change_fibered_product}. Here we give another proof by explicit construction.

    First, there exists a unique $R$-algebra homomorphism
    \begin{align*}
        f\otimes_R g:S\otimes_R S &\longrightarrow A\otimes_R B\\
        s\otimes_R s' &\longmapsto f(s)\otimes_R g(s').
    \end{align*}
    such that the following diagram commutes
    \[
    \begin{tikzcd}
        S \arrow[r, "\iota_{S}"] \arrow[d, "f"']                              & S\otimes_R S \arrow[d, "f\otimes_R g", dashed] & S \arrow[l, "\iota_{S}"'] \arrow[d, "g"] \\[0.7em]
        A  \arrow[r, "\iota_{A}"'] & A\otimes_R B                                  & B \arrow[l, "\iota_{B}"]                
    \end{tikzcd}
    \] 
    This makes $A\otimes_R B$ an $S\otimes_R S$-algebra. Similarly, there exists a unique $R$-algebra homomorphism
    \begin{align*}
        \mu_S:S\otimes_R S &\longrightarrow S\\
        s\otimes_R s' &\longmapsto ss'.
    \end{align*}
    such that the following diagram commutes
    \[
    \begin{tikzcd}
        S \arrow[r, "\iota_{S}"] \arrow[dr, "\mathrm{id}_S"']                              & S\otimes_R S \arrow[d, "\mu_S", dashed] & S \arrow[l, "\iota_{S}"'] \arrow[ld, "\mathrm{id}_S"] \\[0.7em]
        & S  &          
    \end{tikzcd}
    \] 
    This makes $S$ an $S\otimes_R S$-algebra. 

    Given any $S$-algebra $C$ and any $S$-algebra homomorphisms $h_A:A\to C$, $h_B:B\to C$ such that $h_A\circ f=h_B\circ g$, by the universal property of tensor product $A\otimes_R B$, there exists a unique $R$-algebra homomorphism 
    \begin{align*}
        \eta:A\otimes_R B &\longrightarrow C\\
        a\otimes_R b &\longmapsto h_A(a)h_B(b)
    \end{align*}
    such that 
    \[
    h_A = \eta\circ \iota_A,\quad h_B = \eta\circ \iota_B.
    \]
    \[
    \begin{tikzcd}
R \arrow[dd, "f\circ \varphi"'] \arrow[rr, "g \circ \varphi"] \arrow[rd, "\varphi"] &                                  &  [-10pt] B \arrow[dd, "\iota_A"] \arrow[rddd, "h_B", bend left] &   \\[8pt]
                                                                                    & S \arrow[ld, "f"'] \arrow[ru, "g"] &                                                        &   \\[8pt]
A \arrow[rr, "\iota_B"'] \arrow[rrrd, "h_A"', bend right]                           &                                    & A\otimes_R B \arrow[rd, "\eta", dashed]                &   \\[8pt]
                                                                                    &                                    &                                                        & C
\end{tikzcd}
    \]
    Take $C:=A\otimes_S B$ and we get an $R$-algebra homomorphism 
    \begin{align*}
        \eta:A\otimes_R B &\longrightarrow A\otimes_S B\\
        a\otimes_R b &\longmapsto a\otimes_S b.
    \end{align*}
Since for any $s\otimes_R s'\in S\otimes_R S$, we have
\begin{align*}
    \mu\circ (f\otimes_R g)(s\otimes_R s') &= \mu(f(s)\otimes_R g(s'))\\
    &= f(s)\otimes_S g(s')\\
    &=\left(f(s)\otimes_S 1_B\right) \left(1_A\otimes_S g(s')\right)\\
    &=\left(f(s)\otimes_S 1_B\right) \left(f(s')\otimes_S 1_B\right)\\
    &= \left(f(s)f(s')\right)\otimes_S 1_B\\
    &= f(ss')\otimes_S 1_B\\
    &= \iota_{A}'\circ f\circ \mu_S(s\otimes_R s'),
\end{align*}
by the universal property of the tensor product $ \left(A\otimes_{R} B\right) \otimes_{S\otimes_{R} S} S$, there exists a unique $S\otimes_{R} S$-algebra homomorphism
\begin{align*}
    \psi: \left(A\otimes_{R} B\right) \otimes_{S\otimes_{R} S} S &\longrightarrow A\otimes_S B\\
    (a\otimes_R b)\otimes s &\longmapsto (a\otimes_S b) \left(f(s)\otimes_S 1_B\right)= (a\cdot s)\otimes_S b.
\end{align*}
such that the following diagram commutes
\[
\begin{tikzcd}
S\otimes_R S \arrow[d, "f\otimes_R g"'] \arrow[r, "\mu_S"]                      & S \arrow[d, "\iota_S'"] \arrow[rdd, bend left,"\iota_{A}'\circ f"]                                    &                  \\[4em]
A\otimes_R B \arrow[r, "\iota_{A\otimes_R B}"'] \arrow[rrd, "\eta"', bend right] & \left(A\otimes_{R} B\right) \otimes_{S\otimes_{R} S} S \arrow[rd, "\psi", dashed] &                  \\[3.5em] 
                                                                                &                                                                                   &  A\otimes_{S} B 
\end{tikzcd}
\]
Define
\begin{align*}
    j_A:A &\longrightarrow \left(A\otimes_{R} B\right) \otimes_{S\otimes_{R} S} S\\
    a &\longmapsto (a\otimes_R 1_B)\otimes_T 1_S,
\end{align*}
and
\begin{align*}
    j_B:B &\longrightarrow \left(A\otimes_{R} B\right) \otimes_{S\otimes_{R} S} S\\
    b &\longmapsto (1_A\otimes_R b)\otimes_T 1_S.
\end{align*}
Note that for any $s\otimes s'\in S$, we have
\begin{align*}
    (f(s)\otimes_R g(s'))\otimes_T 1_S &=  \iota_{A\otimes_R B}\circ (f\otimes_R g)(s\otimes_R s') \\
    &= \iota_S'\circ \mu_S (s\otimes_R s')\\
     &= \left(1_A\otimes_R 1_B\right)\otimes_T ss'.
\end{align*}
Then for any $s\in S$, we have
\begin{align*}
    j_A\circ f(s) &= ( f(s)\otimes_R 1_B)\otimes_T 1_S\\
    &= ( f(s)\otimes_R g(1_S))\otimes_T 1_S\\
    &= ( 1_A\otimes_R 1_B)\otimes_T (s1_S)\\
    &= ( 1_A\otimes_R 1_B)\otimes_T (1_Ss)\\
    &= (f(1_S)\otimes_R g(s))\otimes_T 1_S\\
    &= (1_A\otimes_R g(s))\otimes_T 1_S\\
    &= j_B\circ g(s).
\end{align*}
By the universal property of the coproduct $A \otimes_{S} B$, there exists a unique $S$-algebra homomorphism
\begin{align*}
    \theta:A\otimes_S B &\longrightarrow \left(A\otimes_{R} B\right) \otimes_{S\otimes_{R} S} S\\
    a\otimes_S b &\longmapsto (a\otimes_R b)\otimes_T 1_S
\end{align*}
such that the following diagram commutes
\[
\begin{tikzcd}
S \arrow[d, "f"'] \arrow[r, "g"]                           & B \arrow[d, "\iota_{B}'"] \arrow[rdd, "j_B", bend left] &                                                        \\[2em]
A \arrow[r, "\iota_{A}'"'] \arrow[rrd, "j_A"', bend right] &  A\otimes_{S} B  \arrow[rd, "\theta", dashed]           &                                                        \\[2em]
                                                           &                                                         & \left(A\otimes_{R} B\right) \otimes_{S\otimes_{R} S} S
\end{tikzcd}
\]
Then it is straightforward to check that $\psi$ and $\theta$ are inverse to each other:
\[
    \psi\circ \theta (a\otimes_S b) = \psi((a\otimes_R b)\otimes_T 1_S) = (a\cdot 1_S)\otimes_S b = a\otimes_S b,
\]
and
\begin{align*}
     \theta\circ \psi((a\otimes_R b)\otimes_T s) &= \theta((a\cdot s)\otimes_S b) \\
     &= \left((a\cdot s)\otimes_S b\right)\otimes_T 1_S\\
     & = \left((a\otimes_S b)\cdot(s\otimes_R 1_S)\right)\otimes_T 1_S \\
     &= \left(a\otimes_S b\right)\otimes_T \left((s\otimes_R 1_S)\cdot 1_S\right)\\
     &=(a\otimes_R b)\otimes_T s.
\end{align*}

\end{prf}


