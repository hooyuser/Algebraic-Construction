
\chapter{Associative Algebra}

\section{Basic Properties}
\begin{definition}{Associative Algebra over Commutative Ring}{}
    Let $R$ be a commutative ring. An \textbf{associative $R$-algebra} is a ring $A$ together with a ring homomorphism $\varphi:R\to Z(A)$, which makes $A$ an $R$-module by defining the scalar multiplication as
    \begin{align*}
        R\times A &\longrightarrow A\\
        (r,a) &\longmapsto r\cdot a:=\varphi(r)a.
    \end{align*}
    $\varphi:R\to Z(A)$ is called the \textbf{structure homomorphism} of $A$.
\end{definition}
\begin{remark}
    We can check that 
    \[
      r\cdot (ab)=\sigma(r)ab=(r\cdot a)b=\sigma(r)ab=a\left(\sigma(r)b\right) =a(r\cdot b),
    \]
    which justifies the naming ``associative". 
\end{remark}

We usually call associative $R$-algebra as $R$-algebra for short.

\begin{proposition}{Commutative Ring homomorphism $R\to S$ induces functor $S\text{-}\mathsf{Alg}\to R\text{-}\mathsf{Alg}$}{}
    Let $R$ and $S$ be commutative rings with a ring homomorphism $f: R\to S$. Then every $S$-algebra $A$ is an $R$-algebra by defining $ra = f(r)a$, or equivalently through $R\to S\to Z(A)$. This defines a functor $F: S\text{-}\mathsf{Alg}\to R\text{-}\mathsf{Alg}$, which is identify map on objects and morphisms.
    \[
        \begin{tikzcd}[ampersand replacement=\&]
            S\text{-}\mathsf{Alg}\&[-25pt]\&[+10pt]\&[-30pt] R\text{-}\mathsf{Alg}\&[-30pt]\&[-30pt] \\ [-15pt] 
            A  \arrow[dd, "g"{name=L, left}] 
            \&[-25pt] \& [+10pt] 
            \& [-30pt] A\arrow[dd, "g"{name=R}] \&[-30pt]\\ [-10pt] 
            \&  \phantom{.}\arrow[r, "F", squigarrow]\&\phantom{.}  \&   \\[-10pt] 
            B \& \& \&  B\&
        \end{tikzcd}
        \]  
\end{proposition}

In particular, commutative ring homomorphism $R\to S$ makes $S$ an $R$-algebra.



\section{Construction}
\subsection{Quotient Object}

\begin{definition}{Quotient Algebra}{}
    Let $A$ be an $R$-algebra and $\mathfrak{a}$ be a two-sided ideal of $A$. Since $\mathfrak{a}$ is an $R$-submodule of $A$, the quotient ring $A/\mathfrak{a}$ can also be endowed with an $R$-module structure, which makes $A/\mathfrak{a}$ an $R$-algebra. We call $A/\mathfrak{a}$ the \textbf{quotient algebra} of $A$ by $\mathfrak{a}$.
\end{definition}


\subsection{Free Object}
\begin{definition}{Free $R$-Algebra}{}
    Let $X$ be a set and $R$ be a commutative ring. The \textbf{free $R$-algebra} on $X$, denoted by $\mathrm{Free}_{R\text{-}\mathsf{Alg}}(X)$, together with a function $\iota:X\to \mathrm{Free}_{R\text{-}\mathsf{Alg}}(X)$, is defined by the following universal property: for any $R$-algebra $A$ and any function $f:X\to A$, there exists a unique $R$-algebra homomorphism $\widetilde{f}:\mathrm{Free}_{R\text{-}\mathsf{Alg}}(X)\to A$ such that the following diagram commutes
    \begin{center}
        \begin{tikzcd}[ampersand replacement=\&]
            \mathrm{Free}_{R\text{-}\mathsf{Alg}}(X)\arrow[r, dashed, "\exists !\,\widetilde{f}"]  \& A \\[0.3cm]
            X\arrow[u, "\iota"] \arrow[ru, "f"'] \&  
        \end{tikzcd}
    \end{center}
    The free $R$-algebra $\mathrm{Free}_{R\text{-}\mathsf{Alg}}(X)$ can be contructed by direct sum of copies of $R$
    \[
        \mathrm{Free}_{R\text{-}\mathsf{Alg}}(X)\cong\bigoplus_{w\in\mathrm{Free}_{\mathsf{Mon}}(X)}Rw.  
    \]
\end{definition}


\subsection{Graded Object}

\begin{definition}{$I$-Graded Algebra over an Graded Commutative Ring}{}
    Let $(I,+)$ be a monoid and $R$ be a $I$-graded commutative ring with grading $(R_i)_{i\in I}$. An \textbf{$I$-graded algebra over graded ring $R$} is an $R$-algebra $A$ together with a family of subalgebras $\left(A_i\right)_{i\in I}$ such that
    \begin{enumerate}[(i)]
        \item $A=\bigoplus_{i\in I}A_i$.
        \item $A_iA_j\subseteq A_{i+j}$ for all $i, j\in I$.
        \item $R_iA_j\subseteq A_{i+j}$ for all $i, j\in I$.
    \end{enumerate}
    Elements in $A_i$ are called \textbf{homogeneous elements of degree $i$}.
\end{definition}


\begin{proposition}{Graded Algebra Quotients out Graded Ideal}{}
    Let $A$ be an $I$-graded algebra over graded ring $R$ with grading $(A_i)_{i\in I}$ and $\mathfrak{a}$ be a graded two-sided ideal of $A$. Then $A/\mathfrak{a}$ has decomposition
    \[
    A/ \mathfrak{a}=\bigoplus_{i\in I}A_i/\left(\mathfrak{a}\cap A_i\right),    
    \]
    which makes $A/\mathfrak{a}$ an $I$-graded algebra.
\end{proposition}



\begin{example}{Polynomial Algebra $R[X_1,\cdots,X_n]$}{}
    Let $R$ be a commutative ring and $X_1,\cdots,X_n$ be indeterminates. Then $R[X_1,\cdots,X_n]$ is an $\mathbb{N}$-graded $R$-algebra with grading $R[X_1,\cdots,X_n]_i$ being the set of homogeneous polynomials of degree $i$.
\end{example}

\subsection{Tensor Product}
\begin{definition}{Tensor Product of Algebras}{}
    Let $R$ be a commutative ring and $A$, $B$ be $R$-algebras. The \textbf{tensor product of $R$-algebras $A$ and $B$} is defined by the following universal property: for any triple $(C, f_A, f_B)$, where $C$ is an $R$-algebra and $f_A:A\to C$, $f_B:B\to C$ are $R$-algebra homomorphisms which satisfy
    \[
        f_A(a)f_B(b)=f_B(b)f_A(a),\quad \forall a\in A, b\in B,
    \]
    the tensor product
    \[
    (A\otimes_R B, \iota_A: A \times B \to A\otimes_R B, \iota_B: A \times B \to A\otimes_R B)
    \]
    is initial among such triples, i.e. there exists a unique $R$-algebra homomorphism \begin{align*}
        \phi: A\otimes_R B &\longrightarrow C
    \end{align*}
    
    such that the following diagram commutes
    \[
        \begin{tikzcd}
        A \arrow[r, "\iota_A"] \arrow[rd, "f_A"'] & A\otimes_R B \arrow[d, "\exists! \phi", dashed] & B \arrow[l, "\iota_B"'] \arrow[ld, "f_B"] \\[0.5em]
                                                & C                                               &                                          
        \end{tikzcd}
    \]

    Concretely, $A\otimes_R B$ can be constructed as the tensor product of $R$-modules $A\otimes_R B$ together with multiplication defined as
    \[
        (a_1\otimes b_1)(a_2\otimes b_2):=(a_1a_2)\otimes(b_1b_2),\quad \forall a_1,a_2\in A, b_1,b_2\in B
    \]
    and unity 
    \[
          1_{A\otimes_R B}:=1_A\otimes 1_B.
    \]
    And the $R$-algebra homomorphisms $\iota_A$, $\iota_B$ are defined as
    \begin{align*}
        \iota_A : A &\longrightarrow A\otimes_R B\\
        a &\longmapsto a\otimes 1_B,
    \end{align*}
    \begin{align*}
        \iota_B : B &\longrightarrow A\otimes_R B\\
        b &\longmapsto 1_A\otimes b.
    \end{align*}
    The unique $R$-algebra homomorphism $\phi:A\otimes_R B\to C$ is defined as
    \begin{align*}
        \phi: A\otimes_R B &\longrightarrow C\\
        a\otimes b &\longmapsto f_A(a)f_B(b).
    \end{align*}
\end{definition}
\begin{remark}
    It is straightforward to check that the multiplication defined above is well-defined and makes $A\otimes_R B$ an $R$-algebra. According to \Cref{th:pure_tensors_generate_tensor_product}, since $(a,b)\mapsto f_A(a)f_B(b)$ is $\mathbb{Z}$-bilinear, $\phi$ is a well-defined abelian group homomorphism. We can further check that $\phi$ is an $R$-algebra homomorphism:
    \[
    \phi(r(a\otimes b))=\phi((r a)\otimes b)=f_A(r a)f_B(b)=r f_A(a)f_B(b)=r \phi(a\otimes b)
    \]
    \[
    \phi\left((a_1\otimes b_1)(a_2\otimes b_2)\right)=\phi\left((a_1a_2)\otimes(b_1b_2)\right)= f_A(a_1a_2)f_B(b_1b_2)=f_A(a_1)f_A(a_2)f_B(b_1)f_B(b_2)=\phi(a_1\otimes b_1)\phi(a_2\otimes b_2)
    \]
    \[
\phi\left(1_{A\otimes_R B}\right)=\phi\left(1_A\otimes 1_B\right)=f_A(1_A)f_B(1_B)=1_C
    \]
\end{remark}

\begin{definition}{Tensor Product of $R$-algebra Homomorphisms}{}
    Let $R$ be a commutative ring and $A_1$, $A_2$, $B_1$, $B_2$ be $R$-algebras. Given two $R$-algebra homomorphisms $f:A_1\to A_2$ and $g:B_1\to B_2$, the \textbf{tensor product of $R$-algebra homomorphisms} is defined as the $R$-algebra homomorphism 
    \begin{align*}
        f\otimes_R g: A_1\otimes_R B_1 &\longrightarrow A_2\otimes_R B_2\\
        a\otimes b &\longmapsto f(a)\otimes g(b).
    \end{align*}
    which is induced by the universal property of tensor product $A_1\otimes_R B_1$ through the following commutative diagram:
    \[
    \begin{tikzcd}
        A_1 \arrow[r, "\iota_{A_1}"] \arrow[d, "f"']                              & A_1\otimes_R B_1 \arrow[d, "f\otimes_R g", dashed] & B_2 \arrow[l, "\iota_{B_1}"'] \arrow[d, "g"] \\[0.7em]
        A_2  \arrow[r, "\iota_{A_2}"'] & A_2\otimes_R B_2                                   & B_2 \arrow[l, "\iota_{B_2}"]                
        \end{tikzcd}
    \] 
\end{definition}


\begin{proposition}{Symmetric Monoidal Structure on $R$-$\mathsf{Alg}$}{}
    Let $R$ be a commutative ring. The tensor product $\otimes_R$ defines a symmetric monoidal structure on the category $R$-$\mathsf{Alg}$, with unit object $R$.
    \begin{enumerate}[(i)]
        \item Tensor product: the tensor product functor is 
        \[
        \begin{tikzcd}[ampersand replacement=\&]
            R\text{-}\mathsf{Alg}\times R\text{-}\mathsf{Alg}\&[-25pt]\&[+10pt]\&[-30pt] R\text{-}\mathsf{Alg}\&[-30pt]\&[-30pt] \\ [-15pt] 
            (A_1, B_1)  \arrow[dd, "f\times g"{name=L, left}] 
            \&[-25pt] \& [+10pt] 
            \& [-30pt] A_1\otimes_R B_1\arrow[dd, "f\otimes_R g"{name=R}] \&[-30pt]\\ [-10pt] 
            \&  \phantom{.}\arrow[r, "\otimes_R", squigarrow]\&\phantom{.}  \&   \\[-10pt] 
            (A_2, B_2) \& \& \&  A_2\otimes_R B_2\&
        \end{tikzcd}
        \]  
        \item Associator: for any $R$-algebras $A$, $B$, $C$, there is a natural isomorphism
        \begin{align*}
            \alpha_{A,B,C}:(A\otimes_R B)\otimes_R C &\xlongrightarrow{\sim}A\otimes_R (B\otimes_R C)\\
            (a\otimes b)\otimes c &\longmapsto a\otimes (b\otimes c)
        \end{align*}
        \item Unit object: $R$.
        \item An isomorphism in $R$-$\mathsf{Alg}$:
        \begin{align*}
            \iota: R \otimes_R R &\xlongrightarrow{\sim} R\\
            r\otimes r' &\longmapsto rr'
        \end{align*}
        \item Symmetry: for any $R$-algebras $A$, $B$, there is a natural isomorphism
        \begin{align*}
            \gamma_{A,B}: A\otimes_R B &\xlongrightarrow{\sim} B\otimes_R A\\
            a\otimes b &\longmapsto b\otimes a
        \end{align*}
    \end{enumerate}
\end{proposition}

\begin{proposition}{}{}
    Let $R$ be a commutative ring and $A_1$, $A_2$ be $R$-algebras. Let $I_1\subseteq A_1$, $I_2\subseteq A_2$ be two-sided ideals of $A_1$, $A_2$ respectively. Then we have an $R$-algebra isomorphism
    \[
        (A_1\otimes_R A_2)/(I_1\otimes_R A_2 + A_1\otimes_R I_2) \cong (A_1/I_1)\otimes_R (A_2/I_2).
    \]
    
\end{proposition}

\subsection{Tensor Algebra}
\begin{definition}{Tensor Algebra $T^{\bullet}(M)$}{}
    Given a $R$-module $M$, the \textbf{$k$-th tensor power of $M$} is defined as
    \begin{align*}
        T^k(M)&=M^{\otimes k}=\underbrace{M\otimes_R\cdots\otimes_R M}_{k\text{ times}},\\
        T^0(M)&=R.
    \end{align*}
    The \textbf{tensor algebra} of $M$ is defined as
    \[
        T^{\bullet}(M)=\bigoplus_{k=0}^{\infty}T^k(M)
    \]
    with multiplication $\otimes$ defined as
    \[
        (m_1\otimes\cdots\otimes m_k)\otimes(m_{k+1}\otimes\cdots\otimes m_{k+l})=m_1\otimes\cdots\otimes m_{k+l}
    \]
    $T^{\bullet}(M)$ is an $\mathbb{N}$-graded $R$-algebra with grading $(T^k(M))_{k\ge 0}$.
\end{definition}


\begin{proposition}{Tensor Algebra Functor $T^{\bullet}:R\text{-}\mathsf{Mod}\to R\text{-}\mathsf{Alg}$}{}
    Let $R$ be a commutative ring. The tensor algebra construction $T^{\bullet}:R\text{-}\mathsf{Mod}\to R\text{-}\mathsf{Alg}$ is a functor. 
    \[
        \begin{tikzcd}[ampersand replacement=\&]
            R\text{-}\mathsf{Mod}\&[-25pt]\&[+10pt]\&[-30pt] R\text{-}\mathsf{Alg}\&[-30pt]\&[-30pt] \\ [-15pt] 
            M  \arrow[dd, "g"{name=L, left}] 
            \&[-25pt] \& [+10pt] 
            \& [-30pt] T^{\bullet}(M)\arrow[dd, "T^{\bullet}(g)"{name=R}] \&[-20pt]\ni\& [+10pt]m_1\otimes\cdots\otimes m_k \arrow[dd, mapsto, "g\otimes g\cdots\otimes g"{name=L, right}] 
            \\ [-10pt] 
            \&  \phantom{.}\arrow[r, "T^{\bullet}", squigarrow]\&\phantom{.}  \&   \\[-10pt] 
            N \& \& \&  T^{\bullet}(N)\&[-0pt]\ni\& g(m_1)\otimes\cdots\otimes g(m_k)
        \end{tikzcd}
        \]  
\end{proposition}


\begin{proposition}{Adjunction $T^{\bullet}\dashv U_{\mathsf{R\text{-}\mathsf{Mod}}}$}{}
    Let $R$ be a commutative ring. Suppose $U:R\text{-}\mathsf{Alg}\to R\text{-}\mathsf{Mod}$ is the forgetful functor. Then the tensor algebra functor $T^{\bullet}:R\text{-}\mathsf{Mod}\to R\text{-}\mathsf{Alg}$ is left adjoint to $U$.
\end{proposition}


\subsection{Exterior Algebra and Symmetric Algebra}
\begin{definition}{Exterior Algebra $\Largewedge^{\bullet} (M)$}{}
    Given an $R$-module $M$, 
    \begin{align*}
        I_{\largewedge}(M) & :=\langle x \otimes x: x \in M\rangle
    \end{align*}
    is a graded two-sided ideal of $T^{\bullet}(M)$. The \textbf{exterior algebra} of $M$ is defined as
    \[
        \largewedge^{\bullet} (M)=T^{\bullet}(M)/I_{\largewedge}(M).
    \]
    The multiplication of $\Largewedge^{\bullet} (M)$ is denoted by $\wedge$ and is called the \textbf{wedge product}.
    The grading of $\Largewedge^{\bullet} (M)$ is given by 
    \begin{align*}
        \largewedge^{\bullet} (M)=\bigoplus_{k=0}^{\infty}\largewedge^k(M),
    \end{align*}
    where 
    \[
        \largewedge^k(M)=T^k(M)/\left(I_{\largewedge}(M)\cap T^k(M)\right)
    \] 
    is called the \textbf{$k$-th exterior power of $M$}.
\end{definition}


\begin{definition}{Exterior Algebra Functor: $\Largewedge^{\bullet}:R\text{-}\mathsf{Mod}\to R\text{-}\mathsf{AcAlg_{\mathbb{Z}}}$}{}
    The exterior algebra construction $\Largewedge^{\bullet}:R\text{-}\mathsf{Mod}\to R\text{-}\mathsf{Alg}$ is a functor defined as follows
    \[
        \begin{tikzcd}[ampersand replacement=\&]
            R\text{-}\mathsf{Mod}\&[-25pt]\&[+10pt]\&[-30pt] R\text{-}\mathsf{AcAlg_{\mathbb{Z}}}\&[-30pt]\&[-30pt] \\ [-15pt] 
            M  \arrow[dd, "g"{name=L, left}] 
            \&[-25pt] \& [+10pt] 
            \& [-30pt] \Largewedge^{\bullet}(M)\arrow[dd, "\Largewedge^{\bullet}(g)"{name=R}] \&[-20pt]\ni\& [+20pt]m_1\wedge\cdots\wedge m_k \arrow[dd, mapsto, "g\wedge g\cdots\wedge g"{name=L, right}] 
            \\ [-10pt] 
            \&  \phantom{.}\arrow[r, "\Largewedge^{\bullet}", squigarrow]\&\phantom{.}  \&   \\[-10pt] 
            N \& \& \&  \Largewedge^{\bullet}(N)\&[-0pt]\ni\& g(m_1)\wedge\cdots\wedge g(m_k)
        \end{tikzcd}
    \]  
\end{definition}

\[
    \begin{tikzcd}[ampersand replacement=\&]
        \mathsf{PoFin}_I^{\mathrm{op}}\&[-25pt]\&[+10pt]\&[-30pt] \mathsf{Mble}\&[-30pt]\&[-30pt] \\ [-15pt] 
        J_2  \arrow[dd, ""{name=L, left}] 
        \&[-25pt] \& [+10pt] 
        \& [-30pt]\prod\limits_{i \in J_2}(\Omega_i,\mathcal{F}_i)\arrow[dd, "\mathrm{pr}_{J_1\subseteq J_2}"{name=R}] \&[-20pt]\ni\& [+20pt](\omega_i)_{i\in J_2} \arrow[dd, mapsto, ""{name=L, right}] 
        \\ [-10pt] 
        \&  \phantom{.}\arrow[r, "U\circ F", squigarrow]\&\phantom{.}  \&   \\[-10pt] 
        J_1\& \& \& \prod\limits_{i \in J_1}(\Omega_i,\mathcal{F}_i)\&[-0pt]\ni\& (\omega_i)_{i\in J_1}
    \end{tikzcd}
\]  


\begin{proposition}{Adjunction $\Largewedge^{\bullet}\dashv U_{\mathsf{R\text{-}\mathsf{Mod}}}$}{}
    Let $R$ be a commutative ring. Suppose $U:R\text{-}\mathsf{AcAlg_{\mathbb{Z}}}\to R\text{-}\mathsf{Mod}$ is the forgetful functor. Then the exterior algebra functor $\Largewedge^{\bullet}:R\text{-}\mathsf{Mod}\to R\text{-}\mathsf{AcAlg_{\mathbb{Z}}}$ is left adjoint to $U$.
\end{proposition}


\begin{proposition}{}{}
    Suppose $R$ is a commutative ring and $M=\bigoplus_{x\in X}Rx$ is a free $R$-module. Then
    \begin{enumerate}[(i)]
        \item $\Largewedge^{\bullet}(M)$ has a basis $\{x_1\wedge\cdots\wedge x_k: x_1,\cdots,x_k\in X, x_i\ne x_j\text{ for all }i\ne j\}$.
        \item If $M$ has a basis $\{x_1,\cdots,x_n\}$, then we have an $R$-linear isomorphism
        \begin{align*}
            \largewedge^{n}(M)&\xlongrightarrow{\sim} R\\
            x_1 \wedge \cdots \wedge x_n& \longmapsto 1_R.
        \end{align*}
        Moreover, we have $\Largewedge^{m}(M)=0$ for all $m>n$.
    \end{enumerate}
\end{proposition}
\section{Integral Element}

\begin{definition}{Integral Element}{}
    Let $R$ be a commutative ring and $A$ be an $R$-algebra with structure homomorphism $\varphi:R\to Z(A)$. An element $x\in A$ is called \textbf{integral} over $R$ if there exists a monic polynomial $f\in R[T]$ such that $\leftindex^{\varphi}\!f(x)=0$.
\end{definition}


\begin{definition}{Generated Subalgebra}{generated_subalgebra}
    Let $R$ be a commutative ring and $A$ be an $R$-algebra. By the universal property of $R\langle T\rangle$, there exists a unique $R$-algebra homomorphism $\psi:R\langle T\rangle\to A$ such that 
    $\psi(T)=x$.
    \begin{center}
        \begin{tikzcd}[ampersand replacement=\&]
            R\langle T\rangle\arrow[r, dashed, "\exists !\,\psi"]  \& A\\[0.3cm]
            \{T\}\arrow[u, "\iota"] \arrow[ru, "\mathrm{const}_x"'] \&  
        \end{tikzcd}
    \end{center}
    The \textbf{$R$-subalgebra of $A$ generated by $x$} is defined as 
    \[
    R[x]:=\psi\left(R\langle T\rangle\right)=\left\{\sum_{k=0}^n r_k x^k \in A\;\middle|\; r_k\in R\right\}.
    \]
\end{definition}

\begin{proposition}{Equivalent Definition of Integral Element}{equivalent_definition_of_integral_element}
    Let $R$ be a commutative ring and $A$ be an $R$-algebra. Let $R[x]$ be the \hyperref[th:generated_subalgebra]{$R$-subalgebra of $A$ generated by $x$}. Then $A$ is an $R[x]$-module. And the following statements are equivalent:
    \begin{enumerate}[(i)]
        \item $x$ is integral over $R$.
        \item $R[x]$ is a finitely generated $R$-module.
        \item There exists a faithful $R[x]$-submodule of $A$ that is finitely generated as an $R$-module and contains $x$.
    \end{enumerate}
\end{proposition}

\section{Trace and Norm}

\begin{lemma}{Left Multiplication Endomorphism}{}
    Let $R$ be a commutative ring and $A$ be an $R$-algebra. For any $a\in A$, we can define the left multiplication endomorphism $l_a\in\mathrm{End}_{R\text{-}\mathsf{Mod}}(A)$ by
    \begin{align*}
        l_a:A &\longrightarrow A\\
        x &\longmapsto ax.
    \end{align*}
    Moreover,
    \begin{align*}
        l_{-}:A &\longrightarrow \mathrm{End}_{R\text{-}\mathsf{Mod}}(A)\\
        a &\longmapsto l_a
    \end{align*}
    is an $R$-algebra homomorphism.
\end{lemma}
\begin{prf}
    For any $r\in R$, $a,b\in A$ and $x\in A$, we have
    \begin{align*}
        l_{ra+b}(x)&=(ra+b)x=r(ax)+bx=(ra)x+bx=l_{ra}(x)+l_b(x),\\
        l_{ab}(x)&=(ab)x=a(bx)=l_a(l_b(x)),\\
        l_{1_A}(x)&=1_Ax=x.
    \end{align*}
    Hence $l_{-}$ is an $R$-algebra homomorphism.
\end{prf}


\begin{definition}{Trace, Norm and Characteristic Polynomial}{trace_norm_and_characteristic_polynomial}
    Let $R$ be a commutative ring and $A$ be an $R$-algebra. Suppose $A$ as an $R$-module is free of finite rank. For any $a\in A$, we can define the left multiplication endomorphism $l_a\in\mathrm{End}_{R\text{-}\mathsf{Mod}}(A)$ by
    \begin{align*}
        l_a:A &\longrightarrow A\\
        x &\longmapsto ax.
    \end{align*}
    \begin{itemize}
        \item The \textbf{trace} of $a\in A$ is defined as the trace of $l_a$, denoted by 
        \[
        \mathrm{Tr}_{A|R}(a):=\mathrm{Tr}(l_a) \in R.
        \]
        That is, $\mathrm{Tr}_{A|R}:A\to R$ is an $R$-module homomorphism through the following composition
        \[
            \mathrm{Tr}_{A|R}: A\xrightarrow{l_{-}} \mathrm{End}_{R\text{-}\mathsf{Mod}}(A)\xrightarrow{\mathrm{Tr}} R.
        \]
        \item The \textbf{norm} of $a\in A$ is defined as the determinant of $l_a$, denoted by
        \[
        \mathrm{N}_{A|R}(a):=\det(l_a) \in R.
        \]
        That is, $\mathrm{N}_{A|R}:A\to R$ is a multiplicative monoid homomorphism through the following composition
        \[
            \mathrm{N}_{A|R}: A\xrightarrow{l_{-}} \mathrm{End}_{R\text{-}\mathsf{Mod}}(A)\xrightarrow{\det} R.
        \]
        \item The \textbf{characteristic polynomial} of $a\in A$ is defined as the characteristic polynomial of $l_a$, denoted by
        \[
        \mathrm{char}_{A|R}(a; X):=\mathrm{char}(l_a; X)=\det(X\cdot\mathrm{id}_A-l_a)= \mathrm{N}_{A[X]|R[X]}(X-a)\in R[X]
        \]
    \end{itemize}
\end{definition}

\begin{proposition}{Trace, Norm, and Characteristic Polynomial under Change of Base Ring}{trace_norm_and_characteristic_polynomial_under_change_of_base_ring}
    Let $R$ be a commutative ring, $A$ be an commutative $R$-algebra and $M$ be a free $R$-module of finite rank. Suppose $A$ as an $R$-module is free of finite rank. Given any $A$-linear transformation $\varphi\in\mathrm{End}_{A\text{-}\mathsf{Mod}}(M)$, by applying the functor of restriction of scalars to $R\to A$, we can regard $\varphi$ as an $R$-linear transformation on $M$ through $\mathrm{End}_{A\text{-}\mathsf{Mod}}(M)\hookrightarrow\mathrm{End}_{R\text{-}\mathsf{Mod}}(M)$. And we have
    \begin{enumerate}[(i)]
        \item $\mathrm{Tr}_{R}(\varphi)=\mathrm{Tr}_{A|R}(\mathrm{Tr}_{A}(\varphi))$.
        \item $\mathrm{N}_{R}(\varphi)=\mathrm{N}_{A|R}(\mathrm{N}_{A}(\varphi))$.
        \item $\mathrm{char}_{R}(\varphi;X)=\mathrm{N}_{A[X]|R[X]}(\mathrm{char}_{A}(\varphi;X))$.
    \end{enumerate}
\end{proposition}
\begin{corollary}{}{}
    Let $R$ be a commutative ring, $A$ be an commutative $R$-algebra and $B$ be an $A$-algebra. Suppose $A$ as an $R$-module is free of finite rank and $B$ as an $A$-module is free of finite rank. Then for any $b\in B$, we have
    \begin{enumerate}[(i)]
        \item $\mathrm{Tr}_{B|R}(b)=\mathrm{Tr}_{A|R}(\mathrm{Tr}_{B|A}(b))$.
        \item $\mathrm{N}_{B|R}(b)=\mathrm{N}_{A|R}(\mathrm{N}_{B|A}(b))$.
        \item $\mathrm{char}_{B|R}(b;X)=\mathrm{N}_{A[X]|R[X]}(\mathrm{char}_{B|A}(b;X))$.
    \end{enumerate}
    
\end{corollary}
\begin{prf}
    This is a direct consequence of \Cref{th:trace_norm_and_characteristic_polynomial_under_change_of_base_ring} by taking $M=A$ and $\varphi=l_b$.
\end{prf}

\section{Algebra over Field}
\begin{lemma}{Nonzero Ring Homomorphism from Field is Injective}{nonzero_ring_homomorphism_from_field_is_injective}
    If $K$ is a field, $R$ is a ring, a ring homomorphism $f:K\to R$ is either injective or the zero map. Furthermore, If $R$ is not a zero ring, then $f$ is injective.
\end{lemma}
\begin{prf}
    Since the only ideals of $K$ are $\{0\}$ and $K$, the kernel of $f$ is either $\{0\}$ or $K$. If $\ker f=\{0\}$, then $f$ is injective. If $\ker f=K$, then $f$ is the zero map. By \Cref{th:kernel_of_ring_homomorphism_is_an_ideal}, if $R$ is not a zero ring, then $\ker f$ is not $K$, so $f$ is injective.
\end{prf}

\begin{corollary}{}{}
    If $K$ is a field and $A$ is a nonzero $K$-algebra, then the ring homomorphism $K\to Z(A)$ is injective.
\end{corollary}
\begin{prf}
    This is a direct consequence of \Cref{th:nonzero_ring_homomorphism_from_field_is_injective}.
\end{prf}


\begin{proposition}{}{}
    Let $K$ be a field, $A$ be a $K$-algebra and $a\in A$. Consider the evaluation ring homomorphism
    \begin{align*}
        \mathrm{ev}_a:K[X] &\longrightarrow A\\
        f &\longmapsto f(a).
    \end{align*}
    Since $K[X]$ is a PID, we can suppose $\ker \mathrm{ev}_a=(P_a)$ for some $P_a\in K[X]$. Since $\operatorname{im}\mathrm{ev}_a=K[a]$, we have the following isomorphism in $K$-$\mathsf{Alg}$
    \[
        K[a]\cong K[X]/(P_a(X)).
    \]
    And it can be divided into two cases:
    \begin{enumerate}[(i)]
        \item If $P_a=0$, then $\mathrm{ev}_a$ is injective and $K[a]\cong K[X]$.
        \item If $P_a\ne 0$, then $\mathrm{ev}_a$ is not injective. If we further assume $A$ is a domain, then $P_a(X)$ is irreducible, $K[a]$ is a field and
        \[
        \left[ K[a]:K \right]=\deg P_a(X).
        \]
    \end{enumerate}
\end{proposition}
\begin{prf}
    \begin{enumerate}[(i)]
        \item If $P_a=0$, then $\ker \mathrm{ev}_a=\{0\}$, so $\mathrm{ev}_a$ is injective. And $K[a]\cong K[X]$.
        \item If $A$ is a domain, then $K[a]$ as a subring of $A$ is an integral domain. This implies $(P_a(X))$ is a nonzero prime ideal of $K[X]$. By \Cref{th:nonzero_prime_ideal_iff_maximal_ideal_in_PID}, $P_a(X)$ is irreducible. Since $K[X]/(P_a(X))$ as $K$-vector space has a basis $\{1,X,X^2,\cdots,X^{\deg P_a(X)-1}\}$, we have $\left[ K[a]:K \right]=\deg P_a(X)$.
    \end{enumerate}
   
\end{prf}

\begin{definition}{Algebraic Element and Transcendental Element}{algebraic_element_and_transcendental_element}
    Let $K$ be a field and $A$ be a $K$-algebra. Consider the evaluation ring homomorphism
    \begin{align*}
        \mathrm{ev}_a:K[X] &\longrightarrow A\\
        f &\longmapsto f(a).
    \end{align*}
    and $\ker \mathrm{ev}_a=(P_a)$ for some $P_a\in K[X]$.
    \begin{itemize}
        \item If $P_a=0$, then $a$ is called a \textbf{transcendental element} over $K$. $a$ is not the root of any nonzero polynomial in $K[X]$. 
        \item If $P_a\ne 0$, then $a$ is called an \textbf{algebraic element} over $K$. Suppose $P_a(X)=\sum_{i=0}^n a_iX^i$. Then $m_a(X)=P_a(X)/a_n$ is called the \textbf{minimal polynomial} of $a$ over $K$.
    \end{itemize}
\end{definition}

Let $K$ be a field and $A$ be a $K$-algebra. Then $a\in A$ is algebraic over $K$ if and only if $a$ is integral over $K$.


\chapter{Commutative Unital Algebra}
\section{Basic Properties}
\begin{definition}{Commutative Algebra}{}
    Let $R$ be a commutative ring. A \textbf{commutative $R$-algebra} is an $R$-algebra where the multiplication is commutative. Or equivalently, a commutative $R$-algebra is a commutative ring $A$ together with a ring homomorphism $R\to A$. 
\end{definition}

\begin{remark}
    There is a category isomorphism $R\text{-}\mathsf{CAlg}\cong \left(R/\mathsf{CRing}\right)$.
\end{remark}

\section{Polynomial Algebra}
\begin{definition}{Polynomial Ring}{}
    Let $R$ be a commutative ring. The \textbf{polynomial ring} in $n$ variables over $R$ is the ring $R[x_1,\cdots,x_n]$ defined as the set of all formal sums $$\sum_{\alpha\in\mathbb{N}^n}a_\alpha x^\alpha$$ where $a_\alpha\in R$ satisfies $a_\alpha=0$ for all but finitely many $\alpha\in\mathbb{N}^n$ and $x^\alpha:=x_1^{\alpha_1}\cdots x_n^{\alpha_n}$ for $\alpha=(\alpha_1,\cdots,\alpha_n)\in\mathbb{N}^n$. The addition and multiplication are defined as follows: $$\sum_{\alpha\in\mathbb{N}^n}a_\alpha x^\alpha+\sum_{\alpha\in\mathbb{N}^n}b_\alpha x^\alpha=\sum_{\alpha\in\mathbb{N}^n}(a_\alpha+b_\alpha)x^\alpha$$ and $$\left(\sum_{\alpha\in\mathbb{N}^n}a_\alpha x^\alpha\right)\left(\sum_{\beta\in\mathbb{N}^n}b_\beta x^\beta\right)=\sum_{\gamma\in\mathbb{N}^n}\left(\sum_{\alpha+\beta=\gamma}a_\alpha b_\beta\right)x^\gamma.$$
\end{definition}


\begin{proposition}{Properties of Polynomial Ring}{properties_of_polynomial_ring}
    Let $R$ be a commutative ring.
    \begin{enumerate}
        \item If $R$ is a UFD, then $R[x_1,\cdots.x_n]$ is a UFD.
        \item $R$ is a field $\iff$ $R[x]$ is a PID $\iff$ $R[x]$ is an Euclidean domain.
        \item $R$ is an integral domain $\iff$ $R[x]$ is an integral domain.
        \item $R$ is Noetherian $\implies$ $R[x]$ is Noetherian.
        \item $R$ is reduced $\implies$ $R[x]$ is reduced.
    \end{enumerate}
\end{proposition}



\begin{proposition}{Division Algorithm in Polynomial Ring}{division_algorithm_in_polynomial_ring}
    Let $R$ be a commutative ring and $f,g\in R[x]$ be nonzero polynomials. If the leading coefficient of $g$ is in $R^\times$, then there exist unique polynomials $q,r\in R[x]$ such that $f=qg+r$ and $\deg r<\deg g$.
\end{proposition}
\begin{prf}
    We can prove this by induction on $\deg f$. The base case is $\deg f=0$. If $g=a_0\in R^\times$, then we can take $q=f/a_0$ and $r=0$. If $\deg g\ge 1$, then we can take $q=0$ and $r=f$.

    Suppose the statement holds for any $h\in R[x]$ with $\deg h < n$. Let $f\in R[x]$ be a polynomial of degree $n$. If $\deg f<\deg g$, then we can take $q=0$ and $r=f$. If $\deg f\ge \deg g$. Suppose
    \[
        f=\sum_{i=0}^n a_ix^i\quad\text{and}\quad g=\sum_{i=0}^m b_ix^i
    \]
    where $a_n,b_m\ne 0$. Let $h=f-\frac{a_n}{b_m} x^{n-m}g$. Then $\deg h<\deg f$. By induction hypothesis, there exist $\tilde{q},\tilde{r}\in R[x]$ such that $h=\tilde{q}g+\tilde{r}$ and $\deg \tilde{r}<\deg g$. Thus there exist 
    \[
        q=\tilde{q}+\frac{a_n}{b_m} x^{n-m}\quad\text{and}\quad r=\tilde{r}
    \]
    such that $f=qg+r$ and $\deg r<\deg g$. If there are $Q$ and $R$ such that $f=Qg+R$ and $\deg R<\deg g$, then we have
    \[
    h=f-\frac{a_n}{b_m} x^{n-m}g=\left(Q-\frac{a_n}{b_m} x^{n-m}\right)g+R=\tilde{q}g+\tilde{r}.
    \]
    By uniqueness, we have $Q-\frac{a_n}{b_m} x^{n-m}=\tilde{q}$ and $R=\tilde{r}$, which implies $Q=q$ and $R=r$.
\end{prf}

\begin{corollary}{Polynomial Remainder Theorem}{polynomial_remainder_theorem}
    Let $R$ be a commutative ring and $f(x)\in R[x]$ be a polynomial. If $a\in R$, then there exist a unique polynomial $q(x)\in R[x]$ such that $f(x)=q(x)(x-a)+f(a)$.
\end{corollary}
\begin{prf}
    This is a direct application of \Cref{th:division_algorithm_in_polynomial_ring}.
\end{prf}


\begin{corollary}{}{}
    Let $R$ be a commutative ring.
    \begin{enumerate}[(i)]
        \item Let $a\in R$ and $f_1(x),\cdots, f_r(x)\in R[x]$ be polynomials. Then we have
        \[
        (f_1(x),\cdots,f_r(x),x-a)=(f_1(a),\cdots,f_r(a),x-a).
        \]
        \item Let $a\in R$ and $f_1(x),\cdots, f_r(x)\in R[x]$ be polynomials. Then we have 
    \[
        \frac{R[x]}{(f_1(x),\cdots,f_r(x),x-a)}\cong \frac{R}{\left(f_1(a),\cdots,f_r(a) \right)}.
    \]
    \item Suppose $a_1,\cdots,a_n\in R$. Then we can define a \textbf{evaluation homomorphism} 
    \begin{align*}
        \mathrm{ev}_{a_1,\cdots,a_n}:R[x_1,\cdots,x_n]&\longrightarrow R\\
        \sum_{\alpha\in\mathbb{N}^n}r_\alpha x^\alpha&\longmapsto \sum_{\alpha\in\mathbb{N}^n}r_\alpha a_1^{\alpha_1}\cdots a_n^{\alpha_n}.
    \end{align*}
    The kernel of $\mathrm{ev}_{a_1,\cdots,a_n}$ is 
    \[
    \ker \mathrm{ev}_{a_1,\cdots,a_n}=(x_1-a_1,\cdots,x_n-a_n)
    \]
    and we have
    \[
    \frac{R[x_1,\cdots,x_n]}{(x_1-a_1,\cdots,x_n-a_n)}\cong R.
    \]
    \end{enumerate}
    
    
\end{corollary}
\begin{prf}
    \begin{enumerate}[(i)]
        \item 
        By the \hyperref[th:polynomial_remainder_theorem]{polynomial remainder theorem}, we have
        \[
        f_i(x)=q_i(x)(x-a)+f_i(a),\quad i=1,2,\cdots,r,
        \]
        which implies
        \[
        (f_1(x),\cdots,f_r(x),x-a)=(f_1(a),\cdots,f_r(a),x-a).
        \]
        \item First we show the kernel of 
        \begin{align*}
            \mathrm{ev}_a:R[x]&\longrightarrow R\\
            f(x)&\longmapsto f(a)
        \end{align*}
        is $\ker \mathrm{ev}_a=(x-a)$.
        By the \hyperref[th:polynomial_remainder_theorem]{polynomial remainder theorem}, we have
        \[
        f(x)=q(x)(x-a)+f(a)
        \]
        Note
        \[
        f(x)\in \ker \mathrm{ev}_{a} \iff f(a)=0\iff f(x)\in (x-a).
        \]
        We have $\ker \mathrm{ev}_a=(x-a)$ and $R[x]/(x-a)\cong R$. 

        From \Cref{ex:coset_of_generated_ideals_quotient_generates_ideals}, we have the following equality of ideals in $R[x]/(x-a)$
        \[
            (f_1(x),\cdots,f_r(x),x-a)/(x-a)=\left(f_1(x)+(x-a),\cdots,f_r(x)+(x-a)\right).
        \]
        By the third isomorphism theorem, we have
        \[
        \frac{R[x]}{(f_1(x),\cdots,f_r(x),x-a)}\cong \frac{R[x]/(x-a)}{\left(f_1(x)+(x-a),\cdots,f_r(x)+(x-a)\right)}.
        \]
        Apply the isomorphism 
        \begin{align*}
            \overline{\mathrm{ev}_a}:R[x]/(x-a)&\longrightarrow R\\
            f(x)+(x-a)&\longmapsto f(a)
        \end{align*}
        we get
        \[
            \frac{R[x]/(x-a)}{\left(f_1(x)+(x-a),\cdots,f_r(x)+(x-a)\right)} \cong \frac{R}{\left(f_1(a),\cdots,f_r(a) \right)}.
        \]
        \item We can prove 
        \[
        \ker \mathrm{ev}_{a_1,\cdots,a_n}=(x_1-a_1,\cdots,x_n-a_n)
        \]
        by induction on $n$. The base case is $n=1$, which has been proved in (ii). Suppose the statement holds for $n-1$. Let $f(x_1,\cdots,x_n)\in R[x_1,\cdots,x_n]$.
        \[
        \ker \mathrm{ev}_{a_1,\cdots,a_n}=\ker \left(\mathrm{ev}_{a_n}\circ \mathrm{ev}_{a_1,\cdots,a_{n-1}}\right)= \mathrm{ev}_{a_n}^{-1}(\left(x_1-a_1,\cdots,x_n-a_n\right)).
        \]
        
        By the \hyperref[th:polynomial_remainder_theorem]{polynomial remainder theorem}, we have
        and we have
    \[
    \frac{R[x_1,\cdots,x_n]}{(x_1-a_1,\cdots,x_n-a_n)}\cong R.
    \]
    \end{enumerate}
\end{prf}

\begin{example}{$R$-algebra Endomorphisms of $R[x]$}{}
    Let $R$ be a commutative ring. By the universal property of free commutative $R$-algebra, we have the following isomorphism
    \[
    \mathrm{End}_{R\text{-}\mathsf{CAlg}}(R[x])\cong \mathrm{Hom}_{\mathsf{Set}}(x,R[x])\cong R[x].
    \]
    For $f\in R[x]$, or for any function $\mathbf{1}_f:\{x\}\to R[x]$, there exists a unique $R$-algebra homomorphism 
    \begin{align*}
        \widetilde{f}:R[x]&\longrightarrow R[x]\\
        \sum_{k=0}^n a_k x^k &\longmapsto \sum_{k=0}^n a_k f(x)^k
    \end{align*}
    
    If $\deg f\le 0$
\end{example}

\section{Construction}

\subsection{Free Object}

\begin{definition}{Free Commutative Algebra}{free_commutative_algebra}
    Let $X$ be a set and $R$ be a commutative ring. The \textbf{free commutative $R$-algebra} on $X$, denoted by $\mathrm{Free}_{R\text{-}\mathsf{CAlg}}(X)$, together with a map $\iota:X\to \mathrm{Free}_{R\text{-}\mathsf{CAlg}}(X)$, is defined by the following universal property: for any commutative $R$-algebra $A$ and any map $f:X\to A$, there exists a unique homomorphism $\widetilde{f}:\mathrm{Free}_{R\text{-}\mathsf{CAlg}}(X)\to A$ such that the following diagram commutes
    \begin{center}
        \begin{tikzcd}[ampersand replacement=\&]
            \mathrm{Free}_{R\text{-}\mathsf{CAlg}}(X)\arrow[r, dashed, "\exists !\,\widetilde{f}"]  \& A\\[0.3cm]
            X\arrow[u, "\iota"] \arrow[ru, "f"'] \&  
        \end{tikzcd}
    \end{center}
    The free commutative $R$-algebra $\mathrm{Free}_{R\text{-}\mathsf{CAlg}}(X)$ can be contructed as the polynomial algebra $R[X]$.

    And we can define a functor
    \[
        \begin{tikzcd}[ampersand replacement=\&]
            \mathsf{CRing}\&[-25pt]\&[+10pt]\&[-30pt] \mathsf{CRing}\&[-30pt]\&[-30pt] \\ [-15pt] 
            R  \arrow[dd, "\varphi"{name=L, left}] 
            \&[-25pt] \& [+10pt] 
            \& [-30pt] R[X]\arrow[dd, "{{}^{\varphi}\!(-)}"{name=R}] \&[-20pt]\ni\& [+10pt]f(X)=\sum\limits_{\beta} a_\beta x^\beta \arrow[dd, mapsto, ""{name=L, right}] 
            \\ [-10pt] 
            \&  \phantom{.}\arrow[r, "{\mathrm{Free}_{\bullet\text{-}\mathsf{CAlg}}(X)}", squigarrow]\&\phantom{.}  \&   \\[-10pt] 
            S\& \& \&  S[X]\&[-0pt]\ni\& ~^\varphi\!f(X)=\sum\limits_{\beta} \varphi(a_\beta) x^\beta
        \end{tikzcd}
        \]  
\end{definition}
\begin{prf}
    We can check that $\mathrm{Free}_{\bullet\text{-}\mathsf{CAlg}}(X)$ is a functor
    \[
        \leftindex^{\psi\circ \varphi}f(X)=\sum\limits_{\beta} (\psi\circ \varphi)(a_\beta) x^\beta=\sum\limits_{\beta} \psi(\varphi(a_\beta)) x^\beta=\leftindex^\psi(\leftindex^\varphi f)(X).
    \]
\end{prf}

\subsection{Coproduct}




