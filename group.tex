
\chapter{Group}
\section{Basic Concepts}
\begin{definition}{Group}{}
    A \textbf{group} is a set $G$ together with a binary operation $\cdot:G\times G\to G$ such that
    \begin{enumerate}[(i)]
        \item (Associativity) $\forall x,y,z\in G$, $(x\cdot y)\cdot z=x\cdot(y\cdot z)$.
        \item (Identity) $\exists e\in G$ such that $\forall x\in G$, $e\cdot x=x\cdot e=x$.
        \item (Inverse) $\forall x\in G$, $\exists x^{-1}\in G$ such that $x\cdot x^{-1}=x^{-1}\cdot x=e$.
    \end{enumerate}
\end{definition}

Since the identity of a group is unique, we denote it by $1_G$ or simply $1$.
\begin{definition}{Opposite Group}{}
    Let $G=(G,*)$ be a group. The \textbf{opposite group} of $G$ is the group $G^{\mathrm{op}}=(G,*^{\mathrm{op}})$, where $*^{\mathrm{op}}:G\times G\to G$ is defined by $x*^{\mathrm{op}}y=y\cdot x$. If we consider $G$ as a category $\mathsf{B}G$, then we have category isomorphism
    \[
        \mathsf{B}G^{\mathrm{op}}\cong (\mathsf{B}G)^{\mathrm{op}}.
    \]
\end{definition}

\begin{proposition}{Group is Isomorphic to Its Opposite group}{}
    $G$ is isomorphic to $G^{\mathrm{op}}$ through the isomorphism $x\mapsto x^{-1}$. This is same as saying that $\mathsf{B}G$ is isomorphic to $(\mathsf{B}G)^{\mathrm{op}}$ through the functor ${}^{\mathrm{op}}$ defined in \Cref{th:functor_op}.
\end{proposition}


\begin{definition}{Subgroup}{}
    Let $G$ be a group. A subset $H$ of $G$ is called a \textbf{subgroup} of $G$ if $H$ is a group with respect to the binary operation of $G$. In this case, we write $H\le G$.
\end{definition}



\section{Group Homomorphism}
\begin{definition}{Group Homomorphism}{}
    Let $G,H$ be groups. A \textbf{group homomorphism} from $G$ to $H$ is a function $\varphi:G\to H$ such that
    \[
        \forall x,y\in G,\quad \varphi(x y)=\varphi(x)\varphi(y).
    \]
\end{definition}

\begin{definition}{Isomorphism}{}
    Let $G,H$ be groups. A group homomorphism $\varphi:G\to H$ is called an \textbf{isomorphism} if $\varphi$ is bijective. In this case, we say that $G$ and $H$ are \textbf{isomorphic} and write $G\cong H$.
\end{definition}

\begin{proposition}{Properties of Group Homomorphisms}{}
    Let $G,H$ be groups and $\varphi:G\to H$ be a group homomorphism. Then
    \begin{enumerate}[(i)]
        \item $\varphi(1_G)=1_H$.
        \item $\forall x\in G$, $\varphi(x^{-1})=\varphi(x)^{-1}$.
        \item $\forall n\in\mathbb{Z}$, $\varphi(x^n)=\varphi(x)^n$.
        \item If $K\le G$ , then $\varphi(K)\le H$.
        \item If $K\le H$, then $\varphi^{-1}(K)\le G$.
    \end{enumerate}
\end{proposition}

\begin{definition}{Kernel of Group Homomorphism}{}
    Let $\varphi:G\to H$ be a group homomorphism. The \textbf{kernel} of $\varphi$ is defined by
    \[
        \ker\varphi=\{x\in G\mid \varphi(x)=1_H\}.
    \]
\end{definition}

\begin{proposition}{Property of Kernel}{}
    Let $G$ be a group. Then
    \begin{enumerate}[(i)]
        \item Let $\varphi:G\to H$ be a group homomorphism. Then $\varphi$ is injective if and only if $\ker\varphi=\{1_G\}$.
    \end{enumerate}
\end{proposition}

\begin{definition}{Normal Subgroup}{}
    Let $G$ be a group. A subgroup $H$ of $G$ is called a \textbf{normal subgroup} if $gHg^{-1}=H$ for all $g\in G$. In this case, we write $H\lhd G$.
\end{definition}

\begin{proposition}{Equivalent Definition of Normal Subgroup}{}
    Let $G$ be a group and $H$ be a subgroup of $G$. Then the following are equivalent:
    \begin{enumerate}[(i)]
        \item $H$ is a normal subgroup of $G$.
        \item $\forall \gamma_g \in\mathrm{Inn}(G)$, $\gamma_g(H)\subseteq H$.
        \item $gHg^{-1}\subseteq H$ for all $g\in G$.
        \item $gHg^{-1}=H$ for all $g\in G$.
        \item $gH=Hg$ for all $g\in G$.
        \item $H$ is a union of conjugacy classes.
        \item $H=\ker\varphi$ for some group homomorphism $\varphi:G\to K$.
    \end{enumerate}
\end{proposition}

\begin{proposition}{Properties of Normal Subgroup}{}
    Let $G$ be a group.
    \begin{enumerate}[(i)]
        \item $\{1_G\}$ and $G$ are normal subgroups of $G$.
        \item If $H\le K\le G$ and $H\lhd G$, then $H\lhd K$.
        \item If $H\lhd_{\rm char} K\lhd G$, then $H\lhd G$.
        \item Normality is preserved under surjective homomorphisms: if $f:G \rightarrow H$ is a surjective group homomorphism and $N\lhd G$, then $f(N)\lhd H$.
        \item Normality is preserved by taking inverse images of homomorphisms: if $f:G \rightarrow H$ is a group homomorphism and $N\lhd H$, then $f^{-1}(N)\lhd G$.
        \item Normality is preserved on taking finite products: if $N_1 \lhd G_1$ and $N_2 \lhd G_2$, then $N_1 \times N_2 \lhd G_1 \times G_2$.
        \item Given two normal subgroups, $N$ and $M$, of $G$, their intersection $N \cap M$ and their product $N M=\{n m: n \in N$ and $m \in M\}$ are also normal subgroups of $G$.
    \end{enumerate}
\end{proposition}

\begin{definition}{Simple Group}{}
    A group $G$ is called \textbf{simple} if $G$ is nontrivial and the only normal subgroups of $G$ are $\{1_G\}$ and $G$.
\end{definition}

\begin{theorem}{Fundamental Theorem on Homomorphisms}{}
    Let $G,H$ be groups and $\varphi:G\to H$ be a group homomorphism. Define natural projection
    \begin{align*}
        \pi:G & \longrightarrow G/\ker\varphi \\
        g     & \longmapsto g\ker\varphi
    \end{align*}
    Then there exists a unique group homomorphism $\overline{\varphi}:G/\ker\varphi\to H$ such that the following diagram commutes
    \begin{center}
        \begin{tikzcd}[ampersand replacement=\&]
            G \arrow[r, "\varphi"] \arrow[d, "\pi"'] \& H \\[0.3cm]
            G/\ker\varphi \arrow[ru, dashed, "\exists!\,\overline{\varphi}"'] \&
        \end{tikzcd}
    \end{center}
    Moreover, $\overline{\varphi}$ is injective and we have $ G/\ker\varphi\cong \mathrm{im}\varphi$.
\end{theorem}

\begin{corollary}{First isomorphism theorem}{}
    Let $G,H$ be groups and $\varphi:G\to H$ be surjective group homomorphism. Then $G/\ker\varphi\cong H$.
\end{corollary}


\begin{proposition}{Universal Property of Quotient Group}{}
    Let $G$ be a group and $N \lhd G$ be a normal subgroup. Suppose $\pi:G\to G/N$ is the natural projection. Then $\pi$ is initial in the category of group homomorphisms $\varphi:G\to H$ such that $N\subseteq \ker\varphi$. \\
    That is, for any group $H$ and group homomorphism $\varphi:G\to H$ such that $N\subseteq \ker\varphi$, there exists a unique group homomorphism $\widetilde{\varphi}:G/N\to H$ such that the following diagram commutes
    \begin{center}
        \begin{tikzcd}[ampersand replacement=\&]
            G \arrow[r, "\varphi"] \arrow[d, "\pi"'] \& H \\[0.3cm]
            G/N \arrow[ru, dashed, "\exists !\,\widetilde{\varphi}"'] \&
        \end{tikzcd}
    \end{center}
\end{proposition}

\proof{
    Since $N\subseteq \ker\varphi$, there is a canonical projection
    \begin{align*}
        p:G/N & \longrightarrow G/\ker\varphi \\
        gN    & \longmapsto g\ker\varphi
    \end{align*}
    According to the following diagram, we can define $\widetilde{\varphi}$ by $\widetilde{\varphi}=\overline{\varphi}\circ p$.
    \begin{center}
        \begin{tikzcd}[ampersand replacement=\&]
            G \arrow[r, "\varphi"] \arrow[d, "\pi"'] \& H \\[0.4cm]
            G/N \arrow[ru, dashed, "\widetilde{\varphi}"'] \arrow[r,"p"'] \&  G/\ker\varphi \arrow[u, dashed, "\overline{\varphi}"']
        \end{tikzcd}
    \end{center}
}
\begin{theorem}{Second Isoomorphism Theorem}{}
    Let $G$ be a group and $H,K$ be subgroups of $G$. Then $HK$ is a subgroup of $G$ and $H\cap K$ is a normal subgroup of $H$. Moreover, we have
    \[
        HK/H\cong K/(H\cap K).
    \]
\end{theorem}

\section{Construction}
\subsection{Free Object}
Let $A$ be set. Define a \textbf{string} over the $A$ to be a finite sequence of elements of $A$. The \textbf{concatenation} of two strings $\overline{a_1\cdots a_n}$ and $\overline{b_1\cdots b_m}$ is a binary operation $\diamond$ defined by
\[
    \overline{a_1\cdots a_n}\diamond \overline{b_1\cdots b_m}=\overline{a_1\cdots a_nb_1\cdots b_m}.
\]
\begin{definition}{Word}{}
    Let $S$ be a set. Define $S^{-1}=\left\{s^{-1}\midv s\in S\right\}$. A \textbf{word} on $S$ is a string over $S\sqcup S^{-1}\sqcup\{1\}$. $\overline{1}$ is called an \textbf{empty word}. The \textbf{length} of a word $w$ is the number of letters in $w$.
\end{definition}

\begin{definition}{Reduced Word}{}
    Let $W(S)$ be the set of all words on $S$. Define a relation $\approx$ on $W(S)$ as follows: for any $s\in S$
    \begin{align*}
        \overline{s^{-1}s}\approx \overline{1},\quad \overline{ss^{-1}}\approx \overline{1}, \quad \overline{1s}\approx s,\quad \overline{s1}\approx \overline{1}
    \end{align*}
    Define $\sim$ to be the equivalence relation generated by $\approx$. Let $\pi:W(S)\to W(S)/\sim, w\mapsto [w]_\sim$ denote the quotient map. It is easy to see that for any $[w]_\sim\in W(S)/\sim$, there exists a unique representative element $\rho(w)\in W(S)$ which has shortest length among all representatives of $[w]_\sim$. A word $w$ is called \textbf{reduced} if $\rho(w)=w$.
\end{definition}


\begin{definition}{Free Group}{}
    Let $S$ be a set. The \textbf{free group} on $S$, denoted by $\mathrm{Free}_{\mathsf{Grp}}(S)$, together with a function $\iota:S\to \mathrm{Free}_{\mathsf{Grp}}(S)$, is defined by the following universal property: for any group $G$ and any function $f:S\to G$, there exists a unique group homomorphism $\widetilde{f}:\mathrm{Free}_{\mathsf{Grp}}(S)\to G$ such that the following diagram commutes
    \begin{center}
        \begin{tikzcd}[ampersand replacement=\&]
            \mathrm{Free}_{\mathsf{Grp}}(S)\arrow[r, dashed, "\exists !\,\widetilde{f}"]  \& G \\[0.3cm]
            S\arrow[u, "\iota"] \arrow[ru, "f"'] \&
        \end{tikzcd}
    \end{center}
    The free group $\mathrm{Free}_{\mathsf{Grp}}(S)$ can be contructed as follows: as a set it consists of all reduced words on $S$. The binary operation $\cdot$ is concatenation with reduction defined by
    \[
        w_1\cdot w_2 = \rho(w_1\diamond w_2).
    \]
    The identity element is the empty word. The inverse of a word is obtained by reversing the order of the letters and replacing each letter by its inverse.
\end{definition}


\begin{example}{Forgetful Functor $U:\mathsf{Grp}\to \mathsf{Set}$}{}
    The forgetful functor $U:\mathsf{Grp}\to \mathsf{Set}$ forgets the group structure of a group and returns the underlying set.
    \begin{enumerate}[(i)]
        \item $U$ is representable by $\left(\mathbb{Z},1\right)$. The natural isomorphism $\phi:\mathrm{Hom}_{\mathsf{Grp}}\left(\mathbb{Z},-\right)\xRightarrow{\sim} U$ is given by
              \begin{align*}
                  \phi_G:\mathrm{Hom}_{\mathsf{Grp}}\left(\mathbb{Z},G\right) & \xlongrightarrow{\sim} U(G) \\
                  f                                                           & \longmapsto f(1).
              \end{align*}
              A group homomorphism from $\mathbb{Z}$ to $G$ is uniquely determined by its action on $1$.
        \item $U$ is faithful but not full.
    \end{enumerate}
\end{example}

\begin{prf}
    \begin{enumerate}[(i)]
        \item $\phi:\mathrm{Hom}_{\mathsf{Grp}}\left(\mathbb{Z},-\right)\xRightarrow{\sim} U$ is the composition of the following natural isomorphisms
              \[
                  \mathrm{Hom}_{\mathsf{Grp}}\left(\mathbb{Z},-\right)\cong\mathrm{Hom}_{\mathsf{Grp}}(\mathrm{Free}_{\mathsf{Grp}}(\{*\}),-)\cong \mathrm{Hom}_{\mathsf{Set}}(\{*\},U(-))\cong U.
              \]
        \item $U$ is not full because not every mapping $f:\mathbb{Z}\to G$ is a group homomorphism.
    \end{enumerate}
\end{prf}


\begin{proposition}{Free-Forgetful Adjunction $\mathrm{Free}_{\mathsf{Grp}}\dashv U$}{}
    The free group functor $\mathrm{Free}_{\mathsf{Grp}}$ is left adjoint to the forgetful functor $U:\mathsf{Grp}\to \mathsf{Set}$
    $$
        \begin{tikzcd}[ampersand replacement=\&]
            \mathsf{Set} \arrow[rr, "\mathrm{Free}_{\mathsf{Grp}}", bend left] \&[-10pt]\bot\&[-10pt] \mathsf{Grp} \arrow[ll, "U", bend left]
        \end{tikzcd}
    $$
    The adjunction isomorphism is given by
    \begin{align*}
        \varphi_{S,G}:\mathrm{Hom}_{\mathsf{Grp}}(\mathrm{Free}_{\mathsf{Grp}}(S),G) & \xlongrightarrow{\sim} \mathrm{Hom}_{\mathsf{Set}}(S,U(G)) \\
        g                                                                            & \longmapsto g\circ \iota
    \end{align*}
\end{proposition}

\begin{prf}
    First we show that $\varphi_{S,G}$ is injective. Suppose $g_1,g_2:\mathrm{Free}_{\mathsf{Grp}}(S)\to G$ are two group homomorphisms such that $g_1\circ \iota=g_2\circ \iota$. By the universal property of free group, we have $g_1=g_2$. Then we show that $\varphi_{S,G}$ is surjective. Suppose $f:S\to U(G)$ is a function. By the universal property there exists a group homomorphism $\widetilde{f}:\mathrm{Free}_{\mathsf{Grp}}(S)\to G$ such that $\varphi_{S,G}(\widetilde{f})=\widetilde{f}\circ \iota=f$. Finally, we show that $\varphi_{S,G}$ is natural in $S$ and $G$. Suppose $h:S_1\to S_2$ is a function and $q:G_1\to G_2$ is a group homomorphism. Then we can check that for any $g\in \mathrm{Hom}_{\mathsf{Grp}}(\mathrm{Free}_{\mathsf{Grp}}(S_2),G_2)$,
    \begin{align*}
        \varphi_{S_1,G_1}(q\circ g\circ \iota_{S_1}) & =(q\circ g\circ \iota_{S_1})\circ \iota_{S_1} \\
                                                     & =q\circ g\circ (\iota_{S_1}\circ \iota_{S_1}) \\
                                                     & =q\circ g\circ \iota_{S_2}                    \\
                                                     & =\varphi_{S_2,G_2}(g\circ \iota_{S_2}).
    \end{align*}
\end{prf}




\subsection{Inverse Limit}
\begin{definition}{Inverse Limit in $\mathsf{Grp}$}{inverse_limit_of_groups}
    Let $\mathsf{I}$ be a \hyperref[th:filtered_category]{filtered} \hyperref[th:thin_category]{thin category} and $F:\mathsf{I}^{\mathrm{op}}\to \mathsf{Grp}$ be a functor. To unpack the information of $F$, denote $I:=\mathrm{Ob}(\mathsf{I})$, $G_i:=F(i)$ and $f_{ij}:=F(i\to j)$. An \textbf{inverse system} is a pair $\left(\left(G_i\right)_{i \in I},\left(f_{i j}\right)_{i \leq j \in I}\right)$ where $f_{i j}: G_{j} \rightarrow G_{i}$ is a group homomorphism for each $i \leq j$ such that
    \begin{enumerate}[(i)]
        \item $f_{i i}=\mathrm{id}_{G_i}$ for all $i \in I$.
        \item $f_{i k}=f_{i j} \circ f_{j k}$ for all $i \leq j \leq k$.
    \end{enumerate}
    The \textbf{inverse limit} of $\left(\left(G_i\right)_{i \in I},\left(f_{i j}\right)_{i \leq j \in I}\right)$ is the cofiltered limit $\varprojlim F$, also denoted by $\varprojlim_{i\in I}G_i$, which can be constructed as a subgroup of $\prod_{i \in I} G_{i}$ as follows
    \[
        \varprojlim_{i\in I}G_i \cong \left\{(x_i)_{i\in I}\in \prod_{i \in I} G_{i}\midv x_i=f_{ij}(x_j) \text{ for all }i\le j \in I\right\}
    \]
    equipped with natural projections $\pi_i:\varprojlim_{i\in I}G_i\to G_i$.\\

\end{definition}


\begin{example}{Inverse Limit $\varprojlim_{i\ge 1}G_i$}{}
    Let $\mathsf{I}=\left(\mathbb{Z}_{\ge 1},\le\right)$ be a filtered thin category and $F:\mathsf{I}^{\mathrm{op}}\to \mathsf{Grp}$ be a functor. To determine an inverse system, it is sufficient to specify $G_i$ and $f_{i,i+1}:G_{i+1}\to G_i$ for all $i\in \mathbb{Z}_{\ge 1}$. The inverse limit of this inverse system is denoted by $\varprojlim_{i\ge 1}G_i$, which we now write as $G$ for simplicity.\\
    $G$ can be imaged as a tree with root layer being $G_0=\{1\}$ and $i$-th layer being $G_i$. Each node in $G_i$ has a unique parent node in $G_{i-1}$, which is determined by $f_{i,i-1}$. An element in $G$ is a path starting from the root and passing through each $G_i$ exactly once along the edges of the tree. The $i$-th component $x_i$ of an element $x\in G$ includes all information of its history path from $G_0$ to $G_i$, which makes $x_1,x_2,\cdots, x_{i-1}$ redundant.
\end{example}


\section{Group Action}
\subsection{Definitions}
\begin{definition}{Symmetric Group}{}
    The \textbf{symmetric group} on a set $X$ is the group whose elements are all bijections from $X$ to $X$, with the group operation of function composition. The symmetric group on $X$ is denoted by $\mathrm{Sym}(X)$ or $\mathrm{Aut}_{\mathsf{Set}}(X)$. If $X=\{1,2,\cdots,n\}$, then we denote $\mathrm{Sym}(X)$ by $S_n$.
\end{definition}

\begin{definition}{Group Action}{}
    Let $G$ be a group and $X$ be a set. A \textbf{group action} of $G$ on $X$ is a group homomorphism
    \begin{align*}
        \sigma:G & \longrightarrow \mathrm{Aut}_{\mathsf{Set}}(X) \\
        g        & \longmapsto \sigma_g
    \end{align*}
    If $G$ acts on $X$ by $\sigma$, we say $(X,\sigma)$ is a \textbf{$G$-set}. If there is no ambiguity, we simply say $X$ is a $G$-set.
\end{definition}



\begin{proposition}{Equivalent Definition of Group Actions}{}
    Let $G$ be a group and $X$ be a set. A group action of $G$ on $X$ can be alternatively defined as a map
    \begin{align*}
        \cdot:G\times X & \longrightarrow X    \\
        (g,x)           & \longmapsto g\cdot x
    \end{align*}
    such that
    \begin{enumerate}[(i)]
        \item $\forall x\in X$, $e\cdot x=x$.
        \item $\forall g,h\in G$, $\forall x\in X$, $(gh)\cdot x=g\cdot(h\cdot x)$.
    \end{enumerate}
    The equivalence of the two definitions is given by
    \[
        \sigma_g(x)=g\cdot x.
    \]
\end{proposition}

We say $X$ is a right $G$-set if $X$ is a left $G^{\mathrm{op}}$-set.

\begin{definition}{$G$-equivariant Map}{}
    Let $G$ be a group and $(X,\sigma)$, $(Y,\sigma')$ be $G$-sets. A map $f:X\to Y$ is called \textbf{$G$-equivariant} if for all $g\in G$ and $x\in X$, we have
    \[
        f(g\cdot x)=g\cdot f(x)    .
    \]
    Equivalently, $f$ is $G$-equivariant if it is a natural transformation $f:\sigma(-)\implies \sigma'(-)$ such that for any $g\in G$, the following naturality diagram commutes
    \[
        \begin{tikzcd}[ampersand replacement=\&, column sep=1.7em, row sep=small]
            \mathsf{B}G  \& \bullet \arrow[rr, "g"]                         \&  \& \bullet                 \\
            \& X \arrow[dd, "f"'] \arrow[rr, "\sigma_g"] \&  \& X \arrow[dd, "f"] \\
            \mathsf{Set} \&                                           \&  \&                   \\
            \& Y \arrow[rr, "\sigma'_g"']                \&  \& Y
        \end{tikzcd}
    \]
\end{definition}

\begin{definition}{Category of $G$-sets}{}
    The categories of left $G$-sets, denoted by $G\text{-}\mathsf{Set}$, are defined as follows:
    \begin{itemize}
        \item Objects: $G$-sets.
        \item Morphisms: $G$-equivariant maps.
        \item Composition of morphisms is the composition of functions.
    \end{itemize}
    $G\text{-}\mathsf{Set}$ can be identified with the functor category $[\mathsf{B}G,\mathsf{Set}]$, given by the following isomorphism of categories
\begin{align*}
    G\text{-}\mathsf{Set} & \stackrel{\sim}{\longrightarrow}[\mathsf{B}G, \mathsf{Set}] \\
    (X,\sigma )               & \longmapsto \left(\bullet \rightarrow X,\;\, \sigma:G\to \mathrm{Aut}_{\mathsf{Set}}(X)\right)
\end{align*}
    
\end{definition}


\begin{example}{Trivial Group Action}{}
    Let $G$ be a group and $X$ be a set. The \textbf{trivial group action} of $G$ on $X$ is defined as $\sigma_g=\mathrm{id}_X$ for all $g\in G$.
\end{example} 


\begin{example}{Actions on $X$ Induce Actions on $2^X$}{acting_on_power_set}
    If a group $G$ acts on a set $X$, then $G$ acts on the power set $2^X$ by
    \[
        g\cdot A=\{ g\cdot x\mid x\in A\}    .
    \]
\end{example}


\begin{definition}{Product of $G$-Sets}{}
    The \textbf{product} of two $G$-sets $X$ and $Y$ is defined as the set $X\times Y$ with the $G$-action
    \[
        g\cdot (x,y)=(g\cdot x, g\cdot y)    .
    \]
    Alternatively, the product of two $G$-sets can be defined as the product of two functors, cf. \Cref{th:ev_functor_preserves_limits}.
\end{definition}

\begin{definition}{Coproduct of $G$-Sets}{}
    The \textbf{coproduct} of two $G$-sets $X$ and $Y$ is defined as the set $X\sqcup Y$ with the $G$-action
    \[
        g\cdot a=\begin{cases}
            g\cdot a & a\in X \\
            g\cdot a & a\in Y
        \end{cases}
    \]
    Alternatively, the coproduct of two $G$-sets can be defined as the coproduct of two functors, cf. \Cref{th:ev_functor_preserves_limits}.
\end{definition}

\begin{example}{$\mathrm{Aut}_\mathsf{C}(X)$ acts on $\mathrm{Hom}_\mathsf{C}(X,Y)$ and $\mathrm{Hom}_\mathsf{C}(Y,X)$}{aut_acts_on_hom}
    Let $X$ and $Y$ be objects in a category $\mathsf{C}$. Then $\mathrm{Aut}_\mathsf{C}(X)$ acts on $\mathrm{Hom}_\mathsf{C}(X,Y)$ by the composition of functors
    \[
        \begin{tikzcd}[ampersand replacement=\&, column sep=5em, row sep=3em]
            \mathsf{B}\mathrm{Aut}_\mathsf{C}(X)\arrow[r,"{(-)}^{-1}"] \&[-3em] \mathsf{B}\mathrm{Aut}_\mathsf{C}(X)^{\mathrm{op}} \arrow[r, hook] \&[-2em] \mathsf{C}^{\mathrm{op}} \arrow[r, "{\mathrm{Hom}_{\mathsf{C}}(-,Y)}"] \& \mathsf{Set}\\[-2.3em]
            \bullet \arrow[r, maps to] \arrow[d, "g"']           \& \bullet \arrow[r, maps to] \arrow[d, "g^{-1}"]                           \& X \arrow[d, "g^{-1}"] \arrow[r, maps to]                               \& {\mathrm{Hom}(X,Y)} \arrow[d, "\left(g^{-1}\right)^*"] \\
            \bullet \arrow[r, maps to]                           \& \bullet \arrow[r, maps to]                                               \& X \arrow[r, maps to]                                                   \& {\mathrm{Hom}(X,Y)}
        \end{tikzcd}
    \]
    Writing explicily, the action is given by
    \begin{align*}
        \mathrm{Aut}_\mathsf{C}(X)\times \mathrm{Hom}_\mathsf{C}(X,Y) & \longrightarrow \mathrm{Hom}_\mathsf{C}(X,Y) \\
        (g,f)                                                         & \longmapsto f\circ g^{-1}
    \end{align*}
    Similarly, $\mathrm{Aut}_\mathsf{C}(Y)$ acts on $\mathrm{Hom}_\mathsf{C}(Y,X)$ by
    \[
        \begin{tikzcd}[ampersand replacement=\&, column sep=5em, row sep=3em]
            \mathsf{B}\mathrm{Aut}_\mathsf{C}(X)\arrow[r, hook] \&[-1em] \mathsf{C} \arrow[r, "{\mathrm{Hom}_{\mathsf{C}}(Y,-)}"] \& \mathsf{Set}\\[-2.3em]
            \bullet \arrow[r, maps to] \arrow[d, "g"']           \&  X \arrow[d, "g"] \arrow[r, maps to]                               \& {\mathrm{Hom}(Y,X)} \arrow[d, "g_*"] \\
            \bullet \arrow[r, maps to]                           \& X \arrow[r, maps to]                                                   \& {\mathrm{Hom}(X,Y)}
        \end{tikzcd}
    \]
    Writing explicily, the action is given by
    \begin{align*}
        \mathrm{Aut}_\mathsf{C}(Y)\times \mathrm{Hom}_\mathsf{C}(Y,X) & \longrightarrow \mathrm{Hom}_\mathsf{C}(Y,X) \\
        (g,f)                                                         & \longmapsto g\circ f
    \end{align*}
\end{example}

\begin{example}{Actions on $X$ Induce Actions on $\mathrm{Hom}_{\mathsf{Set}}(X,Y)$}{acting_on_functions}
    If $G$ acts on $X$ through a functor $\sigma(-):\mathsf{B}G\to\mathsf{Set}$, then it also acts on $\mathrm{Hom}_{\mathsf{Set}}(X,Y)$ for any set $Y$ by the composition of functors
    \[
        \begin{tikzcd}[ampersand replacement=\&, column sep=5em, row sep=3em]
            \mathsf{B}G \arrow[r,"{(-)}^{-1}"] \&[-2em] \mathsf{B}G^{\mathrm{op}} \arrow[r, "\sigma(-)^{\mathrm{op}}"] \&[-2.2em] \mathsf{Set}^{\mathrm{op}} \arrow[r, "{\mathrm{Hom}_{\mathsf{Set}}(-,Y)}"] \& \mathsf{Set}
        \end{tikzcd}
    \]
    The left action on $\mathrm{Hom}_{\mathsf{Set}}(X,Y)$ is given explicitly as follows: for all $g\in G$, $f\in \mathrm{Hom}_{\mathsf{Set}}(X,Y)$ and $x\in X$,
    \[
        (g\cdot f)(x)=f(g^{-1}\cdot x).
    \]
    Equivalently, the right action $\star$ on $\mathrm{Hom}_{\mathsf{Set}}(X,Y)$ is given by
    \[
        (f\star g)(x)=f(g\cdot x).
    \]


\end{example}

\begin{prf}
    We can check that
    \begin{align*}
        (g_1\cdot (g_2\cdot f))(x) & =\left(g_2\cdot f\right)\left(g_1^{-1}\cdot x\right)     \\
                                   & =\left(g_2\cdot f\right)\left(g_1^{-1}\cdot x\right)     \\
                                   & =f\left(g_2^{-1}\cdot\left(g_1^{-1}\cdot x\right)\right) \\
                                   & =f\left(\left(g_2^{-1} g_1^{-1}\right)\cdot x\right)     \\
                                   & =\left(\left(g_1g_2\right)\cdot f\right)(x).
    \end{align*}
    and also check that
    \begin{align*}
        ((f\star g_1)\star g_2)(x)
         & =\left(f\star g_1\right)\left(g_2\cdot x\right) \\
         & =f\left(g_1\cdot\left(g_2\cdot x\right)\right)  \\
         & =f\left((g_1g_2)\cdot x\right)                  \\
         & =\left(f\star (g_1 g_2)\right)(x).
    \end{align*}
\end{prf}

\begin{definition}{Orbit of a Group Action}{}
    Let $G$ be a group acting on a set $X$. For $x\in X$, the \textbf{orbit} of $x$ is defined as
    \[
        G x=\{ g\cdot x\mid g\in G\}    .
    \]
\end{definition}


\begin{definition}{Orbit Space}{}
    Let $G$ be a group acting on a set $X$. The \textbf{orbit space} of $G$ acting on $X$ is defined as
    \[
        G\backslash X=\{ Gx\mid x\in X\}.
    \]
\end{definition}

If $G$ acts on $X$, then $G$ acts on $G\backslash X$ trivially by $g\cdot Gx=Gx$.
\begin{proposition}{Orbit Decomposition}{}
    Let $G$ be a group acting on a set $X$. We define a equivalence relation $\sim$ on $X$ by
    \[
        x\sim y \iff Gx=Gy.
    \]
    Then the equivalence class of $x$ is exactly $Gx$. The quotient set $X/\sim$ is exactly the orbit space $G\backslash X$.
    And we have a partition of $X$ by the orbits of $G$ acting on $X$
    \[
        X=\bigsqcup_{Gx\in G\backslash X}Gx.
    \]
\end{proposition}

\begin{prf}
    We can check that the equivalence class of $x$ is $Gx$. If $y\sim x$, then $y\in Gy=Gx$. If $y\in Gx$, then $Gy\subseteq Gx$ and $x\in Gy$. Note $x\in Gy$ implies $Gx\subseteq Gy$. We have $Gx=Gy$, i.e. $x\sim y$.
\end{prf}


If $G$ acts on $X$, then $G$ acts on $G\backslash X$ trivially.
\begin{definition}{$G$-invariant element }{}
    Let $G$ be a group acting on a set $X$. An element $x\in X$ is called \textbf{$G$-invariant} if $Gx=\{ x\}$ or equivalently $|Gx|=1$. The set of all $G$-invariant elements is denoted by $X^G$
    \[
        X^G=\{ x\in X\mid Gx=\{ x\} \} =  \{ x\in X\mid \forall g\in G, g\cdot x=x \} .
    \]
\end{definition}


\begin{definition}{Stabilizer Subgroup}{}
    Let $G$ be a group acting on a set $X$. For $x\in X$, the \textbf{stabilizer subgroup} of of $G$ with respect to $x$ is defined as
    \[
        \mathrm{Stab}_G(x)=\{ g\in G\mid g\cdot x=x\}    .
    \]
    It is easy to see that $\mathrm{Stab}_G(x)$ is a subgroup of $G$.
\end{definition}

\begin{proposition}{Properties of Stabilizer Subgroup}{properties_of_stabilizer_subgroup}
    Let $G$ be a group acting on a set $X$. For $x\in X$, the stabilizer subgroup $\mathrm{Stab}_G(x)$ has the following properties
    \begin{enumerate}[(i)]
        \item $x\in X^G\iff \mathrm{Stab}_G(x)=G$.
        \item $\ker \left(G\to \mathrm{Aut}_{\mathsf{Set}}(X)\right)=\bigcap\limits_{x\in X}\mathrm{Stab}_G(x)$.
        \item $\mathrm{Stab}_G(gx)=g\mathrm{Stab}_G(x)g^{-1}$ for any $g\in G$. Hence, $\{\mathrm{Stab}_G(x)\mid x\in X\}$ is a conjugacy class in $G$.
    \end{enumerate}
    If $X \curvearrowleft G$ is a right action, then we have $\mathrm{Stab}_G(xg)=g^{-1}\mathrm{Stab}_G(x)g$ for any $g\in G$.
\end{proposition}
\begin{proof}
    \begin{enumerate}[(i)]
        \item 
        $$
        x \in X^G \iff \forall g \in G, gx = x \iff \mathrm{Stab}_G(x) = G.
        $$
        \item 
        $$
        g \in \ker \left(G \to \mathrm{Aut}_{\mathsf{Set}}(X) \right) \iff  \forall x \in X ,\; gx = x \iff g \in \bigcap_{x \in X} \mathrm{Stab}_G(x).
        $$    
        \item  Let $h \in \mathrm{Stab}_G(gx)$, meaning $h(gx) = gx$.
        Applying $g^{-1}$ to both sides, we get $g^{-1}h(gx) = g^{-1}(gx)=x$. Thus, $g^{-1}hg \in \mathrm{Stab}_G(x)$, meaning $h \in g \mathrm{Stab}_G(x) g^{-1}$.
        Conversely, if $h \in g \mathrm{Stab}_G(x) g^{-1}$, then $h = g k g^{-1}$ for some $k \in \mathrm{Stab}_G(x)$. Therefore, $h(gx) = g(kx) = gx$, so $h \in \mathrm{Stab}_G(gx)$.
        Thus, $\mathrm{Stab}_G(gx) = g \mathrm{Stab}_G(x) g^{-1}$.
    \end{enumerate}
\end{proof}


\begin{definition}{Faithful Group Action}{}
    Let $G$ be a group acting on a set $X$. The action is called \textbf{faithful} if any of the following equivalent conditions holds
    \begin{enumerate}[(i)]
        \item $G\to \mathrm{Aut}_{\mathsf{Set}}(X)$ is injective.
        \item $\bigcap\limits_{x\in X}\mathrm{Stab}_G(x)=\{ 1_G\}$.
        \item $\forall x\in X,\;g\cdot x=x\implies g=1_G$.
    \end{enumerate}
\end{definition}


\begin{definition}{Free Group Action}{}
    Let $G$ be a group acting on a set $X$. The action is called \textbf{free} if any of the following equivalent conditions holds
    \begin{enumerate}[(i)]
        \item For all $x\in X$, $\mathrm{Stab}_G(x)=\{ 1_G\}$ .
        \item $\exists x\in X,\;g\cdot x=x\implies g=1_G$.
    \end{enumerate}
\end{definition}

It is clear that a free action is faithful, but the converse does not hold in general.

\begin{definition}{Transitive Group Action}{}
    Let $G$ be a group acting on a set $X$. The action is called \textbf{transitive} if any of the following equivalent conditions holds
    \begin{enumerate}[(i)]
        \item For any $x,y\in X$, there exists $g\in G$ such that $g\cdot x=y$.
        \item $X$ has only one orbit, i.e. $X= Gx$ for any $x\in X$.
    \end{enumerate}
    If $G$ acts transitively on $X$, then $X$ is called a \textbf{homogeneous space} for $G$.
\end{definition}

The following proposition shows that we can understand a group action on a set $X$ by studying the group action on each $G$-orbit $Gx$ separately.

\begin{proposition}{$G$ Acts on Orbit $Gx$ Transitively}{}
    Let $G$ be a group acting on a set $X$ and $x\in X$. Then $G$ acts on the orbit $Gx$ by left multiplication transitively. And we have a $G$-set isomorphism
    \[
        X\cong \bigsqcup_{Gx\in G\backslash X}Gx,
    \]
    which decomposes any $G$-set into coproduct of transitive $G$-sets.
\end{proposition}

\begin{example}{The Orbit Decomposition of Subgroup Action}{}
    Let $G$ be a group acting on a set $X$ with orbit decomposition
    \[
        X=\bigsqcup_{i \in I} G x_i
    \]
    Suppose $H$ be a subgroup of $G$ and $G$ has right coset decomposition
    \[
        G=\bigsqcup_{j \in J} Hg_j
    \]
    Then $H$ also acts on $X$ and each $G$-orbit is disjoint union of some $H$-orbits, which can written as
    \[
        Gx_i =\bigsqcup_{k \in K} H s_{ik}.
    \]
    More concretely, $Gx_i$ is the union of the cosets $H (g_jx_i)\;(j \in J)$,
    \[
        Gx_i =\bigcup_{j \in J} H g_j x_i.
    \]
    But $H (g_{j}x_i)$ may coincide with $H (g_{j'}x_i)$ for $j\ne j'$. We can duplicate $H (g_jx_i)\;(j \in J)$ by checking if there exists $h \in H$ such that $h g_j s_i=g_{j'} s_i$. Suppose $a \sim_H b$ iff $a$ and $b$ in the same $H$-orbit. Then we get
    \[
        \{s_{ik}\mid k \in K\}=\{g_j x_i \mid j \in J\}/\sim_H.
    \]
\end{example}

\begin{definition}{Regular Group Action}{}
    Let $G$ be a group acting on a set $X$. The action is called \textbf{regular} if any of the following equivalent conditions holds
    \begin{enumerate}[(i)]
        \item The action is transitive and free.
        \item For any $x,y\in X$, there exists unique $g\in G$ such that $g\cdot x=y$.
    \end{enumerate}
    If $G$ acts regularly on $X$, then $X$ is called a \textbf{principal homogeneous space} for $G$ or a $G$-\textbf{torsor}.
\end{definition}

\subsection{Coset}
\begin{example}{Left Multiplication Action}{}
    Let $G$ be a group. The \textbf{left multiplication action} of $G$ on itself is defined as
    \begin{align*}
        m^L:G & \longrightarrow \mathrm{Aut}(G) \\
        g     & \longmapsto ( x\longmapsto gx)
    \end{align*}
\end{example}

\begin{example}{Right Multiplication Action}{}
    Let $G$ be a group. The \textbf{right multiplication action} of $G$ on itself is defined as
    \begin{align*}
        m^R:G^\circ & \longrightarrow \mathrm{Aut}(G) \\
        g           & \longmapsto ( x\longmapsto xg)
    \end{align*}
\end{example}

\begin{definition}{Left Cosets}{}
    Let $G$ be a group and $H$ be a subgroup of $G$. $H^\circ$ can act on $G$ through $H^\circ\hookrightarrow G^\circ\stackrel{m^R}{\longrightarrow} \mathrm{Aut}(G)$, namely
    \begin{align*}
        H^\circ & \longrightarrow \mathrm{Aut}(G) \\
        h       & \longmapsto  (g\longmapsto gh)
    \end{align*}
    The orbit of $g$ under $H^\circ$ is called the \textbf{left coset} of $H$ containing $g$, denoted by $gH$
    \[
        gH = H^\circ g = \{ gh\mid h\in H\}.
    \]
    The set of all left cosets of $H$ is denoted by $G/H$, called the left coset space of $G$ modulo $H$. $G/H$ is the orbit space of $G$ under the right multiplication action of $H$.
\end{definition}

\begin{example}{$G$ Acts on $G/H$ Transitively}{}
    Let $G$ be a group and $H$ be a subgroup of $G$. $G$ acts on $G/H$ through
    \begin{align*}
        G & \longrightarrow \mathrm{Aut}(G/H) \\
        g & \longmapsto  (xH\longmapsto gxH)
    \end{align*}
    For any $xH,yH\in G/H$, we have $yH=gxH$ for some $g=yx^{-1}\in G$. Thus $G$ acts on $G/H$ transitively.
\end{example}

\begin{definition}{Right Cosets}{}
    Let $G$ be a group and $H$ be a subgroup of $G$. $H$ can act on $G$ through $H\hookrightarrow G\stackrel{m^L}{\longrightarrow} \mathrm{Aut}(G)$, namely
    \begin{align*}
        H & \longrightarrow \mathrm{Aut}(G) \\
        h & \longmapsto  (g\longmapsto hg)
    \end{align*}
    The orbit of $g$ under $H$ is called the \textbf{right coset} of $H$ containing $g$, denoted by $Hg$
    \[
        Hg = \{ hg\mid h\in H\},
    \]
    which matches notation of orbit. The set of all right cosets of $H$ is denoted by $H\backslash G$, called the right coset space of $G$ modulo $H$.
\end{definition}

\begin{definition}{Index of Subgroup}{}
    Let $G$ be a group and $H$ be a subgroup of $G$. The \textbf{index} of $H$ in $G$ is defined as the cardinality of $G/H$ or $H\backslash G$, denoted by $[G:H]$.
\end{definition}

\begin{theorem}{Lagrange's Theorem}{}
    Let $G$ be a finite group and $H$ be a subgroup of $G$. Then $|G|=|H|[G:H]$.
\end{theorem}


\begin{proposition}{$G$-Set Isomorphism $G/\mathrm{Stab}_G(x)\cong Gx$}{iso_stab_orbit}
    Let $G$ be a group acting on a set $X$ and $x\in X$. Then the map
    \begin{align*}
        F:G/\mathrm{Stab}_G(x)          & \longrightarrow Gx   \\
        g\hspace{1pt}\mathrm{Stab}_G(x) & \longmapsto g\cdot x
    \end{align*}
    is a $G$-set isomorphism.
\end{proposition}

\begin{prf}
    The map is well-defined since for any $h\in \mathrm{Stab}_G(x)$, we have $(gh)\cdot x=g\cdot (h\cdot x)=g\cdot x$. The map is a $G$-set homomorphism since for any $g_1,g_2\in G$,
    \begin{align*}
        F\left(g_1\cdot g_2\mathrm{Stab}_G(x)\right)=(g_1g_2)\cdot x=g_1\cdot(g_2\cdot x)=g_1\cdot F\left(g_2\mathrm{Stab}_G(x)\right).
    \end{align*}
    The map is injective since for any $g,h\in G$, if $g\cdot x=h\cdot x$, then $h^{-1}g\in \mathrm{Stab}_G(x)$. Hence $g\mathrm{Stab}_G(x)=h\mathrm{Stab}_G(x)$. The map is surjective because for any $g\cdot x\in Gx$, we have $F\left(g\hspace{1pt}\mathrm{Stab}_G(x)\right)=g\cdot x$.
\end{prf}

\begin{theorem}{Orbit-Stabilizer Theorem}{}
    Let $G$ be a group acting on a set $X$. For $x\in X$, we have
    \[
        |G|=|Gx|\cdot |\mathrm{Stab}_G(x)|    .
    \]
\end{theorem}

\begin{prf}
    According to \Cref{th:iso_stab_orbit}, we have
    \[
        \left|Gx\right|=\left|G/\mathrm{Stab}_G(x)\right|=|G|/\left|\mathrm{Stab}_G(x)\right|.
    \]
\end{prf}

\begin{theorem}{Burnside's Lemma}{Burnside's_lemma}
    Let $G$ be a finite group acting on a finite set $X$. Then the number of orbits of $G$ on $X$ is equal to
    \[
        |G\backslash X|=\frac{1}{|G|}\sum_{g\in G}|X^g|    ,
    \]
    where $X^g=\{x\in X\mid g\cdot x=x\}$ is the set of fixed points of $g$.
\end{theorem}

\begin{prf}
    \begin{align*}
        \sum_{g \in G}\left|X^g\right| & =|\{(g, x) \in G \times X \mid g \cdot x=x\}|                                        \\
                                       & =\sum_{x \in X}\left|\mathrm{Stab}_G(x)\right|                                       \\
                                       & =\sum_{x \in X}\frac{|G|}{\left|G x\right|} \quad \text{by Orbit-Stabilizer Theorem} \\
                                       & =|G| \sum_{G y \in G \backslash X}\; \sum_{x \in Gy}\frac{1}{\left|G x\right|}       \\
                                       & =|G| \sum_{G y \in G \backslash X} \left|Gy\right|\frac{1}{\left|G y\right|}         \\
                                       & =|G| \sum_{G y \in G \backslash X} 1                                                 \\
                                       & =|G| \cdot|G \backslash X|.
    \end{align*}

\end{prf}



\subsection{Conjugacy Action}
\begin{definition}{Conjugacy Action and Inner Automorphism Group}{conjugacy_action}
    Let $G$ be a group. The \textbf{conjugacy action} of $G$ on itself is defined as a group homomorphism
    \begin{align*}
        \gamma:G & \longrightarrow \mathrm{Aut}_{\mathsf{Grp}}(G) \\
        g        & \longmapsto (\gamma_g: x\longmapsto gxg^{-1})
    \end{align*}
    The \textbf{inner automorphism group} of $G$ is defined as the image of $\gamma$
    $$
        \mathrm{Inn}(G)=\operatorname{im}\gamma=\{ \gamma_g\mid g\in G\}.
    $$
    And we have inclusion relation $\mathrm{Inn}(G)\hookrightarrow\mathrm{Aut}_{\mathsf{Grp}}(G)\hookrightarrow\mathrm{Aut}_{\mathsf{Set}}(G)$.
\end{definition}

\begin{definition}{Conjugate Subgroups}{}
    From \Cref{ex:acting_on_power_set}, we see conjugacy action on $G$ induces an action on its power set $2^G$: \begin{align*}
        G\times 2^G & \longrightarrow 2^G  \\
        (g,E)       & \longmapsto gEg^{-1}
    \end{align*}
    If $H$ is a subgroup of $G$, then $gHg^{-1}$ is also a subgroup of $G$. We say $H$ and $gHg^{-1}$ are \textbf{conjugate subgroups} of $G$.
\end{definition}

\begin{proposition}{Equivalent
        Characterization of Inner Automorphisms}{}
    Let $G$ be a group and $\varphi \in \mathrm{Aut}(G)$. Then $\varphi \in \mathrm{Inn}(G)$ if and only if $\varphi$ satisfies the property:
    \[
        \text{$G$ is embedded in a group $H$} \implies \text{$\varphi$ extends to an automorphism of $H$}.
    \]
    To be specific, the property can be stated as: for any monomophism $\iota:G\hookrightarrow H$,
    there exists $\psi\in \mathrm{Aut}(H)$ such that the following diagram commutes
    \[
        \begin{tikzcd}[ampersand replacement=\&]
            G \arrow[r, hook, "\iota"] \arrow[d, "\varphi"'] \& H \arrow[d, "\psi"] \\
            G \arrow[r, hook, "\iota"']                      \& H
        \end{tikzcd}
    \]
\end{proposition}




\begin{definition}{Outer Automorphism Group}{}
    Let $G$ be a group. Then we have $\mathrm{Inn}(G) \lhd \mathrm{Aut}_{\mathsf{Grp}}(G)$. And the \textbf{outer automorphism group} of $G$ is defined as
    $$
        \mathrm{Out}(G)=\mathrm{coker}\,\gamma=\mathrm{Aut}_{\mathsf{Grp}}(G)/\mathrm{Inn}(G).
    $$
\end{definition}

\begin{definition}{Characteristic Subgroup}{}
    Let $G$ be a group. A subgroup $H\le G$ is called a \textbf{characteristic subgroup} if
    \[
        \forall \varphi \in\mathrm{Aut}_{\mathsf{Grp}}(G),\; \varphi(H)\subseteq H.
    \]
    It would be equivalent to require the stronger condition that $\forall \varphi \in\mathrm{Aut}_{\mathsf{Grp}}(G)$, $\varphi(H)= H$, because
    \[
        \varphi(H)\subseteq H\implies \varphi^{-1}(H)\subseteq H\implies H\subseteq \varphi(H).
    \]
\end{definition}

\begin{definition}{Fully Characteristic Subgroup}{}
    Let $G$ be a group. A subgroup $H\le G$ is called a \textbf{fully characteristic subgroup} if
    \[
        \forall \varphi \in\mathrm{End}_{\mathsf{Grp}}(G),\; \varphi(H)\subseteq H.
    \]
\end{definition}

\begin{definition}{Word Map}{}
    Suppose $G$ is a group and
    $$
        x=x_{i_1}^{\alpha_{1}}\cdots x_{i_m}^{\alpha_{m}}\in F\langle x_1,\cdots,x_n\rangle
    $$
    is a reduced word in a free group of rank $n$, where $\alpha_k\in\mathbb{Z}-\{0\}$ for $k=1,2,\cdots,m$. The \textbf{word map} induced by $x$ is defined as a map
    \begin{align*}
        w_x:G^m          & \longrightarrow G                                        \\
        (g_1,\cdots,g_m) & \longmapsto g_{i_1}^{\alpha_1}\cdots g_{i_m}^{\alpha_m}.
    \end{align*}
\end{definition}

\begin{definition}{Verbal Subgroup}{}
    Let $G$ be a group and $\mathcal{W}$ be a collection of word maps. A subgroup $H\le G$ is called a \textbf{verbal subgroup} if $H$ is the subgroup generated by
    $$
        \left\{ w(g_1,\cdots,g_n)\mid w\in\mathcal{W},\; g_i\in G \right\}.
    $$
\end{definition}

\begin{definition}{Commutator}{}
    Let $G$ be a group. The word map induced by $xyx^{-1}y^{-1}$ is a binary operation defined on $G$, denoted by
    \begin{align*}
        [\cdot,\cdot]:G\times G & \longrightarrow G                \\
        (x,y)                   & \longmapsto [x,y]=xyx^{-1}y^{-1}
    \end{align*}
    $[x,y]$ is called the \textbf{commutator} of $x$ and $y$.
\end{definition}

\begin{proposition}{Properties of Commutator}{}
    Let $G$ be a group. Then
    \begin{enumerate}[(i)]
        \item $x$ commutes with $y$ if and only if $[x,y]=1_G$.
        \item $[x,y]^{-1}=[y,x]$.
        \item For any homomorphism $f:G\to H$, $f([x,y])=[f(x),f(y)]$.
    \end{enumerate}
\end{proposition}

\begin{proposition}{}{}
    According to the extent that a subgroup is preserved by endomorphisms, we have the following inclusions
    \[
        \left\{\text{verbal subgroups}\right\}\subseteq \left\{\text{fully characteristic subgroups}\right\}\subseteq \left\{\text{characteristic subgroups}\right\}\subseteq\left\{\text{normal subgroups}\right\}.
    \]
\end{proposition}


\begin{definition}{Commutator Subgroup}{}
    Let $G$ be a group. The \textbf{commutator subgroup} or \textbf{derived subgroup} of $G$ is the subgroup generated by all the commutators, denoted by
    $$
        [G,G]=\langle \left\{[x,y]\mid x,y\in G \right\}\rangle.
    $$
\end{definition}

\begin{proposition}{Properties of Commutator Subgroup}{}
    Let $G$ be a group. Then
    \begin{enumerate}[(i)]
        \item $[G,G]$ is a verbal subgroup with $\mathcal{W}=\left\{ [\cdot,\cdot] \right\}$. Hence $[G,G]\lhd G$.
        \item $[G,G]$ is the smallest normal subgroup of $G$ such that $G/[G,G]$ is abelian.
        \item $[G,G]=\{1_G\}$ if and only if $G$ is abelian.
    \end{enumerate}
\end{proposition}


\begin{definition}{Abelianization}{}
    Let $G$ be a group. The \textbf{abelianization} of $G$ is defined as the quotient group
    $$
        G^{\mathrm{ab}}=G/[G,G].
    $$
\end{definition}

\begin{proposition}{Universal Property of Abelianization}{}
    Let $G$ be a group and $A$ be an abelian group. Then any group homomorphism $f:G\to A$ factors through $G^{\mathrm{ab}}$ uniquely, that is, there exists a unique homomorphism $\bar{f}:G^{\mathrm{ab}}\to A$ such that the following diagram commutes
    $$
        \begin{tikzcd}[ampersand replacement=\&]
            G \arrow[rr, "f"] \arrow[rd] \&  \& A \\
            \& G^{\mathrm{ab}} \arrow[ru, "\bar{f}"'] \&
        \end{tikzcd}
    $$
\end{proposition}

\begin{definition}{Normalizer}{}
    Let $G$ be a group and $S$ be a subset of $G$. The \textbf{normalizer} of $S$ in $G$ is defined as
    $$
        \mathrm{N}_G(S)=\left\{ g\in G\mid gSg^{-1}=S \right\}.
    $$
    Let $G$ acts on $2^G$ by conjugation (c.f. \Cref{th:conjugacy_action}). Then $\mathrm{N}_G(S)=\mathrm{Stab}_G(S)\le G$.
\end{definition}

\begin{proposition}{Normalizer of a Subgroup is the Largest Subgroup in which the Subgroup is Normal}{}
    Let $G$ be a group and $H$ be a subgroup of $G$. Then $H\lhd\mathrm{N}_G(H)$. Moreover, $\mathrm{N}_G(H)$ is the largest subgroup of $G$ in which $H$ is normal, i.e. $$H\lhd K\le G\implies H\lhd K\le \mathrm{N}_G(H)\le G.$$
\end{proposition}

\begin{prf}
    For all $n\in \mathrm{N}_G(H)$, we have $nHn^{-1}=H$, which implies $H \lhd \mathrm{N}_G(H)$. Suppose $H\lhd K\le G$. Then $\forall k\in K$, $kHk^{-1}=H$. Hence $k\in \mathrm{N}_G(H)$. Therefore we prove the maximality of $\mathrm{N}_G(H)$.
\end{prf}

\begin{definition}{Centralizer}{}
    Let $G$ be a group and $S$ be a subset of $G$. The \textbf{centralizer} of $S$ in $G$ is defined as
    \[
        \mathrm{C}_G(S)=\left\{g\in G\mid  \forall s\in S,\;gs=sg\right\}.
    \]
    The centralizer of $\{x\}$ is the stabilizer subgroup of $x$ under conjugacy action, denoted by
    \[
        \mathrm{C}_G(x)=\{ g\in G\mid gx=xg \}=\{ g\in G\mid gxg^{-1}=x \}= \mathrm{Stab}_G(x)=\mathrm{N}_G(x).
    \]

\end{definition}



\begin{definition}{Center of a Group}{}
    Let $G$ be a group. The \textbf{center} of $G$ is defined as the centralizer of $G$ in $G$, denoted by
    \[
        Z_G=\mathrm{C}_G(G)=\{ g\in G\mid \forall x\in G,\; gx=xg\}.
    \]
\end{definition}



\begin{proposition}{Normalizer $\mathrm{N}_G(S)$ Acts on $S$ by Conjugation}{normalizer_conjugation_action}
    Let $G$ be a group and $S\subseteq G$. Then $\mathrm{N}_G(S)$ acts on $S$ by conjugation, i.e. by the group homomorphism
    \begin{align*}
        \Psi_S: \mathrm{N}_G(S) & \longrightarrow \mathrm{Aut}_{\mathsf{Set}}(S) \\
        g                       & \longmapsto \gamma_g|_S
    \end{align*}
    where $\gamma_g|_S(s)=gsg^{-1}$ for all $s\in S$. Moreover, we have $\ker\Psi_S=\mathrm{C}_G(S)\lhd \mathrm{N}_G(S)$.
\end{proposition}

\proof{
$\Psi_S$ is obtained from restriction $\Psi_S:\mathrm{N}_G(S)\hookrightarrow G\xrightarrow{\gamma}\mathrm{Aut}_{\mathsf{Set}}(G)$. Since for any $g\in \mathrm{N}_G(S)$, $\gamma_g|_S(S)=\left\{gsg^{-1}\in G\mid s \in S\right\}\subseteq S$, we see $\Psi_S(g)=\gamma_g|_S\in \mathrm{Aut}_{\mathsf{Set}}(S)$. The kernel of $\Psi_S$ is
\begin{align*}
    \ker\Psi_S & = \left\{ n\in \mathrm{N}_G(S)\mid \gamma_n|_S=\mathrm{id}_S \right\}= \left\{ n\in \mathrm{N}_G(S)\mid \forall s\in S, nsn^{-1}=s \right\}=\mathrm{N}_G(S)\cap \mathrm{C}_G(S) = \mathrm{C}_G(S).
\end{align*}
}

\begin{theorem}{N/C Theorem}{N/C_theorem}
    Let $G$ be a group and $H$ be a subgroup of $G$. By \Cref{th:normalizer_conjugation_action}, $\mathrm{N}_G(H)$ acts on $H$ by conjugation through the group homomorphism $\Psi_H:\mathrm{N}_G(H)\to \mathrm{Aut}_{\mathsf{Set}}(H)$. We assert that
    \[
        \mathrm{N}_G(H)/\mathrm{C}_G(H) \cong \mathrm{Im}\Psi_H\le \mathrm{Aut}_{\mathsf{Grp}}(H).
    \]
    Hence it is legal to define
    \begin{align*}
        \Psi_H: \mathrm{N}_G(H) & \longrightarrow \mathrm{Aut}_{\mathsf{Grp}}(H) \\
        g                       & \longmapsto \gamma_g|_H
    \end{align*}
\end{theorem}

\begin{corollary}{Kernel and Cokernel of Conjugation Action $\gamma:G\to \mathrm{Aut}_{\mathsf{Grp}}(G)$}{kernel_cokernel_of_conjugation}
    Let $G$ be a group. Then $G/Z_G\cong \mathrm{Inn}(G)$. That means the conjugation action of a central element is trivial. The kernel and cokernel of the conjugation $\gamma:G\to \mathrm{Aut}_{\mathsf{Grp}}(G)$ can be connected by the following exact sequence
    \[
        1\longrightarrow Z_G\longrightarrow G\xrightarrow{\hspace{5pt}\gamma\hspace{5pt}}\mathrm{Aut}_{\mathsf{Grp}}(G)\longrightarrow \mathrm{Out}(G)\longrightarrow 1.
    \]
\end{corollary}

\begin{prf}
    Take $H=G$ in \Cref{th:N/C_theorem} we can get $G/Z_G\cong \mathrm{Inn}(G)$.
\end{prf}





\begin{proposition}{Properties of Centralizer}{}
    Let $G$ be a group and $S$ be a subset of $G$. Then
    \begin{enumerate}[(i)]
        \item $S\subseteq\mathrm{C}_G(\mathrm{C}_G(S))$.
    \end{enumerate}
\end{proposition}

\begin{proposition}{Properties of Center}{}
    Let $G$ be a group. Then
    \begin{enumerate}[(i)]
        \item $Z_G=\bigcap_{g \in G} C_G(g)$.
        \item $Z_G\lhd_{\mathrm{char}} G$.
        \item A group is Abelian if and only if $Z_G=G$.
    \end{enumerate}
\end{proposition}
\begin{prf}
    \begin{enumerate}[(i)]
        \item According to \Cref{th:kernel_cokernel_of_conjugation} and \Cref{th:properties_of_stabilizer_subgroup} (ii), we have
              \[
                  x\in Z_G\iff x\in \ker \gamma\iff x\in
                  \bigcap\limits_{g\in G}\mathrm{C}_G(g).
              \]
    \end{enumerate}
\end{prf}

\begin{definition}{Conjugacy Class}{}
    Let $G$ be a group. The orbit of $a$ under conjugacy action is called the \textbf{conjugacy class} of $a$, denoted by
    \[
        \mathrm{Cl}(a)=\{ gag^{-1}\mid x\in G   \}.
    \]
    Two elements $a,b\in G$ are called \textbf{conjugate} if $\mathrm{Cl}(a)=\mathrm{Cl}(b)$.
\end{definition}


\begin{proposition}{Conjugacy Class of Element of Center is Singleton}{}
    Consider a group $G$ under conjugacy action. The $G$-invariant elements under conjugacy action are elements in the center of $G$:
    \[
        a\in Z_G\iff \mathrm{Cl}(a)=\{ a\}\iff \mathrm{C}_G(a)=\mathrm{Stab}_G(a)=G.
    \]
\end{proposition}
\begin{prf}
    According to \Cref{th:properties_of_stabilizer_subgroup} (i), $\mathrm{Cl}(a)=\{ a\}\iff \mathrm{Stab}_G(a)=G$.
    $$
        \begin{aligned}
            a  \in Z_G \iff & \forall g \in G,\;g a  =a g     \\
            \iff            & \forall g \in G,\;g a g^{-1} =a \\
            \iff            & \mathrm{C}_G(a) =\{a\}
        \end{aligned}
    $$
\end{prf}

\begin{proposition}{Conjugacy Class Equation}{}
    Suppose $G$ is a finite group. If the distinct conjugacy classes of $G$ which are not singletons are $\mathrm{Cl}(x_1),\cdots,\mathrm{Cl}(x_m)$, then we have the \textbf{conjugacy class equation}
    \[
        |G|=|Z(G)|+\sum_{j=1}^m\left[G: Z_G\left(x_j\right)\right]
    \]
\end{proposition}
\begin{prf}
    With the orbit decomposition of $G$ under conjugacy action, then by orbit-stabilizer theorem we have
    \[
        |G|=\sum_{Gx\in G\backslash G}\left|\mathrm{Cl}(x)\right|=\sum_{Gx\in G\backslash G}\left[G: \mathrm{Stab}_G(x)\right]=|Z(G)|+\sum_{j=1}^m\left[G: Z_G\left(x_j\right)\right].
    \]
\end{prf}

\begin{proposition}{}{}
    Let $G$ acts transitively on a set $S$. Choose $s \in S$, and let $H = \mathrm{Stab}_G(s)$. Then we have group isomorphism $\mathrm{N}_G(H)/H\cong \mathrm{Aut}_{G\text{-}\mathsf{Set}}(S)$.
\end{proposition}

\begin{prf}
    For any $n \in \mathrm{N}_G(H)$ with image $\bar{n} \in \mathrm{N}_G(H)/H$, since $G$ acts transitively on $S$, we can define a map
    \begin{align*}
        \phi(\bar{n}):S & \longrightarrow S          \\
        g\cdot s        & \longmapsto gn^{-1}\cdot s
    \end{align*}
    and check $\phi(\bar{n})\in\mathrm{Aut}_{G\text{-}\mathsf{Set}}(S)$ by
    \begin{itemize}
        \item $\phi(\bar{n})\in\mathrm{Aut}_{G\text{-}\mathsf{Set}}(S)$. $\phi(\bar{n})$ is well-defined:
              \begin{align*}
                           & \bar{m}=\bar{n}\in\mathrm{N}_G(H)/H                                              \\
                  \implies & m^{-1}\in n^{-1}H                                                                \\
                  \implies & \exists h\in H,m^{-1}=n^{-1}h                                                    \\
                  \implies & \phi(\bar{m})(s)=gm^{-1}\cdot s=gn^{-1}h\cdot s=gn^{-1}\cdot s=\phi(\bar{n})(s).
              \end{align*}
        \item $\phi(\bar{n})$ is a $G$-set morphism: $\phi(\bar{n})(g\cdot s)=gn^{-1}\cdot s=g\cdot\phi(\bar{n})(s)$
        \item $\phi(\bar{n})$ is a bijection: $\phi(\bar{n}^{-1})$ is the inverse of $\phi(\bar{n})$.
    \end{itemize}
    For any automorphism $\psi\in \mathrm{Aut}_{G\text{-}\mathsf{Set}}(S)$, by transitivity we have $\psi(s)=n^{-1}\cdot s$ for some $n \in G$. For $h \in H$, $hn^{-1}\cdot s=h\cdot\psi(s)=\psi(h\cdot s)=\psi(s)=n^{-1}\cdot s$, hence $n h n^{-1} \in H$ and $n \in \mathrm{N}_G(H)$. This gives a well-defined map
    \begin{align*}
        \eta:\mathrm{Aut}_{G\text{-}\mathsf{Set}}(S) & \longrightarrow \mathrm{N}_G(H)/H \\
        \psi                                         & \longmapsto \bar{n}
    \end{align*}
    Clearly $\eta(\phi(\bar{n}))=\bar{n}$. Suppose $\psi\in\mathrm{Aut}_{G\text{-}\mathsf{Set}}(S)$ and $\psi(s)=n^{-1}\cdot s$. Then $\phi(\eta(\psi))(g\cdot s)=g n^{-1}\cdot s=g\psi(s)=\psi(g\cdot s)$, which imples $\phi(\eta(\psi))=\psi$. Therefore, $\phi:\mathrm{N}_G(H)/H\to\mathrm{Aut}_{G\text{-}\mathsf{Set}}$ is bijective. And it is easy to check that this $\phi$ is an isomorphism of groups,
    \[
        \phi(\bar{n}\bar{m})(g\cdot s)=\phi(\overline{nm})(g\cdot s)=g(nm)^{-1}\cdot s=gm^{-1}n^{-1}\cdot s=\phi(\bar{n})(gm^{-1}\cdot s)=\phi(\bar{n})\circ\phi(\bar{m})(g\cdot s).
    \]
\end{prf}


\begin{proposition}{$G$-maps between Left Coset Spaces}{}
    Suppose $H,K$ are subgroups of $G$. Then we know $G$ left acts on $G/H$ and $G/K$. A $G$-map $\alpha: G / H \longrightarrow G / K$ has the form $\alpha(g H)=g r K$, where the element $r \in G$ satisfies $r^{-1} H r\subseteq K$.
\end{proposition}

\begin{prf}
    Note
    \[
        \alpha(g H)=\alpha(g 1_G H)=g \alpha(1_G H).
    \]
    Taking $\alpha(1_G H)=r K$, then we obtain $\alpha(g H)=g r K$. Since for all $h\in H$ we have
    $$
        r K=\alpha(h1_G H)=h \alpha(1_G H)=hr K\implies r^{-1} h r \in K,
    $$
    we show $r^{-1} H r\subseteq K$.
\end{prf}


\section{Symmetric Groups}

\begin{definition}{Symmetric Group}{}
    The \textbf{symmetric group} on a set $X$ is the group of all permutations of $X$, denoted by $S_X=\mathrm{Aut}_{\mathsf{Set}}(X)$. If $X=\{1,2,\cdots,n\}$, then we denote $S_X$ by $S_n$.
\end{definition}

\begin{definition}{$k$-Cycle}{}
    Let $n\ge 2$ and $k\ge 1$ be integers. A \textbf{$k$-cycle} in $S_n$ is a permutation $\sigma\in S_n$ such that there exist a subset of $P_n=\{1,2,\cdots,n\}$, denoted by $A=\{a_1,a_2,\cdots,a_k\}$, satisfying
    \begin{enumerate}[(i)]
        \item $\sigma(a_i)=a_{i+1}\text{ for }i=1,2,\cdots,k-1$,
        \item $\sigma(a_k)=a_1$,
        \item $\sigma(x)=x\text{ for }x\in P_n-A$.
    \end{enumerate}
\end{definition}

\begin{definition}{Cycle Decomposition}{}
    Let $n\ge 2$ and $\sigma\in S_n$. A \textbf{cycle decomposition} of $\sigma$ is a product of disjoint cycles $\sigma=\sigma_1\sigma_2\cdots\sigma_r$ such that $\sigma_i$ is a $k_i$-cycle for $i=1,2,\cdots,r$ and $k_1+k_2+\cdots+k_r=n$. $(k_1,k_2,\cdots,k_r)$ is called the \textbf{cycle type} of $\sigma$. Equivalently, a \textbf{cycle decomposition} of $\sigma$ is the decomposition of $P_n$ into orbits under the action of $\langle\sigma\rangle$.
\end{definition}


\begin{theorem}{Pólya Enumeration Theorem (Unweighted)}{Polya_enumeration_unweighted}
    Let $X, Y$ be finite sets, where $X=\{1,2,\cdots,n\}$ is the set of points to be colored and $Y$ is the set of colors. Suppose a group $G$ acts on $X$ through $\sigma:G\to\mathrm{Aut}_{\mathsf{Set}}(X)$. Then it also acts on $Y^X$ by \Cref{ex:acting_on_functions}. Define that a coloring configuration of $(X,Y,\sigma)$ is an orbit in $Y^X/G$. Then the number of essentially distinct coloring configurations is
    $$
        \left|Y^X / G\right|=\frac{1}{|G|} \sum_{g \in G}|Y|^{c(g)},
    $$
    where $c(g):=\left|\langle g\rangle \backslash X\right|=\left|\langle \sigma_g\rangle \backslash X\right|$ denotes the number of cycles in the cycle decomposition of $\sigma_g \in S_n$.
\end{theorem}

\begin{prf}
    We apply Burnside's lemma \ref{th:Burnside's_lemma}, which states that
    $$
        \left|Y^X / G\right|=\frac{1}{|G|} \sum_{g \in G}\left|\left(Y^X\right)^g\right|
    $$
    where
    \begin{align*}
        \left(Y^X\right)^g & =\left\{ f \in Y^X \mid f(g \cdot x)=f (x)\text{ for all }x \in X\right\}                  \\
                           & =\left\{ f \in Y^X \mid \langle g\rangle x = \langle g\rangle y\implies f(x)=f(y)\right\}.
    \end{align*}
    Define an equivalence relation $\sim_g$ on $X$ by $x \sim_g y\iff\langle g\rangle x = \langle g\rangle y$. Then we have $X/\sim_g\;=\langle g\rangle \backslash X$. By the universal property of the quotient set, for any map $f:X\to Y$ satisfying $x\sim y\implies f(x)=f(y)$, there exists a unique map $\overline{f}:\langle \sigma_g\rangle \backslash X\to Y$ such that the following diagram commutes
    \[
        \begin{tikzcd}[ampersand replacement=\&]
            X \arrow[rr, "f"] \arrow[rd,"\pi"'] \&  \& Y \\
            \& \langle \sigma_g\rangle \backslash X \arrow[ru, "\overline{f}"',dashed] \&
        \end{tikzcd}
    \]
    And we have a natural bijection between $\left(Y^X\right)^g$ and $Y^{\langle g\rangle \backslash X}$, which implies that $\left|\left(Y^X\right)^g\right|=|Y|^{c(g)}$.
\end{prf}

\begin{definition}{Cycle Index Polynomial}{}
    Let $G$ be a subgroup of $S_n$. The \textbf{cycle index polynomial} of $G$ is defined as
    $$
        Z(t_1,t_2,\cdots,t_n;G)=\frac{1}{|G|}\sum_{g\in G}t_1^{c_1(g)}t_2^{c_2(g)}\cdots t_n^{c_n(g)},
    $$
    where $c_k(g)$ denotes the number of $k$-cycles in the cycle decomposition of $\sigma_g$. $|G|Z(t_1,t_2,\cdots,t_n;G)$ can be seen as a generating function, where the coefficient of $t_1^{c_1}t_2^{c_2}\cdots t_n^{c_n}$ represents the number of permutations in $G$ with excatly $c_k$ $k$-cycles for $k=1,2,\cdots,n$.
\end{definition}

\begin{theorem}{Pólya Enumeration Theorem (Weighted)}{}
    Let $X, Y$ be finite sets, where $X=\{1,2,\cdots,n\}$ is the set of points to be colored and $Y$ is the set of colors.  Suppose $w:Y\to\mathbb{Z}_{\ge 0}^m$ is a weight function which assigns a weight $w(y)=(w_1(y),w_2(y),\cdots,w_m(y))$ to each color $y\in Y$. Consider the genrating function
    \[
        q(x_1,\cdots,x_m)=\sum_{y\in Y}x_1^{w_1(y)}x_2^{w_2(y)}\cdots x_m^{w_m(y)},
    \]
    where the coefficient of the term $x_1^{a_1}x_2^{a_2}\cdots x_m^{a_m}$ is the number of colors with weight $(a_1,a_2,\cdots,a_m)$. For each coloring map $f:X\to Y$, define the its weight as $W(f)$, where
    \begin{align*}
        W:Y^X & \longrightarrow\mathbb{Z}_{\ge 0}^m \\
        f     & \longmapsto\sum_{x\in X}w(f(x))
    \end{align*}
    Let $G$ be a subgroup of $S_n$. Given any $g\in G$ and $f\in X^Y$, we can check the action of $g$ on $f$ does not change its weight
    \begin{align*}
        W(f\star g)=\sum_{x\in X}w(( f\star g)(x))=\sum_{x\in X}w(f(g\cdot x))=\sum_{x\in X}w(f(x))=W(f),
    \end{align*}
    which implies that for each $\omega=(\omega_1,\cdots,\omega_m)\in\mathbb{Z}_{\ge 0}^m$, the fiber $W^{-1}(\omega)$ is $G$-invariant.

    Define a coloring configuration of $(X,Y,G)$ as an orbit in $Y^X/G$. The generating function for the number of essentially distinct coloring configurations with weight $\omega$ can be expressed as
    \begin{align*}
        \mathrm{CGF}\left(x_1, \cdots,x_m\right) & =\sum_{\omega \in \mathbb{Z}_{\ge 0}^m}\left|G\backslash W^{-1}(\omega)\right| x_1^{\omega_1} \cdots x_m^{\omega_m}            \\
                                                 & = Z\left(q\left(x_1, \cdots,x_m\right), q\left(x_1^2, \cdots,x_m^2\right), \cdots, q\left(x_1^n, \cdots,x_m^n\right);G\right).
    \end{align*}
\end{theorem}

\begin{prf}
    For any $\omega=(\omega_1,\cdots,\omega_m)\in\mathbb{Z}_{\ge 0}^m$, by applying Burnside's lemma \ref{th:Burnside's_lemma} to the $G$-set $W^{-1}(\omega)$, we have
    \[
        \left|G\backslash W^{-1}(\omega)\right| =\frac{1}{|G|}\sum_{g\in G}\left|W^{-1}(\omega)^g\right|.
    \]
    Similar to \Cref{th:Polya_enumeration_unweighted}, we have a bijection between
    \[
        W^{-1}(\omega)^g=\left\{f\in \left(Y^X\right)^g\mid W(f)=\omega\right\}\quad\text{and}\quad\left\{\overline{f}\in Y^{\langle g\rangle\backslash X}\mid W\left(\,\overline{f}\circ\pi\right)=\omega\right\},
    \]
    Suppose $\langle g\rangle\backslash X=\{\langle g\rangle x_1,\cdots,\langle g\rangle x_r\}$. We can give a coloring configuration by $r$ consecutive steps. In the $i$-th step, we just need to choose a color in $y\in Y$ for the $i$-orbit $\langle g\rangle\backslash x_i$, which will contribute a term
    $$
        x_1^{|\langle g\rangle x_i| w_1(y)}x_2^{|\langle g\rangle x_i|w_2(y)}\cdots x_m^{|\langle g\rangle x_i| w_m(y)}.
    $$
    Thus we have
    \begin{align*}
        \sum_{\omega\in\mathbb{Z}_{\ge 0}^m} \left |W^{-1}(\omega)^g\right |x_1^{\omega_1} \cdots x_m^{\omega_m} & = \prod_{\langle g\rangle x_i \in \langle g\rangle\backslash X }\left(\sum_{y\in Y}x_1^{|\langle g\rangle x_i|w_1(y)}x_2^{|\langle g\rangle x_i|w_2(y)}\cdots x_m^{|\langle g\rangle x_i|w_m(y)}\right) \\
                                                                                                                 & = \prod_{\langle g\rangle x_i \in \langle g\rangle\backslash X } q\left (x_1^{|\langle g\rangle x_i|}, x_2^{|\langle g\rangle x_i|}, \cdots,x_m^{|\langle g\rangle x_i|}\right )                        \\
                                                                                                                 & = q(x_1,  \cdots,x_m)^{c_1(g)} q \left (x_1^2, \cdots,x_m^2 \right )^{c_2(g)} \cdots q \left (x_1^n, \cdots,x_m^n\right )^{c_n(g)}.
    \end{align*}
    where
    $$
        c_k(g)=\sum_{\langle g\rangle x_i \in \langle g\rangle\backslash X}\mathbf{1}_{|\langle g\rangle x_i |=k}
    $$
    denotes the number of orbits with size $k$. With this equality, we can rewrite the generating function as
    \begin{align*}
        \mathrm{CGF}\left(x_1, \cdots,x_m\right) & =\sum_{\omega \in \mathbb{Z}_{\ge 0}^m}\left|G\backslash W^{-1}(\omega)\right| x_1^{\omega_1} \cdots x_m^{\omega_m}                                          \\
                                                 & =\sum_{\omega \in \mathbb{Z}_{\ge 0}^m}\frac{1}{|G|}\sum_{g \in G}\left|W^{-1}(\omega)^g\right| x_1^{\omega_1} \cdots x_m^{\omega_m}                         \\
                                                 & =\frac{1}{|G|}\sum_{g \in G}\sum_{\omega \in \mathbb{Z}_{\ge 0}^m}\left|W^{-1}(\omega)^g\right| x_1^{\omega_1} \cdots x_m^{\omega_m}                         \\
                                                 & =\frac{1}{|G|}\sum_{g \in G}q(x_1,  \cdots,x_m)^{c_1(g)} q \left (x_1^2, \cdots,x_m^2 \right )^{c_2(g)} \cdots q \left (x_1^n, \cdots,x_m^n\right )^{c_n(g)} \\
                                                 & = Z\left(q\left(x_1, \cdots,x_m\right), q\left(x_1^2, \cdots,x_m^2\right), \cdots, q\left(x_1^n, \cdots,x_m^n\right);G\right).
    \end{align*}
\end{prf}

\begin{example}{Counting the Isomers of Chlorobenzene}{}
    Replacing the H in a benzene ring with Cl, one can consider coloring the 6 vertices of the benzene ring with two colors: H and Cl. The group action is $D_{12}$, which includes 6 rotations: 0°, 60°, ..., 300°, and 6 reflections: 3 along the opposing sides and 3 along the opposing diagonals.

    Compute the cycle decomposition for each \(g\):
    \begin{itemize}
        \item Identity: 6 1-cycles, corresponding to \(t_1^6\).
        \item Rotation by 1 or 5 times 60°: 1 6-cycle, corresponding to \(t_6^1\).
        \item Rotation by 2 or 4 times 60°: 2 3-cycles, corresponding to \(t_3^2\).
        \item Rotation by 3 times 60°: 3 2-cycles, corresponding to \(t_2^3\).
        \item 3 kinds of opposing side reflections: 3 2-cycles, corresponding to \(t_2^3\).
        \item 3 kinds of opposing diagonal reflections: 2 1-cycles and 2 2-cycles, corresponding to \(t_1^2t_2^2\).
    \end{itemize}

    Writing out \(Z(t_1,\cdots,t_6;D_{12})\):
    \[
        Z(t_1,\cdots,t_6;D_{12})=\frac{1}{12}\left(t_1^6+3 t_1^2 t_2^2+4 t_2^3+2 t_3^2+2 t_6\right)
    \]

    Assigning a weight of (1,0) to H and a weight of (0,1) to Cl, the corresponding generating function is \(q(\text{H},\text{Cl})=\text{H}+\text{Cl}\). Finally, the generating function for the number of essentially distinct coloring configurations is:

    \[
        \begin{aligned}
             & \quad \mathrm{CGF}(\text{H},\text{Cl})                                                                                                                                                                                           \\
             & =\frac{1}{12}\left((\text{H}+\text{Cl})^6+3(\text{H}+\text{Cl})^2\left(\text{H}^2+\text{Cl}^2\right)^2+4\left(\text{H}^2+\text{Cl}^2\right)^3+2\left(\text{H}^3+\text{Cl}^3\right)^2+2\left(\text{H}^6+\text{Cl}^6\right)\right) \\
             & =\text{H}^6+\text{H}^5 \text{Cl}+3 \text{H}^4 \text{Cl}^2+3 \text{H}^3 \text{Cl}^3+3 \text{H}^2 \text{Cl}^4+\text{H} \text{Cl}^5+\text{Cl}^6
        \end{aligned}
    \]

    The coefficients give the number of isomers for various chlorobenzene compounds. For instance, looking at the term \(3 \text{H}^4 \text{Cl}^2\), with a weight of $(4,2)$, it can only be achieved using 4 H atoms and 2 Cl atoms. The coefficient 3 indicates that there are 3 isomers for dichlorobenzene. The 3 isomers are 1,2-dichlorobenzene, 1,3-dichlorobenzene, and 1,4-dichlorobenzene, plotted as follows:

    \begin{center}
        \setchemfig{atom sep=1.5em}
        \chemfig[scale=0.1]{*6(=-(-Cl)=(-Cl)-=-)} \qquad % 1,2-dichlorobenzene
        \chemfig[scale=0.1]{*6(=-=(-Cl)-=(-Cl)-=)} \qquad % 1,3-dichlorobenzene
        \chemfig[scale=0.1]{*6(=-(-Cl)=-=(-Cl)-)}     % 1,4-dichlorobenzene
    \end{center}
\end{example}



\section{Abelian Group}
\begin{definition}{Abelian Group}{}
    An \textbf{abelian group} is a group $G$ such that $G$ is commutative. That is, for all $a,b\in G$, $ab=ba$.
\end{definition}


An Abelian group is a $\mathbb{Z}$-module and we have category isomorphism $\mathsf{Ab}\cong\mathbb{Z}\raisebox{0.22ex}{-}\mathsf{Mod}$.

\begin{example}{Forgetful Functor $U_{\mathsf{Grp}}: \mathsf{Ab}\to \mathsf{Grp}$}{}
    The forgetful functor $U_{\mathsf{Grp}}: \mathsf{Ab}\to \mathsf{Grp}$ is a functor that sends an abelian group to its underlying group. It is a fully faithful functor.
\end{example}


\begin{example}{Forgetful Functor $U: \mathsf{Ab}\to \mathsf{Set}$}{}
    The forgetful functor $U: \mathsf{Ab}\to \mathsf{Set}$ forgets the group structure and sends an abelian group to its underlying set.
    \begin{enumerate}[(i)]
        \item $U$ is representable by $\left(\mathbb{Z}, 1_\mathbb{Z}\right)$.
        \item $U$ is full but not faithful.
    \end{enumerate}
\end{example}


\begin{definition}{Free Abelian Group}{}
    A \textbf{free abelian group} generated by a set $X$ is an abelian group denoted by $\mathbb{Z}^{\oplus X}$ which satisfies the following universal property: for any group $G$ and any funtion $f:X\to G$, there exists a unique group homomorphism $\varphi:\mathbb{Z}^{\oplus X}\to G$ such that the following diagram commutes
    \[
        \begin{tikzcd}[ampersand replacement=\&]
            \mathbb{Z}^{\oplus X} \arrow[r, "\exists!\varphi", dashed] \& G \\[+10pt]
            X \arrow[u, "\iota"] \arrow[ru, "f"']                      \&
        \end{tikzcd}
    \]
    where $\iota:X\to \mathbb{Z}^{\oplus X}$ is the canonical injection.

\end{definition}






