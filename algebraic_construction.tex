\documentclass{report}

%%%%%%%%%%%%%%%%%%%%%%%%%%%%%%%%%
% PACKAGE IMPORTS
%%%%%%%%%%%%%%%%%%%%%%%%%%%%%%%%%


\usepackage[tmargin=2cm,rmargin=1in,lmargin=1in,margin=0.85in,bmargin=1.5cm,footskip=.2in]{geometry}
\usepackage{amsmath,amsfonts,amsthm,amssymb,mathtools}

% \usepackage[bb=dsserif]{mathalpha}
% \usepackage{bm}
\usepackage{dsfont}
\usepackage{bookmark} 
\usepackage{booktabs}
\usepackage{enumerate}
\usepackage[shortlabels]{enumitem}
\usepackage{hyperref,theoremref}
\usepackage[table]{xcolor}
\usepackage[most,many,breakable]{tcolorbox}
\usepackage{xcolor}
\usepackage{graphicx}
\graphicspath{ {./images/} }
\usepackage{varwidth}
\usepackage{varwidth}
\usepackage{etoolbox}
%\usepackage{authblk}
\usepackage{nameref}
\usepackage{multicol,array}
\usepackage{tikz-cd}
\usepackage{cancel}
\usepackage{pgfplots}
\pgfplotsset{compat=newest}
\usepgfplotslibrary{patchplots}
\usepackage{anyfontsize}
\usepackage{sectsty}
%\usepackage{import}
%\usepackage{xifthen}
%\usepackage{pdfpages}
%\usepackage{transparent}
\usepackage[nameinlink]{cleveref}
\usepackage{fancyhdr}


%%%%%%%%%%%%%%%%%%%%%%%%%%%%%%
% HEADER AND FOOTER
%%%%%%%%%%%%%%%%%%%%%%%%%%%%%%
\pagestyle{fancy}

% Clear all header and footer fields
\fancyhf{}

% Redefine the chaptermark
\renewcommand{\chaptermark}[1]{%
  \markboth{\thechapter.\ \MakeUppercase{#1}}{}%
}

% Chapter name on the left of the header
\fancyhead[L]{\leftmark}

% Section name on the right of the header
\fancyhead[R]{\rightmark}

% Page number on the bottom right in the format "- 34 -"
\fancyhead[C]{\raisebox{-0pt}{-\hspace{4pt}\thepage\hspace{4pt}-}}

% Line at the top of the header
\renewcommand{\headrulewidth}{0.0pt}

\fancypagestyle{plain}{%
  \fancyhf{} % clear all header and footer fields
  \renewcommand{\headrulewidth}{0pt} % remove the header rule
  \renewcommand{\footrulewidth}{0pt} % remove the footer rule
}
%%%%%%%%%%%%%%%%%%%%%%%%%%%%%%
% SELF MADE COLORS
%%%%%%%%%%%%%%%%%%%%%%%%%%%%%%

\usetikzlibrary{shapes.geometric}
\usetikzlibrary{calc}
\usepackage{anyfontsize}

\definecolor{myg}{RGB}{56, 140, 70}
\definecolor{myb}{RGB}{45, 111, 177}
\definecolor{myr}{RGB}{199, 68, 64}
\definecolor{mytheorembg}{HTML}{fdf8ea} %orange
\definecolor{mytheoremfr}{HTML}{f19000}
\definecolor{myexamplebg}{HTML}{F2FBF8}
\definecolor{myexamplefr}{HTML}{88D6D1}
\definecolor{myexampleti}{HTML}{2A7F7F}
\definecolor{mydefinitbg}{HTML}{F2F2F9} %blue
\definecolor{mydefinitfr}{HTML}{00007B} 
\definecolor{mypropbg}{RGB}{56, 140, 70} %green
\definecolor{mypropfr}{RGB}{56, 140, 70}
\definecolor{mylemmabg}{RGB}{169, 144, 126}
\definecolor{mylemmafr}{RGB}{169, 144, 126}
\definecolor{notesgreen}{RGB}{0,162,0}
\definecolor{myp}{RGB}{197, 92, 212}
\definecolor{mygr}{HTML}{2C3338}
\definecolor{myred}{RGB}{127,0,0}
\definecolor{myyellow}{RGB}{169,121,69}
\definecolor{OrangeRed}{HTML}{ED135A}
\definecolor{Dandelion}{HTML}{FDBC42}
\definecolor{light-gray}{gray}{0.95}
\definecolor{Emerald}{HTML}{00A99D}
\definecolor{RoyalBlue}{HTML}{0071BC}

% \definecolor{mydefnewbg}{HTML}{FFF0E0}
% \definecolor{mydefnewfr}{HTML}{FF9900}
\definecolor{myRoyalBlue}{RGB}{50,60,150}
\definecolor{myGreen}{RGB}{45,100,45}
\definecolor{mynavy}{RGB}{56, 102, 148} % RGB color model

%%%%%%%%%%%%%%%%%%%%%%%%%%%%%%
% HYPERREF SETUP
%%%%%%%%%%%%%%%%%%%%%%%%%%%%%%

\hypersetup{
	colorlinks=true,
	linkcolor=mynavy,
	citecolor=BurntOrange,
	bookmarksnumbered=true,
	bookmarksopen=true,
}

%%%%%%%%%%%%%%%%%%%%%%%%%%%%
% TCOLORBOX SETUPS
%%%%%%%%%%%%%%%%%%%%%%%%%%%%


%================================
% NEW DEFINITION BOX
%================================


\newtcbtheorem[number within=section]{definition}{Definition}
{%
    enhanced
    ,breakable
    ,colback = mydefinitbg
    ,frame hidden
    ,boxrule = 0sp
    ,borderline west = {2pt}{0pt}{mydefinitfr}
    ,sharp corners
    ,detach title
    ,before upper = \tcbtitle\par\smallskip
    ,coltitle = mydefinitfr!85!black
    ,fonttitle = \bfseries\sffamily
    ,description font = \mdseries
    ,separator sign none
    ,segmentation style={solid, mydefinitfr!85!black}
	,label type=definition
}
{th}

% \newtcbtheorem[number within=chapter]{definition}{Definition}
% {%
%     enhanced
%     ,breakable
%     ,colback = mydefinitbg
%     ,frame hidden
%     ,boxrule = 0sp
%     ,borderline west = {2pt}{0pt}{mydefinitfr}
%     ,sharp corners
%     ,detach title
%     ,before upper = \tcbtitle\par\smallskip
%     ,coltitle = mydefinitfr!85!black
%     ,fonttitle = \bfseries\sffamily
%     ,description font = \mdseries
%     ,separator sign none
%     ,segmentation style={solid, mydefinitfr!85!black}
% 	,label type=definition
% }
% {th}

\setlength{\parindent}{0.5cm}
%================================
% THEOREM BOX
%================================

\tcbuselibrary{theorems,skins,hooks}
\newtcbtheorem[use counter from=definition, number within=section]{theorem}{Theorem}
{%
	enhanced,
	breakable,
	colback = mytheorembg,
	frame hidden,
	boxrule = 0sp,
	borderline west = {2pt}{0pt}{mytheoremfr},
	sharp corners,
	detach title,
	before upper = \tcbtitle\par\smallskip,
	coltitle = mytheoremfr,
	fonttitle = \bfseries\sffamily,
	description font = \mdseries,
	separator sign none,
	segmentation style={solid, mytheoremfr},
	label type=theorem
}
{th}

% \tcbuselibrary{theorems,skins,hooks}
% \newtcbtheorem[number within=chapter]{theorem}{Theorem}
% {%
% 	enhanced,
% 	breakable,
% 	colback = mytheorembg,
% 	frame hidden,
% 	boxrule = 0sp,
% 	borderline west = {2pt}{0pt}{mytheoremfr},
% 	sharp corners,
% 	detach title,
% 	before upper = \tcbtitle\par\smallskip,
% 	coltitle = mytheoremfr,
% 	fonttitle = \bfseries\sffamily,
% 	description font = \mdseries,
% 	separator sign none,
% 	segmentation style={solid, mytheoremfr},
% 	label type=theorem
% }
% {th}


\tcbuselibrary{theorems,skins,hooks}
\newtcolorbox{Theoremcon}
{%
	enhanced
	,breakable
	,colback = mytheorembg
	,frame hidden
	,boxrule = 0sp
	,borderline west = {2pt}{0pt}{mytheoremfr}
	,sharp corners
	,description font = \mdseries
	,separator sign none
}


%================================
% Corollery
%================================
\tcbuselibrary{theorems,skins,hooks}
\newtcbtheorem[use counter from=definition, number within=section]{corollary}{Corollary}
{%
	enhanced
	,breakable
	,colback = myp!10
	,frame hidden
	,boxrule = 0sp
	,borderline west = {2pt}{0pt}{myp!85!black}
	,sharp corners
	,detach title
	,before upper = \tcbtitle\par\smallskip
	,coltitle = myp!85!black
	,fonttitle = \bfseries\sffamily
	,description font = \mdseries
	,separator sign none
	,segmentation style={solid, myp!85!black}
	,label type=corollary
}
{th}
% \tcbuselibrary{theorems,skins,hooks}
% \newtcbtheorem[number within=chapter]{corollary}{Corollary}
% {%
% 	enhanced
% 	,breakable
% 	,colback = myp!10
% 	,frame hidden
% 	,boxrule = 0sp
% 	,borderline west = {2pt}{0pt}{myp!85!black}
% 	,sharp corners
% 	,detach title
% 	,before upper = \tcbtitle\par\smallskip
% 	,coltitle = myp!85!black
% 	,fonttitle = \bfseries\sffamily
% 	,description font = \mdseries
% 	,separator sign none
% 	,segmentation style={solid, myp!85!black}
% 	,label type=corol
% }
% {th}

%================================
% CLAIM
%================================

\tcbuselibrary{theorems,skins,hooks}
\newtcbtheorem[number within=section]{claim}{Claim}
{%
	enhanced
	,breakable
	,colback = myg!10
	,frame hidden
	,boxrule = 0sp
	,borderline west = {2pt}{0pt}{myg}
	,sharp corners
	,detach title
	,before upper = \tcbtitle\par\smallskip
	,coltitle = myg!85!black
	,fonttitle = \bfseries\sffamily
	,description font = \mdseries
	,separator sign none
	,segmentation style={solid, myg!85!black}
	,label type=claim
}
{th}


\newtcbtheorem[number within=chapter]{Claim}{Claim}
{%
	enhanced
	,breakable
	,colback = myg!10
	,frame hidden
	,boxrule = 0sp
	,borderline west = {2pt}{0pt}{myg}
	,sharp corners
	,detach title
	,before upper = \tcbtitle\par\smallskip
	,coltitle = myg!85!black
	,fonttitle = \bfseries\sffamily
	,description font = \mdseries
	,separator sign none
	,segmentation style={solid, myg!85!black}
	,label type=claim
}
{th}
%================================
% PROPOSITION
%================================

\tcbuselibrary{theorems,skins,hooks}
\newtcbtheorem[use counter from=definition, number within=section]{proposition}{Proposition}
{%
	enhanced
	,breakable
	,colback = mypropbg!10
	,frame hidden
	,boxrule = 0sp
	,borderline west = {2pt}{0pt}{mypropfr!85!black}
	,sharp corners
	,detach title
	,before upper = \tcbtitle\par\smallskip
	,coltitle = mypropfr!85!black
	,fonttitle = \bfseries\sffamily
	,description font = \mdseries
	,separator sign none
	,segmentation style={solid, mypropfr!85!black}
	,label type=proposition
}
{th}

% \newtcbtheorem[number within=chapter]{Proposition}{Proposition}
% {%
% 	enhanced
% 	,breakable
% 	,colback = mypropbg!10
% 	,frame hidden
% 	,boxrule = 0sp
% 	,borderline west = {2pt}{0pt}{mypropfr!85!black}
% 	,sharp corners
% 	,detach title
% 	,before upper = \tcbtitle\par\smallskip
% 	,coltitle = mypropfr!85!black
% 	,fonttitle = \bfseries\sffamily
% 	,description font = \mdseries
% 	,separator sign none
% 	,segmentation style={solid, mypropfr!85!black}
% 	,label type=proposition
% }
% {th}
%================================
% LEMMA
%================================

\tcbuselibrary{theorems,skins,hooks}
\newtcbtheorem[use counter from=definition, number within=section]{lemma}{Lemma}
{%
	enhanced
	,breakable
	,colback = mylemmabg!10
	,frame hidden
	,boxrule = 0sp
	,borderline west = {2pt}{0pt}{mylemmafr!85!black}
	,sharp corners
	,detach title
	,before upper = \tcbtitle\par\smallskip
	,coltitle = mylemmafr!85!black
	,fonttitle = \bfseries\sffamily
	,description font = \mdseries
	,separator sign none
	,segmentation style={solid, mylemmafr!85!black}
	,label type=lemma
}
{th}

% \newtcbtheorem[number within=chapter]{Lemma}{Lemma}
% {%
% 	enhanced
% 	,breakable
% 	,colback = mylemmabg!10
% 	,frame hidden
% 	,boxrule = 0sp
% 	,borderline west = {2pt}{0pt}{mylemmafr!85!black}
% 	,sharp corners
% 	,detach title
% 	,before upper = \tcbtitle\par\smallskip
% 	,coltitle = mylemmafr!85!black
% 	,fonttitle = \bfseries\sffamily
% 	,description font = \mdseries
% 	,separator sign none
% 	,segmentation style={solid, mylemmafr!85!black}
% 	,label type=lemma
% }
% {th}
%================================
% EXAMPLE BOX
%================================

\newtcbtheorem[number within=section]{example}{Example}
{%
	colback = myexamplebg
	,breakable
	,colframe = myexamplefr
	,coltitle = myexampleti
	,boxrule = 1pt
	,sharp corners
	,detach title
	,before upper=\tcbtitle\par\smallskip
	,fonttitle = \bfseries\sffamily
	%,fonttitle = \bfseries
	,description font = \mdseries\sffamily
	%,description font = \mdseries
	,separator sign none
	%,description delimiters parenthesis
	,label type=example
}
{ex}

% \newtcbtheorem[number within=chapter]{example}{Example}
% {%
% 	colback = myexamplebg
% 	,breakable
% 	,colframe = myexamplefr
% 	,coltitle = myexampleti
% 	,boxrule = 1pt
% 	,sharp corners
% 	,detach title
% 	,before upper=\tcbtitle\par\smallskip
% 	,fonttitle = \bfseries
% 	,description font = \mdseries
% 	,separator sign none
% 	,description delimiters parenthesis
% 	,label type=example
% }
% {ex}

%================================
% DEFINITION BOX
%================================

% \newtcbtheorem[number within=section]{Definition}{Definition}{enhanced,
% 	before skip=2mm,after skip=2mm, colback=red!5,colframe=red!80!black,boxrule=0.5mm,
% 	attach boxed title to top left={xshift=1cm,yshift*=1mm-\tcboxedtitleheight}, varwidth boxed title*=-3cm,
% 	boxed title style={frame code={
% 					\path[fill=tcbcolback]
% 					([yshift=-1mm,xshift=-1mm]frame.north west)
% 					arc[start angle=0,end angle=180,radius=1mm]
% 					([yshift=-1mm,xshift=1mm]frame.north east)
% 					arc[start angle=180,end angle=0,radius=1mm];
% 					\path[left color=tcbcolback!60!black,right color=tcbcolback!60!black,
% 						middle color=tcbcolback!80!black]
% 					([xshift=-2mm]frame.north west) -- ([xshift=2mm]frame.north east)
% 					[rounded corners=1mm]-- ([xshift=1mm,yshift=-1mm]frame.north east)
% 					-- (frame.south east) -- (frame.south west)
% 					-- ([xshift=-1mm,yshift=-1mm]frame.north west)
% 					[sharp corners]-- cycle;
% 				},interior engine=empty,
% 		},
% 	fonttitle=\bfseries,
% 	title={#2},#1}{def}
% \newtcbtheorem[number within=chapter]{definition}{Definition}{enhanced,
% 	before skip=2mm,after skip=2mm, colback=red!5,colframe=red!80!black,boxrule=0.5mm,
% 	attach boxed title to top left={xshift=1cm,yshift*=1mm-\tcboxedtitleheight}, varwidth boxed title*=-3cm,
% 	boxed title style={frame code={
% 					\path[fill=tcbcolback]
% 					([yshift=-1mm,xshift=-1mm]frame.north west)
% 					arc[start angle=0,end angle=180,radius=1mm]
% 					([yshift=-1mm,xshift=1mm]frame.north east)
% 					arc[start angle=180,end angle=0,radius=1mm];
% 					\path[left color=tcbcolback!60!black,right color=tcbcolback!60!black,
% 						middle color=tcbcolback!80!black]
% 					([xshift=-2mm]frame.north west) -- ([xshift=2mm]frame.north east)
% 					[rounded corners=1mm]-- ([xshift=1mm,yshift=-1mm]frame.north east)
% 					-- (frame.south east) -- (frame.south west)
% 					-- ([xshift=-1mm,yshift=-1mm]frame.north west)
% 					[sharp corners]-- cycle;
% 				},interior engine=empty,
% 		},
% 	fonttitle=\bfseries,
% 	title={#2},#1}{def}



%================================
% OPEN QUESTION BOX
%================================

\newtcbtheorem[number within=section]{open}{Open Question}{enhanced,
	before skip=2mm,after skip=2mm, colback=myp!5,colframe=myp!80!black,boxrule=0.5mm,
	attach boxed title to top left={xshift=1cm,yshift*=1mm-\tcboxedtitleheight}, varwidth boxed title*=-3cm,
	boxed title style={frame code={
			\path[fill=tcbcolback]
			([yshift=-1mm,xshift=-1mm]frame.north west)
			arc[start angle=0,end angle=180,radius=1mm]
			([yshift=-1mm,xshift=1mm]frame.north east)
			arc[start angle=180,end angle=0,radius=1mm];
			\path[left color=tcbcolback!60!black,right color=tcbcolback!60!black,
			middle color=tcbcolback!80!black]
			([xshift=-2mm]frame.north west) -- ([xshift=2mm]frame.north east)
			[rounded corners=1mm]-- ([xshift=1mm,yshift=-1mm]frame.north east)
			-- (frame.south east) -- (frame.south west)
			-- ([xshift=-1mm,yshift=-1mm]frame.north west)
			[sharp corners]-- cycle;
		},interior engine=empty,
	},
	fonttitle=\bfseries,
	title={#2},#1}{def}
\newtcbtheorem[number within=chapter]{Open}{Open Question}{enhanced,
	before skip=2mm,after skip=2mm, colback=myp!5,colframe=myp!80!black,boxrule=0.5mm,
	attach boxed title to top left={xshift=1cm,yshift*=1mm-\tcboxedtitleheight}, varwidth boxed title*=-3cm,
	boxed title style={frame code={
			\path[fill=tcbcolback]
			([yshift=-1mm,xshift=-1mm]frame.north west)
			arc[start angle=0,end angle=180,radius=1mm]
			([yshift=-1mm,xshift=1mm]frame.north east)
			arc[start angle=180,end angle=0,radius=1mm];
			\path[left color=tcbcolback!60!black,right color=tcbcolback!60!black,
			middle color=tcbcolback!80!black]
			([xshift=-2mm]frame.north west) -- ([xshift=2mm]frame.north east)
			[rounded corners=1mm]-- ([xshift=1mm,yshift=-1mm]frame.north east)
			-- (frame.south east) -- (frame.south west)
			-- ([xshift=-1mm,yshift=-1mm]frame.north west)
			[sharp corners]-- cycle;
		},interior engine=empty,
	},
	fonttitle=\bfseries,
	title={#2},#1}{def}



%================================
% EXERCISE BOX
%================================

\makeatletter
\newtcbtheorem{question}{Question}{enhanced,
	breakable,
	colback=white,
	colframe=myb!80!black,
	attach boxed title to top left={yshift*=-\tcboxedtitleheight},
	fonttitle=\bfseries,
	title={#2},
	boxed title size=title,
	boxed title style={%
			sharp corners,
			rounded corners=northwest,
			colback=tcbcolframe,
			boxrule=0pt,
		},
	underlay boxed title={%
			\path[fill=tcbcolframe] (title.south west)--(title.south east)
			to[out=0, in=180] ([xshift=5mm]title.east)--
			(title.center-|frame.east)
			[rounded corners=\kvtcb@arc] |-
			(frame.north) -| cycle;
		},
	#1
}{def}
\makeatother

%================================
% SOLUTION BOX
%================================

\makeatletter
\newtcolorbox{solution}{enhanced,
	breakable,
	colback=white,
	colframe=myg!80!black,
	attach boxed title to top left={yshift*=-\tcboxedtitleheight},
	title=Solution,
	boxed title size=title,
	boxed title style={%
			sharp corners,
			rounded corners=northwest,
			colback=tcbcolframe,
			boxrule=0pt,
		},
	underlay boxed title={%
			\path[fill=tcbcolframe] (title.south west)--(title.south east)
			to[out=0, in=180] ([xshift=5mm]title.east)--
			(title.center-|frame.east)
			[rounded corners=\kvtcb@arc] |-
			(frame.north) -| cycle;
		},
}
\makeatother

%================================
% Question BOX
%================================

\makeatletter
\newtcbtheorem{qstion}{Question}{enhanced,
	breakable,
	colback=white,
	colframe=mygr,
	attach boxed title to top left={yshift*=-\tcboxedtitleheight},
	fonttitle=\bfseries,
	title={#2},
	boxed title size=title,
	boxed title style={%
			sharp corners,
			rounded corners=northwest,
			colback=tcbcolframe,
			boxrule=0pt,
		},
	underlay boxed title={%
			\path[fill=tcbcolframe] (title.south west)--(title.south east)
			to[out=0, in=180] ([xshift=5mm]title.east)--
			(title.center-|frame.east)
			[rounded corners=\kvtcb@arc] |-
			(frame.north) -| cycle;
		},
	#1
}{def}
\makeatother

\newtcbtheorem[number within=chapter]{wconc}{Wrong Concept}{
	breakable,
	enhanced,
	colback=white,
	colframe=myr,
	arc=0pt,
	outer arc=0pt,
	fonttitle=\bfseries\sffamily\large,
	colbacktitle=myr,
	attach boxed title to top left={},
	boxed title style={
			enhanced,
			skin=enhancedfirst jigsaw,
			arc=3pt,
			bottom=0pt,
			interior style={fill=myr}
		},
	#1
}{def}



%================================
% NOTE BOX
%================================

\usetikzlibrary{arrows,calc,shadows.blur}
\tcbuselibrary{skins}
\newtcolorbox{note}[1][]{%
	enhanced jigsaw,
	colback=gray!20!white,%
	colframe=gray!80!black,
	size=small,
	boxrule=1pt,
	title=\textbf{Note:-},
	halign title=flush center,
	coltitle=black,
	breakable,
	drop shadow=black!50!white,
	attach boxed title to top left={xshift=1cm,yshift=-\tcboxedtitleheight/2,yshifttext=-\tcboxedtitleheight/2},
	minipage boxed title=1.5cm,
	boxed title style={%
			colback=white,
			size=fbox,
			boxrule=1pt,
			boxsep=2pt,
			underlay={%
					\coordinate (dotA) at ($(interior.west) + (-0.5pt,0)$);
					\coordinate (dotB) at ($(interior.east) + (0.5pt,0)$);
					\begin{scope}
						\clip (interior.north west) rectangle ([xshift=3ex]interior.east);
						\filldraw [white, blur shadow={shadow opacity=60, shadow yshift=-.75ex}, rounded corners=2pt] (interior.north west) rectangle (interior.south east);
					\end{scope}
					\begin{scope}[gray!80!black]
						\fill (dotA) circle (2pt);
						\fill (dotB) circle (2pt);
					\end{scope}
				},
		},
	#1,
}

%%%%%%%%%%%%%%%%%%%%%%%%%%%%%%
% SELF MADE COMMANDS
%%%%%%%%%%%%%%%%%%%%%%%%%%%%%%

\NewDocumentCommand{\EqM}{ m O{black} m}{%
	\tikz[remember picture, baseline, anchor=base] 
	\node[inner sep=0pt, outer sep=3pt, text=#2] (#1) {%
		\ensuremath{#3}%
	};    
}

% \newcommand{\thm}[3][]{\begin{theorem}{#2}{#1}#3\end{theorem}}
% \newcommand{\thmc}[3][]{\begin{theorem}{#2}{#1}#3\end{theorem}}
% \newcommand{\cor}[3][]{\begin{corollary}{#2}{#1}#3\end{corollary}}
% \newcommand{\corc}[3][]{\begin{corollary}{#2}{#1}#3\end{corollary}}
% \newcommand{\clm}[3][]{\begin{claim}{#2}{#1}#3\end{claim}}
% \newcommand{\prop}[3][]{\begin{proposition}{#2}{#1}#3\end{proposition}}
% \newcommand{\lemm}[3][]{\begin{lemma}{#2}{#1}#3\end{lemma}}
% \newcommand{\wc}[3][]{\begin{wconc}{#2}{#1}\setlength{\parindent}{1cm}#3\end{wconc}}
% \newcommand{\thmcon}[1]{\begin{Theoremcon}{#1}\end{Theoremcon}}
% \newcommand{\ex}[3][]{\begin{example}{#2}{#1}#3\end{example}}
% \newcommand{\exc}[3][]{\begin{example}{#2}{#1}#3\end{example}}
% \newcommand{\dfn}[3][]{\begin{definition}{#2}{#1}#3\end{definition}}
% \newcommand{\dfn}[3][]{\begin{Definition}[colbacktitle=red!75!black]{#2}{#1}#3\end{Definition}}
% \newcommand{\dfnc}[3][]{\begin{definition}[colbacktitle=red!75!black]{#2}{#1}#3\end{definition}}
% \newcommand{\opn}[3][]{\begin{open}[colbacktitle=myp!75!black]{#2}{#1}#3\end{open}}
% \newcommand{\opnc}[3][]{\begin{Open}[colbacktitle=myp!75!black]{#2}{#1}#3\end{Open}}
% \newcommand{\qs}[3][]{\begin{question}{#2}{#1}#3\end{question}}
%\newcommand{\pf}[2]{\begin{myproof}[#1]#2\end{myproof}}
% \newcommand{\pf}[2][Proof]{\begin{prf}[#1]#2\end{prf}}
% \newcommand{\nt}[1]{\begin{note}#1\end{note}}

\newcommand*\circled[1]{\tikz[baseline=(char.base)]{
		\node[shape=circle,draw,inner sep=1pt] (char) {#1};}}
\newcommand\getcurrentref[1]{%
	\ifnumequal{\value{#1}}{0}
	{??}
	{\the\value{#1}}%
}
\newcommand{\getCurrentSectionNumber}{\getcurrentref{section}}
\newenvironment{prf}[1][\proofname]{%
	\proof[#1.]%
}{\endproof}
\newcounter{mylabelcounter}

\makeatletter
\newcommand{\setword}[2]{%
	\phantomsection
	#1\def\@currentlabel{\unexpanded{#1}}\label{#2}%
}
\makeatother

\newenvironment{remark}[1][Remark]{%
    \par\noindent\textit{#1.}\hspace{1ex}%
}{\qed\par}


\tikzset{
	symbol/.style={
			draw=none,
			every to/.append style={
					edge node={node [sloped, allow upside down, auto=false]{$#1$}}}
		}
}

%\usepackage{framed}
%\usepackage{titletoc}
%\usepackage{etoolbox}
%\usepackage{lmodern}


%\patchcmd{\tableofcontents}{\contentsname}{\sffamily\contentsname}{}{}

%\renewenvironment{leftbar}
%{\def\FrameCommand{\hspace{6em}%
%		{\color{myyellow}\vrule width 2pt depth 6pt}\hspace{1em}}%
%	\MakeFramed{\parshape 1 0cm \dimexpr\textwidth-6em\relax\FrameRestore}\vskip2pt%
%}
%{\endMakeFramed}

%\titlecontents{chapter}
%[0em]{\vspace*{2\baselineskip}}
%{\parbox{4.5em}{%
%		\hfill\Huge\sffamily\bfseries\color{myred}\thecontentspage}%
%	\vspace*{-2.3\baselineskip}\leftbar\textsc{\small\chaptername~\thecontentslabel}\\\sffamily}
%{}{\endleftbar}
%\titlecontents{section}
%[8.4em]
%{\sffamily\contentslabel{3em}}{}{}
%{\hspace{0.5em}\nobreak\itshape\color{myred}\contentspage}
%\titlecontents{subsection}
%[8.4em]
%{\sffamily\contentslabel{3em}}{}{}  
%{\hspace{0.5em}\nobreak\itshape\color{myred}\contentspage}



%%%%%%%%%%%%%%%%%%%%%%%%%%%%%%%%%%%%%%%%%%%
% TABLE OF CONTENTS
%%%%%%%%%%%%%%%%%%%%%%%%%%%%%%%%%%%%%%%%%%%


\usepackage{titletoc}
\usepackage{sectsty} % For customizing section titles if needed

\renewcommand{\contentsname}{\fontfamily{times}\selectfont\Huge\bfseries\sffamily Contents}

% Define custom colors as per tstextbook.cls
\definecolor{tssteelblue}{RGB}{70,130,180}
\definecolor{tsorange}{RGB}{255,138,88}

% Customizing the Table of Contents

\contentsmargin{1cm}

% Chapter titles
\titlecontents{chapter}[0.25cm]
{\addvspace{2ex}\Large\bfseries\sffamily} % Spacing and font adjustments before the entry
{\hypersetup{linkcolor=tssteelblue}\color{tssteelblue}\contentslabel[\Large\thecontentslabel]{1.0cm}} % Label (number) formatting
{\hypersetup{linkcolor=tssteelblue}} % For unnumbered chapters
{\color{tssteelblue}\titlerule*[1pc]{.}\contentspage} % Dot leader and page number

% Section titles
\titlecontents{section}[1.5cm] % Indentation
{\addvspace{0.7ex}\large\bfseries\sffamily} % Spacing and font adjustments before the entry
{\hypersetup{linkcolor=black}\contentslabel[\thecontentslabel]{1.25cm}} % Label (number) formatting
{}
{\sffamily\hfill\color{black}\contentspage}[]

\titlecontents{subsection}[2.65cm] % Adjust the left margin to align with subsection indent
{\addvspace{0.25em}\sffamily}
{\hypersetup{linkcolor=black!85}\contentslabel[\thecontentslabel]{3.2em}} % Subsection label formatting
{} % Unnumbered format
{\hfill\color{black}\contentspage} % Page number


% \usepackage{tikz}
% \definecolor{doc}{RGB}{0,60,110}
% \usepackage{titletoc}
% \contentsmargin{0cm}
% \titlecontents{chapter}[3.7pc]
% {\addvspace{30pt}%
% 	\begin{tikzpicture}[remember picture, overlay]%
% 		\draw[fill=doc!60,draw=doc!60] (-7,-.1) rectangle (-0.7,.5);%
% 		\pgftext[left,x=-3.6cm,y=0.2cm]{\color{white}\Large\sc\bfseries Chapter\ \thecontentslabel};%
% 	\end{tikzpicture}\color{doc!60}\large\sc\bfseries}%
% {}
% {}
% {\;\titlerule\;\large\sc\bfseries Page \thecontentspage
% 	\begin{tikzpicture}[remember picture, overlay]
% 		\draw[fill=doc!60,draw=doc!60] (2pt,0) rectangle (4,0.1pt);
% 	\end{tikzpicture}}%
% \titlecontents{section}[3.7pc]
% {\addvspace{2pt}}
% {\contentslabel[\thecontentslabel]{2pc}}
% {}
% {\hfill\small \thecontentspage}
% []
% \titlecontents*{subsection}[3.7pc]
% {\addvspace{-1pt}\small}
% {}
% {}
% {\ --- \small\thecontentspage}
% [ \textbullet\ ][]

% \makeatletter
% \renewcommand{\tableofcontents}{%
% 	\chapter*{%
% 	  \vspace*{-20\p@}%
% 	  \begin{tikzpicture}[remember picture, overlay]%
% 		  \pgftext[right,x=15cm,y=0.2cm]{\color{doc!60}\Huge\sc\bfseries \contentsname};%
% 		  \draw[fill=doc!60,draw=doc!60] (13,-.75) rectangle (20,1);%
% 		  \clip (13,-.75) rectangle (20,1);
% 		  \pgftext[right,x=15cm,y=0.2cm]{\color{white}\Huge\sc\bfseries \contentsname};%
% 	  \end{tikzpicture}}%
% 	\@starttoc{toc}}
% \makeatother

% \newcommand{\mytitlea}[4]{
% 	\begin{tikzpicture}[remember picture,overlay]
% 		%%%%%%%%%%%%%%%%%%%% Background %%%%%%%%%%%%%%%%%%%%%%%%
% 		\fill[orange] (current page.south west) rectangle (current page.north east);
		
		
		
		
% 		%%%%%%%%%%%%%%%%%%%% Background Polygon %%%%%%%%%%%%%%%%%%%%
		
% 		\foreach \i in {2.5,...,22}
% 		{
% 			\node[rounded corners,orange!60,draw,regular polygon,regular polygon sides=6, minimum size=\i cm,ultra thick] at ($(current page.west)+(2.5,-5)$) {} ;
% 		}
		
% 		\foreach \i in {0.5,...,22}
% 		{
% 			\node[rounded corners,orange!60,draw,regular polygon,regular polygon sides=6, minimum size=\i cm,ultra thick] at ($(current page.north west)+(2.5,0)$) {} ;
% 		}
		
% 		\foreach \i in {0.5,...,22}
% 		{
% 			\node[rounded corners,orange!90,draw,regular polygon,regular polygon sides=6, minimum size=\i cm,ultra thick] at ($(current page.north east)+(0,-9.5)$) {} ;
% 		}
		
		
% 		\foreach \i in {21,...,6}
% 		{
% 			\node[orange!85,rounded corners,draw,regular polygon,regular polygon sides=6, minimum size=\i cm,ultra thick] at ($(current page.south east)+(-0.2,-0.45)$) {} ;
% 		}
		
		
% 		%%%%%%%%%%%%%%%%%%%% Title of the Report %%%%%%%%%%%%%%%%%%%% 
% 		\node[left,black,minimum width=0.625*\paperwidth,minimum height=3cm, rounded corners] at ($(current page.north east)+(0,-9.5)$)
% 		{
% 			{\fontsize{25}{30} \selectfont \bfseries #1}
% 		};
		
% 		%%%%%%%%%%%%%%%%%%%% Subtitle %%%%%%%%%%%%%%%%%%%% 
% 		\node[left,black,minimum width=0.625*\paperwidth,minimum height=2cm, rounded corners] at ($(current page.north east)+(0,-11)$)
% 		{
% 			{\huge \textit{#2}}
% 		};
		
% 		%%%%%%%%%%%%%%%%%%%% Author Name %%%%%%%%%%%%%%%%%%%% 
% 		\node[left,black,minimum width=0.625*\paperwidth,minimum height=2cm, rounded corners] at ($(current page.north east)+(0,-13)$)
% 		{
% 			{\Large \textsc{#3}}
% 		};
		
% 		%%%%%%%%%%%%%%%%%%%% Year %%%%%%%%%%%%%%%%%%%% 
% 		\node[rounded corners,fill=orange!70,text =black,regular polygon,regular polygon sides=6, minimum size=2.5 cm,inner sep=0,ultra thick] at ($(current page.west)+(2.5,-5)$) {\LARGE \bfseries #4};
		
% 	\end{tikzpicture}
% }
% \newcommand{\mytitleb}[4]{\begin{tikzpicture}[overlay,remember picture]
		
% 		% Background color
% 		\fill[
% 		black!2]
% 		(current page.south west) rectangle (current page.north east);
		
% 		% Rectangles
% 		\shade[
% 		left color=Dandelion, 
% 		right color=Dandelion!40,
% 		transform canvas ={rotate around ={45:($(current page.north west)+(0,-6)$)}}] 
% 		($(current page.north west)+(0,-6)$) rectangle ++(9,1.5);
		
% 		\shade[
% 		left color=lightgray,
% 		right color=lightgray!50,
% 		rounded corners=0.75cm,
% 		transform canvas ={rotate around ={45:($(current page.north west)+(.5,-10)$)}}]
% 		($(current page.north west)+(0.5,-10)$) rectangle ++(15,1.5);
		
% 		\shade[
% 		left color=lightgray,
% 		rounded corners=0.3cm,
% 		transform canvas ={rotate around ={45:($(current page.north west)+(.5,-10)$)}}] ($(current page.north west)+(1.5,-9.55)$) rectangle ++(7,.6);
		
% 		\shade[
% 		left color=orange!80,
% 		right color=orange!60,
% 		rounded corners=0.4cm,
% 		transform canvas ={rotate around ={45:($(current page.north)+(-1.5,-3)$)}}]
% 		($(current page.north)+(-1.5,-3)$) rectangle ++(9,0.8);
		
% 		\shade[
% 		left color=red!80,
% 		right color=red!80,
% 		rounded corners=0.9cm,
% 		transform canvas ={rotate around ={45:($(current page.north)+(-3,-8)$)}}] ($(current page.north)+(-3,-8)$) rectangle ++(15,1.8);
		
% 		\shade[
% 		left color=orange,
% 		right color=Dandelion,
% 		rounded corners=0.9cm,
% 		transform canvas ={rotate around ={45:($(current page.north west)+(4,-15.5)$)}}]
% 		($(current page.north west)+(4,-15.5)$) rectangle ++(30,1.8);
		
% 		\shade[
% 		left color=RoyalBlue,
% 		right color=Emerald,
% 		rounded corners=0.75cm,
% 		transform canvas ={rotate around ={45:($(current page.north west)+(13,-10)$)}}]
% 		($(current page.north west)+(13,-10)$) rectangle ++(15,1.5);
		
% 		\shade[
% 		left color=lightgray,
% 		rounded corners=0.3cm,
% 		transform canvas ={rotate around ={45:($(current page.north west)+(18,-8)$)}}]
% 		($(current page.north west)+(18,-8)$) rectangle ++(15,0.6);
		
% 		\shade[
% 		left color=lightgray,
% 		rounded corners=0.4cm,
% 		transform canvas ={rotate around ={45:($(current page.north west)+(19,-5.65)$)}}]
% 		($(current page.north west)+(19,-5.65)$) rectangle ++(15,0.8);
		
% 		\shade[
% 		left color=OrangeRed,
% 		right color=red!80,
% 		rounded corners=0.6cm,
% 		transform canvas ={rotate around ={45:($(current page.north west)+(20,-9)$)}}] 
% 		($(current page.north west)+(20,-9)$) rectangle ++(14,1.2);
		
% 		% Year
% 		\draw[ultra thick,gray]
% 		($(current page.center)+(5,2)$) -- ++(0,-3cm) 
% 		node[
% 		midway,
% 		left=0.25cm,
% 		text width=5cm,
% 		align=right,
% 		black!75
% 		]
% 		{
% 			{\fontsize{25}{30} \selectfont \bf  Lecture\\[10pt] Notes}
% 		} 
% 		node[
% 		midway,
% 		right=0.25cm,
% 		text width=6cm,
% 		align=left,
% 		orange]
% 		{
% 			{\fontsize{72}{86.4} \selectfont #4}
% 		};
		
% 		% Title
% 		\node[align=center] at ($(current page.center)+(0,-5)$) 
% 		{
% 			{\fontsize{60}{72} \selectfont {{#1}}} \\[1cm]
% 			{\fontsize{16}{19.2} \selectfont \textcolor{orange}{ \bf #2}}\\[3pt]
% 			#3};
% \end{tikzpicture}
% }

\usepackage{chemfig}

\definecolor{arrowBlue}{RGB}{66, 135, 245}
\definecolor{arrowRed}{RGB}{245, 93, 66}

\tikzcdset{arrow style=tikz,
    squigarrow/.style={
        decoration={
        snake, 
        amplitude=.25mm,
        segment length=2mm
        }, 
        rounded corners=.1pt,
        decorate
        }
    }

\newcolumntype{P}[1]{>{\centering\arraybackslash}p{#1}}

\begin{document}
\begin{center}
	~\\
	\vspace{6em}
	{\fontsize{34}{48}\selectfont\textsc{Algebraic Construction}}
	~\\
	\vspace{2.5em}
	{\Large }
	~\\
	\vspace{6em}
	\textsf{\Large Huyi Chen}
	~\\
	\vspace{5in}
	{\large Latest Update: \today}
\end{center}

\makeatletter
\MHInternalSyntaxOn
\def\MT_leftarrow_fill:{%
  \arrowfill@\leftarrow\relbar\relbar}
\def\MT_rightarrow_fill:{%
  \arrowfill@\relbar\relbar\rightarrow}
\newcommand{\xrightleftarrows}[2][]{\mathrel{%
  \raise.55ex\hbox{%
    $\ext@arrow 0359\MT_rightarrow_fill:{\phantom{#1}}{#2}$}%
  \setbox0=\hbox{%
    $\ext@arrow 3095\MT_leftarrow_fill:{#1}{\phantom{#2}}$}%
  \kern-\wd0 \lower.55ex\box0}}
\MHInternalSyntaxOff
\makeatother
\newcommand{\spec}{\operatorname{Spec}}
\newcommand{\midv}{\,\middle\vert\,}
\newpage
% table of contents
\tableofcontents

% Your document content here

\chapter{Category Theory}
\section{Limits and Colimits}

\newpage
\begin{center}
\Large
\renewcommand{\arraystretch}{1.5}
\begin{tabular}{|P{4.4cm}|m{5.3cm}|m{5.3cm}|}
    \hline
    {\bf Diagram} &\multicolumn{1}{c|}{\bf  Limit }& \multicolumn{1}{c|}{\bf Colimit } \\ 
    \hline\hline
    Empty Category $\varnothing$ & Terminal Object
    \begin{center}
        \begin{tikzcd}[every arrow/.append style={-latex, line width=1.2pt}]
            \circ \arrow[r, dash pattern=on 4pt off 2pt, draw=arrowRed] & \varprojlim\\[-2em]
        \end{tikzcd}
    \end{center}
    & Initial Object
    \begin{center}
        \begin{tikzcd}[every arrow/.append style={-latex, line width=1.2pt}]
             \varinjlim\arrow[r, dash pattern=on 4pt off 2pt, draw=arrowRed]&\circ  \\[-2em]
        \end{tikzcd}
    \end{center}
    \\ 
    \hline
    Discrete Category  $\bullet\;\bullet\;\bullet\;\bullet\;\cdots$& Product & Coproduct \\
    \hline
    $\bullet\qquad\bullet$ & Finite Product
    \begin{center}
        \begin{tikzcd}[every arrow/.append style={-latex, line width=1.2pt}]
            & \circ \arrow[d, dash pattern=on 4pt off 2pt, draw=arrowRed] \arrow[ld, draw=cyan] \arrow[rd, draw=cyan] &         \\
    \bullet & \varprojlim \arrow[l, draw=arrowBlue] \arrow[r, draw=arrowBlue]               & \bullet
    \end{tikzcd}
    \end{center} & Finite Coproduct 
    \begin{center}
        \begin{tikzcd}[every arrow/.append style={-latex, line width=1.2pt}]
            & \circ   &         \\
    \bullet \arrow[r, draw=arrowBlue]  \arrow[ru, draw=cyan] & \varinjlim  \arrow[u, dash pattern=on 4pt off 2pt, draw=arrowRed]             & \bullet\arrow[l, draw=arrowBlue]\arrow[lu, draw=cyan]
    \end{tikzcd}
    \end{center}\\
    \hline
        \begin{tikzcd}[every arrow/.append style={-latex, line width=1.2pt}]
            \bullet \arrow[r] &[-0.8em] \bullet &[-0.8em] \bullet \arrow[l]
        \end{tikzcd}
    & Pullback    
    \begin{center}
        \begin{tikzcd}[every arrow/.append style={-latex, line width=1.2pt}]
        \circ \arrow[rd, dash pattern=on 4pt off 2pt, draw=arrowRed] \arrow[rrd, draw=cyan, bend left] \arrow[rdd, draw=cyan, bend right] &[-2em]                                 &  [-0.4em]                  \\[-0.3cm]
                                                                                & \varprojlim \arrow[d, draw=arrowBlue] \arrow[r, draw=arrowBlue] & \bullet \arrow[d] \\
                                                                                & \bullet \arrow[r]               & \bullet          
        \end{tikzcd} 
    \end{center}
    & Pushout 
    \begin{center}
        \begin{tikzcd}[every arrow/.append style={-latex, line width=1.2pt}]
        \circ   &[-2em]                                 &  [-0.4em]                  \\[-0.3cm]
                                                                                & \varinjlim\arrow[lu, dash pattern=on 4pt off 2pt, draw=arrowRed]  & \bullet  \arrow[llu, draw=cyan, bend right] \arrow[l, draw=arrowBlue]\\
                                                                                & \bullet \arrow[u, draw=arrowBlue]   \arrow[luu, draw=cyan, bend left]           & \bullet  \arrow[l] \arrow[u]      
        \end{tikzcd} 
    \end{center} \\
    \hline
    $\begin{tikzcd}[every arrow/.append style={-latex, line width=1.2pt}]
        \bullet \arrow[r, shift left] \arrow[r, shift right] & \bullet
        \end{tikzcd}$ & Equalizer  \begin{center}
            \begin{tikzcd}[every arrow/.append style={-latex, line width=1.2pt}]
                \circ \arrow[rd, draw=cyan] \arrow[d, dash pattern=on 4pt off 2pt, draw=arrowRed] &                                                      &         \\
                \varprojlim \arrow[r, draw=arrowBlue]    & \bullet \arrow[r, shift left] \arrow[r, shift right] & \bullet
                \end{tikzcd}
        \end{center}  & Coequalizer \begin{center}
            \begin{tikzcd}[every arrow/.append style={-latex, line width=1.2pt}]
                &                              & \circ                        \\
            \bullet \arrow[r, shift left] \arrow[r, shift right] & \bullet \arrow[r, draw=arrowBlue] \arrow[ru, draw=cyan] & \varinjlim \arrow[u, dash pattern=on 4pt off 2pt, draw=arrowRed]
            \end{tikzcd}
        \end{center}\\
    \hline
        \begin{tikzcd}[every arrow/.append style={-latex, line width=1.2pt, shorten <=-3.5pt, shorten >=-3.5pt}]
            \bullet &[-1.5em] \bullet \arrow[l] &[-1.5em] \bullet \arrow[l] &[-1.5em] \cdots \arrow[l]
        \end{tikzcd}
    
        & Inverse Limit
    \begin{center}
        \begin{tikzcd}[every arrow/.append style={-latex, line width=1.2pt}]
            &   [-1.5em]                &[-1em] \circ \arrow[rdd, draw=cyan] \arrow[ldd, draw=cyan] \arrow[d, dash pattern=on 4pt off 2pt, draw=arrowRed] & [-1em]                  \\
            &                   & \varprojlim \arrow[ld, draw=arrowBlue, shorten <=-4pt] \arrow[rd, draw=arrowBlue, shorten <=-3pt]               &                   \\[-0.3cm]
    \bullet & \bullet \arrow[l, shorten <=-3pt, shorten >=-3pt] &                                                 & \cdots \arrow[ll]
    \end{tikzcd}
\end{center}
    & Direct Limit 
    \begin{center}
        \begin{tikzcd}[every arrow/.append style={-latex, line width=1.2pt}]
            &   [-1.5em]                &[-1em] \circ  & [-1em]                  \\
            &                   & \varinjlim \arrow[u, dash pattern=on 4pt off 2pt, draw=arrowRed]       &                   \\[-0.3cm]
    \bullet \arrow[r, shorten <=-3pt, shorten >=-3pt]& \bullet  \arrow[ru, draw=arrowBlue, shorten >=-4.7pt]        \arrow[rr] \arrow[ruu, draw=cyan] &                                                 & \cdots \arrow[lu, draw=arrowBlue, shorten >=-2.5pt] \arrow[luu, draw=cyan]
    \end{tikzcd}
\end{center}\\
    \hline
    \end{tabular}
\end{center}

\prop[ev_functor_preserves_limits]{Evaluation Functor Preserves Limits}{
    Let $\mathsf{A}$ be a small category and $\mathsf{C}$ be a category. Given a diagram $F: \mathsf{J} \to\left[\mathsf{A},\mathsf{C}\right]$ with $\mathsf{J}$ small, if for any $a\in \mathsf{A}$, the diagram 
    \[
        \mathrm{ev}_a \circ F:\mathsf{J}\xrightarrow{\quad F\quad}\left[\mathsf{A},\mathsf{C}\right]\xrightarrow{\quad\mathrm{ev}_a\quad}\mathsf{C}
    \]
    has a limit, then
    \begin{enumerate}[(i)]
        \item $\varprojlim F$ exists.
        \item For any $a\in \mathsf{A}$, $\mathrm{ev}_a$ preserves $\varprojlim F$.
    \end{enumerate}
}
\chapter{Group}
\section{Basic Concepts}
\dfn{Group}{
    A \textbf{group} is a set $G$ together with a binary operation $\cdot:G\times G\to G$ such that
    \begin{enumerate}[(i)]
        \item (Associativity) $\forall x,y,z\in G$, $(x\cdot y)\cdot z=x\cdot(y\cdot z)$.
        \item (Identity) $\exists e\in G$ such that $\forall x\in G$, $e\cdot x=x\cdot e=x$.
        \item (Inverse) $\forall x\in G$, $\exists x^{-1}\in G$ such that $x\cdot x^{-1}=x^{-1}\cdot x=e$.
    \end{enumerate}
}
Since the identity of a group is unique, we denote it by $1_G$ or simply $1$.
\dfn{Opposite Group}{
    Let $G=(G,*)$ be a group. The \textbf{opposite group} of $G$ is the group $G^{\mathrm{op}}=(G,*^{\mathrm{op}})$, where $*^{\mathrm{op}}:G\times G\to G$ is defined by $x*^{\mathrm{op}}y=y\cdot x$. If we consider $G$ as a category $\mathsf{B}G$, then we have category isomorphism
    \[
        \mathsf{B}G^{\mathrm{op}}\cong (\mathsf{B}G)^{\mathrm{op}}.
    \]
}
\dfn{Subgroup}{
    Let $G$ be a group. A subset $H$ of $G$ is called a \textbf{subgroup} of $G$ if $H$ is a group with respect to the binary operation of $G$. In this case, we write $H\le G$.
}


\section{Group Homomorphism}
\dfn{Group Homomorphism}{
    Let $G,H$ be groups. A \textbf{group homomorphism} from $G$ to $H$ is a function $\varphi:G\to H$ such that
    \[
        \forall x,y\in G,\quad \varphi(x y)=\varphi(x)\varphi(y).
    \]
}
\dfn{Isomorphism}{
    Let $G,H$ be groups. A group homomorphism $\varphi:G\to H$ is called an \textbf{isomorphism} if $\varphi$ is bijective. In this case, we say that $G$ and $H$ are \textbf{isomorphic} and write $G\cong H$.
}
\prop{Properties of Group Homomorphisms}{
    Let $G,H$ be groups and $\varphi:G\to H$ be a group homomorphism. Then
    \begin{enumerate}[(i)]
        \item $\varphi(1_G)=1_H$.
        \item $\forall x\in G$, $\varphi(x^{-1})=\varphi(x)^{-1}$.
        \item $\forall n\in\mathbb{Z}$, $\varphi(x^n)=\varphi(x)^n$.
        \item If $K\le G$ , then $\varphi(K)\le H$.
        \item If $K\le H$, then $\varphi^{-1}(K)\le G$.
    \end{enumerate}
}
\dfn{Kernel of Group Homomorphism}{
    Let $\varphi:G\to H$ be a group homomorphism. The \textbf{kernel} of $\varphi$ is defined by
    \[
        \ker\varphi=\{x\in G\mid \varphi(x)=1_H\}.
    \]
}
\prop{Property of Kernel}{
    Let $G$ be a group. Then
    \begin{enumerate}[(i)]
        \item Let $\varphi:G\to H$ be a group homomorphism. Then $\varphi$ is injective if and only if $\ker\varphi=\{1_G\}$.
    \end{enumerate}
}
\dfn{Normal Subgroup}{
    Let $G$ be a group. A subgroup $H$ of $G$ is called a \textbf{normal subgroup} if $gHg^{-1}=H$ for all $g\in G$. In this case, we write $H\lhd G$.
}
\prop{Equivalent Definition of Normal Subgroup}{
    Let $G$ be a group and $H$ be a subgroup of $G$. Then the following are equivalent:
    \begin{enumerate}[(i)]
        \item $H$ is a normal subgroup of $G$.
        \item $\forall \gamma_g \in\mathrm{Inn}(G)$, $\gamma_g(H)\subseteq H$.
        \item $gHg^{-1}\subseteq H$ for all $g\in G$.
        \item $gHg^{-1}=H$ for all $g\in G$.
        \item $gH=Hg$ for all $g\in G$.
        \item $H$ is a union of conjugacy classes.
        \item $H=\ker\varphi$ for some group homomorphism $\varphi:G\to K$.
    \end{enumerate}
}
\prop{Properties of Normal Subgroup}{
    Let $G$ be a group.
    \begin{enumerate}[(i)]
        \item $\{1_G\}$ and $G$ are normal subgroups of $G$.
        \item If $H\le K\le G$ and $H\lhd G$, then $H\lhd K$.
        \item If $H\lhd_{\rm char} K\lhd G$, then $H\lhd G$.
        \item Normality is preserved under surjective homomorphisms: if $f:G \rightarrow H$ is a surjective group homomorphism and $N\lhd G$, then $f(N)\lhd H$.
        \item Normality is preserved by taking inverse images of homomorphisms: if $f:G \rightarrow H$ is a group homomorphism and $N\lhd H$, then $f^{-1}(N)\lhd G$.
        \item Normality is preserved on taking finite products: if $N_1 \lhd G_1$ and $N_2 \lhd G_2$, then $N_1 \times N_2 \lhd G_1 \times G_2$.
        \item Given two normal subgroups, $N$ and $M$, of $G$, their intersection $N \cap M$ and their product $N M=\{n m: n \in N$ and $m \in M\}$ are also normal subgroups of $G$.
    \end{enumerate}
}
\dfn{Simple Group}{
    A group $G$ is called \textbf{simple} if $G$ is nontrivial and the only normal subgroups of $G$ are $\{1_G\}$ and $G$.
}
\thm{Fundamental Theorem on Homomorphisms}{
    Let $G,H$ be groups and $\varphi:G\to H$ be a group homomorphism. Define natural projection 
    \begin{align*}
        \pi:G&\longrightarrow G/\ker\varphi\\
        g&\longmapsto g\ker\varphi
    \end{align*}
    Then there exists a unique group homomorphism $\overline{\varphi}:G/\ker\varphi\to H$ such that the following diagram commutes
    \begin{center}
        \begin{tikzcd}[ampersand replacement=\&]
            G \arrow[r, "\varphi"] \arrow[d, "\pi"'] \& H \\[0.3cm]
            G/\ker\varphi \arrow[ru, dashed, "\exists!\,\overline{\varphi}"'] \&  
        \end{tikzcd}
    \end{center}
    Moreover, $\overline{\varphi}$ is injective and we have $ G/\ker\varphi\cong \mathrm{im}\varphi$.
}
\cor{First isomorphism theorem}{
    Let $G,H$ be groups and $\varphi:G\to H$ be surjective group homomorphism. Then $G/\ker\varphi\cong H$.
}

\prop{Universal Property of Quotient Group}{
    Let $G$ be a group and $N \lhd G$ be a normal subgroup. Suppose $\pi:G\to G/N$ is the natural projection. Then $\pi$ is initial in the category of group homomorphisms $\varphi:G\to H$ such that $N\subseteq \ker\varphi$. \\
    That is, for any group $H$ and group homomorphism $\varphi:G\to H$ such that $N\subseteq \ker\varphi$, there exists a unique group homomorphism $\widetilde{\varphi}:G/N\to H$ such that the following diagram commutes
    \begin{center}
        \begin{tikzcd}[ampersand replacement=\&]
            G \arrow[r, "\varphi"] \arrow[d, "\pi"'] \& H \\[0.3cm]
            G/N \arrow[ru, dashed, "\exists !\,\widetilde{\varphi}"'] \&  
        \end{tikzcd}
    \end{center}
}
\proof{
    Since $N\subseteq \ker\varphi$, there is a canonical projection 
    \begin{align*}
        p:G/N &\longrightarrow G/\ker\varphi\\
        gN &\longmapsto g\ker\varphi
    \end{align*}
    According to the following diagram, we can define $\widetilde{\varphi}$ by $\widetilde{\varphi}=\overline{\varphi}\circ p$.
    \begin{center}
        \begin{tikzcd}[ampersand replacement=\&]
            G \arrow[r, "\varphi"] \arrow[d, "\pi"'] \& H \\[0.4cm]
            G/N \arrow[ru, dashed, "\widetilde{\varphi}"'] \arrow[r,"p"'] \&  G/\ker\varphi \arrow[u, dashed, "\overline{\varphi}"']
        \end{tikzcd}
    \end{center}
}
\thm{Second Isoomorphism Theorem}{
    Let $G$ be a group and $H,K$ be subgroups of $G$. Then $HK$ is a subgroup of $G$ and $H\cap K$ is a normal subgroup of $H$. Moreover, we have
    \[
        HK/H\cong K/(H\cap K).
    \]
}

\section{Group Action}
\subsection{Definitions}
\dfn{Symmetric Group}{
    The \textbf{symmetric group} on a set $X$ is the group whose elements are all bijections from $X$ to $X$, with the group operation of function composition. The symmetric group on $X$ is denoted by $\mathrm{Sym}(X)$ or $\mathrm{Aut}_{\mathsf{Set}}(X)$. If $X=\{1,2,\cdots,n\}$, then we denote $\mathrm{Sym}(X)$ by $S_n$.
}
\dfn{Group Action}{
    Let $G$ be a group and $X$ be a set. A \textbf{group action} of $G$ on $X$ is a group homomorphism
    \begin{align*}
        \sigma:G&\longrightarrow \mathrm{Aut}_{\mathsf{Set}}(X)\\
        g&\longmapsto \sigma_g
    \end{align*}
    If $G$ acts on $X$ by $\sigma$, we say $(X,\sigma)$ is a \textbf{$G$-set}. If there is no ambiguity, we simply say $X$ is a $G$-set.
}
The $G$-sets and $G$-maps form a category $G\text{-}\mathsf{Set}$ and we have category isomorphism
\begin{align*}
    G\text{-}\mathsf{Set}&\stackrel{\sim}{\longrightarrow} \mathrm{Fun}(\mathsf{B}G, \mathsf{Set})    \\
    \sigma&\longmapsto \sigma(-)
\end{align*}
\prop{Equivalent Definition of Group Actions}{
    Let $G$ be a group and $X$ be a set. A group action of $G$ on $X$ can be alternatively defined as a map
    \begin{align*}
        \cdot:G\times X&\longrightarrow X\\
        (g,x)&\longmapsto g\cdot x
    \end{align*}
    such that
    \begin{enumerate}[(i)]
        \item $\forall x\in X$, $e\cdot x=x$.
        \item $\forall g,h\in G$, $\forall x\in X$, $(gh)\cdot x=g\cdot(h\cdot x)$.
    \end{enumerate}
    The equivalence of the two definitions is given by
    \[
    \sigma_g(x)=g\cdot x.
    \]
}
We say $X$ is a right $G$-set if $X$ is a left $G^{\mathrm{op}}$-set.
\ex{Trivial Group Action}{
    Let $G$ be a group and $X$ be a set. The \textbf{trivial group action} of $G$ on $X$ is defined as $\sigma_g=\mathrm{id}_X$ for all $g\in G$.
}
\ex[acting_on_power_set]{Actions on $X$ Induce Actions on $2^X$}{
    If a group $G$ acts on a set $X$, then $G$ acts on the power set $2^X$ by
    \[
    g\cdot A=\{ g\cdot x\mid x\in A\}    .
    \]
}
\dfn{Equivariant Map}{
    Let $G$ be a group and $X,Y$ be $G$-sets. A map $f:X\to Y$ is called \textbf{equivariant} if for all $g\in G$ and $x\in X$, we have
    \[
    f(g\cdot x)=g\cdot f(x)    .
    \]
    Equivalently, $f$ is equivariant if it is a natural transformation $f:\sigma(-)\implies \sigma'(-)$
    \[
        \begin{tikzcd}[ampersand replacement=\&, column sep=1.7em, row sep=small]
            \mathsf{B}G  \& \bullet \arrow[rr, "g"]                         \&  \& \bullet                 \\
                         \& X \arrow[dd, "f"'] \arrow[rr, "\sigma_g"] \&  \& X \arrow[dd, "f"] \\
            \mathsf{Set} \&                                           \&  \&                   \\
                         \& Y \arrow[rr, "\sigma'_g"']                \&  \& Y                
            \end{tikzcd}
    \]
}
\dfn{Product of $G$-Sets}{
    The \textbf{product} of two $G$-sets $X$ and $Y$ is defined as the set $X\times Y$ with the $G$-action
    \[
    g\cdot (x,y)=(g\cdot x, g\cdot y)    .
    \]
    Alternatively, the product of two $G$-sets can be defined as the product of two functors, cf. \Cref{th:ev_functor_preserves_limits}.
}
\dfn{Coproduct of $G$-Sets}{
    The \textbf{coproduct} of two $G$-sets $X$ and $Y$ is defined as the set $X\sqcup Y$ with the $G$-action
    \[
    g\cdot a=\begin{cases}
        g\cdot a & a\in X\\
        g\cdot a & a\in Y
    \end{cases} 
    \]
    Alternatively, the coproduct of two $G$-sets can be defined as the coproduct of two functors, cf. \Cref{th:ev_functor_preserves_limits}.
}
\ex[aut_acts_on_hom]{$\mathrm{Aut}_\mathsf{C}(X)$ acts on $\mathrm{Hom}_\mathsf{C}(X,Y)$ and $\mathrm{Hom}_\mathsf{C}(Y,X)$}{
    Let $X$ and $Y$ be objects in a category $\mathsf{C}$. Then $\mathrm{Aut}_\mathsf{C}(X)$ acts on $\mathrm{Hom}_\mathsf{C}(X,Y)$ by the composition of functors
    \[
    \begin{tikzcd}[ampersand replacement=\&, column sep=5em, row sep=3em]
    \mathsf{B}\mathrm{Aut}_\mathsf{C}(X)\arrow[r] \&[-3em] \mathsf{B}\mathrm{Aut}_\mathsf{C}(X)^{\mathrm{op}} \arrow[r, hook] \&[-2em] \mathsf{C}^{\mathrm{op}} \arrow[r, "{\mathrm{Hom}_{\mathsf{C}}(-,Y)}"] \& \mathsf{Set}\\[-2.3em]
    \bullet \arrow[r, maps to] \arrow[d, "g"']           \& \bullet \arrow[r, maps to] \arrow[d, "g^{-1}"]                           \& X \arrow[d, "g^{-1}"] \arrow[r, maps to]                               \& {\mathrm{Hom}(X,Y)} \arrow[d, "\left(g^{-1}\right)^*"] \\
    \bullet \arrow[r, maps to]                           \& \bullet \arrow[r, maps to]                                               \& X \arrow[r, maps to]                                                   \& {\mathrm{Hom}(X,Y)}  
    \end{tikzcd}
    \]
    Writing explicily, the action is given by
    \begin{align*}
        \mathrm{Aut}_\mathsf{C}(X)\times \mathrm{Hom}_\mathsf{C}(X,Y)&\longrightarrow \mathrm{Hom}_\mathsf{C}(X,Y)\\
        (g,f)&\longmapsto f\circ g^{-1}
    \end{align*}
    Similarly, $\mathrm{Aut}_\mathsf{C}(Y)$ acts on $\mathrm{Hom}_\mathsf{C}(Y,X)$ by
    \[
    \begin{tikzcd}[ampersand replacement=\&, column sep=5em, row sep=3em]
    \mathsf{B}\mathrm{Aut}_\mathsf{C}(X)\arrow[r, hook] \&[-1em] \mathsf{C} \arrow[r, "{\mathrm{Hom}_{\mathsf{C}}(Y,-)}"] \& \mathsf{Set}\\[-2.3em]
    \bullet \arrow[r, maps to] \arrow[d, "g"']           \&  X \arrow[d, "g"] \arrow[r, maps to]                               \& {\mathrm{Hom}(Y,X)} \arrow[d, "g_*"] \\
    \bullet \arrow[r, maps to]                           \& X \arrow[r, maps to]                                                   \& {\mathrm{Hom}(X,Y)}  
    \end{tikzcd}
    \]
    Writing explicily, the action is given by
    \begin{align*}
        \mathrm{Aut}_\mathsf{C}(Y)\times \mathrm{Hom}_\mathsf{C}(Y,X)&\longrightarrow \mathrm{Hom}_\mathsf{C}(Y,X)\\
        (g,f)&\longmapsto g\circ f
    \end{align*}
}
\ex[acting_on_functions]{Actions on $X$ Induce Actions on $Y^X$}{
    If $G$ acts on $X$ through a functor $\sigma(-):\mathsf{B}G\to\mathsf{Set}$, then it also acts on $Y^X$ for any set $Y$ by the composition of functors 
    \[
    \begin{tikzcd}[ampersand replacement=\&, column sep=5em, row sep=3em]
        \mathsf{B}G \arrow[r] \&[-2em] \mathsf{B}G^{\mathrm{op}} \arrow[r, "\sigma(-)^{\mathrm{op}}"] \&[-2.2em] \mathsf{Set}^{\mathrm{op}} \arrow[r, "{\mathrm{Hom}_{\mathsf{Set}}(-,Y)}"] \& \mathsf{Set}
        \end{tikzcd}
    \]
    The action on $Y^X$ is given explicitly as follows: for all $g\in G$, $f\in Y^X$ and $x\in X$,
    \[
        (g\cdot f)(x)=f(g^{-1}\cdot x).
    \]
    
}
\pf{
    We can check that
    \begin{align*}
        (g_1\cdot (g_2\cdot f))(x)&=\left(g_2\cdot f\right)\left(g_1^{-1}\cdot x\right)\\
        &=\left(g_2\cdot f\right)\left(g_1^{-1}\cdot x\right)\\
        &=f\left(g_2^{-1}\cdot\left(g_1^{-1}\cdot x\right)\right)\\
        &=f\left(\left(g_2^{-1} g_1^{-1}\right)\cdot x\right)\\
        &=\left(\left(g_1g_2\right)\cdot f\right)(x).
    \end{align*}
}
\dfn{Orbit of a Group Action}{
    Let $G$ be a group acting on a set $X$. For $x\in X$, the \textbf{orbit} of $x$ is defined as 
    \[
    G x=\{ g\cdot x\mid g\in G\}    .
    \]
}

\dfn{Orbit Space}{
    Let $G$ be a group acting on a set $X$. The \textbf{orbit space} of $G$ acting on $X$ is defined as 
    \[
    X/G=\{ Gx\mid x\in X\}.
    \]
}
If $G$ acts on $X$, then $G$ acts on $X/G$ trivially.
\prop{Orbit Decomposition}{
    Let $G$ be a group acting on a set $X$. We define a equivalence relation $\sim$ on $X$ by
    \[
    x\sim y \iff Gx=Gy.
    \]
    Then the equivalence class of $x$ is exactly $Gx$. The quotient set $X/\sim$ is exactly the orbit space $X/G$.
    And we have a partition of $X$ by the orbits of $G$ acting on $X$
    \[
    X=\bigsqcup_{Gx\in X/G}Gx.
    \]
}
\pf{
We can check that the equivalence class of $x$ is $Gx$. If $y\sim x$, then $y\in Gy=Gx$. If $y\in Gx$, then $Gy\subseteq Gx$ and $x\in Gy$. Note $x\in Gy$ implies $Gx\subseteq Gy$. We have $Gx=Gy$, i.e. $x\sim y$. 
}

If $G$ acts on $X$, then $G$ acts on $X/G$ trivially.
\dfn{$G$-invariant element }{
    Let $G$ be a group acting on a set $X$. An element $x\in X$ is called \textbf{$G$-invariant} if $Gx=\{ x\}$. The set of all $G$-invariant elements is denoted by $X^G$
    \[
      X^G=\{ x\in X\mid Gx=\{ x\} \}  .  
    \]
}

\dfn{Stabilizer Subgroup}{
    Let $G$ be a group acting on a set $X$. For $x\in X$, the \textbf{stabilizer subgroup} of of $G$ with respect to $x$ is defined as 
    \[
    \mathrm{Stab}_G(x)=\{ g\in G\mid g\cdot x=x\}    .
    \]
    It is easy to see that $\mathrm{Stab}_G(x)$ is a subgroup of $G$.
}

We can see $x\in X^G$ if and only if $\mathrm{Stab}_G(x)=G$.

\dfn{Faithful Group Action}{
    Let $G$ be a group acting on a set $X$. The action is called \textbf{faithful} if any of the following equivalent conditions holds
    \begin{enumerate}[(i)]
        \item $G\to \mathrm{Aut}_{\mathsf{Set}}(X)$ is injective.
        \item $\bigcap\limits_{x\in X}\mathrm{Stab}_G(x)=\{ 1_G\}$.
        \item $\forall x\in X,\;g\cdot x=x\implies g=1_G$.
    \end{enumerate}
}

\dfn{Free Group Action}{
    Let $G$ be a group acting on a set $X$. The action is called \textbf{free} if any of the following equivalent conditions holds
    \begin{enumerate}[(i)]
        \item For all $x\in X$, $\mathrm{Stab}_G(x)=\{ 1_G\}$ .
        \item $\exists x\in X,\;g\cdot x=x\implies g=1_G$.
    \end{enumerate}
}

\dfn{Transitive Group Action}{
    Let $G$ be a group acting on a set $X$. The action is called \textbf{transitive} if any of the following equivalent conditions holds
    \begin{enumerate}[(i)]
        \item For any $x,y\in X$, there exists $g\in G$ such that $g\cdot x=y$.
        \item $X$ has only one orbit, i.e. $X= Gx$ for any $x\in X$.
    \end{enumerate}
    If $G$ acts transitively on $X$, then $X$ is called a \textbf{homogeneous space} for $G$.
}
\ex{$G$ Acts on Orbit $Gx$ Transitively}{
    Let $G$ be a group acting on a set $X$ and $x\in X$. Then $G$ acts on the orbit $Gx$ by left multiplication transitively. And we have a $G$-set isomorphism
    \[
    X\cong \bigsqcup_{Gx\in X/G}Gx,
    \]
    which decomposes any $G$-set into coproduct of transitive $G$-sets.
}
\dfn{Regular Group Action}{
    Let $G$ be a group acting on a set $X$. The action is called \textbf{regular} if any of the following equivalent conditions holds
    \begin{enumerate}[(i)]
        \item The action is transitive and free.
        \item For any $x,y\in X$, there exists unique $g\in G$ such that $g\cdot x=y$.
    \end{enumerate}
    If $G$ acts regularly on $X$, then $X$ is called a \textbf{principal homogeneous space} for $G$ or a $G$-\textbf{torsor}.
}
\subsection{Coset}
\ex{Left Multiplication Action}{
    Let $G$ be a group. The \textbf{left multiplication action} of $G$ on itself is defined as 
    \begin{align*}
        m^L:G &\longrightarrow \mathrm{Aut}(G)\\
        g &\longmapsto ( x\longmapsto gx)
    \end{align*}
}
\ex{Right Multiplication Action}{
    Let $G$ be a group. The \textbf{right multiplication action} of $G$ on itself is defined as 
    \begin{align*}
        m^R:G^\circ &\longrightarrow \mathrm{Aut}(G)\\
        g &\longmapsto ( x\longmapsto xg)
    \end{align*}
}
\dfn{Left Cosets}{
    Let $G$ be a group and $H$ be a subgroup of $G$. $H^\circ$ can act on $G$ through $H^\circ\hookrightarrow G^\circ\stackrel{m^R}{\longrightarrow} \mathrm{Aut}(G)$, namely
    \begin{align*}
        H^\circ &\longrightarrow \mathrm{Aut}(G)\\
        h &\longmapsto  (g\longmapsto gh)
    \end{align*}
    The orbit of $g$ under $H^\circ$ is called the \textbf{left coset} of $H$ containing $g$, denoted by $gH$
    \[
        gH = H^\circ g = \{ gh\mid h\in H\}.
    \]
    The set of all left cosets of $H$ is denoted by $G/H^L$, called the left coset space of $G$ modulo $H$. $G/H^L$ is the orbit space of $G$ under the right multiplication action of $H$.
}
\ex{$G$ Acts on $G/H^L$ Transitively}{
    Let $G$ be a group and $H$ be a subgroup of $G$. $G$ acts on $G/H^L$ through
    \begin{align*}
        G &\longrightarrow \mathrm{Aut}(G/H^L)\\
        g &\longmapsto  (xH\longmapsto gxH)
    \end{align*}
    For any $xH,yH\in G/H^L$, we have $yH=gxH$ for some $g=yx^{-1}\in G$. Thus $G$ acts on $G/H^L$ transitively.
}
\dfn{Right Cosets}{
    Let $G$ be a group and $H$ be a subgroup of $G$. $H$ can act on $G$ through $H\hookrightarrow G\stackrel{m^L}{\longrightarrow} \mathrm{Aut}(G)$, namely
    \begin{align*}
        H &\longrightarrow \mathrm{Aut}(G)\\
        h &\longmapsto  (g\longmapsto hg)
    \end{align*}
    The orbit of $g$ under $H$ is called the \textbf{right coset} of $H$ containing $g$, denoted by $Hg$
    \[
        Hg = \{ hg\mid h\in H\},
    \]
    which matches notation of orbit. The set of all right cosets of $H$ is denoted by $G/H^R$, called the right coset space of $G$ modulo $H$.
}
\dfn{Index of Subgroup}{
    Let $G$ be a group and $H$ be a subgroup of $G$. The \textbf{index} of $H$ in $G$ is defined as the cardinality of $G/H^L$ or $G/H^R$, denoted by $[G:H]$.
}
\thm{Lagrange's Theorem}{
    Let $G$ be a finite group and $H$ be a subgroup of $G$. Then $|G|=|H|[G:H]$.
}

\prop[iso_stab_orbit]{$G$-Set Isomorphism $G/\mathrm{Stab}_G(x)^L\cong Gx$}{
    Let $G$ be a group acting on a set $X$ and $x\in X$. Then the map
    \begin{align*}
        F:G/\mathrm{Stab}_G(x)^L &\longrightarrow Gx\\
        g\hspace{1pt}\mathrm{Stab}_G(x) &\longmapsto g\cdot x
    \end{align*}
    is a $G$-set isomorphism.
}
\pf{ 
    The map is well-defined since for any $h\in \mathrm{Stab}_G(x)$, we have $(gh)\cdot x=g\cdot (h\cdot x)=g\cdot x$. The map is a $G$-set homomorphism since for any $g_1,g_2\in G$, 
    \begin{align*}
        F\left(g_1\cdot g_2\mathrm{Stab}_G(x)\right)=(g_1g_2)\cdot x=g_1\cdot(g_2\cdot x)=g_1\cdot F\left(g_2\mathrm{Stab}_G(x)\right).
    \end{align*}
     The map is injective since for any $g,h\in G$, if $g\cdot x=h\cdot x$, then $h^{-1}g\in \mathrm{Stab}_G(x)$. Hence $g\mathrm{Stab}_G(x)=h\mathrm{Stab}_G(x)$. The map is surjective because for any $g\cdot x\in Gx$, we have $F\left(g\hspace{1pt}\mathrm{Stab}_G(x)\right)=g\cdot x$.
}
\thm{Orbit-Stabilizer Theorem}{
    Let $G$ be a group acting on a set $X$. For $x\in X$, we have
    \[
    |G|=|Gx|\cdot |\mathrm{Stab}_G(x)|    .
    \]
}
\pf{ 
    According to \Cref{th:iso_stab_orbit}, we have
    \[
\left|Gx\right|=\left|G/\mathrm{Stab}_G(x)^L\right|=|G|/\left|\mathrm{Stab}_G(x)\right|.
    \]
}
\thm[Burnside's_lemma]{Burnside's Lemma}{
    Let $G$ be a finite group acting on a finite set $X$. Then the number of orbits of $G$ on $X$ is equal to
    \[
    |X/G|=\frac{1}{|G|}\sum_{g\in G}|X^g|    ,
    \]
    where $X^g=\{x\in X\mid g\cdot x=x\}$ is the set of fixed points of $g$.
}
\pf{ 
    \begin{align*}
    \sum_{g \in G}\left|X^g\right|&=|\{(g, x) \in G \times X \mid g \cdot x=x\}|\\
    &=\sum_{x \in X}\left|\mathrm{Stab}_G(x)\right|\\
    &=\sum_{x \in X}\frac{|G|}{\left|G x\right|} \quad \text{by Orbit-Stabilizer Theorem}\\
    &=|G| \sum_{G y \in X / G}\; \sum_{x \in Gy}\frac{1}{\left|G x\right|}\\
    &=|G| \sum_{G y \in X / G} \left|Gy\right|\frac{1}{\left|G y\right|}\\
    &=|G| \sum_{G y \in X / G} 1\\
    &=|G| \cdot|X / G|.
    \end{align*}

}


\subsection{Conjugacy Action}
\dfn[conjugacy_action]{Conjugacy Action and Inner Automorphism Group}{
    Let $G$ be a group. The \textbf{conjugacy action} of $G$ on itself is defined as 
    \begin{align*}
        \gamma:G &\longrightarrow \mathrm{Aut}_{\mathsf{Grp}}(G)\\
        g &\longmapsto (\gamma_g: x\longmapsto gxg^{-1})
    \end{align*}
    The \textbf{inner automorphism group} of $G$ is defined as 
    $$
    \mathrm{Inn}(G)=\gamma(G)=\{ \gamma_g\mid g\in G\},
    $$ 
    which is a subgroup of $\mathrm{Aut}(G)$.
}
\dfn{Conjugate Subgroups}{
From \Cref{ex:acting_on_power_set}, we see conjugacy action on $G$ induces an action on its power set $2^G$: \begin{align*}
    G\times 2^G &\longrightarrow 2^G\\
    (g,E)&\longmapsto gEg^{-1}
\end{align*}
If $H$ is a subgroup of $G$, then $gHg^{-1}$ is also a subgroup of $G$. We say $H$ and $gHg^{-1}$ are \textbf{conjugate subgroups} of $G$.
}
\prop{Equivalent 
Characterization of Inner Automorphisms}{
    Let $G$ be a group and $\varphi \in \mathrm{Aut}(G)$. Then $\varphi \in \mathrm{Inn}(G)$ if and only if $\varphi$ satisfies the property: 
    \[
        \text{$G$ is embedded in a group $H$} \implies \text{$\varphi$ extends to an automorphism of $H$}.
    \]
    To be specific, the property can be stated as: for any monomophism $\iota:G\hookrightarrow H$,
    there exists $\psi\in \mathrm{Aut}(H)$ such that the following diagram commutes
    \[
        \begin{tikzcd}[ampersand replacement=\&]
            G \arrow[r, hook, "\iota"] \arrow[d, "\varphi"'] \& H \arrow[d, "\psi"] \\
            G \arrow[r, hook, "\iota"']                      \& H                             
        \end{tikzcd}
    \]
}
\dfn{Conjugacy Class}{
    Let $G$ be a group. The orbit of $a$ under conjugacy action is called the \textbf{conjugacy class} of $a$, denoted by 
    \[
      \mathrm{Cl}(a)=\{ gag^{-1}\mid x\in G   \}.
    \]
    Two elements $a,b\in G$ are called \textbf{conjugate} if $\mathrm{Cl}(a)=\mathrm{Cl}(b)$.
}

\prop{Properties of Conjugacy Class}{
    Let $G$ be a group. Then the following are true:
    \begin{enumerate}[(i)]
        \item $a\sim b$ if and only if $\mathrm{Cl}(a)=\mathrm{Cl}(b)$.
        \item $\mathrm{Cl}(a)=\{ a\}$ if and only if $a\in Z(G)$.
    \end{enumerate}
}


\dfn{Outer Automorphism Group}{
    Let $G$ be a group. Then we have $\mathrm{Inn}(G) \lhd \mathrm{Aut}(G)$. And the \textbf{outer automorphism group} of $G$ is defined as 
    $$
    \mathrm{Out}(G)=\mathrm{Aut}(G)/\mathrm{Inn}(G).
    $$ 
}
\dfn{Characteristic Subgroup}{
    Let $G$ be a group. A subgroup $H\le G$ is called a \textbf{characteristic subgroup} if 
    \[
        \forall \varphi \in\mathrm{Aut}(G),\; \varphi(H)\subseteq H.
    \]
    It would be equivalent to require the stronger condition that $\forall \varphi \in\mathrm{Aut}(G)$, $\varphi(H)= H$, because 
    \[
        \varphi(H)\subseteq H\implies \varphi^{-1}(H)\subseteq H\implies H\subseteq \varphi(H).
    \]
}
\dfn{Fully Characteristic Subgroup}{
    Let $G$ be a group. A subgroup $H\le G$ is called a \textbf{fully characteristic subgroup} if 
    \[
        \forall \varphi \in\mathrm{End}(G),\; \varphi(H)\subseteq H.
    \]
}
\dfn{Word Map}{
    Suppose $G$ is a group and 
    $$
    x=x_{i_1}^{\alpha_{1}}\cdots x_{i_m}^{\alpha_{m}}\in F\langle x_1,\cdots,x_n\rangle
    $$
    is a reduced word in a free group of rank $n$, where $\alpha_k\in\mathbb{Z}-\{0\}$ for $k=1,2,\cdots,m$. The \textbf{word map} induced by $x$ is defined as a map
    \begin{align*}
        w_x:G^m &\longrightarrow G\\
        (g_1,\cdots,g_m) &\longmapsto g_{i_1}^{\alpha_1}\cdots g_{i_m}^{\alpha_m}.
    \end{align*}
}
\dfn{Verbal Subgroup}{
    Let $G$ be a group and $\mathcal{W}$ be a collection of word maps. A subgroup $H\le G$ is called a \textbf{verbal subgroup} if $H$ is the subgroup generated by
    $$
    \left\{ w(g_1,\cdots,g_n)\mid w\in\mathcal{W},\; g_i\in G \right\}.
    $$
}
\dfn{Commutator}{
    Let $G$ be a group. The word map induced by $xyx^{-1}y^{-1}$ is a binary operation defined on $G$, denoted by
    \begin{align*}
        [\cdot,\cdot]:G\times G &\longrightarrow G\\
        (x,y) &\longmapsto [x,y]=xyx^{-1}y^{-1}
    \end{align*}
    $[x,y]$ is called the \textbf{commutator} of $x$ and $y$.
}
\prop{Properties of Commutator}{
    Let $G$ be a group. Then
    \begin{enumerate}[(i)]
        \item $x$ commutes with $y$ if and only if $[x,y]=1_G$.
        \item $[x,y]^{-1}=[y,x]$.
        \item For any homomorphism $f:G\to H$, $f([x,y])=[f(x),f(y)]$.
    \end{enumerate}
}
\prop{}{
    According to the extent that a subgroup is preserved by endomorphisms, we have the following inclusions
    \[
        \left\{\text{verbal subgroups}\right\}\subseteq \left\{\text{fully characteristic subgroups}\right\}\subseteq \left\{\text{characteristic subgroups}\right\}\subseteq\left\{\text{normal subgroups}\right\}.
    \]
}

\dfn{Commutator Subgroup}{
    Let $G$ be a group. The \textbf{commutator subgroup} or \textbf{derived subgroup} of $G$ is the subgroup generated by all the commutators, denoted by
    $$
    [G,G]=\langle \left\{[x,y]\mid x,y\in G \right\}\rangle.
    $$
}
\prop{Properties of Commutator Subgroup}{
    Let $G$ be a group. Then
    \begin{enumerate}[(i)]
        \item $[G,G]$ is a verbal subgroup with $\mathcal{W}=\left\{ [\cdot,\cdot] \right\}$. Hence $[G,G]\lhd G$.
        \item $[G,G]$ is the smallest normal subgroup of $G$ such that $G/[G,G]$ is abelian.
        \item $[G,G]=\{1_G\}$ if and only if $G$ is abelian.
    \end{enumerate}
}

\dfn{Abelianization}{
    Let $G$ be a group. The \textbf{abelianization} of $G$ is defined as the quotient group
    $$
    G^{\mathrm{ab}}=G/[G,G].
    $$
}
\prop{Universal Property of Abelianization}{
    Let $G$ be a group and $A$ be an abelian group. Then any group homomorphism $f:G\to A$ factors through $G^{\mathrm{ab}}$ uniquely, that is, there exists a unique homomorphism $\bar{f}:G^{\mathrm{ab}}\to A$ such that the following diagram commutes
    $$
    \begin{tikzcd}[ampersand replacement=\&]
        G \arrow[rr, "f"] \arrow[rd] \&  \& A \\
        \& G^{\mathrm{ab}} \arrow[ru, "\bar{f}"'] \& 
    \end{tikzcd}
    $$
}
\dfn{Normalizer}{
    Let $G$ be a group and $S$ be a subset of $G$. The \textbf{normalizer} of $S$ in $G$ is defined as 
    $$
    \mathrm{N}_G(S)=\left\{ g\in G\mid gSg^{-1}=S \right\}.
    $$ 
    Let $G$ acts on $2^G$ by conjugation (c.f. \Cref{th:conjugacy_action}). Then $\mathrm{N}_G(S)=\mathrm{Stab}_G(S)\le G$.
}
\prop{Normalizer of a Subgroup}{
    Let $G$ be a group and $H$ be a subgroup of $G$. Then $H\lhd\mathrm{N}_G(H)$. Moreover, $\mathrm{N}_G(H)$ is the largest subgroup of $G$ in which $H$ is normal, i.e. $$H\lhd K\le G\implies H\lhd K\le \mathrm{N}_G(H)\le G.$$
}
\pf{
    For all $n\in \mathrm{N}_G(H)$, we have $nHn^{-1}=H$, which implies $H \lhd \mathrm{N}_G(H)$. Suppose $H\lhd K\le G$. Then $\forall k\in K$, $kHk^{-1}=H$. Hence $k\in \mathrm{N}_G(H)$. Therefore we prove the maximality of $\mathrm{N}_G(H)$.
}
\dfn{Centralizer}{
    Let $G$ be a group and $S$ be a subset of $G$. The \textbf{centralizer} of $S$ in $G$ is defined as 
    \[
        \mathrm{C}_G(S)=\left\{g\in G\mid  \forall s\in S,\;gs=sg\right\}.
    \]
    The centralizer of $\{x\}$ is the stabilizer subgroup of $x$ under conjugacy action, denoted by 
    \[
      \mathrm{C}_G(x)=\{ g\in G\mid gx=xg \}=\{ g\in G\mid gxg^{-1}=x \},
    \]
    which equals $\mathrm{N}_G(x)$.
}

\dfn{Center of a Group}{
    Let $G$ be a group. The \textbf{center} of $G$ is defined as the centralizer of $G$ in $G$, denoted by
    \[
        Z_G=\mathrm{C}_G(G)=\{ g\in G\mid \forall x\in G,\; gx=xg\}.
    \]
}
\prop[normalizer_conjugation_action]{Normalizer $\mathrm{N}_G(S)$ Acts on $S$ by Conjugation}{
    Let $G$ be a group and $S\subseteq G$. Then $\mathrm{N}_G(S)$ acts on $S$ by conjugation, i.e. by the group homomorphism
    \begin{align*}
        \Psi_S: \mathrm{N}_G(S) &\longrightarrow \mathrm{Aut}_{\mathsf{Set}}(S)\\
        g &\longmapsto \gamma_g|_S
    \end{align*}
    where $\gamma_g|_S(s)=gsg^{-1}$ for all $s\in S$. Moreover, we have $\ker\Psi_S=\mathrm{C}_G(S)\lhd \mathrm{N}_G(S)$.
}
\proof{
    $\Psi_S$ is obtained from restriction $\Psi_S:\mathrm{N}_G(S)\hookrightarrow G\xrightarrow{\gamma}\mathrm{Aut}_{\mathsf{Set}}(G)$. Since for any $g\in \mathrm{N}_G(S)$, $\gamma_g|_S(S)=\left\{gsg^{-1}\in G\mid s \in S\right\}\subseteq S$, we see $\Psi_S(g)=\gamma_g|_S\in \mathrm{Aut}_{\mathsf{Set}}(S)$. The kernel of $\Psi_S$ is 
    \begin{align*}
        \ker\Psi_S &= \left\{ n\in \mathrm{N}_G(S)\mid \gamma_n|_S=\mathrm{id}_S \right\}= \left\{ n\in \mathrm{N}_G(S)\mid \forall s\in S, nsn^{-1}=s \right\}=\mathrm{N}_G(S)\cap \mathrm{C}_G(S) = \mathrm{C}_G(S).
    \end{align*}
}

\thm{N/C Theorem}{
    Let $G$ be a group and $H$ be a subgroup of $G$. By \Cref{th:normalizer_conjugation_action}, $\mathrm{N}_G(H)$ acts on $H$ by conjugation through the group homomorphism $\Psi_H:\mathrm{N}_G(H)\to \mathrm{Aut}_{\mathsf{Set}}(H)$. We assert that 
    \[
        \mathrm{N}_G(H)/\mathrm{C}_G(H) \cong \mathrm{Im}\Psi_H\le \mathrm{Aut}_{\mathsf{Grp}}(H).
    \]
    Hence it is legal to define
    \begin{align*}
        \Psi_H: \mathrm{N}_G(H) &\longrightarrow \mathrm{Aut}_{\mathsf{Grp}}(H)\\
        g &\longmapsto \gamma_g|_H
    \end{align*}
}

\prop{Properties of Centralizer}{
    Let $G$ be a group and $S$ be a subset of $G$. Then
    \begin{enumerate}[(i)]
        \item $\mathrm{C}_G(S)\lhd \mathrm{N}_G(S)\le G$.
    \end{enumerate}
}


\prop{}{
    Let $G$ acts transitively on a set $S$. Choose $s \in S$, and let $H = \mathrm{Stab}_G(s)$. Then we have group isomorphism $\mathrm{N}_G(H)/H\cong \mathrm{Aut}_{G\text{-}\mathsf{Set}}(S)$.
}
\pf{
    For any $n \in \mathrm{N}_G(H)$ with image $\bar{n} \in \mathrm{N}_G(H)/H$, since $G$ acts transitively on $S$, we can define a map 
    \begin{align*}
        \phi(\bar{n}):S &\longrightarrow S\\
        g\cdot s &\longmapsto gn^{-1}\cdot s
    \end{align*}
    and check $\phi(\bar{n})\in\mathrm{Aut}_{G\text{-}\mathsf{Set}}(S)$ by
    \begin{itemize}
        \item $\phi(\bar{n})\in\mathrm{Aut}_{G\text{-}\mathsf{Set}}(S)$. $\phi(\bar{n})$ is well-defined: 
        \begin{align*}
               &\bar{m}=\bar{n}\in\mathrm{N}_G(H)/H\\
               \implies &m^{-1}\in n^{-1}H\\
              \implies & \exists h\in H,m^{-1}=n^{-1}h\\
              \implies&\phi(\bar{m})(s)=gm^{-1}\cdot s=gn^{-1}h\cdot s=gn^{-1}\cdot s=\phi(\bar{n})(s).
        \end{align*}
        \item $\phi(\bar{n})$ is a $G$-set morphism: $\phi(\bar{n})(g\cdot s)=gn^{-1}\cdot s=g\cdot\phi(\bar{n})(s)$
        \item $\phi(\bar{n})$ is a bijection: $\phi(\bar{n}^{-1})$ is the inverse of $\phi(\bar{n})$.
    \end{itemize}
    For any automorphism $\psi\in \mathrm{Aut}_{G\text{-}\mathsf{Set}}(S)$, by transitivity we have $\psi(s)=n^{-1}\cdot s$ for some $n \in G$. For $h \in H$, $hn^{-1}\cdot s=h\cdot\psi(s)=\psi(h\cdot s)=\psi(s)=n^{-1}\cdot s$, hence $n h n^{-1} \in H$ and $n \in \mathrm{N}_G(H)$. This gives a well-defined map 
    \begin{align*}
        \eta:\mathrm{Aut}_{G\text{-}\mathsf{Set}}(S) &\longrightarrow \mathrm{N}_G(H)/H\\
        \psi &\longmapsto \bar{n}
    \end{align*}
    Clearly $\eta(\phi(\bar{n}))=\bar{n}$. Suppose $\psi\in\mathrm{Aut}_{G\text{-}\mathsf{Set}}(S)$ and $\psi(s)=n\cdot s$. Then $\phi(\eta(\psi))(g\cdot s)=g n\cdot s=g\psi(s)=\psi(g\cdot s)$, which imples $\phi(\eta(\psi))=\psi$. Therefore, $\phi:\mathrm{N}_G(H)/H\to\mathrm{Aut}_{G\text{-}\mathsf{Set}}$ is bijective. And it is easy to check that this $\phi$ is an isomorphism of groups,
    \[
    \phi(\bar{n}\bar{m})(g\cdot s)=\phi(\overline{nm})(g\cdot s)=gnm\cdot s=\phi(\bar{n})(gm\cdot s)=\phi(\bar{n})\circ\phi(\bar{m})(g\cdot s).
    \]
}
\section{Symmetric Groups}

\dfn{$k$-Cycle}{
    Let $n\ge 2$ and $k\ge 1$ be integers. A \textbf{$k$-cycle} in $S_n$ is a permutation $\sigma\in S_n$ such that there exist a subset of $P_n=\{1,2,\cdots,n\}$, denoted by $A=\{a_1,a_2,\cdots,a_k\}$, satisfying
    \begin{enumerate}[(i)]
        \item $\sigma(a_i)=a_{i+1}\text{ for }i=1,2,\cdots,k-1$,
        \item $\sigma(a_k)=a_1$,
        \item $\sigma(x)=x\text{ for }x\in P_n-A$.
    \end{enumerate}
}
\dfn{Cycle Decomposition}{
    Let $n\ge 2$ and $\sigma\in S_n$. A \textbf{cycle decomposition} of $\sigma$ is a product of disjoint cycles $\sigma=\sigma_1\sigma_2\cdots\sigma_r$ such that $\sigma_i$ is a $k_i$-cycle for $i=1,2,\cdots,r$ and $k_1+k_2+\cdots+k_r=n$. $(k_1,k_2,\cdots,k_r)$ is called the \textbf{cycle type} of $\sigma$. Equivalently, a \textbf{cycle decomposition} of $\sigma$ is the decomposition of $P_n$ into orbits under the action of $\langle\sigma\rangle$.
}

\thm[Polya_enumeration_unweighted]{Pólya Enumeration Theorem (Unweighted)}{
    Let $X, Y$ be finite sets, where $X=\{1,2,\cdots,n\}$ is the set of points to be colored and $Y$ is the set of colors. Suppose a group $G$ acts on $X$ through $\sigma:G\to\mathrm{Aut}_{\mathsf{Set}}(X)$. Then it also acts on $Y^X$ by \Cref{ex:acting_on_functions}. Define that a coloring configuration of $(X,Y,\sigma)$ is an orbit in the $G$-set $Y^X$. Then the number of essentially distinct coloring configurations is
$$
\left|Y^X / G\right|=\frac{1}{|G|} \sum_{g \in G}|Y|^{c(g)},
$$
where $c(g):=\left|X/\langle \sigma_g\rangle\right|$ denotes the number of cycles in the cycle decomposition of $\sigma_g \in \mathrm{Aut}_{\mathsf{Set}}(X)$.
}
\pf{ 
    We apply Burnside's lemma \ref{th:Burnside's_lemma}, which states that
$$
\left|Y^X / G\right|=\frac{1}{|G|} \sum_{g \in G}\left|\left(Y^X\right)^g\right| .
$$
It remains to show that $\left|\left(Y^X\right)^g\right|=|Y|^{c(g)}$. For any map $f:X\to Y$, if 
\[
f(g\cdot x)=f(x)   ,\quad\forall x\in X, 
\]
then $f(x)=f(y)\iff Gx=Gy$. By the universal property of the quotient set, there exists a unique map $\overline{f}:X/G\to Y$ such that the following diagram commutes
\[
\begin{tikzcd}[ampersand replacement=\&]
    X \arrow[rr, "f"] \arrow[rd,"\pi"'] \&  \& Y \\
    \& X/\langle \sigma_g\rangle \arrow[ru, "\overline{f}"'] \&
\end{tikzcd}
\]
Conversely, for any map $h:X/\langle \sigma_g\rangle\to Y$, we can define a map $f:X\to Y$ and checking that $f(g\cdot x)=f(x)$ for all $x\in X$. Hence we have a bijection between $\left(Y^X\right)^g$ and $Y^{X/\langle \sigma_g\rangle}$, which implies that $\left|\left(Y^X\right)^g\right|=|Y|^{c(g)}$.
}
\dfn{Cycle Index Polynomial}{
    Let $G$ be a subgroup of $S_n$. The \textbf{cycle index polynomial} of $G$ is defined as
    $$
    Z(t_1,t_2,\cdots,t_n;G)=\frac{1}{|G|}\sum_{g\in G}t_1^{c_1(g)}t_2^{c_2(g)}\cdots t_n^{c_n(g)},
    $$
    where $c_k(g)$ denotes the number of $k$-cycles in the cycle decomposition of $\sigma_g$. $|G|Z(t_1,t_2,\cdots,t_n;G)$ can be seen as a generating function, where the coefficient of $t_1^{c_1}t_2^{c_2}\cdots t_n^{c_n}$ represents the number of permutations in $G$ with excatly $c_k$ $k$-cycles for $k=1,2,\cdots,n$.
}
\thm{Pólya Enumeration Theorem (Weighted)}{
Let $X, Y$ be finite sets, where $X=\{1,2,\cdots,n\}$ is the set of points to be colored and $Y$ is the set of colors.  Suppose $w:Y\to\mathbb{Z}_{\ge 0}^m$ is a weight function which assigns a weight $w(y)=(w_1(y),w_2(y),\cdots,w_m(y))$ to each color $y\in Y$. Consider the genrating function 
\[
q(x_1,\cdots,x_m)=\sum_{y\in Y}x_1^{w_1(y)}x_2^{w_2(y)}\cdots x_m^{w_m(y)},
\]
where the coefficient of the term $x_1^{a_1}x_2^{a_2}\cdots x_m^{a_m}$ is the number of colors with weight $(a_1,a_2,\cdots,a_m)$. For each coloring configuration $f:X\to Y$, define the its weight as $W(f)$, where
\begin{align*}
    W:Y^X&\longrightarrow\mathbb{Z}_{\ge 0}^m\\
f&\longmapsto\sum_{x\in X}w(f(x))
\end{align*}
Let $G$ be subgroup of $S_n$. Since for all $g\in G$ and $f\in X^Y$,
\begin{align*}
    W(g\cdot f)=\sum_{x\in X}w((g\cdot f)(x))=\sum_{x\in X}w(f(g^{-1}\cdot x))=\sum_{x\in X}w(f(x))=W(f),
\end{align*}
we see $G$ acts on $W^{-1}(\omega)$ for each $\omega=(\omega_1,\cdots,\omega_m)\in\mathbb{Z}_{\ge 0}^m$. The generating function for the number of essentially distinct coloring configurations with weight $\omega$ can be expressed as
\begin{align*}
    \mathrm{CGF}\left(x_1, \cdots,x_m\right) &=\sum_{\omega \in \mathbb{Z}_{\ge 0}^m}\left|W^{-1}(\omega)/G\right| x_1^{\omega_1} \cdots x_m^{\omega_m}\\
     &= Z\left(q\left(x_1, \cdots,x_m\right), q\left(x_1^2, \cdots,x_m^2\right), \cdots, q\left(x_1^n, \cdots,x_m^n\right);G\right).
\end{align*}
}
\pf{ 
    For any $\omega=(\omega_1,\cdots,\omega_m)\in\mathbb{Z}_{\ge 0}^m$, by applying Burnside's lemma \ref{th:Burnside's_lemma} to the $G$-set $W^{-1}(\omega)$, we have
    \[
        \left|W^{-1}(\omega)/G\right| =\frac{1}{|G|}\sum_{g\in G}\left|W^{-1}(\omega)^g\right|.
    \]
    Similar to \Cref{th:Polya_enumeration_unweighted}, we have a bijection between 
    \[
        W^{-1}(\omega)^g=\left\{f\in \left(Y^X\right)^g\mid W(f)=\omega\right\}\quad\text{and}\quad\left\{\overline{f}\in Y^{X/\langle g\rangle}\mid W\left(\,\overline{f}\circ\pi\right)=\omega\right\},
    \]
    which implies coloring points of $X$ essentially distinctly with colors in $Y$ with weight $\omega=(\omega_1,\cdots,\omega_m)$ is equivalent to coloring orbits $g_1,g_2,\cdots,g_r$ of $X/\langle g\rangle$ with colors in $Y$ with weight $\omega=(|g_1|\omega_{v_1},\cdots,|g_r|\omega_{v_r})$. Hence
    \begin{align*}
        \sum_{\omega\in\mathbb{Z}_{\ge 0}^m} \left |W^{-1}(\omega)^g\right |x_1^{\omega_1} \cdots x_m^{\omega_m} &= \prod_{g_i \in X/\langle g\rangle } q\left (x_1^{|g_i|}, x_2^{|g_i|}, \cdots,x_m^{|g_i|}\right )\\
&= q(x_1,  \cdots,x_m)^{c_1(g)} q \left (x_1^2, \cdots,x_m^2 \right )^{c_2(g)} \cdots q \left (x_1^n, \cdots,x_m^n\right )^{c_n(g)}.
    \end{align*}
    Therefore, we have
    \begin{align*}
        \mathrm{CGF}\left(x_1, \cdots,x_m\right) &=\sum_{\omega \in \mathbb{Z}_{\ge 0}^m}\left|W^{-1}(\omega)/G\right| x_1^{\omega_1} \cdots x_m^{\omega_m}\\
        &=\sum_{\omega \in \mathbb{Z}_{\ge 0}^m}\frac{1}{|G|}\sum_{g \in G}\left|W^{-1}(\omega)^g\right| x_1^{\omega_1} \cdots x_m^{\omega_m}\\
        &=\frac{1}{|G|}\sum_{g \in G}\sum_{\omega \in \mathbb{Z}_{\ge 0}^m}\left|W^{-1}(\omega)/G\right| x_1^{\omega_1} \cdots x_m^{\omega_m}\\
        &=\frac{1}{|G|}\sum_{g \in G}q(x_1,  \cdots,x_m)^{c_1(g)} q \left (x_1^2, \cdots,x_m^2 \right )^{c_2(g)} \cdots q \left (x_1^n, \cdots,x_m^n\right )^{c_n(g)}\\
     &= Z\left(q\left(x_1, \cdots,x_m\right), q\left(x_1^2, \cdots,x_m^2\right), \cdots, q\left(x_1^n, \cdots,x_m^n\right);G\right).
    \end{align*}
}
\ex{Counting the Isomers of Chlorobenzene}{
    Replacing the H in a benzene ring with Cl, one can consider coloring the 6 vertices of the benzene ring with two colors: H and Cl. The group action is $D_{12}$, which includes 6 rotations: 0°, 60°, ..., 300°, and 6 reflections: 3 along the opposing sides and 3 along the opposing diagonals.

    Compute the cycle decomposition for each \(g\):
    \begin{itemize}
        \item Identity: 6 1-cycles, corresponding to \(t_1^6\).
        \item Rotation by 1 or 5 times 60°: 1 6-cycle, corresponding to \(t_6^1\).
        \item Rotation by 2 or 4 times 60°: 2 3-cycles, corresponding to \(t_3^2\).
        \item Rotation by 3 times 60°: 3 2-cycles, corresponding to \(t_2^3\).
        \item 3 kinds of opposing side reflections: 3 2-cycles, corresponding to \(t_2^3\).
        \item 3 kinds of opposing diagonal reflections: 2 1-cycles and 2 2-cycles, corresponding to \(t_1^2t_2^2\).
    \end{itemize}

    Writing out \(Z(t_1,\cdots,t_6;D_{12})\):
\[
    Z(t_1,\cdots,t_6;D_{12})=\frac{1}{12}\left(t_1^6+3 t_1^2 t_2^2+4 t_2^3+2 t_3^2+2 t_6\right)
\]

Assigning a weight of (1,0) to H and a weight of (0,1) to Cl, the corresponding generating function is \(q(\text{H},\text{Cl})=\text{H}+\text{Cl}\). Finally, the generating function for the number of essentially distinct colorings is:

\[
\begin{aligned}
&\quad \mathrm{CGF}(\text{H},\text{Cl})\\
& =\frac{1}{12}\left((\text{H}+\text{Cl})^6+3(\text{H}+\text{Cl})^2\left(\text{H}^2+\text{Cl}^2\right)^2+4\left(\text{H}^2+\text{Cl}^2\right)^3+2\left(\text{H}^3+\text{Cl}^3\right)^2+2\left(\text{H}^6+\text{Cl}^6\right)\right) \\
& =\text{H}^6+\text{H}^5 \text{Cl}+3 \text{H}^4 \text{Cl}^2+3 \text{H}^3 \text{Cl}^3+3 \text{H}^2 \text{Cl}^4+\text{H} \text{Cl}^5+\text{Cl}^6
\end{aligned}
\]

The coefficients give the number of isomers for various chlorobenzene compounds. For instance, looking at the term \(3 \text{H}^4 \text{Cl}^2\), with a weight of $(4,2)$, it can only be achieved using 4 H atoms and 2 Cl atoms. The coefficient 3 indicates that there are 3 isomers for dichlorobenzene. The 3 isomers are 1,2-dichlorobenzene, 1,3-dichlorobenzene, and 1,4-dichlorobenzene, plotted as follows:

\begin{center}
    \setchemfig{atom sep=1.5em}
    \chemfig[scale=0.1]{*6(=-(-Cl)=(-Cl)-=-)} \qquad % 1,2-dichlorobenzene
    \chemfig[scale=0.1]{*6(=-=(-Cl)-=(-Cl)-=)} \qquad % 1,3-dichlorobenzene
    \chemfig[scale=0.1]{*6(=-(-Cl)=-=(-Cl)-)}     % 1,4-dichlorobenzene
\end{center}
}


\chapter{Abelian Group}
\section{Basic Concepts}
\dfn{Abelian Group}{
    An \textbf{abelian group} is a group $G$ such that $G$ is commutative. That is, for all $a,b\in G$, $ab=ba$.
}

An Abelian group is a $\mathbb{Z}$-module and we have category isomorphism $\mathsf{Ab}\cong\mathbb{Z}\raisebox{0.22ex}{-}\mathsf{Mod}$.

\chapter{Ring}
\section{Basic Concepts}
\dfn{Ring}{
    A \textbf{ring} is a set $R$ together with two binary operations $+$ and $\cdot$ on $R$ such that
    \begin{enumerate}[(i)]
        \item $(R,+)$ is an abelian group.
        \item $(R,\cdot)$ is a monoid.
        \item $\cdot$ is distributive over $+$,
        \begin{align*}
            a\cdot(b+c)=a\cdot b+a\cdot c\\
            (a+b)\cdot c=a\cdot c+b\cdot c
        \end{align*}
    \end{enumerate}
}
A ring is a monoid object in the category $\mathsf{Ab}$. In other words, a ring is an $\mathsf{Ab}$-enriched category with only one object.\\
A ring $R$ is an $R$-module over itself.\\
A ring $R$ is a $Z(R)$-algebra and also a $\mathbb{Z}$-algebra.\\
\dfn{Unit Group of a Ring}{
    Let $R$ be a ring. The \textbf{unit group} of $R$ is the group of invertible elements of $R$ under multiplication, denoted by $R^\times$.
}

\dfn{Reduced Ring}{
    A ring $R$ is called \textbf{reduced} if it has no nonzero nilpotent elements, or equivalently, if for any $x\in R$, $x^2=0\implies x=0$.
}

\prop{Examples of Reduced Ring}{
    \begin{enumerate}[(i)]
        \item Subrings, products, and localizations of reduced rings are again reduced rings.
        \item Every integral domain is reduced.
        \item $\mathbb{Z}/n\mathbb{Z}$ is reduced if and only if $n=0$ or $n$ is square-free.
    \end{enumerate}
}

\dfn{Local Ring}{
    A ring $R$ is called \textbf{local} if it has a unique maximal ideal.
}

\chapter{Commutative Ring}
\section{Basic Concepts}
\prop{Equivalent Definition for Local Ring}{
    Let $R$ be a commutative ring. Then the following are equivalent:
    \begin{enumerate}[(i)]
        \item $R$ is a local ring.
        \item $R$ has a unique maximal ideal.
        \item $R$ has a maximal idea $\mathfrak{m}$ and $R - \mathfrak{m}\subseteq R^{\times}$.
        \item $R$ is not the zero ring and for every $x \in R$, $x\in R^{\times}$ or $1-x\in R^{\times}$.
        \item $R$ is not the zero ring and if $\sum_{i=1}^n r_i\in R^{\times}$, then there exist some $i$ such that $r_i\in R^{\times}$. 
        \item $R$ is not the zero ring and the sum of any two non-units in $R$ is a non-unit.
    \end{enumerate}
}

\subsection{Ideals}
\dfn{Ideal}{
    Let $R$ be a ring. A subset $I\subseteq R$ is called an \textbf{ideal} if
    \begin{enumerate}[(i)]
        \item $I$ is a subgroup of $(R,+)$.
        \item $I$ is closed under multiplication, i.e. $a\in I$ and $b\in R$ implies $ab\in I$.
    \end{enumerate}
}

\prop{Ideal as Submodule}{
    Let $R$ be a ring and $I\subseteq R$ be a subset of $R$. Then $I$ is an ideal of $R$ if and only if $I$ is a submodule of $R$ as an $R$-module.
}

\dfn{Prime Ideal}{
    Let $R$ be a commutative ring. An ideal $I\subseteq R$ is called \textbf{prime} if  and 
    \begin{enumerate}[(i)]
        \item $I\neq R$, i.e. $I$ is a proper ideal.
        \item $ab\in I\implies a\in I\text{ or }b\in I$, i.e. there exist no two elements in $R$ whose product is in $I$ but neither of them is in $I$.
    \end{enumerate}    
}
\prop{Prime Ideal Equivalent Definition}{
    Let $R$ be a commutative ring and $I\subseteq R$ be an ideal. Then $I$ is prime if and only if $R/I$ is an integral domain.
}
\dfn{Maximal Ideal}{
    Let $R$ be a commutative ring. An ideal $I\subseteq R$ is called \textbf{maximal} if
    \begin{enumerate}[(i)]
        \item $I\ne R$, i.e. $I$ is a proper ideal.
        \item There exists no ideal $J\subseteq R$ such that $I\subsetneq J\subsetneq R$.
    \end{enumerate}
}
\prop{Maximal Ideal Equivalent Definition}{
    Let $R$ be a commutative ring and $I\subseteq R$ be an ideal. Then $I$ is maximal if and only if $R/I$ is a field.  
}
\dfn{Ideal generated from subset}{
    Let $R$ be a commutative ring and $\mathcal I(R)$ be the set of all ideals of $R$. Suppose $S\subseteq R$ be a subset. The \textbf{ideal generated by $S$}, denoted by $(S)$, is the smallest ideal of $R$ containing $S$, i.e. 
    \[
        (S)=\bigcap_{\substack{ I\in \mathcal I(R)\\S\subseteq I}}I.
    \]
    If $S=\{a_1,\dots,a_n\}$, we write 
    \[
        (S)=(a_1,\dots,a_n)=\left\{\sum_{i=1}^n r_ia_i\midv  r_i\in R\right\}.
    \]
}
\dfn{Ideal Operations}{
    \begin{enumerate}[(i)]
        \item Sum: $$I+J=\left\{a+b\mid a\in I,b\in J\right\}=\left(I\cup J\right),$$
        $$
        \sum_{t \in T} I_t=\left\{a_{t_1}+ \cdots +a_{t_n}\mid n\in\mathbb{Z}_{+},t_i\in T,a_{t_i}\in I_{t_i}\right\}.
        $$
        \item Product: $$IJ=\left\{\sum_{i=1}^n a_ib_i\midv n\in\mathbb{Z}_{+},a_i\in I,b_i\in J\right\}=\left(\{ab\mid a\in I,b\in J\}\right).$$
        \item Power: $I^0=R$,
        \[
            I^n=\underbrace{I\cdots I}_{n\text{ times}}=\left(\{a^n\midv a\in I\}\right), 
            \]
        \item Radical: \[
            \sqrt{I} = \left\{ r \in R \mid r^n \in I \text{ for some } n \in \mathbb{Z}_{+} \right\} = \bigcap_{\substack{\mathfrak{p} \in \mathrm{Spec} R \\ I \subseteq \mathfrak{p}}} \mathfrak{p}
            \].
    \end{enumerate}
}

\prop{Properties of Ideal Operations}{
    \begin{enumerate}[(i)]
        \item $(I\cap J)^2 \subseteq I J \subseteq I \cap J \subseteq I+J$
        \item ${I} \cap({J}+{K}) \supseteq {I} \cap {J}+{I} \cap {K}$
        \item ${I} ({J}+{K}) = {I}  {J}+{I}  {K}$
        \item $$
        \begin{gathered}
        \left(\sum_{t \in T} I_t\right) J=\sum_{t \in T}\left(I_t J\right), \quad J\left(\sum_{t \in T} I_t\right)=\sum_{t \in T} J I_t.
        \end{gathered}
        $$
        \item $I(J K)=(I J) K$
        \item $I^0 \supseteq \sqrt{I} \supseteq I \supseteq I^2 \supseteq I^3 \supseteq \cdots$
        \item $\sqrt{\sqrt{I}} = \sqrt{I}$,
        \item $\sqrt{I^n}=\sqrt{I}$, $\sqrt{I J}=\sqrt{I \cap J}=\sqrt{I} \cap \sqrt{J}$
    \end{enumerate}
}
\proof{
    \begin{enumerate}[(i)]
        \item Since $\{ab\mid a\in I,b\in J\}\subseteq I\cap J$, we see $IJ=\left(\{ab\mid a\in I,b\in J\}\right)\subseteq I\cap J$. Also we can check $I \cap J \subseteq I \cup J\subseteq (I \cup J)=I+J$.
    \end{enumerate}
}

\dfn{Radical Ideal}{
    An ideal $I$ is called a \textbf{radical ideal} if $I=\sqrt{I}$.
}
\prop{Radical Ideal Equivalent Definition}{
    Let $R$ be a commutative ring and $I\subseteq R$ be an ideal. Then $I$ is radical if and only if $R/I$ is reduced.
}
\dfn{Nilradical}{
    The \textbf{nilradical} of $R$, denoted by $\mathfrak{N}_R$, is the radical ideal $\sqrt{0}$ consisting of all the nilpotent elements of $R$. We have
    \[
        \mathfrak{N}_R=\sqrt{0}=\left\{ r \in R \mid r^n=0 \text{ for some } n \in \mathbb{Z}_{+} \right\} = \bigcap_{\substack{\mathfrak{p} \in \mathrm{Spec} R }} \mathfrak{p}
            \]
}


\prop{Properties of Radical Ideal}{
    \begin{enumerate}
        \item For any ideal $I$, $\sqrt{0}\subseteq \sqrt{I}$.
        \item $\sqrt{I}$ is the smallest radical ideal containing $I$.
        \item $\sqrt{\mathfrak{p}^n}=\sqrt{\mathfrak{p}}=\mathfrak{p}$ for any prime ideal $\mathfrak{p}$, which means prime ideals are radical.
        \item Suppose the natural projection $\pi: R\to R/I$ induces a bijection between the set of ideals of $R$ containing $I$ and the set of ideals of $R/I$, denoted by $\tilde{\pi}:\mathcal{I}(R)\to\mathcal{I}(R/I)$. Then $\tilde{\pi}$ maps $\sqrt{I}$ to $\mathfrak{N}_{R/I}$.
        \item A commutative ring $R$ is reduced if and only if $\mathfrak{N}_R=(0)$. 
    \end{enumerate}
}
In summary, we have the following chain of inclusions:
\[
\left\{\text{maximal ideals of }R\right\} \subseteq \left\{\text{prime ideals of }R\right\} \subseteq \left\{\text{radical ideals of }R\right\} \subseteq \left\{\text{ideals of }R\right\}.
\]
\prop{Quotient Preserves Radical, Prime, Maximal Ideals}{
    Let $R$ be a commutative ring and $I\subseteq R$ be a proper ideal. Then we have bijections between the following sets:
    \begin{align*}
        \left\{\text{ideals of }R\text{ containing }I\right\}&\longleftrightarrow\left\{\text{ideals of }R/I\right\}\\
        J&\longmapsto J/I
    \end{align*}
    The ideal $J\supseteq I$ is radical, prime, or maximal if and only if $J/I$ is radical, prime, or maximal respectively.
}

\subsection{Prime Elements}
\dfn{Divisibility}{
    Let $R$ be a commutative ring and $a,b\in R$. We say $a$ \textbf{divides} $b$ if there exists $c\in R$ such that $b=ac$, denoted by $a\mid b$. If $a\mid b$. $a$ is called a \textbf{divisor} of $b$, and $b$ is called a \textbf{multiple} of $a$.
}

\prop{}{
    Let $R$ be a commutative ring.
    \begin{enumerate}[(i)]
        \item $a \mid b\iff(b) \subseteq (a)$.
        \item $u\in R^\times \iff (u) = R  \iff \forall r\in R,\,u\mid r$.
    \end{enumerate}
}
\dfn{Prime Element}{
    Let $R$ be a commutative ring. An element $a\in R$ is called \textbf{prime} if
    \begin{enumerate}[(i)]
        \item $a\ne 0$.
        \item $a\notin R^\times$, i.e. $a$ is not a unit.
        \item $a\mid bc\implies a\mid b\text{ or }a\mid c$.
    \end{enumerate}
}


\prop{Prime Element and Prime Ideal}{
    Suppose $R$ is a commutative ring and $a\in R$. Then
    \[
        a\text{ is prime }\iff (a)\text{ is a nonzero prime ideal}.
    \]
}
\proof{
    \begin{align*}
        a\text{ is prime }\iff &a\ne 0\text{ and }a\notin R^\times\text{ and }a\mid bc\implies a\mid b\text{ or }a\mid c\\
        \iff &(a)\ne 0\text{ and }(a)\ne R^\times\text{ and }bc\in (a)\implies b\in (a)\text{ or }c\in (a)\\
        \iff &(a)\text{ is a nonzero prime ideal}.
    \end{align*}
}



\section{Integral Domain}
\dfn{Associate}{
    Let $R$ be an integal domain. Two elements $a,b\in R$ are called \textbf{associates} if one of the following equivalent conditions holds:
    \begin{enumerate}[(i)]
        \item $a=ub$ for some $u\in R^\times$.
        \item $a\mid b$ and $b\mid a$, i.e. $(a)=(b)$.
    \end{enumerate}
}
If $R$ is a general commutative ring, then we only have the implication $(\mathrm i)\implies (\mathrm{ii})$. The converse is not true in general. For example, in $\mathbb{C}[x,y,z]/(x-xyz)$, $\overline{x}\mid \overline{xy}$ and $\overline{xy}\mid \overline{x}$, but there exists no unit $u$ such that $\overline{x}=u\overline{xy}$.

Associatedness can also be described in terms of the action of $R^\times$ on $R$ via multiplication: two elements of $R$ are associates if they are in the same $R^\times$-orbit.
\dfn{Irreducible Element}{
    Let $R$ be an integal domain. An element $a\in R$ is called \textbf{irreducible} if
    \begin{enumerate}[(i)]
        \item $a\notin R^\times$, i.e. $a$ is not a unit.
        \item $a=bc\implies b\in R^\times\text{ or }c\in R^\times$.
    \end{enumerate}    
}
0 is never an irreducible element.
\prop{Prime Element $\implies$ Irreducible Element in Integral Domain}{
    Let $R$ be an integal domain. Then every prime element in $R$ is irreducible.
}
\proof{
    Let $a\in R$ be a prime element. Suppose $a=bc$ for some $b,c\in R$. Then $a\mid bc$. Since $a$ is prime, there must be $a\mid b$ or $a\mid c$. Without loss of generality, we can assume $a\mid b$. Then $b=ad$ for some $d\in R$. Thus we have $$a=bc=adc\implies a(1-dc)=0\implies dc=1\implies c\in R^\times.$$ That implies $a$ is irreducible.
}
\prop{Prime Ideal Equivalent Definition}{
    Let $R$ be a commutative ring. An ideal $I\subseteq R$ is prime if and only if $R/I$ is an integal domain.
}

\section{Unique Factorization Domain}
\dfn{Unique Factorization Domain}{
    An integral domain $R$ is called a \textbf{unique factorization domain} (UFD) if
    \begin{enumerate}[(i)]
        \item every nonzero nonunit element of $R$ can be written as a product of irreducible elements of $R$.
        \item if $p_1\cdots p_n=q_1\cdots q_m$ for some irreducible elements $p_1,\cdots,p_n,q_1,\cdots,q_m\in R$, then $n=m$ and there exists a permutation $\sigma\in S_n$ such that $p_i$ is an associate of $q_{\sigma(i)}$ for all $i=1,\cdots,n$.
    \end{enumerate}
}
\prop{Irreducible Element $\iff$ Prime Element in UFD}{
    Let $R$ be a UFD. Then every irreducible element in $R$ is prime.
}
\proof{
    Let $a\in R$ be an irreducible element. Suppose $a\mid bc$ for some $b,c\in R$. Then $bc=ad$ for some $d\in R$. Since $R$ is a UFD, we can write $b=p_1\cdots p_n$ and $c=q_1\cdots q_m$ for some irreducible elements $p_1,\cdots,p_n,q_1,\cdots,q_m\in R$. Then we have $$ad=bc=p_1\cdots p_nq_1\cdots q_m.$$ Since $a$ is irreducible, $a$ must be an associate of one of the $p_i$'s or $q_j$'s. Without loss of generality, we can assume $a\sim p_1$. Then $a\mid b$. That implies $a$ is prime.
}

\section{Principal Ideal Domain}
\dfn{Principal Ideal Domain}{
    An integral domain $R$ is called a \textbf{principal ideal domain} (PID) if every ideal of $R$ is principal.
}

\prop{PID $\implies$ UFD}{
    Every PID is a UFD.
}

\prop{Prime Ideal $\iff$ Maximal Ideal in PID}{
    Let $R$ be a PID. Then every nonzero prime ideal in $R$ is maximal.
}
\proof{
    Let $I\subseteq R$ be a prime ideal. We only need to show $R/I$ is a field. Let $\overline{a}\in R/I$ be a nonzero element. Then $a\notin I$. Since $I$ is prime, $a$ is not a multiple of any prime element in $R$. Thus $a$ is irreducible. Since $R$ is a PID, $a$ is prime. Thus $\overline{a}$ is prime in $R/I$. Since $R/I$ is an integral domain, $\overline{a}$ is a maximal ideal in $R/I$. That implies $R/I$ is a field.
}

\section{Polynomial Ring}
\dfn{Polynomial Ring}{
    Let $R$ be a commutative ring. The \textbf{polynomial ring} in $n$ variables over $R$ is the ring $R[x_1,\cdots,x_n]$ defined as the set of all formal sums $$\sum_{\alpha\in\mathbb{N}^n}a_\alpha x^\alpha$$ where $a_\alpha\in R$ satisfies $a_\alpha=0$ for all but finitely many $\alpha\in\mathbb{N}^n$ and $x^\alpha=x_1^{\alpha_1}\cdots x_n^{\alpha_n}$ for $\alpha=(\alpha_1,\cdots,\alpha_n)\in\mathbb{N}^n$. The addition and multiplication are defined as follows: $$\sum_{\alpha\in\mathbb{N}^n}a_\alpha x^\alpha+\sum_{\alpha\in\mathbb{N}^n}b_\alpha x^\alpha=\sum_{\alpha\in\mathbb{N}^n}(a_\alpha+b_\alpha)x^\alpha$$ and $$\left(\sum_{\alpha\in\mathbb{N}^n}a_\alpha x^\alpha\right)\left(\sum_{\beta\in\mathbb{N}^n}b_\beta x^\beta\right)=\sum_{\gamma\in\mathbb{N}^n}\left(\sum_{\alpha+\beta=\gamma}a_\alpha b_\beta\right)x^\gamma.$$
}

\prop{Properties of Polynomial Ring}{
    Let $R$ be a commutative ring.
    \begin{enumerate}
        \item If $R$ is a UFD, then $R[x_1,\cdots.x_n]$ is a UFD.
        \item $R$ is a field $\iff$ $R[x]$ is a PID $\iff$ $R[x]$ is an Euclidean domain.
    \end{enumerate}
}

\section{Construction}
\dfn{Multuplicative Subset}{
    Let $R$ be a commutative ring. A subset $S\subseteq R$ is called \textbf{multiplicative} if $S$ is monoid under the multiplication of $R$, i.e.
    \begin{enumerate}[(i)]
        \item $1\in S$.
        \item $a,b\in S\implies ab\in S$.
    \end{enumerate}
}


\dfn{Localization of a Ring}{
    Let $R$ be a commutative ring and $S\subseteq R$ be a multiplicative subset. The \textbf{localization} of $R$ at $S$ is the ring $S^{-1}R$ defined as the set of equivalence classes of the relation $\sim$ on $R\times S$ defined by $$(a,s)\sim (b,t)\iff \exists u\in S\text{ such that }u(at-bs)=0.$$
    The equivalence class of $(a,s)$ is denoted by $\frac{a}{s}$. The addition and multiplication on $S^{-1}R$ are defined as follows:
    \begin{align*}
        \frac{a}{s}+\frac{b}{t}&=\frac{at+bs}{st}\\
        \frac{a}{s}\cdot\frac{b}{t}&=\frac{ab}{st}
    \end{align*}
    The addition identify is $\frac{0}{1}$ and the multiplication identity is $\frac{1}{1}$.
}

\prop{Universal Property of Localization}{
    Let $R$ be a commutative ring and $S\subseteq R$ be a multiplicative subset. The ring homomorphism
    \begin{align*}
        \varphi:R&\longrightarrow S^{-1}R\\
         r&\longmapsto \frac{r}{1}
    \end{align*}
    satisfies the following universal property: for any ring homomorphism $\psi:R\to T$ such that $\psi(S)\subseteq T^\times$, there exists a unique ring homomorphism 
    \begin{align*}
        \psi':S^{-1}R&\longrightarrow T\\
        \frac{a}{s}&\longmapsto \psi(a)(\psi(s))^{-1}
    \end{align*}
    such that the following diagram commutes
    \begin{center}
        \begin{tikzcd}[ampersand replacement=\&]
         
            S^{-1}R\arrow[rr, "\psi'", dashed]\&\& T \&  \\             
            \&R \arrow[ru, "\psi"'] \arrow[lu, "\varphi"] \&                          
        \end{tikzcd}
    \end{center}
}
\proof{
    First let's check $\psi'$ is well-defined. Suppose $\dfrac{a}{s}=\dfrac{b}{t}$. Then there exists $u\in S$ such that $u(at-bs)=0$. Since $\psi$ is a ring homomorphism, we have 
    $$
    0=\psi(u(at-bs))=\psi(u)\left(\psi(a)\psi(t)-\psi(b)\psi(s)\right).
    $$
    Since $u\in S$ and $\psi(S)\subseteq T^\times$, we have $\psi(u)\in T^\times$. Thus
    $$
    \psi'\left(\frac{a}{s}\right)=\psi(a)(\psi(t))^{-1}=\psi(b)(\psi(s))^{-1}= \psi'\left(\frac{b}{t}\right).
    $$ 
    That implies $\psi'$ is well-defined. It is easy to check $\psi'$ is a ring homomorphism
    \[
        \psi'\left(\frac{a}{s}+\frac{b}{t}\right)=\psi'\left(\frac{at+bs}{st}\right)=\psi(at+bs)(\psi(st))^{-1}=\psi(a)(\psi(s))^{-1}+\psi(b)(\psi(t))^{-1}=\psi'\left(\frac{a}{s}\right)+\psi'\left(\frac{b}{t}\right).  
    \]
    The multiplication is similar. The diagram commutes since
    \[
        \psi'\circ\varphi(r)=\psi'\left(\frac{r}{1}\right)=\psi(r)(\psi(1))^{-1}=\psi(r).
    \]
    Now we show $\psi'$ is unique. Suppose there exists another ring homomorphism $\psi'':S^{-1}R\to T$ such that the diagram commutes. Then for any $\frac{a}{s}\in S^{-1}R$, we have 
    \[
        \psi''\left(\frac{a}{s}\right) = \psi''\left(\frac{a}{1}\frac{1}{s}\right) = \psi''\left(\frac{a}{1}\right)\psi''\left(\frac{1}{s}\right) = \psi''\left(\frac{a}{1}\right)\left(\psi''\left(\frac{s}{1}\right)\right)^{-1} = \psi\left(a\right)(\psi(s))^{-1} =    \psi'\left(\frac{a}{s}\right).
    \]
    That implies $\psi''=\psi'$. Thus $\psi'$ is unique.
}

Localization is the most economical way to make a multiplicative subset invertible.

\prop{}{
    Let $R$ be a commutative ring and $S \subseteq R$ be a multiplicative subset. The category of $S^{-1} R$ modules is equivalent to the category of $R$-modules $M$ with the property that every $s \in S$ acts as an automorphism on $M$. The following functor $F$ gives a equivalence of categories: 
    \[
        \begin{tikzcd}[ampersand replacement=\&]
            S^{-1} R\text{-}\mathsf{Mod}\&[-25pt]\&[+10pt]\&[-30pt] R\text{-}\mathsf{Mod}\text{ where }S\text{ act as automorphisms}\&[-30pt]\&[-30pt] \\ [-15pt] 
            M  \arrow[dd, "f"{name=L, left}] 
            \&[-25pt] \& [+10pt] 
            \& [-30pt] M\arrow[dd, "f"{name=R}] \&[-30pt]\\ [-10pt] 
            \&  \phantom{.}\arrow[r, "F", squigarrow]\&\phantom{.}  \&   \\[-10pt] 
            N \& \& \&  N\&
        \end{tikzcd}
        \]
}
\proof{
    Assume $S$ is a multiplicative subset of communitative ring $R$ and the localization map is $\varphi:R\to S^{-1}R$. Then $R$ can acts on $S^{-1}R$-module $M$ through 
    \[
        R\xrightarrow{\varphi}S^{-1}R\xrightarrow{\sigma_M'}\mathrm{End}_{\mathsf{Ab}}(M),
    \]
    which enables us to regard $M$ as an $R$-module. Furthermore, since 
    \[
        \sigma_M'(\varphi(S))\subseteq \sigma_M' \left(\left(S^{-1}R\right)^\times\right)\subseteq \left(\mathrm{End}_{\mathsf{Ab}}(M)\right)^\times=\mathrm{Aut}_{\mathsf{Ab}}(M),
    \]
    every $s \in S$ acts as an automorphism on $M$.\\
    Conversely, if $M$ is an $R$-module such that every $s\in S$ acts as an automorphism on $M$, i.e. $\sigma_M:R\to\mathrm{End}_{\mathsf{Ab}}(M)$ satisfies $\sigma_M(S)\subseteq \mathrm{Aut}_{\mathsf{Ab}}(M)$, then by unversal property
    \begin{center}
        \begin{tikzcd}[ampersand replacement=\&]
         
            S^{-1}R\arrow[rr, "\sigma_M'", dashed]\&\& \mathrm{End}_{\mathsf{Ab}}(M) \&  \\             
            \&R \arrow[ru, "\sigma_M"'] \arrow[lu, "\varphi"] \&                          
        \end{tikzcd}
    \end{center}
    we can define a $S^{-1}R$-module structure on $M$ by lifting $\sigma_M$ to $\sigma_M'$. It is easy to check that these two functors are quasi-inverse to each other.
}
\prop{Properties of Localization of Rings}{
    Let $R$ be a commutative ring and $S\subseteq R$ be a multiplicative subset. Then
    \begin{enumerate}[(i)]
        \item $S^{-1}R=0$ if and only if $0\in S$.
        \item If $0\notin S$, then $\frac{a}{s}$ is invertible in $S^{-1}R$ if and only if there exists $r\in R$ such that $ra\in S$.
        \item If $0\notin S$, the localization map $\varphi:R\to S^{-1}R$ is injective if and only if $S$ contains no zero divisors.
        \item If $R$ is an integral domain, then $S^{-1}R$ is also an integral domain.
    \end{enumerate}
}
\proof{
    \begin{enumerate}[(i)]
        \item \[
            S^{-1}R=0\iff \frac{1}{1}=\frac{0}{1}\iff \exists s\in S\text{ such that }s\cdot 1=0\iff 0\in S.
            \]
        \item Suppose $0\notin S$. If $\frac{a}{s}$ is invertible in $S^{-1}R$, then there exists $\frac{b}{t}\in S^{-1}R$ such that $\frac{a}{s}\cdot\frac{b}{t}=\frac{1}{1}$, which implies there exists $u\in S$ such that $u(ab-st)=0$. Let $r=ub\in R$ and then we see $ra=ust\in S$. Conversely, suppose there exists $r\in R$ such that $ra\in S$. Then $\frac{a}{s}\cdot\frac{rs}{ra}=\frac{1}{1}$, which implies $\frac{a}{s}$ is invertible.
        \item Suppose $0\notin S$. Given the localization map $\varphi:R\to S^{-1}R$, we have
        \[
            \varphi(r)=0\iff \frac{r}{1}=\frac{0}{1}\iff \exists s\in S\text{ such that }s\cdot r=0.
        \]
        Thus 
        $$
        \varphi\text{ is injective}\iff \ker \varphi=\{0\}\iff \forall s\in S,\forall r\in R-\{0\},sr\ne 0\iff S\text{ contains no zero divisors}.
        $$
    \end{enumerate}
}

\dfn{Total Ring of Fractions}{
    Let $R$ be a commutative ring. Then $S=\left\{r\in R\mid r\text{ is not a zero divisor}\right\}$ is a multiplicative subset. The \textbf{total ring of fractions} of $R$ is the localization $S^{-1}R$, denoted by $\mathrm{Frac}(R)$. The localization map $\varphi:R\to \mathrm{Frac}(R)$ is an injective ring homomorphism.
}

\dfn{Field of Fractions}{
    If $R$ be an integral domain, the total ring of fractions $\mathrm{Frac}(R)$ is a field, call the \textbf{field of fractions} of $R$.
}

\dfn{Localization of an Ideal}{
    Let $R$ be a commutative ring, $S$ be a multiplicative set in $R$, and $I$ be an ideal of $R$. If we regard $I$ as a $R$-module, the \textbf{localization of the ideal} $I$ by $S$, denoted $S^{-1}I$, is the localization of the module $I$ by $S$. That is,
    \[
        S^{-1}I=\left\{\frac{a}{s}\midv a\in I, s\in S\right\}.
    \]
    $S^{-1}I$ is a $S^{-1}R$-submodule of $S^{-1}R$. Suppose the localization map is $\varphi:R\to S^{-1}R$, $S^{-1}I$ can also defined as the ideal generated by $\varphi(I)$ in $S^{-1}R$
    \[
        S^{-1}I=\langle \varphi(I)\rangle=\left\{\frac{r}{s}\frac{a}{1}\midv a\in I, \frac{r}{s}\in S^{-1}R\right\}.
    \]
}
\prop{Properties of localization of Ideals}{
    Let $R$ be a commutative ring, $S$ be a multiplicative set in $R$, and $0\notin S$. Suppose the localization map is $\varphi:R\to S^{-1}R$. Then we have maps between the sets of ideals of $R$ and $S^{-1}R$:
    \begin{align*}
        \mathcal{I}(R)=\left\{\text{ideals of }R\right\}\xrightleftarrows[\varphi^{-1}]{\quad S^{-1}\quad}
         \left\{\text{ideals of }S^{-1}R\right\}=\mathcal{I}(S^{-1}R)
    \end{align*}
    \begin{enumerate}[(i)]
        \item $S^{-1}\circ \varphi^{-1}=\mathrm{id}_{\mathcal{I}(S^{-1}R)}$. As a result, $S^{-1}$ is surjective and $\varphi^{-1}$ is injective.
        \item For any ideal $J$ of $S^{-1}R$, there exists an ideal $I$ of $R$ such that $S^{-1}I=J$. 
        \item If $I$ is a ideal of $R$, then $S^{-1}I=S^{-1}R\iff I\cap S\ne\varnothing$.
        \item $\varphi$ induces a bijection between the set of prime ideals of $R$ that do not intersect $S$ and the set of prime ideals of $S^{-1}R$. That is, the following restriction of $S^{-1}$ and $\varphi^{-1}$ are bijections:
        \begin{align*}
            \{I \in \operatorname{Spec} R: I \cap S=\varnothing\} \xrightleftarrows[\varphi^{-1}]{\quad S^{-1}\quad}
           \spec S^{-1}R
        \end{align*}
    \end{enumerate}
}
\proof{
    \begin{enumerate}[(i)]
        \item Let $J$ be an ideal of $S^{-1}R$. We have
        \[
            S^{-1}\varphi^{-1}(J)=\left\{\frac{x}{s}\midv x\in \varphi^{-1}(J),s\in S\right\}=\left\{\frac{x}{s}\midv \frac{x}{1}\in J,s\in S\right\}=\left\{\frac{1}{s}\frac{x}{1}\midv \frac{x}{1}\in J,s\in S\right\}=J.
        \]
        \item It is a direct consequence of the surjectivity of $S^{-1}$.
        \item Let $I$ be an ideal of $R$. We have
        \[
            S^{-1}I=S^{-1}R\iff \frac{1}{1} \in S^{-1}I \iff \exists t,s\in S,a\in I, t(a-s)=0\iff ta=ts\in I\cap S\ne\varnothing \iff I\cap S\ne\varnothing.
        \]
    \end{enumerate}
}

\ex{Localization at a Prime Ideal}{
    Let $R$ be a commutative ring and $\mathfrak{p}$ be a prime ideal of $R$. Then $S=R-\mathfrak{p}$ is a multiplicative set. The localization $S^{-1}R$ is called the \textbf{localization of $R$ at $\mathfrak{p}$}, denoted by $R_\mathfrak{p}$. $R_\mathfrak{p}$ is a local ring with unique maximal ideal 
    \[
    \mathfrak{p}R_\mathfrak{p}=S^{-1}\mathfrak{p}=\left\{\frac{x}{s}\midv x\in \mathfrak{p}, s\in R-\mathfrak{p}\right\}.
    \]
    And we have field isomorphism $R_\mathfrak{p}/\mathfrak{p}R_\mathfrak{p}\cong \mathrm{Frac}(R/\mathfrak{p})$.
}
\proof{
    Since for any ideal $I\in  \{I \in \operatorname{Spec} R: I \cap S=\varnothing\}$, we have
    \[
        I\subseteq \mathfrak{p}\implies S^{-1}I\subseteq  S^{-1}\mathfrak{p}.
    \]
    Thus we see $S^{-1}\mathfrak{p}$ is the unique maximal ideal of $S^{-1}R$.
}

\ex{}{
    Let $R$ be a commutative ring and $f\in R$. Let $S=\{1,f,f^2,\cdots\}$ be the monoid generated by $f$. Then $S$ is a multiplicative set. The localization $S^{-1}R$ is called the \textbf{localization of $R$ at $f$}, denoted by $R_f$. $R_f=0$ if and only if $f$ is nilpotent.
}
\proof{
    $R_f=0\iff 0\in S\iff \exists n\in\mathbb{Z}_{\ge0},\;f^n=0$.
}

\chapter{Module}
\section{Basic Concepts}
\dfn{Module}{
    Let $R$ be a ring. An left \textbf{$R$-module} is an abelian group $M$ with a binary operation $R\times M\to M$ such that
    \begin{enumerate}[(i)]
        \item $r(m+n)=rm+rn$ for all $r\in R$ and $m, n\in M$.
        \item $(r+s)m=rm+sm$ for all $r, s\in R$ and $m\in M$.
        \item $(rs)m=r(sm)$ for all $r, s\in R$ and $m\in M$.
        \item $1m=m$ for all $m\in M$.
    \end{enumerate}
    If $R$ is a commutative ring, then $M$ is called a \textbf{commutative $R$-module}.
}

\dfn{Homomorphism of $R$-modules}{
    Let $R$ be a ring and $M, N$ be $R$-modules. A map $f:M\to N$ is called an \textbf{$R$-module homomorphism} if
    \begin{enumerate}[(i)]
        \item $f(m+n)=f(m)+f(n)$ for all $m, n\in M$.
        \item $f(rm)=rf(m)$ for all $r\in R$ and $m\in M$.
    \end{enumerate}
    If $f$ is bijective, then $f$ is called an \textbf{$R$-module isomorphism}. If $M=N$, then $f$ is called an \textbf{$R$-module endomorphism}. If $f$ is bijective, then $f$ is called an \textbf{$R$-module automorphism}. Another name for a homomorphism of $R$-modules is an  \textbf{$R$-linear map}.
}

\prop{Ring Action on an Abelian Group}{
From the perspective of representation theory, a module is a ring action on an abelian group. To be more precise, a ring $R$ can be regarded as an $\mathsf{Ab}$-enriched category with only one object, called the delooping of $R$, denoted $\mathsf{B}R$. $\mathsf{Ab}$ itself is an $\mathsf{Ab}$-enriched category. Thus a left $R$-module $M$ is a functor between $\mathsf{Ab}$-enriched categories $\mathcal{M}:\mathsf{B}R\to \mathsf{Ab}$. 
\[
    \begin{tikzcd}[ampersand replacement=\&]
        \mathsf{B}R\&[-25pt]\&[+10pt]\&[-30pt] \mathsf{Ab}\&[-30pt]\&[-30pt] \\ [-15pt] 
        *  \arrow[dd, "r\in R"{name=L, left}] 
        \&[-25pt] \& [+10pt] 
        \& [-30pt] M\arrow[dd, "r_M\in \mathrm{End}_{\mathsf{Ab}}(M)"{name=R}] \&[-30pt]\\ [-10pt] 
        \&  \phantom{.}\arrow[r, "\mathcal{M}", squigarrow]\&\phantom{.}  \&   \\[-10pt] 
        * \& \& \&  M\&
    \end{tikzcd}
\]
As a map between objects, $\mathcal{M}$ specifies an abelian group $M$. and a ring homomorphism $R\to \mathrm{End}_{\mathsf{Ab}}(M)$, which is the ring action of $R$ on $M$. As a map between morphisms, $\mathcal{M}$ specifies a ring homomorphism $R\to \mathrm{End}_{\mathsf{Ab}}(M)$, which is $\mathbb{Z}$-bilinear, i.e. $r(m+n)=rm+rn$ for all $r\in R$ and $m, n\in M$. Define the ring representation category 
\[
\mathrm{Rep}_{\mathsf{Ab}}(R)=\mathrm{Fun}_{\mathsf{Ab}}(\mathsf{B}R, \mathsf{Ab})    
\]
to be the category of all functors between $\mathsf{Ab}$-enriched categories $\mathsf{B}R$ and $\mathsf{Ab}$. Then we have category isomorphism 
\[
    \mathrm{Rep}(R)\cong R\text{-}\mathsf{Mod}
\]
}
\prop{Ring homomorphism $R\to S$ induces functor $S\text{-}\mathsf{Mod}\to R\text{-}\mathsf{Mod}$}{
    Let $R$ and $S$ be rings with a ring homomorphism $f: R\to S$. Then every $S$-module $M$ is an $R$-module by defining $rm = f(r)m$, or equivalently through $R\to S\to \mathrm{End}_{\mathsf{Ab}}(M)$. This defines a functor $F: S\text{-}\mathsf{Mod}\to R\text{-}\mathsf{Mod}$, which is identify map on objects and morphisms.
    \[
        \begin{tikzcd}[ampersand replacement=\&]
            S\text{-}\mathsf{Mod}\&[-25pt]\&[+10pt]\&[-30pt] R\text{-}\mathsf{Mod}\&[-30pt]\&[-30pt] \\ [-15pt] 
            M  \arrow[dd, "g"{name=L, left}] 
            \&[-25pt] \& [+10pt] 
            \& [-30pt] M\arrow[dd, "g"{name=R}] \&[-30pt]\\ [-10pt] 
            \&  \phantom{.}\arrow[r, "F", squigarrow]\&\phantom{.}  \&   \\[-10pt] 
            N \& \& \&  N\&
        \end{tikzcd}
        \]  
}
In particular, homomorphism $R\to S$ makes $S$ an $R$-module.

\section{Construction}
\dfn{Localization of a Module}{
    Let $R$ be a commutative ring, $S$ be a multiplicative set in $R$, and $M$ be an $R$-module. The \textbf{localization of the module} $M$ by $S$, denoted $S^{-1}M$, is an $S^{-1}R$-module that is constructed exactly as the localization of $R$, except that the numerators of the fractions belong to $M$. That is, as a set, it consists of equivalence classes, denoted $\frac{m}{s}$, of pairs $(m, s)$, where $m\in M$ and $s\in S$, and two pairs $(m, s)$ and $(n, t)$ are equivalent if there is an element $u$ in $S$ such that
    \[u(sn-tm)=0.\]
    Addition and scalar multiplication are defined as for usual fractions (in the following formula, $r\in R$, $s,t\in S$, and $m,n\in M$):
    \[\frac{m}{s} + \frac{n}{t} = \frac{tm+sn}{st},\]
    \[\frac{r}{s} \frac{m}{t} = \frac{r m}{st}.\]
}
\prop{Universal Property of Localization}{
    Let $R$ be a commutative ring and $S\subseteq R$ be a multiplicative subset, an $M$ be an $R$-module. The $R$-linear map
    \begin{align*}
        \varphi:M&\longrightarrow S^{-1}M\\
         m&\longmapsto \frac{m}{1}
    \end{align*}
    satisfies the following universal property: for any $R$-linear map $\psi:M\to N$ such that $S$ act as automorphisms on $N$ (i.e. the induced ring homomorphism $\sigma_{N}:R\to\mathrm{End}_{\mathsf{Ab}}(N)$ satisfies $\sigma_{N}(S)\subseteq \mathrm{Aut}_{\mathsf{Ab}}(N)$), there exists a unique $R$-linear map
    \begin{align*}
        \psi':S^{-1}M&\longrightarrow N\\
        \frac{m}{s}&\longmapsto s^{-1}\psi(m)=\sigma_{N}(s)^{-1}(\psi(m))
    \end{align*}
    such that the following diagram commutes
    \begin{center}
        \begin{tikzcd}[ampersand replacement=\&]
         
            S^{-1}M\arrow[rr, "\psi'", dashed]\&\& N \&  \\             
            \&M \arrow[ru, "\psi"'] \arrow[lu, "\varphi"] \&                          
        \end{tikzcd}
    \end{center}
}
\prop{Localization is a Left Adjoint Functor}{ 
    Let $R$ be a commutative ring, $S$ be a multiplicative set in $R$, and $M$ be an $R$-module. Define the localization functor as follows
    \[
        \begin{tikzcd}[ampersand replacement=\&]
           R\text{-}\mathsf{Mod}\&[-25pt]\&[+10pt]\&[-30pt] S^{-1}R\text{-}\mathsf{Mod}\&[-30pt]\&[-30pt] \\ [-15pt] 
            M  \arrow[dd, "f"{name=L, left}] 
            \&[-25pt] \& [+10pt] 
            \& [-30pt] S^{-1}M\arrow[dd, "S^{-1}(f)"{name=R}] \&[-30pt]\ni
            \&[-30pt]\frac{m}{s}\arrow[dd,mapsto]\&[-30pt]\\ [-10pt] 
            \&  \phantom{.}\arrow[r, "S^{-1}", squigarrow]\&\phantom{.}  \&   \\[-10pt] 
            N \& \& \&  S^{-1}N\&[-30pt]\ni
            \&[-30pt]\frac{f(m)}{s}
        \end{tikzcd}
        \]  
        where $S^{-1}(f)$ is defiend as the composition $S^{-1}M\xrightarrow{f'} N\to S^{-1}N$. \\
        Let $F: S^{-1}R\text{-}\mathsf{Mod}\to R\text{-}\mathsf{Mod}$ be the functor that regards $S^{-1}R$-modules as $R$-modules. Then we have a pair of adjoint functors $S^{-1}\dashv F: R\text{-}\mathsf{Mod}\leftrightarrows S^{-1}R\text{-}\mathsf{Mod}$ and natural isomorphism
        \[
        \mathrm{Hom}_{S^{-1}R\text{-}\mathsf{Mod}}(S^{-1}M, N)\cong \mathrm{Hom}_{R\text{-}\mathsf{Mod}}(M, F(N)).    
        \]
}
\prop{Localization is an Exact Functor}{
    Let $R$ be a commutative ring, $S$ be a multiplicative set in $R$. If $L\xrightarrow {u} M\xrightarrow {v} N$ is an exact sequence of $R$-module, then $S^{-1}L\xrightarrow {S^{-1}(u)} S^{-1}M\xrightarrow {S^{-1}(v)} S^{-1}N$ is an exact sequence of $S^{-1}R$-module.
}
\proof{
    Suppose $\frac{m}{s}\in \ker S^{-1}(v) $. Then we have
    \[
        S^{-1}(v)\left(\frac{m}{s}\right)=\frac{v(m)}{s}=\frac{0}{1},
    \]
    which imples that there exists $t\in S$ such that $tv(m)=v(tm)=0$. Thus we have $tm\in \ker v$. By exactness, there exists $l\in L$ such that $u(l)=tm$. Since 
    \[
        S^{-1}(u)\left(\frac{l}{ts}\right)=\frac{u(l)}{ts}=\frac{tm}{ts}=\frac{m}{s},
    \]
    we see that $\frac{m}{s}\in \operatorname{im}S^{-1}(u)$, which means $\operatorname{im}S^{-1}(u)=\ker S^{-1}(v)$. Hence $S^{-1}$ is exact.
}
\prop{Localization Respects Quotients}{
    Let $M$ be an $R$-module and $N$ be a submodule of $M$. Then we have an isomorphism $S^{-1}(M/N)\cong (S^{-1}M)/(S^{-1}N)$.
}
\proof{
    From the exact sequence
    \[
        0\longrightarrow N\longrightarrow M\longrightarrow M/N\longrightarrow 0,
    \]
    we have the exact sequence
    \[
        0\longrightarrow S^{-1}N\longrightarrow S^{-1}M\longrightarrow S^{-1}(M/N)\longrightarrow 0.
    \]
}
\prop{Localization as colimit}{
    Let $R$ be a commutative ring, $S$ be a multiplicative set in $R$, and $M$ be an $R$-module. Then we have an isomorphism
    \[
        S^{-1}M\cong \varinjlim_{f\in S}M_f,
    \]
    where $M_f$ is the localization of $M$ by the multiplicative set $S_f=\{f^n\mid n\in \mathbb{Z}_{\ge0}\}$. 
    
    Formally, $S$ can be endowed with a preorder relation: $f\mid g$ if and only if $fh=g$ for some $h\in S$, which makes $S$ a thin category $\mathsf{S}$. Then we can define a functor $G:\mathsf{S}\to R\text{-}\mathsf{Mod}$
    \[
        \begin{tikzcd}[ampersand replacement=\&]
            \mathsf{S}\&[-25pt]\&[+10pt]\&[-30pt] R\text{-}\mathsf{Mod}\&[-30pt]\&[-30pt] \\ [-15pt] 
            f  \arrow[dd, ""{name=L, left}] 
            \&[-25pt] \& [+10pt] 
            \& [-30pt]M_f\arrow[dd, "\varphi_g'"{name=R}] \&[-30pt]\ni
            \&[-30pt]\frac{m}{f^n}\arrow[dd,mapsto]\&[-30pt]\\ [-10pt] 
            \&  \phantom{.}\arrow[r, "G", squigarrow]\&\phantom{.}  \&   \\[-10pt] 
            g \&\hspace{-3pt}=fh \& \&  M_g\&[-30pt]\ni
            \&[-30pt]\frac{mh^n}{g^n}
        \end{tikzcd}
        \]  
        where $\varphi_g'$ is given by the following  universal property
        \begin{center}
            \begin{tikzcd}[ampersand replacement=\&]
             
                M_f\arrow[rr, "\varphi_g'", dashed]\&\& M_g \&  \\             
                \&M \arrow[ru, "\varphi_g"'] \arrow[lu, "\varphi_f"] \&                          
            \end{tikzcd}
        \end{center}
        And we have
        \[
            S^{-1}M\cong \varinjlim G  
        \]
}
\proof{
First let's show that the $\varphi'_g$ induced by universal property can be writen as $\varphi'_g:\frac{m}{f^n}\mapsto\frac{mh^n}{g^n}$. Suppose $R$ acts on $M_g$ through
\[
    \sigma_{M_g}:R\xrightarrow{}S^{-1}_gR\xrightarrow{\sigma_{M_g'}}\mathrm{End}_{\mathsf{Ab}}(M_g),
\]
Then we can check for any $f^n \in S_f$,
\[
    \sigma_{M_g}(f^n)\sigma_{M_g'}\left(\frac{h^n}{g^n}\right)=\sigma_{M_g'}\left(\frac{f^nh^n}{g^n}\right)=\sigma_{M_g'}\left(1\right)=1\implies \sigma_{M_g}(f^n)\in \mathrm{Aut}_{\mathsf{Ab}}(M_g),
\]
which means $\sigma_{M_g}(S_f)\subseteq  \mathrm{Aut}_{\mathsf{Ab}}(M_g)$. Thus by universal property of $M_f$, we have
\[
    \varphi'_g\left(\frac{m}{f^n}\right)=\sigma_{M_g}(f^n)^{-1}(m)=\sigma_{M_g'}\left(\frac{h^n}{g^n}\right)(m)=\frac{mh^n}{g^n}.
\]
In a similar way, we can check that $S_f$ can act on $S^{-1}M$ as automorphisms and induce $\psi_f$ by the following universal property
\begin{center}
    \begin{tikzcd}[ampersand replacement=\&]
     
        M_f\arrow[rr, "\psi_f", dashed]\&\& S^{-1}M\&  \\             
        \&M \arrow[ru, "\varphi_S"'] \arrow[lu, "\varphi_f"] \&                          
    \end{tikzcd}
\end{center}
And we are going to show that $\left(\psi_f:M_f\to S^{-1}M\right)_{f\in S}$ is the colimit of $G$. 
\[
\begin{tikzcd}
        & N                                 &                                                             \\[+15pt]
        & S^{-1}M \arrow[u, "\nu"', dashed] &                                                             \\[+10pt]
M_f \arrow[rr, "\varphi_g'"] \arrow[ru, "\psi_f"] \arrow[ruu, "\mu_f", bend left] &                                   & M_g \arrow[lu, "\psi_g"'] \arrow[luu, "\mu_g"', bend right]
\end{tikzcd}
\]
We can prove
\[
   \psi_f=\psi_g\circ \varphi_g'
\]
by checking
\[
    \left(\psi_g\circ \varphi_g'\right)\circ \varphi_f=\psi_g\circ\varphi_g =\varphi_S=\psi_f\circ \varphi_f
\]
and utilizing the uniqueness of the universal property. \\
Given any $\left(\mu_f:M_f\to S^{-1}M\right)_{f\in S}$ such that $\mu_f=\mu_g\circ \varphi_g'$, note that $\mu_f\circ \varphi_f=\mu_f\circ \mu_g\circ \varphi_g'=\mu_g\circ \varphi_g$. Thus we can define $\nu$ to be the unique map such that $\nu\circ \varphi_S=\mu_f\circ \varphi_f$.
\[
\begin{tikzcd}
    & N                                                  &                                    \\[+12pt]
M_f \arrow[rr, "\psi_f"] \arrow[ru, "\mu_f"] &                                                    & S^{-1}M \arrow[lu, "\nu"', dashed] \\[+12pt]
    & M \arrow[lu, "\varphi_f"] \arrow[ru, "\varphi_S"'] &                                   
\end{tikzcd}
\]
Hence we have 
\[
\left(\nu\circ \psi_f\right)\circ \varphi_f=\nu\circ\varphi_S =\mu_f\circ \varphi_f.
\]
By the uniqueness of the universal property of $M_f$, we have $\mu_f=\nu\circ \psi_f$. If there exists another $\nu'$ such that $\mu_f=\nu'\circ \psi_f$, there must be $\nu'\circ \varphi_S=\nu'\circ \psi_f \circ\varphi_f=\mu_f \circ\varphi_f=\nu\circ \varphi_S$. The uniqueness of such $\nu$ forces $\nu=\nu'$.\\
Therefore we show that $ S^{-1}M\cong \varinjlim_{f\in S}M_f$.
}

\chapter{Associate Algebra}

\section{Basic Properties}
\dfn{Associative $R$-algebra}{
    Let $R$ be a commutative ring. An \textbf{associative $R$-algebra} is a ring $A$ together with a ring homomorphism $\varphi:R\to Z(A)$, which makes $A$ an $R$-module by defining 
    $$
    r\cdot a=\varphi(r)a
    $$
    for all $r\in R$ and $a\in A$.
    We usually call associative $R$-algebra as $R$-algebra for short.
}
We can check that 
\[
  r\cdot (ab)=\varphi(r)ab=(r\cdot a)b=a\varphi(r)b =a(r\cdot b).
\]



% \begin{Theorem}{Title}{label}
%     This is the statement of the theorem.
% \end{Theorem}

% \begin{corollary}{Title}{label}
%     This is the statement of the corollary.
% \end{corollary}

% \begin{claim}{Title}{label}
%     This is the statement of the claim.
% \end{claim}

% \begin{Example}{Title}{label}
%     This is an example.
% \end{Example}

% \begin{Definition}{Title}{label}
%     This is a definition.
% \end{Definition}
\end{document}