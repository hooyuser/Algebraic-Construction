
\chapter{Number Theory}
\section{$p$-adic Numbers}


\dfn{$p$-adic Integer}{
    Let $p$ be a prime number. The $p$-adic integer topological ring is defined as
    \[
    \mathbb{Z}_p=\varprojlim_{n}\mathbb{Z}/p^n\mathbb{Z}.    
    \]
    The universal property of $\mathbb{Z}_p$ is given by the following commutative diagram
    \[
        \begin{tikzcd}[ampersand replacement=\&, row sep=30pt]
            \&                                    \&                  \&                                    \&                  \& G \arrow[d, dashed] \arrow[rdd]                                                    \&                  \\
            \&                                    \&                  \&                                    \&                  \& \mathbb{Z}_p \arrow[d] \arrow[lld] \arrow[lllld] \arrow[llllld] \arrow[rd] \&                  \\
\mathbb{Z}/p\mathbb{Z} \& \mathbb{Z}/p^2\mathbb{Z} \arrow[l] \& \cdots \arrow[l] \& \mathbb{Z}/p^n\mathbb{Z} \arrow[l] \& \cdots \arrow[l] \& \mathbb{Z}/p^m\mathbb{Z} \arrow[l]                                                 \& \cdots \arrow[l]
\end{tikzcd}
\]

}
A $p$-adic integer can be represented as a sequence
$$
x=\left(x_1 \bmod p, x_2 \bmod p^2, x_3 \bmod p^3, \ldots\right)
$$
such that $x_{n+1}\equiv x_n \bmod p^n$ for all $n\in \mathbb{Z}_{\ge 1}$. Writing in base-$p$ form, $x_n \bmod p^n$ can be represented as 
\[
x_n =a_0+a_1p+\cdots+a_{n-1}p^{n-1}= \left(\,\overline{ a_{n-1}\cdots a_1a_0}\,\right)_p.    
\]
The condition $x_{n+1}\equiv x_n \bmod p^n$ means that $x_{n+1}$ and $x_n$ have the same last $n$ digits in base-$p$ form. Thus $p$-adic integer $x$ can be thought as a base-$p$ interger with infinite digits
\[
x =a_0+a_1p+\cdots+a_{n}p^{n}+\cdots= \left(\,\overline{\cdots a_{n}\cdots a_1a_0}\,\right)_p.
\]
The projection map $\pi_n:\mathbb{Z}_p\to \mathbb{Z}/p^n\mathbb{Z}$ is the  truncation map that truncates the last $n$ digits of $x$ in base-$p$ form.

\dfn{$p$-adic Valuation}{
    Let $p$ be a prime number. The \textbf{$p$-adic valuation} is defined as
    \begin{align*}
        v_p:\mathbb{Q}&\longrightarrow \mathbb{Z}\cup\{\infty\}\\
        \frac{a}{b}&\longmapsto \begin{cases}
            \infty & \text{if }a=0,\\
            \max\{n\in \mathbb{Z}\mid p^n\mid a\}-\max\{n\in \mathbb{Z}\mid p^n\mid b\} & \text{if }a\ne 0.
        \end{cases}
    \end{align*}

}

The \textbf{$p$-adic absolute value} is defined as
\begin{align*}
    \left|\,\cdot\,\right|_p:\mathbb{Q}&\longrightarrow \mathbb{R}_{\ge 0}\\
    x&\longmapsto \begin{cases}
        0 & \text{if }a=0,\\
        p^{-v_p(x)} & \text{if }x\ne 0.
    \end{cases}
\end{align*}

\dfn{Global Field}{
    A \textbf{global field} is a field isomorphic to one of the following:
    \begin{itemize}
        \item a \textbf{number field}: finite extension of $\mathbb{Q}$,
        \item a \textbf{function field} over a finite field $\mathbb{F}_q$: finite extension of $\mathbb{F}_q(t)$.
    \end{itemize}
}
\dfn{Local Field}{
    A \textbf{local field} is a field isomorphic to one of the following:
    \begin{itemize}
        \item (Character zero): $\mathbb{R}$, $\mathbb{C}$ or a finite extension of $\mathbb{Q}_p$,
        \item (Character $p>1$): the field of formal Laurent series  $\mathbb{F}_q((t))$, where $q=p^n$.
    \end{itemize}
}
\section{Dirichlet Charater}
\dfn{Euler's Totient Function}{
    The \textbf{Euler's totient function} is defined as
    \begin{align*}
        \varphi:\mathbb{N}&\longrightarrow \mathbb{N}\\
        n&\longmapsto \left|\left\{a\in \mathbb{N}\mid 1\le a\le n, (a,n)=1\right\}\right|.
    \end{align*}
}
\prop{Euler's Product Formula}{
    For any $n\in \mathbb{N}$, we have
    \[
        \varphi(n)=\sum_{k=1}^n  \mathds{1}_{(k,n)=1}=n\prod_{p\mid n}\left(1-\frac{1}{p}\right).
    \]
}
\prop{Properties of Euler's Totient Function}{
    For any $m,n\in \mathbb{N}$, we have
    \begin{enumerate}[(i)]
        \item $\varphi(mn)=\varphi(m)\varphi(n)$ if $(m,n)=1$.
        \item $\varphi(n)\mid n$.
        \item $\varphi(n)=n$ if and only if $n=1$.
        \item $\varphi(p^k)=p^k-p^{k-1}$ for any prime $p$ and $k\in \mathbb{N}$.
        \item $\varphi(n)\le n-\sqrt{n}$ for any $n\in \mathbb{N}$.
    \end{enumerate}
}

\dfn{Dirichlet Character}{
    Given any group homomorphism $\rho_m:(\mathbb{Z} / m \mathbb{Z})^{\times} \rightarrow \mathbb{C}^{\times}$, we can define a function $\chi:\mathbb{Z} \rightarrow \mathbb{C}$ by
    \begin{align*}
        \chi_m(a)=\left\{\begin{array}{lll}
            0 & \text { if }[a] \notin(\mathbb{Z} / m \mathbb{Z})^{\times} & \text {i.e. }(a, m)>1 \\
            \rho([a]) & \text { if }[a] \in(\mathbb{Z} / m \mathbb{Z})^{\times} & \text {i.e. }(a, m)=1
            \end{array}\right.
    \end{align*}
    Such function $\chi_m$ is called a \textbf{Dirichlet character modulo $m$}.
}
\dfn{Principal Dirichlet Character}{
    The \textbf{principal Dirichlet character modulo $m$} is the simplest Dirichlet character defined by
    \begin{align*}
        \chi_0\left(a\right)=\left\{\begin{array}{lll}
            0 & \text { if } a \neq 1 \\
            1 & \text { if } a=1
            \end{array}\right.
    \end{align*}
}